

 
%%%%%%%%%%%%%%%%%%%%%%%%%%%%%%%%%%%%%%%%%%%%%%%%%%%%%%%%%%%%%%%%%%%%%%%%%%%%%
%%%%%%%%%%%%%%%%%%%%%%%%%%%%%%%%%%%%%%%%%%%%%%%%%%%%%%%%%%%%%%%%%%%%%%%%%%%%%
%%%%%%%%%%%%%%%%%%%%%%%%%%%%%%%%%%%%%%%%%%%%%%%%%%%%%%%%%%%%%%%%%%%%%%%%%%%%%
 

 
\section{Final discussion}
\label{S:disc}

We have described an OOP system for Haskell that supports stateful
objects, inheritance and subtype polymorphism. The Shapes Benchmark
demonstrated that our encoding is very close to the textbook OO code
(usually given in C++ or Java tutorials), with pleasant
deviations. The re-implementation of examples from the OCaml object
tutorial demonstrated the faithfulness of our realisation of
objects. (We have opted for OCaml because it is a leading object
system in a functional setting.) We have implemented parameterised
classes, constructor methods, abstract classes, pure virtual methods,
single and multiple inheritance with flexible rules of sharing or
separation of superclasses. Major byproducts are these: extensive type
inference, first-class classes, sharing of the code of methods among
even non-related classes, implicit polymorphism of classes, fully
static type checking.

We have clarified the relation between subtyping and type-class based
polymorphism: the latter can encode the former. We have implemented
OOHaskell with the existing Haskell implementation (GHC), requiring no
extra extensions beyond the commonly implemented ones: multi-parameter
type classes with functional dependencies. It is quite pleasant that
the existing OOHaskell code does not seem to be a burden to
write~---~even in the absence of any syntactic sugar.  Once we
consider open recursion in our setup, it turns out that the object
constructor (i.e., `new') is merely the monadic fix-point operation
(i.e., `mfix'). Just as people have known all the time~---~OO and
recursion are intertwined.

There exists a large body of literature,
e.g.,~\cite{Poll97,AC96,Ohori95,Remy94a,PT94,BM92}. Most often
discussed are pure functional objects. Most often the type systems of
object models are variants of system $F_{\leq}$ (polymorphic
lambda-calculus plus subtyping). Most often objects with open
recursion are represented either with recursive types, or with
existentially quantified types. In the present paper we demonstrated
the encoding of imperative objects with inheritance and polymorphism
in Haskell~---~that is, polymorphic lambda-calculus plus
multi-parameter type classes with functional dependencies. Unlike
$F_{\leq}$, there is no built-in subtyping relation. It is remarkable
that our encoding of imperative objects avoids both recursive types
and existentially quantified types. A particularly interesting case
are functional objects, which seem to require recursive or
existentially-quantified types. Functional objects and preservation of
type inference is to be discussed in a forthcoming paper.  Our current
implementation has strong similarities with prototype-based systems
(such as Self~\cite{Self}) in that mutable fields and method
`pointers' are a part of the same record. This does not have to be the
case~---~and in fact, in the forthcoming paper on pure-functional
objects we separate the two tables (in the manner similar to object
realisations in C++ or Java).

There are some further idioms that complement OOHaskell as a faithful
OOP system.  For instance, we need to take measures to do
instantiation checking for classes at declaration time. Otherwise some
type errors would go unnoticed until the first attempt to programme an
instantiation. We can also model various forms of private and
protected methods and other access modifiers. We can clone objects, we
can let methods return `self', we can operate in the ST monad rather
than the IO monad. The article comes with an extensive collection of
source code~\cite{OOHaskell}, where these and other issues are
covered. The source code also illustrates some cases of depth
subtyping~\cite{Poll97} and statically safe argument
covariance. However, the extent and limitations of our handling of
depth subtyping remains the subject of ongoing research.

We are currently working on some elaborations and advanced topics.
Simple syntactic sugar would make OOP more convenient in Haskell.
Extra effort is needed to provide OOP-like error messages.  This is a
sticking issue that requires major effort, but there is a line of
research being carried out by Sulzmann and others~\cite{SSW04}. A
non-trivial case study is required to demonstrate the scalability of
the approach. The mere compilation time of OOHaskell programs and
their runtime efficiency is challenged by the huge dictionaries that
are implied by our type-class-based approach.  It seems that some
dedicated optimisations will be needed in order to handle the
\HList/OOHaskell style of programming efficiently. An interesting
advanced topic is reflective programming. A simple form of reflection
is readily provided in terms of the type-level encoding of
records. One can iterate over records and their components in a
generic fashion. Other forms of reflection, such as iteration over the
object pool, as needed for dynamic aspect-oriented programming,
requires further effort. Another challenge is to capture reusable
solutions for design problems (as part of design patterns) in Haskell.



%%%%%%%%%%%%%%%%%%%%%%%%%%%%%%%%%%%%%%%%%%%%%%%%%%%%%%%%%%%%%%%%%%%%%%%%%%%%%
%%%%%%%%%%%%%%%%%%%%%%%%%%%%%%%%%%%%%%%%%%%%%%%%%%%%%%%%%%%%%%%%%%%%%%%%%%%%%
%%%%%%%%%%%%%%%%%%%%%%%%%%%%%%%%%%%%%%%%%%%%%%%%%%%%%%%%%%%%%%%%%%%%%%%%%%%%%
