


%%%%%%%%%%%%%%%%%%%%%%%%%%%%%%%%%%%%%%%%%%%%%%%%%%%%%%%%%%%%%%%%%%%%%%%%%%%%%
%%%%%%%%%%%%%%%%%%%%%%%%%%%%%%%%%%%%%%%%%%%%%%%%%%%%%%%%%%%%%%%%%%%%%%%%%%%%%
%%%%%%%%%%%%%%%%%%%%%%%%%%%%%%%%%%%%%%%%%%%%%%%%%%%%%%%%%%%%%%%%%%%%%%%%%%%%%



\section{Heterogeneous collections}
\label{S:HList}

The OOHaskell approach adopts typeful heterogeneous
collections~\cite{HLIST-HW04} for the representation of objects, and
for other purposes. In fact, all sorts of heterogeneous collections
are derived from heterogeneous list. This observation gave the name to
the Haskell library for heterogeneous collections: \HList. The
heterogeneity of these lists allows us to line up objects with
components of different types. The typefulness of these lists allows
us to enforce constraints such as uniqueness of labels and subtyping
for record extension. We will rehash the essentials of \HList{}s here,
while focusing on the bits that are most relevant for OOHaskell. We
refer to~\cite{HLIST-HW04} for a more general treatment.



%%%%%%%%%%%%%%%%%%%%%%%%%%%%%%%%%%%%%%%%%%%%%%%%%%%%%%%%%%%%%%%%%%%%%%%%%%%%%
%%%%%%%%%%%%%%%%%%%%%%%%%%%%%%%%%%%%%%%%%%%%%%%%%%%%%%%%%%%%%%%%%%%%%%%%%%%%%
%%%%%%%%%%%%%%%%%%%%%%%%%%%%%%%%%%%%%%%%%%%%%%%%%%%%%%%%%%%%%%%%%%%%%%%%%%%%%



\subsection{Heterogeneous lists}

These are the basic constructors of the \HList\ library:

\begin{code}
 data HNil      = HNil       deriving (Eq,Show,Read)
 data HCons e l = HCons e l  deriving (Eq,Show,Read)
\end{code}

\noindent
That is, there are two datatypes constructors @HNil@ and @HCons@
corresponding to the two constructors @[]@ and @(:)@ of the normal
list datatype. The chosen style of type parameterisation allows for
list elements of different types. In fact, the two datatype
constructors are isomorphic to the type constructors for empty
products and binary products, but fresh symbols or chosen to avoid
confusion. We only want to use @HNil@ and @HCons@ for the construction
of nested, binary, right-associative products~---~as expressed by the
following type class for \HList{}s:

\begin{code}
 class HList l
 instance HList HNil
 instance HList l => HList (HCons e l)
\end{code}

\noindent
Using type-level programming (aka type-class-based programming), we can
reify all kinds of list-processing functions on \HList{}s. Each such
reified operation is placed in a dedicated type class whose instances
discriminate on the type structure in the same way as the normal
operations were defined by case discrimination (pattern matching) on
algebraic data. For instance, a type-level append gives rise to a
class like this:

\begin{code}
 class (HList l, HList l', HList l'')
    =>  HAppend l l' l'' | l l' -> l''
  where hAppend :: l -> l' -> l''
\end{code}

\noindent
(We omit the two instances, which will perform induction on the first HList.)



%%%%%%%%%%%%%%%%%%%%%%%%%%%%%%%%%%%%%%%%%%%%%%%%%%%%%%%%%%%%%%%%%%%%%%%%%%%%%
%%%%%%%%%%%%%%%%%%%%%%%%%%%%%%%%%%%%%%%%%%%%%%%%%%%%%%%%%%%%%%%%%%%%%%%%%%%%%
%%%%%%%%%%%%%%%%%%%%%%%%%%%%%%%%%%%%%%%%%%%%%%%%%%%%%%%%%%%%%%%%%%%%%%%%%%%%%

\oleg{I propose to merge this and the following section. Here's the
  outline of the new section.}

\subsection{Record calculus and subtyping}

OOHaskell's objects are modelled as records. The current version of
the \HList\ library models records simply as lists of label-value
pairs. (Other realisations, e.g., a pair of lists or a tree, are
certainly possible.) Labels are distinguished by their
\emph{types}. The \HList\ library offers several models of labels. We
will pick a model shortly. We use an extra newtype @Record@ to tag
lists of pairs as records:

\begin{code}
 -- Non-public constructor
 newtype Record r = Record r
 -- Public constructor with constraints
 mkRecord :: HRLabelSet r => r -> Record r
 mkRecord = Record
 emptyRecord = mkRecord $ HNil
\end{code}

\noindent
The set property for labels (cf.\ class @HRLabelSet@) \emph{statically}
assures that all labels occur in the list of record's label-value
pairs exactly once.
We use an infix operator @.=.@ to form a label-value pair.  We use
another infix operator @.*.@ to build up records. That is, suppose
@l1@, @l2@ are labels, and @v1@, @v2@ are values, then we construct a
record as follows:

\begin{code}
 myRecord =  l1 .=. v1
         .*. l2 .=. v2
         .*. emptyRecord
\end{code}

\noindent
The operator @.=.@ coincides with pair construction: |l .=. v = (l,v)|.
The operator @.*.@ models record extension:

\begin{code}
 (l,v) .*. (Record r) = mkRecord (HCons (l,v) r)
\end{code}

\noindent
We can define all normal record operations: look-up, update, label
renaming, and others. (Cf.\ App.~\ref{A:hTPupdateAtLabel} for sample
definition: a type-preserving update operation.) We can also define
(different kinds of) subtyping on records:

\begin{code}
 class SubType l l'
 instance ( HZip ls vs r'
      , HProjectByLabels ls (Record r) (Record r') )
  =>    SubType (Record r) (Record r')
\end{code}

\noindent
That is, a record type $r$ is a subtype of some record type $r'$ if
$r$ contains at least the labels of $r'$, and the value types for the
shared labels are the same. Subtyping is defined in terms of
projection on records, while labels are used to control projection.
We do not consider any sort of co-/contra-variance for the value types
here, but we could do so without ado, including iteration of arrow
types. Thereby, we can accommodate variance properties as needed for
OO subtyping with co- or contra-variant method arguments. For
instance, we can encode OCaml's approach to variance.


\subsection{Oleg: This is the old text for the two sections.}

\subsection{Representation of records}

OOHaskell's objects are going to be modelled as records. The current
version of the \HList\ library models records simply as lists of
label-value pairs. (Other realisations, e.g., a pair of lists or a
tree, are certainly possible.) Labels are distinguished by their
\emph{types}. The \HList\ library offers several models of labels. We
will pick a model shortly.

Records can be zipped from two lists: one list for the labels, another
list for the values. Records can also be unzipped. These are just two
reified list-processing functions hosted by the class @HZip@:

\begin{code}
 class HZip x y l | x y -> l, l -> x y
  where hZip   :: x -> y -> l
        hUnzip :: l -> (x,y)
\end{code}

\noindent
We use an extra newtype @Record@ to tag lists of pairs as records:

\begin{code}
 -- Non-public constructor
 newtype Record r = Record r
\end{code}
\begin{code}
 -- Public constructor with constraints
 mkRecord :: (HZip ls vs r, HLabelSet ls) => r -> Record r
 mkRecord = Record
\end{code}
\begin{code}
 -- Empty record construction
 emptyRecord = mkRecord $ hZip HNil HNil
\end{code}

\noindent
The set property for labels (cf.\ class @HLabelSet@) is defined by
iteration over the list of labels such that the heading label does not
occur in the tail of the list, and the tail meets the set property as
well. That is:

\begin{code}
 class HLabelSet ls
 instance HLabelSet HNil
 instance (HMember x ls HFalse, HLabelSet ls)
       =>  HLabelSet (HCons x ls)
 -- Membership-test uses type-level equality on labels.
\end{code}



%%%%%%%%%%%%%%%%%%%%%%%%%%%%%%%%%%%%%%%%%%%%%%%%%%%%%%%%%%%%%%%%%%%%%%%%%%%%%
%%%%%%%%%%%%%%%%%%%%%%%%%%%%%%%%%%%%%%%%%%%%%%%%%%%%%%%%%%%%%%%%%%%%%%%%%%%%%
%%%%%%%%%%%%%%%%%%%%%%%%%%%%%%%%%%%%%%%%%%%%%%%%%%%%%%%%%%%%%%%%%%%%%%%%%%%%%



\subsection{Record calculus and subtyping}

We use an infix operator @.=.@ to form a label-value pair.  We use
another infix operator @.*.@ to line up records. That is, suppose
@l1@, @l2@ are labels, and @v1@, @v2@ are values, then we construct a
record as follows:

\begin{code}
 myRecord =  l1 .=. v1
         .*. l2 .=. v2
         .*. emptyRecord
\end{code}

\noindent
The operator @.=.@ coincides with pair construction:
\begin{code}
 l .=. v = (l,v) 
\end{code}

\noindent
The operator @.*.@ models record extension:

\begin{code}
 (l,v) .*. (Record r) = mkRecord r'
    where
     (ls,vs) = hUnzip r
     r'      = hZip (HCons l ls) (HCons v vs)
\end{code}

\noindent
We can define all normal record operations: look-up, update, label
renaming, and others. (Cf.\ App.~\ref{A:hTPupdateAtLabel} for sample
definition: a type-preserving update operation.) We can also define
(different kinds of) subtyping on records:

\begin{code}
 class SubType l l'
 instance ( HZip ls vs r'
      , HProjectByLabels ls (Record r) (Record r') )
  =>    SubType (Record r) (Record r')
\end{code}

\noindent
That is, a record type $r$ is a subtype of some record type $r'$ if
$r$ contains at least the labels of $r'$, and the value types for the
shared labels are the same. Subtyping is defined in terms of
projection on records, while labels are used to control projection.
We do not consider any sort of co-/contra-variance for the value types
here, but we could do so without ado, including iteration of arrow
types. Thereby, we can accommodate variance properties as needed for
OO subtyping with co- or contra-variant method arguments. For
instance, we can encode OCaml's approach to variance.



%%%%%%%%%%%%%%%%%%%%%%%%%%%%%%%%%%%%%%%%%%%%%%%%%%%%%%%%%%%%%%%%%%%%%%%%%%%%%
%%%%%%%%%%%%%%%%%%%%%%%%%%%%%%%%%%%%%%%%%%%%%%%%%%%%%%%%%%%%%%%%%%%%%%%%%%%%%
%%%%%%%%%%%%%%%%%%%%%%%%%%%%%%%%%%%%%%%%%%%%%%%%%%%%%%%%%%%%%%%%%%%%%%%%%%%%%



\subsection{Declaration of first-class labels}

Generally, labels are distinguished by their \emph{types}. The \HList\
library readily offers 4 different models of labels. A very simple
model of generating label types is to employ type-level naturals:
@HZero@, @HSucc HZero@, @HSucc (HSucc HZero)@, \ldots.  For the sake
of the programmer's convenience, it is preferable to use label types
whose type names are readily meaningful. We will consider here one
such option.

We model labels as very carefully typed \emph{undefined} values that
are proxies for distinguished and trivial datatypes. Let's suppose we
need record types that involve labels @xbase@, @ybase@, and
@radius@. Using the prime model of this paper, we declare these labels
as follows.

\begin{code}
 data Xbase;  xbase  = proxy::Proxy Xbase
 data Ybase;  ybase  = proxy::Proxy Ybase
 data Radius; radius = proxy::Proxy Radius
\end{code}

\noindent
(We note that simple syntactic sugar can reduce the length of these
one-liners dramatically in case this is considered an issue.)  That
is, we introduce dedicated types @Xbase@, @Ybase@, @Radius@ for the
different labels. These types are `empty' in the sense that
\undefined\ is the only inhabitant of these types. (There are no
constructors!) We use the phantom type @Proxy@ for type proxies to
represent labels as proxies for empty types. The \HList\ library
defines type proxies as follows:

\begin{Verbatim}[fontsize=\small,commandchars=\\\{\}]
 data Proxy e; proxy :: Proxy e; proxy = \undefined
\end{Verbatim}

\noindent
The explicit declaration of labels blends perfectly with Haskell's
scoping rules and its module concept. If different modules with
various record types want to share labels, then they have to agree on
a declaration site that they all import. All models of \HList\ labels
support labels as first-class citizens~---~we can pass them to
functions etc.



%%%%%%%%%%%%%%%%%%%%%%%%%%%%%%%%%%%%%%%%%%%%%%%%%%%%%%%%%%%%%%%%%%%%%%%%%%%%%
%%%%%%%%%%%%%%%%%%%%%%%%%%%%%%%%%%%%%%%%%%%%%%%%%%%%%%%%%%%%%%%%%%%%%%%%%%%%%
%%%%%%%%%%%%%%%%%%%%%%%%%%%%%%%%%%%%%%%%%%%%%%%%%%%%%%%%%%%%%%%%%%%%%%%%%%%%%
