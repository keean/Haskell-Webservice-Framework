\documentclass[onecolumn,11pt,preprint]{sigplanconf}

% \documentclass{llncs}
% \documentclass{article}

\usepackage{amsmath}
\usepackage{times}
\usepackage{comment}
\usepackage{url}
\usepackage{fancyvrb}
\usepackage{graphics}
\usepackage{boxedminipage}
\usepackage{code}

\usepackage[bookmarks=false,%
            colorlinks,linkcolor=black,urlcolor=blue,%
            pdfauthor={Oleg and Keean and Ralf},%
            pdftitle={OOHaskell}]{hyperref}


\DefineShortVerb{\|}
%\DefineVerbatimEnvironment{code}{Verbatim}{xleftmargin=\mathindent,commandchars=\\\{\},fontsize=\small}
% \setlength{\parskip}{0pt}
%\newlength{\mathindent}
%\setlength{\mathindent}{1em}


% Macros for notes to each other in the text
\newcommand{\keean}[1]{{\it [Keean says: #1]}}
\newcommand{\oleg}[1]{{\it [Oleg says: #1]}}
\newcommand{\ralf}[1]{{\it [Ralf says: #1]}}

\newcommand{\mysize}{\small}
\newcommand{\myskip}{\smallskip}
\newcommand{\antiskip}{\vspace{-25\in}}
\newcommand{\myinput}[1]{The content of this section is not yet released.}
\newcommand{\HList}{\textsc{HList}}
\newcommand{\undefined}{\ensuremath{\bot}}
\newcommand{\Forall}{\ensuremath{\forall}}

\setlength{\parskip}{0pt}
\setlength{\parsep}{0pt}
 


\begin{document}

%\conferenceinfo{WXYZ '05}{date, City.} 
%\copyrightyear{2005} 
%\copyrightdata{supplied by printer} 
 
\title{Haskell's overlooked object system\\
{\small ---~DRAFT OF \today~---}\vspace{-77\in}}

\authorinfo{Oleg Kiselyov}{FNMO Center, Monterey, CA}{}
\authorinfo{Ralf L{\"a}mmel}{Microsoft Corp., Redmond, WA}{}
\authorinfo{Keean Schupke}{Imperial College, London, UK}{}


\maketitle

\begin{abstract}

Haskell provides type-class-based bounded polymorphism as opposed to
subtype polymorphism of object-oriented languages such as Java and
OCaml. It is a contentious question whether Haskell (alone or with
extensions) can support conventional object-oriented programming with
encapsulation, inheritance, overriding, statically checked subtyping,
and so on.

\medskip

Recently it has been shown that multi-parameter type classes with
functional dependencies alone support the record calculus for
extensible polymorphic records with first-class labels and
subtyping. This paper builds upon this result and shows that Haskell
readily supports objects encapsulating methods and mutable fields,
inheritance, overriding, constructor methods, abstract classes,
lexically scoped classes, in fact, first-class classes, and others. We
provide open recursion, and the ability to close it selectively, thus
giving the programmer full control over sharing or isolation of base
classes in multiple-inheritance hierarchies. It is especially pleasing
that the class instantiator is just the (monadic) fix-point
combinator.

\medskip

We work through a series of working OO examples as they are commonly
found in OO textbooks and programming language tutorials. In
particular, we elucidate a translation of OCaml's object syntax to
Haskell, where we do not perturb the OCaml code: we translate
expression-by-expression without a need for global parameters. More
generally, we aim to provide OO idioms in Haskell in an OO
intuition-preserving way. The resulting combination of OO idioms,
higher-order functional programming, and type inference is
comparatively powerful. Hence, `OOHaskell' lends itself as a prime
environment for typed OO language design.

\end{abstract}

\makeatactive



%%%%%%%%%%%%%%%%%%%%%%%%%%%%%%%%%%%%%%%%%%%%%%%%%%%%%%%%%%%%%%%%%%%%%%%%%%%%%
%%%%%%%%%%%%%%%%%%%%%%%%%%%%%%%%%%%%%%%%%%%%%%%%%%%%%%%%%%%%%%%%%%%%%%%%%%%%%
%%%%%%%%%%%%%%%%%%%%%%%%%%%%%%%%%%%%%%%%%%%%%%%%%%%%%%%%%%%%%%%%%%%%%%%%%%%%%



\section{Introduction}

Is Haskell fit for OOP? This topic is raised time and again on
functional programming mailing lists, on websites, and in verbal
communication with remarkable regularity. There appear to be several
reasons for such popularity:

\smallskip

\begin{itemize}

\item In an \emph{intellectual} sense, we may wonder whether Haskell's
advanced type system is expressive enough to model object types,
inheritance, subtyping, virtual methods, etc. There is an established
scientific pessimism regarding this question.

\smallskip

\item In a \emph{practical} sense, can we faithfully transport
imperative OO designs from, say, C++, Java, C\#, VB to Haskell --
without totally rewriting the design and without foreign-language
interfacing?

\smallskip

\item From a \emph{language design} perspective, we know how to use
Haskell for prototyping semantics and for encoding various abstraction
mechanisms, but can we also leverage Haskell as a sandbox for design
of typed object-oriented languages so that we can play with new ideas
without the immediate need to write or modify a compiler.

\smallskip

\item In an \emph{educational} sense, is there anything important that
more or less advanced functional and object-oriented programmers can
learn about Haskell's type system and about OO by looking into the
pros and cons of different OO encoding options?

\end{itemize}

\smallskip

This paper delivers substantiated, positive answers to these
questions. We describe OOHaskell~---~a Haskell-based library for (as
of today: imperative) OO programming in Haskell. OOHaskell uncovers
Haskell's overlooked object system. The key to this result is a good
deal of exploitation of Haskell's breath-taking type system
\emph{combined} with a careful identification of a suitable object
encoding. OOHaskell builds upon our previous work on heterogeneous
collections~\cite{HLIST-HW04} (the HList library), while we had to
discover several new techniques to make non-trivial objects work.


\medskip

\subsection*{Road-map of this paper}

\begin{itemize}
\item Sec.~\ref{S:shapes}: We walk through an illustrative OO example.
\item Sec.~\ref{S:rationale}: We give a detailed rationale for OOHaskell.
\item Sec.~\ref{S:basics}: We describe basic OOHaskell programming idioms.
\item Sec.~\ref{S:types}: We discuss typing issues~---~inference, errors, etc.
\item Sec.~\ref{S:strengths}: We highlight some of the strengths of OOHaskell.
\item Sec.~\ref{S:related}: We review related work.
\item Sec.~\ref{S:concl}: The paper is concluded.
\end{itemize}
%
There is an extended code distribution coming with this paper~\cite{OOHaskell}.

\begin{comment}

In Sec.~\ref{S:HList}, we briefly review the \HList\
library~\cite{HLIST-HW04}, which provides extensible polymorphic
heterogeneous records with first-class labels.  In
Sec.~\ref{S:simple}, we introduce more basic OO notions such as
objects and constructors. In Sec.~\ref{S:self}, we describe open
recursion, which allows us to cover rich forms of inheritance.  In
Sec.~\ref{S:shapes}, we handle a prototypical scenario for subtype
polymorphism in detail. In~\ref{S:disc}, we very briefly discuss all
remaining issues~---~including some technicalities, conclusions, and
directions for future work.

\end{comment}



%%%%%%%%%%%%%%%%%%%%%%%%%%%%%%%%%%%%%%%%%%%%%%%%%%%%%%%%%%%%%%%%%%%%%%%%%%%%%
%%%%%%%%%%%%%%%%%%%%%%%%%%%%%%%%%%%%%%%%%%%%%%%%%%%%%%%%%%%%%%%%%%%%%%%%%%%%%
%%%%%%%%%%%%%%%%%%%%%%%%%%%%%%%%%%%%%%%%%%%%%%%%%%%%%%%%%%%%%%%%%%%%%%%%%%%%%



\section{Heterogeneous collections}
\label{S:HList}

The OOHaskell approach adopts typeful heterogeneous
collections~\cite{HLIST-HW04} for the representation of objects, and
for other purposes. In fact, all sorts of heterogeneous collections
are derived from heterogeneous list. This observation gave the name to
the Haskell library for heterogeneous collections: \HList. The
heterogeneity of these lists allows us to line up objects with
components of different types. The typefulness of these lists allows
us to enforce constraints such as uniqueness of labels and subtyping
for record extension. We will rehash the essentials of \HList{}s here,
while focusing on the bits that are most relevant for OOHaskell. We
refer to~\cite{HLIST-HW04} for a more general treatment.



%%%%%%%%%%%%%%%%%%%%%%%%%%%%%%%%%%%%%%%%%%%%%%%%%%%%%%%%%%%%%%%%%%%%%%%%%%%%%
%%%%%%%%%%%%%%%%%%%%%%%%%%%%%%%%%%%%%%%%%%%%%%%%%%%%%%%%%%%%%%%%%%%%%%%%%%%%%
%%%%%%%%%%%%%%%%%%%%%%%%%%%%%%%%%%%%%%%%%%%%%%%%%%%%%%%%%%%%%%%%%%%%%%%%%%%%%



\subsection{Heterogeneous lists}

These are the basic constructors of the \HList\ library:

\begin{code}
 data HNil      = HNil       deriving (Eq,Show,Read)
 data HCons e l = HCons e l  deriving (Eq,Show,Read)
\end{code}

\noindent
That is, there are two datatypes constructors @HNil@ and @HCons@
corresponding to the two constructors @[]@ and @(:)@ of the normal
list datatype. The chosen style of type parameterisation allows for
list elements of different types. In fact, the two datatype
constructors are isomorphic to the type constructors for empty
products and binary products, but fresh symbols or chosen to avoid
confusion. We only want to use @HNil@ and @HCons@ for the construction
of nested, binary, right-associative products~---~as expressed by the
following type class for \HList{}s:

\begin{code}
 class HList l
 instance HList HNil
 instance HList l => HList (HCons e l)
\end{code}

\noindent
Using type-level programming (aka type-class-based programming), we can
reify all kinds of list-processing functions on \HList{}s. Each such
reified operation is placed in a dedicated type class whose instances
discriminate on the type structure in the same way as the normal
operations were defined by case discrimination (pattern matching) on
algebraic data. For instance, a type-level append gives rise to a
class like this:

\begin{code}
 class (HList l, HList l', HList l'')
    =>  HAppend l l' l'' | l l' -> l''
  where hAppend :: l -> l' -> l''
\end{code}

\noindent
(We omit the two instances, which will perform induction on the first HList.)



%%%%%%%%%%%%%%%%%%%%%%%%%%%%%%%%%%%%%%%%%%%%%%%%%%%%%%%%%%%%%%%%%%%%%%%%%%%%%
%%%%%%%%%%%%%%%%%%%%%%%%%%%%%%%%%%%%%%%%%%%%%%%%%%%%%%%%%%%%%%%%%%%%%%%%%%%%%
%%%%%%%%%%%%%%%%%%%%%%%%%%%%%%%%%%%%%%%%%%%%%%%%%%%%%%%%%%%%%%%%%%%%%%%%%%%%%

\oleg{I propose to merge this and the following section. Here's the
  outline of the new section.}

\subsection{Record calculus and subtyping}

OOHaskell's objects are modelled as records. The current version of
the \HList\ library models records simply as lists of label-value
pairs. (Other realisations, e.g., a pair of lists or a tree, are
certainly possible.) Labels are distinguished by their
\emph{types}. The \HList\ library offers several models of labels. We
will pick a model shortly. We use an extra newtype @Record@ to tag
lists of pairs as records:

\begin{code}
 -- Non-public constructor
 newtype Record r = Record r
 -- Public constructor with constraints
 mkRecord :: HRLabelSet r => r -> Record r
 mkRecord = Record
 emptyRecord = mkRecord $ HNil
\end{code}

\noindent
The set property for labels (cf.\ class @HRLabelSet@) \emph{statically}
assures that all labels occur in the list of record's label-value
pairs exactly once.
We use an infix operator @.=.@ to form a label-value pair.  We use
another infix operator @.*.@ to build up records. That is, suppose
@l1@, @l2@ are labels, and @v1@, @v2@ are values, then we construct a
record as follows:

\begin{code}
 myRecord =  l1 .=. v1
         .*. l2 .=. v2
         .*. emptyRecord
\end{code}

\noindent
The operator @.=.@ coincides with pair construction: |l .=. v = (l,v)|.
The operator @.*.@ models record extension:

\begin{code}
 (l,v) .*. (Record r) = mkRecord (HCons (l,v) r)
\end{code}

\noindent
We can define all normal record operations: look-up, update, label
renaming, and others. (Cf.\ App.~\ref{A:hTPupdateAtLabel} for sample
definition: a type-preserving update operation.) We can also define
(different kinds of) subtyping on records:

\begin{code}
 class SubType l l'
 instance ( HZip ls vs r'
      , HProjectByLabels ls (Record r) (Record r') )
  =>    SubType (Record r) (Record r')
\end{code}

\noindent
That is, a record type $r$ is a subtype of some record type $r'$ if
$r$ contains at least the labels of $r'$, and the value types for the
shared labels are the same. Subtyping is defined in terms of
projection on records, while labels are used to control projection.
We do not consider any sort of co-/contra-variance for the value types
here, but we could do so without ado, including iteration of arrow
types. Thereby, we can accommodate variance properties as needed for
OO subtyping with co- or contra-variant method arguments. For
instance, we can encode OCaml's approach to variance.


\subsection{Oleg: This is the old text for the two sections.}

\subsection{Representation of records}

OOHaskell's objects are going to be modelled as records. The current
version of the \HList\ library models records simply as lists of
label-value pairs. (Other realisations, e.g., a pair of lists or a
tree, are certainly possible.) Labels are distinguished by their
\emph{types}. The \HList\ library offers several models of labels. We
will pick a model shortly.

Records can be zipped from two lists: one list for the labels, another
list for the values. Records can also be unzipped. These are just two
reified list-processing functions hosted by the class @HZip@:

\begin{code}
 class HZip x y l | x y -> l, l -> x y
  where hZip   :: x -> y -> l
        hUnzip :: l -> (x,y)
\end{code}

\noindent
We use an extra newtype @Record@ to tag lists of pairs as records:

\begin{code}
 -- Non-public constructor
 newtype Record r = Record r
\end{code}
\begin{code}
 -- Public constructor with constraints
 mkRecord :: (HZip ls vs r, HLabelSet ls) => r -> Record r
 mkRecord = Record
\end{code}
\begin{code}
 -- Empty record construction
 emptyRecord = mkRecord $ hZip HNil HNil
\end{code}

\noindent
The set property for labels (cf.\ class @HLabelSet@) is defined by
iteration over the list of labels such that the heading label does not
occur in the tail of the list, and the tail meets the set property as
well. That is:

\begin{code}
 class HLabelSet ls
 instance HLabelSet HNil
 instance (HMember x ls HFalse, HLabelSet ls)
       =>  HLabelSet (HCons x ls)
 -- Membership-test uses type-level equality on labels.
\end{code}



%%%%%%%%%%%%%%%%%%%%%%%%%%%%%%%%%%%%%%%%%%%%%%%%%%%%%%%%%%%%%%%%%%%%%%%%%%%%%
%%%%%%%%%%%%%%%%%%%%%%%%%%%%%%%%%%%%%%%%%%%%%%%%%%%%%%%%%%%%%%%%%%%%%%%%%%%%%
%%%%%%%%%%%%%%%%%%%%%%%%%%%%%%%%%%%%%%%%%%%%%%%%%%%%%%%%%%%%%%%%%%%%%%%%%%%%%



\subsection{Record calculus and subtyping}

We use an infix operator @.=.@ to form a label-value pair.  We use
another infix operator @.*.@ to line up records. That is, suppose
@l1@, @l2@ are labels, and @v1@, @v2@ are values, then we construct a
record as follows:

\begin{code}
 myRecord =  l1 .=. v1
         .*. l2 .=. v2
         .*. emptyRecord
\end{code}

\noindent
The operator @.=.@ coincides with pair construction:
\begin{code}
 l .=. v = (l,v) 
\end{code}

\noindent
The operator @.*.@ models record extension:

\begin{code}
 (l,v) .*. (Record r) = mkRecord r'
    where
     (ls,vs) = hUnzip r
     r'      = hZip (HCons l ls) (HCons v vs)
\end{code}

\noindent
We can define all normal record operations: look-up, update, label
renaming, and others. (Cf.\ App.~\ref{A:hTPupdateAtLabel} for sample
definition: a type-preserving update operation.) We can also define
(different kinds of) subtyping on records:

\begin{code}
 class SubType l l'
 instance ( HZip ls vs r'
      , HProjectByLabels ls (Record r) (Record r') )
  =>    SubType (Record r) (Record r')
\end{code}

\noindent
That is, a record type $r$ is a subtype of some record type $r'$ if
$r$ contains at least the labels of $r'$, and the value types for the
shared labels are the same. Subtyping is defined in terms of
projection on records, while labels are used to control projection.
We do not consider any sort of co-/contra-variance for the value types
here, but we could do so without ado, including iteration of arrow
types. Thereby, we can accommodate variance properties as needed for
OO subtyping with co- or contra-variant method arguments. For
instance, we can encode OCaml's approach to variance.



%%%%%%%%%%%%%%%%%%%%%%%%%%%%%%%%%%%%%%%%%%%%%%%%%%%%%%%%%%%%%%%%%%%%%%%%%%%%%
%%%%%%%%%%%%%%%%%%%%%%%%%%%%%%%%%%%%%%%%%%%%%%%%%%%%%%%%%%%%%%%%%%%%%%%%%%%%%
%%%%%%%%%%%%%%%%%%%%%%%%%%%%%%%%%%%%%%%%%%%%%%%%%%%%%%%%%%%%%%%%%%%%%%%%%%%%%



\subsection{Declaration of first-class labels}

Generally, labels are distinguished by their \emph{types}. The \HList\
library readily offers 4 different models of labels. A very simple
model of generating label types is to employ type-level naturals:
@HZero@, @HSucc HZero@, @HSucc (HSucc HZero)@, \ldots.  For the sake
of the programmer's convenience, it is preferable to use label types
whose type names are readily meaningful. We will consider here one
such option.

We model labels as very carefully typed \emph{undefined} values that
are proxies for distinguished and trivial datatypes. Let's suppose we
need record types that involve labels @xbase@, @ybase@, and
@radius@. Using the prime model of this paper, we declare these labels
as follows.

\begin{code}
 data Xbase;  xbase  = proxy::Proxy Xbase
 data Ybase;  ybase  = proxy::Proxy Ybase
 data Radius; radius = proxy::Proxy Radius
\end{code}

\noindent
(We note that simple syntactic sugar can reduce the length of these
one-liners dramatically in case this is considered an issue.)  That
is, we introduce dedicated types @Xbase@, @Ybase@, @Radius@ for the
different labels. These types are `empty' in the sense that
\undefined\ is the only inhabitant of these types. (There are no
constructors!) We use the phantom type @Proxy@ for type proxies to
represent labels as proxies for empty types. The \HList\ library
defines type proxies as follows:

\begin{Verbatim}[fontsize=\small,commandchars=\\\{\}]
 data Proxy e; proxy :: Proxy e; proxy = \undefined
\end{Verbatim}

\noindent
The explicit declaration of labels blends perfectly with Haskell's
scoping rules and its module concept. If different modules with
various record types want to share labels, then they have to agree on
a declaration site that they all import. All models of \HList\ labels
support labels as first-class citizens~---~we can pass them to
functions etc.



%%%%%%%%%%%%%%%%%%%%%%%%%%%%%%%%%%%%%%%%%%%%%%%%%%%%%%%%%%%%%%%%%%%%%%%%%%%%%
%%%%%%%%%%%%%%%%%%%%%%%%%%%%%%%%%%%%%%%%%%%%%%%%%%%%%%%%%%%%%%%%%%%%%%%%%%%%%
%%%%%%%%%%%%%%%%%%%%%%%%%%%%%%%%%%%%%%%%%%%%%%%%%%%%%%%%%%%%%%%%%%%%%%%%%%%%%




%%%%%%%%%%%%%%%%%%%%%%%%%%%%%%%%%%%%%%%%%%%%%%%%%%%%%%%%%%%%%%%%%%%%%%%%%%%%%
%%%%%%%%%%%%%%%%%%%%%%%%%%%%%%%%%%%%%%%%%%%%%%%%%%%%%%%%%%%%%%%%%%%%%%%%%%%%%
%%%%%%%%%%%%%%%%%%%%%%%%%%%%%%%%%%%%%%%%%%%%%%%%%%%%%%%%%%%%%%%%%%%%%%%%%%%%%



\section{Simple objects and classes}
\label{S:simple}

We start with the very bascis of class-based OO: objects as capsules
of mutable data and methods. Objects are constructed from object
templates (or classes). The construction process can be parameterised,
which eventually leads to the concept of constructor methods.  Object
templates (or constructor methods) can perform extra computations at
object creation time, and they can even maintain state on their own.

\myskip

\noindent
The following batch of examples is adopted from the first section of
the OCaml tutorial~\cite[\S\,3.1]{OCaml}. We are keen to mimic the
OCaml in some cases because this suggests a direct, local translation.
The source code distribution for this paper contains many additional
examples~\cite{OOHaskell}.



%%%%%%%%%%%%%%%%%%%%%%%%%%%%%%%%%%%%%%%%%%%%%%%%%%%%%%%%%%%%%%%%%%%%%%%%%%%%%
%%%%%%%%%%%%%%%%%%%%%%%%%%%%%%%%%%%%%%%%%%%%%%%%%%%%%%%%%%%%%%%%%%%%%%%%%%%%%
%%%%%%%%%%%%%%%%%%%%%%%%%%%%%%%%%%%%%%%%%%%%%%%%%%%%%%%%%%%%%%%%%%%%%%%%%%%%%



\subsection{Packaging mutable data and methods}

Quoting from~\cite[\S\,3.1]{OCaml}:\footnote{Throughout the paper and
the source code distribution, we took the liberty to rename some
identifiers and to massage some subminor details while quoting
portions of the OCaml tutorial.}

\begin{quote}\itshape
``The class @point@ below defines one instance variable @x@ and two
methods @getX@ and @move@. The initial value of the instance variable
is @0@. The variable @x@ is declared mutable, so the method @move@ can
change its value.''
\end{quote}

\antiskip

\begin{code}
 class point =
   object
     val mutable x = 0
     method getX   = x
     method move d = x <- x + d
   end;;
\end{code}

\noindent
The transcription to Haskell starts with the declaration of all the
labels that occur in the OCaml code. Here is the sugar-free version of
these declarations:

\begin{code}
 data MutableX; mutableX = proxy::Proxy MutableX
 data GetX;     getX     = proxy::Proxy GetX
 data Move;     move     = proxy::Proxy Move
\end{code}

\noindent
Then, the @point@ class is defined as the following Haskell value:

\begin{code}
 point = 
   do
      x <- newIORef 0
      returnIO
        $  mutableX .=. x
       .*. getX     .=. readIORef x
       .*. move     .=. (\d -> do modifyIORef x ((+) d))
       .*. emptyRecord
\end{code}

\noindent
Note how the Haskell code mimics the OCaml code. We use Haskell's
@IORefs@ to model mutable variables. (We do not use any magic of the
IO monad. We could as well use the simpler ST monad, which is very
well formalised~\cite{LPJ95}. In fact, the code distribution for the
paper explores this option.) The @point@ class stands revealed as a
monadic @do@ sequence that first creates an @IORef@ for the mutable
variable, and then returns a record for the new @point@ object. In
general, such records provide access to the public methods of the
object and to the @IORefs@ for public mutable variables.

Let's now instantiate the @point@ class and invoke some methods. To
provide a reference, we include the log of an OCaml session, which
shows some inputs and the responses of the OCaml interpreter:

\begin{code}
 let p = new point;;
 val p : point = <obj>
\end{code}

\begin{code}
 p#getX;;
 - : int = 0
\end{code}

\begin{code}
 p#move 3;;
 - : unit = ()
\end{code}
 
\begin{code}
 p#getX;;
 - : int = 3
\end{code}

\noindent
In Haskell, we capture this program in a monadic @do@ sequence because
we employ the @IO@ monad with its @IORefs@ for the mutable fields. So
object construction and method invocations look as follows:

\begin{code}
 myFirstOOP =
  do
     p <- point -- no need for new!
     p # getX >>= Prelude.print
     p # move $ 3
     p # getX >>= Prelude.print
\end{code}

\noindent
When we run this Haskell program, we get the same result as OCaml:

\begin{code}
 ghci> myFirstOOP
 0
 3
\end{code}



%%%%%%%%%%%%%%%%%%%%%%%%%%%%%%%%%%%%%%%%%%%%%%%%%%%%%%%%%%%%%%%%%%%%%%%%%%%%%
%%%%%%%%%%%%%%%%%%%%%%%%%%%%%%%%%%%%%%%%%%%%%%%%%%%%%%%%%%%%%%%%%%%%%%%%%%%%%
%%%%%%%%%%%%%%%%%%%%%%%%%%%%%%%%%%%%%%%%%%%%%%%%%%%%%%%%%%%%%%%%%%%%%%%%%%%%%



\subsection{Trivial access control}

\noindent
We note that the field @mutableX@ is public~---~just as in the OCaml
code above.\\
Hence, we can manipulate @mutableX@ directly:

\begin{code}
 mySecondOOP =
  do 
     p <- point
     writeIORef (p # mutableX) 42
     p # getX >>= Prelude.print
\end{code}

\begin{code}
 ghci> mySecondOOP
 0
 42
\end{code}

\noindent
Making the mutable variable private is no problem at all. We simply do
not provide direct access to the IORef in the record, i.e., we omit
the field @mutableX@. Using the delete operation of record calculus,
we can also restrict access after the fact. (More interesting options
for access control arise once we cope with the notion of \emph{self}.)



%%%%%%%%%%%%%%%%%%%%%%%%%%%%%%%%%%%%%%%%%%%%%%%%%%%%%%%%%%%%%%%%%%%%%%%%%%%%%
%%%%%%%%%%%%%%%%%%%%%%%%%%%%%%%%%%%%%%%%%%%%%%%%%%%%%%%%%%%%%%%%%%%%%%%%%%%%%
%%%%%%%%%%%%%%%%%%%%%%%%%%%%%%%%%%%%%%%%%%%%%%%%%%%%%%%%%%%%%%%%%%%%%%%%%%%%%



\subsection{Parameterised classes}

Quoting from~\cite[\S\,3.1]{OCaml}:

\begin{quote}\itshape
``The class @point@ can also be abstracted over the initial values of
the @x@ coordinate.  The parameter @x_init@ is, of course, visible in
the whole body of the definition, including methods. For instance, the
method @getOffset@ in the class below returns the position of the
object relative to its initial position.''
\end{quote}

\antiskip

\begin{code}
 class para_point x_init =
   object
     val mutable x    = x_init
     method getX      = x
     method getOffset = x - x_init
     method move d    = x <- x + d
   end;;
\end{code}

\noindent
Non-parameterised classes are represented as computations in Haskell.
Consequently, parameterised classes are represented as monadic
functions (i.e., functions that return computations). Hence, the
parameter @x_init@ ends up as a plain function argument:

\begin{code}
 para_point x_init
   = do
        x <- newIORef x_init
        returnIO
          $  mutableX  .=. x
         .*. getX      .=. readIORef x
         .*. getOffset .=. queryIORef x (\v -> v - x_init)
         .*. move      .=. (\d -> modifyIORef x ((+) d))
         .*. emptyRecord
\end{code}
%%% $


%%%%%%%%%%%%%%%%%%%%%%%%%%%%%%%%%%%%%%%%%%%%%%%%%%%%%%%%%%%%%%%%%%%%%%%%%%%%%
%%%%%%%%%%%%%%%%%%%%%%%%%%%%%%%%%%%%%%%%%%%%%%%%%%%%%%%%%%%%%%%%%%%%%%%%%%%%%
%%%%%%%%%%%%%%%%%%%%%%%%%%%%%%%%%%%%%%%%%%%%%%%%%%%%%%%%%%%%%%%%%%%%%%%%%%%%%



\subsection{Constructor functionality}

Quoting from~\cite[\S\,3.1]{OCaml}:

\begin{quote}\itshape
``Expressions can be evaluated and bound before defining the object
body of the class. This is useful to enforce invariants. For instance,
points can be automatically adjusted to the nearest point on a grid,
as follows:''
\end{quote}

\antiskip

\begin{code}
 class adjusted_point x_init =
   let origin = (x_init / 10) * 10 in
   object
     val mutable x    = origin
     method getX      = x
     method getOffset = x - origin
     method move d    = x <- x + d
   end;;
\end{code}

\noindent
This ability is akin to functionality in constructor methods known
from mainstream OO languages. Using the OCaml notation as given above,
one can also define different constructors for the same type of
object. The Haskell transcription simply takes the following sentence
from the OCaml tutorial literally: ``Expressions can be evaluated and
bound before defining the object body of the class''. That is, we use
local lets prior to returning the construced object:

\begin{code}
 adjusted_point x_init
   = do
        let origin = (x_init `div` 10) * 10
        x <- newIORef origin
        returnIO
          $  mutableX  .=. x
         .*. getX      .=. readIORef x
         .*. getOffset .=. queryIORef x (\v -> v - origin)
         .*. move      .=. (\d -> modifyIORef x ((+) d))
         .*. emptyRecord
\end{code}
%%% $
\noindent
(The fact whether such lets are computed \emph{at all} depends, of
course, on the strictness of the program due to Haskell's lazyness. So
`before' is not meant in a temporal sense.)



%%%%%%%%%%%%%%%%%%%%%%%%%%%%%%%%%%%%%%%%%%%%%%%%%%%%%%%%%%%%%%%%%%%%%%%%%%%%%
%%%%%%%%%%%%%%%%%%%%%%%%%%%%%%%%%%%%%%%%%%%%%%%%%%%%%%%%%%%%%%%%%%%%%%%%%%%%%
%%%%%%%%%%%%%%%%%%%%%%%%%%%%%%%%%%%%%%%%%%%%%%%%%%%%%%%%%%%%%%%%%%%%%%%%%%%%%



\subsection{Nested object templates}

Quoting from~\cite[\S\,3.1]{OCaml}:

\begin{quote}\itshape
``The evaluation of the body of a class only takes place at object
creation time.  Therefore, in the following example, the instance
variable @x@ is initialized to different values for two different
objects.''
\end{quote}

\antiskip

\begin{code}
 let x0 = ref 0;;
 val x0 : int ref = {contents = 0}
\end{code}

\begin{code}
 class incrementing_point :
   object
     val mutable x = incr x0; !x0
     method getX   = x
     method move d = x <- x + d
   end;;
\end{code}

\begin{code}
 new incrementing_point#getX;;
 - : int = 1
\end{code}

\begin{code}
 new incrementing_point#getX;;
 - : int = 2
\end{code}

\noindent
Before we transcribe the use of this OCaml idiom to Haskell, we
observe that we can view the body of a class as the body of a
constructor method. Then, any mutable variable that is used along
subsequent invocations of the constructor functionality can be viewed
as belonging to a class object. So we arrive at a nested object
template.

\begin{code}
 incrementing_point = 
   do 
      x0 <- newIORef 0
      returnIO (
        do modifyIORef x0 (+1)
           x <- readIORef x0 >>= newIORef
           returnIO
             $  mutableX .=. x
            .*. getX     .=. readIORef x
            .*. move     .=. (\d -> modifyIORef x ((+) d))
            .*. emptyRecord
       )
\end{code}
%%% $

\noindent
At the outer level, we do the computation for the point template.  At
the inner level, we perform the computation that constructs points
themselves. This value deserves a more OOP-biased name:

\begin{code}
 makeIncrementingPointClass = incrementing_point
\end{code}

\noindent
A demo is worthwhile; it will reveal a powerful feature of OOHaskell:

\begin{code}
 myNestedOOP =
   do
      localClass <- makeIncrementingPointClass
      localClass >>= ( # getX ) >>= Prelude.print
      localClass >>= ( # getX ) >>= Prelude.print
\end{code}

\begin{code}
 ghci> myNestedOOP
 1
 2
\end{code}

\noindent
We effectively created a class in a scope, and then exported it,
closing over a locally-scoped variable. We can't do such a class
closure in Java! Java supports anonymous objects, but not anonymous
first-class classes. C++ is nowhere close to such an ability.  

\oleg{A reviewer commented: ``It seems to me that Java does have
  anonymous classes (even if they have a name, this name may be used
  more than once and cannot be used globally to access the class)}


%%%%%%%%%%%%%%%%%%%%%%%%%%%%%%%%%%%%%%%%%%%%%%%%%%%%%%%%%%%%%%%%%%%%%%%%%%%%%
%%%%%%%%%%%%%%%%%%%%%%%%%%%%%%%%%%%%%%%%%%%%%%%%%%%%%%%%%%%%%%%%%%%%%%%%%%%%%
%%%%%%%%%%%%%%%%%%%%%%%%%%%%%%%%%%%%%%%%%%%%%%%%%%%%%%%%%%%%%%%%%%%%%%%%%%%%%






%%%%%%%%%%%%%%%%%%%%%%%%%%%%%%%%%%%%%%%%%%%%%%%%%%%%%%%%%%%%%%%%%%%%%%%%%%%%%
%%%%%%%%%%%%%%%%%%%%%%%%%%%%%%%%%%%%%%%%%%%%%%%%%%%%%%%%%%%%%%%%%%%%%%%%%%%%%
%%%%%%%%%%%%%%%%%%%%%%%%%%%%%%%%%%%%%%%%%%%%%%%%%%%%%%%%%%%%%%%%%%%%%%%%%%%%%



\section{Selfish objects and classes}
\label{S:self}

In the previous section we have avoided one important tenet of OO: the
ability of methods to send messages to `self'.  One \emph{could}
simulate such selfish messages by implementing methods as regular
mutually recursive functions. Sending a message to `self' will then
clearly look differently than sending a message to another object. A
more important problem with that naive solution is the closed nature
of the method recursion. Overriding methods in subclasses will not
affect this recursion, drastically limiting subtype polymorphism.

So we need to bind `self' explicitly. Consequently, object templates
need to be in the style of `open recursion': they take self and
construct (some part of) self. This makes object templates amenable to
inheritance including subtype polymorphism. The following batch of
examples is mostly adopted from the OCaml tutorial. The source code
distribution for this paper contains many additional
examples~\cite{OOHaskell}.



%%%%%%%%%%%%%%%%%%%%%%%%%%%%%%%%%%%%%%%%%%%%%%%%%%%%%%%%%%%%%%%%%%%%%%%%%%%%%
%%%%%%%%%%%%%%%%%%%%%%%%%%%%%%%%%%%%%%%%%%%%%%%%%%%%%%%%%%%%%%%%%%%%%%%%%%%%%
%%%%%%%%%%%%%%%%%%%%%%%%%%%%%%%%%%%%%%%%%%%%%%%%%%%%%%%%%%%%%%%%%%%%%%%%%%%%%



\subsection{Object creation boils down to fix-point computation}

Quoting from~\cite[\S\,3.2]{OCaml}:

\begin{quote}\itshape
``A method or an initialiser can send messages to self (that is, the
current object). For that, self must be explicitly bound, here to the
variable @s@ (@s@ could be any identifier, even though we will often
choose the name @self@.)

...

\noindent
Dynamically, the variable s is bound at the invocation of a method. In
particular, when the class @printable_point@ is inherited, the
variable @s@ will be correctly bound to the object of the subclass.''
\end{quote}

\antiskip

\begin{code}
 class printable_point x_init =
   object (s)
     val mutable x = x_init
     method getX = x
     method move d = x <- x + d
     method print = print_int s#getX
   end;;
\end{code}

\noindent
Again, this OCaml code is transcribed to Haskell very directly. A
noteworthy and appreciated deviation is that @s@ ends up as just an
\emph{ordinary} argument of the monadic function for constructing
printable point objects:

\begin{code}
 printable_point x_init s =
   do
      x <- newIORef x_init
      returnIO
        $  mutableX .=. x
       .*. getX     .=. readIORef x
       .*. move     .=. (\d -> modifyIORef x ((+) d))
       .*. print    .=. ((s # getX ) >>= Prelude.print)
       .*. emptyRecord
\end{code}

\noindent
Object creation and invocation looks as follows in OCaml:

\begin{code}
 let p = new printable_point 7;;
 val p : printable_point = <obj>
\end{code}
 
\begin{code}
 p#move 2;;
 - : unit = ()
\end{code}

\begin{code}
 p#print;;
 9- : unit = ()
\end{code}

\noindent
We note that @s@ does not show up in the OCaml line that constructs a
point @p@ with the @new@ construct, but it is clear that the recursive
knot is tied right there. The Haskell code makes this really explicit.
We do not use any special @new@ construct. We simply use the (monadic)
fix-point operation as is:

\begin{code}
 mySelfishOOP =
   do
      p <- mfix (printable_point 7)
      p # move $ 2
      p # print
\end{code}

\begin{code}
 ghci> mySelfishOOP
 9
\end{code}



%%%%%%%%%%%%%%%%%%%%%%%%%%%%%%%%%%%%%%%%%%%%%%%%%%%%%%%%%%%%%%%%%%%%%%%%%%%%%
%%%%%%%%%%%%%%%%%%%%%%%%%%%%%%%%%%%%%%%%%%%%%%%%%%%%%%%%%%%%%%%%%%%%%%%%%%%%%
%%%%%%%%%%%%%%%%%%%%%%%%%%%%%%%%%%%%%%%%%%%%%%%%%%%%%%%%%%%%%%%%%%%%%%%%%%%%%



\subsection{Some MAJOR byproducts of using Haskell}

We would like to highlight some pieces of expressiveness and some
issues of convenience that are implied by our use of Haskell as the
base language for OO programming. The combined list that follows is
not covered by any mainstream OO language. This strongly suggests that
(OO)Haskell lends itself as a language-design environment.



\paragraph*{Type inference}

This claim stands: we haven't seen \emph{any} type annotations for
classes or methods or attributes so far. We will need annotations for
some purposes later. The reader may be interested how the inferred
types look like; we show the inferred type for a class inherited from
|point_class| below.



\paragraph*{First-class classes}

Using higher-order functional programming, we can programme functions
that take classes as arguments, compute them, instantiate them, use
them. The first-class status of classes is easily illustrated as
follows:

\begin{code}
 myFirstClassOOP point_class =
   do
      p <- mfix (point_class 7)
      p # move $ 35
      p # print
\end{code}
%%% $
\begin{code}
 > myFirstClassOOP printable_point
 42
\end{code}

\noindent
That is, we have parameterised @myFirstClassOOP@ in a class.



\paragraph*{Reusable methods}

Using some bits of parameterisation and higher-order functional
programming, we can programme methods outside of any hosting
class. Such methods can be reused across classes without any
inheritance relationship.  For instance, let's identify a method
@print_getX@ that can be shared by all objects that have at least the
method @getX@ of type @Show@~@a@~@=>@~@IO@~@a@~---~regardless of any
inheritance relationships:

\begin{code}
 print_getX self = ((self # getX ) >>= Prelude.print)
\end{code}

\noindent
We can update the corresponding line of @printable_point@ as follows:

\begin{code}
 -- before
 ... .*. print    .=. ((s # getX ) >>= Prelude.print)
 -- after
 ... .*. print    .=. print_getX s
\end{code}



\paragraph*{Implicit polymorphism}

The class of printable points, just given above, is polymorphic with
regard to the point's coordinate~---~without our contribution. This is
a fine difference between the OCaml model and our Haskell
transcription. In OCaml's definition of @printable_point@, the
parameter @x_init@ was of the type @int@~---~because the operation
@(+)@ in Ocaml can deal with integers only. Our points are
polymorphic~---~a point's coordinate can be any @Num@-ber, for
example, an @Int@ or a @Double@. Here is an example to illustrate
that:

\begin{code}
 myPolyOOP =
   do
      p  <- mfix (printable_point (1::Int))
      p' <- mfix (printable_point (1::Double))
      p  # move $ 2
      p' # move $ 2.5
      p  # print
      p' # print
\end{code}

\noindent
Our points are actually \emph{bounded} polymorphic. The point
coordinate may be of any type that implements addition. Until very
recently, one could not express this in Java and in C\#. Expressing
bounded polymorphism in C++ is possible with significant
contortions. In (OO)Haskell, we didn't have to do anything at
all. Bounded polymorphism (aka, generics) are available in Ada95,
Eiffel and a few other languages. However, in those languages, the
polymorphic type and the type bounds must be declared
\emph{explicitly}. In (OO)Haskell, the type system \emph{infers} the
(bounded) polymorphism on its own.



\paragraph*{Fully static type checking}

Although our points are polymorphic, they are still statically
checked. (This is unlike the poor men's implementation of polymorphic
collections, e.g., in Java @<@ 1.5, which up-casts all the items to
the most general type, @Object@, when inserting elements into the
collection, and which attempts runtime-checked downcasts when
accessing elements.) Indeed, if we confuse @Int@s and @Double@s in the
above code, say we attempt ``@p@~@#@~@move@~@$@~@2.5@'', then we get a
type error saying that @Int@ is not the same as @Double@.
%%% $


%%%%%%%%%%%%%%%%%%%%%%%%%%%%%%%%%%%%%%%%%%%%%%%%%%%%%%%%%%%%%%%%%%%%%%%%%%%%%
%%%%%%%%%%%%%%%%%%%%%%%%%%%%%%%%%%%%%%%%%%%%%%%%%%%%%%%%%%%%%%%%%%%%%%%%%%%%%
%%%%%%%%%%%%%%%%%%%%%%%%%%%%%%%%%%%%%%%%%%%%%%%%%%%%%%%%%%%%%%%%%%%%%%%%%%%%%



\subsection{Single inheritance boils down to record extension or update}

(We should note that we only consider width subtyping at this stage
but see App.~\ref{A:deep} for a discussion of deep subtyping.)

\medskip

\noindent
Quoting from~\cite[\S\,3.7]{OCaml}:

\begin{quote}\itshape
``We illustrate inheritance by defining a class of colored points that
inherits from the class of points. This class has all instance
variables and all methods of class @point@, plus a new instance
variable @color@, and a new method @get_color@.''
\end{quote}

\antiskip

\begin{code}
 class colored_point x (color : string) =
   object
     inherit point x
     val color = color
     method get_color = color
   end;;
\end{code}

\begin{code}
 let p' = new colored_point 5 "red";;
 val p' : colored_point = <obj>
\end{code}

\begin{code} 
 p'#getX, p'#get_color;;
 - : int * string = (5, "red")
\end{code}

\noindent
The Haskell version does not refer to a special @inherit@
construct. We rather compose a computation. That is, to construct a
colored point, we instantiate the superclass while maintaining open
recursion, and the obtained record is extended by the new method:

\begin{code}
 colored_point x_init (color::String) self =
   do
        p <- printable_point x_init self
        returnIO $ getColor .=. (returnIO color) .*. p
\end{code}
%%% $

\begin{code}
 myColoredOOP =
   do
      p' <- mfix (colored_point 5 "red")
      x  <- p' # getX
      c  <- p' # getColor
      Prelude.print (x,c)
\end{code}

\begin{code}
 ghci>  myColoredOOP
 (5,"red")
\end{code}

\noindent
It is illustrative to examine the type \emph{inferred} for the class 
|colored_point|. It is:
\begin{code}
colored_point :: forall a r a1.
		 (HRLabelSet (HCons
				  (Proxy GetColor, IO String)
				  (HCons
				       (Proxy MutableX, IORef a)
				       (HCons
					    (Proxy GetX, IO a)
					    (HCons
						 (Proxy Move, a -> IO ())
						 (HCons (Proxy Print, IO ()) HNil))))),
		  HRLabelSet (HCons
				  (Proxy MutableX, IORef a)
				  (HCons
				       (Proxy GetX, IO a)
				       (HCons
					    (Proxy Move, a -> IO ())
					    (HCons (Proxy Print, IO ()) HNil)))),
		  HRLabelSet (HCons
				  (Proxy GetX, IO a)
				  (HCons
				       (Proxy Move, a -> IO ()) (HCons (Proxy Print, IO ()) HNil))),
		  HRLabelSet (HCons
				  (Proxy Move, a -> IO ()) (HCons (Proxy Print, IO ()) HNil)),
		  Num a,
		  HasField (Proxy GetX) r (IO a1),
		  Show a1) =>
		 a
		 -> String
		    -> r
		       -> IO  (Record
				   (HCons
					(Proxy GetColor, IO String)
					(HCons
					     (Proxy MutableX, IORef a)
					     (HCons
						  (Proxy GetX, IO a)
						  (HCons
						       (Proxy Move, a -> IO ())
						       (HCons (Proxy Print, IO ()) HNil))))))


\end{code}
\oleg{remove some HRLabelSet constraints}.
The constraints |HRLabelSet| are apparent (we elided a few of
them). It is obvious that these constraints are all satsified, no
matter how the type variable |a| is instantiated. Alas, GHC is
(perhaps, too) lazy in resolving such constraints. This is the issue
we intend to bring to the GHC implementors. If the constraints
|HRLabelSet| are eliminated (as they could be), the rest of the type
looks very reasonable. The type explicitly lists all the fields and
the types of their values. The type is actually quite
readable. Because the type lists both the new fields and all inherited
fields, the type can be used for a class browser (aka
Eclipse/Smalltalk?), which would greatly simplify the design of such
an IDE (the class browser doesn't need to figure out the set of all
methods in a class by itself: the compiler has already doen that, and
expressed in the type). We also note the bounded polymorphism of the
|colored_point|: the x coordinate can be any |Num|. 


We can also override methods and refer to the implementation of a
method in the superclass (akin to the @super@ construct in OCaml and
other languages). This is illustrated with a subclass of
@colored_point@ whose @print@ method is more informative:

\begin{code}
 colored_point' x_init color self =
   do
      p <- colored_point x_init color self
      return $  print .=. (
              do putStr "so far - "; p # print
                 putStr "color  - "; Prelude.print color )
            .<. p
\end{code}
%%% $

\noindent
The first step in the monadic @do@ sequence constructs an
old-fashioned colored point, and binds it to @p@ for further
reference. (We do not need any @super@ construct, but we can simply
refer to @p@.) The second step in the monadic @do@ sequence returns
@p@ but updated as far as the @print@ method is concerned. The \HList\
operation ``@.<.@'' denotes type-preserving record update as opposed
to record extension. (Cf.\ to App~\ref{A:hTPupdateAtLabel} for its
routine definition.) This operation ``@.<.@'' rather than the familiar
``@.*.@'' makes the overriding explicit (as it is in |C#|, for
example). We could also provide a single operation,
which carries out extension in case the given label does not yet occur
in the given record, while it falls back to type-preserving update
otherwise. The latter operation would let us model the implicit
overriding in C++ and Java.

Here is a demo of inheritance with override:

\begin{code}
 myOverridingOOP =
   do
      p  <- mfix (colored_point' 5 "red")
      p  # print
\end{code}

\begin{code}
 ghci> myOverridingOOP
 so far - 5
 color  - "red"
\end{code}

\noindent
Incidentally, the (advanced) colored point we produced is still a
printable point as far as our previous first-class OOP is concerned:

\begin{code}
 ghci> myFirstClassOOP $ flip colored_point' "red"
 so far - 42
 color  - "red"
\end{code}
%%% $

\noindent
We pass @myFirstClassOOP@ a constructor function (a `class'), which,
when instantiated, makes an object compatible (responding to the same
messages) as the earlier class @printable_point@ declared above. The
compiler has statically verified that the new point indeed has a slot
@getX@ of the required type.



%%%%%%%%%%%%%%%%%%%%%%%%%%%%%%%%%%%%%%%%%%%%%%%%%%%%%%%%%%%%%%%%%%%%%%%%%%%%%
%%%%%%%%%%%%%%%%%%%%%%%%%%%%%%%%%%%%%%%%%%%%%%%%%%%%%%%%%%%%%%%%%%%%%%%%%%%%%
%%%%%%%%%%%%%%%%%%%%%%%%%%%%%%%%%%%%%%%%%%%%%%%%%%%%%%%%%%%%%%%%%%%%%%%%%%%%%



\subsection{Pure virtuals boil down to constrained self arguments}

In OCaml, one can declare a method without actually defining it, using
the keyword @virtual@. A class containing virtual methods must be
flagged @virtual@, and cannot be instantiated. Virtual methods will be
implemented in subclasses. Virtual classes still define type
abbreviations. Here is a virtual class:

\begin{code}
 class virtual abstract_point x_init =
   object (self)
     val mutable x = x_init
     method print = print_int self#getX
     method virtual getX : int
     method virtual move : int -> unit
   end;;
\end{code}

\noindent
In C++, one calls such methods \emph{pure} virtual methods and classes
that cannot be instantiated are called abstract. In Java, we can flag
classes as being abstract. In Haskell, we do not need any special
constructs. A virtual method is simply not defined, and that's it:

\begin{code}
 abstract_point x_init self =
   do
      x <- newIORef x_init
      returnIO $
           mutableX  .=. x
       .*. print     .=. (self # getX >>= Prelude.print )
       .*. emptyRecord
\end{code}
%%% $

\noindent
This specific class cannot be instantiated with @mfix@ because @getX@
is used but not defined. It is worth quoting an error message:
\begin{code}
    No instance for (HasField (Proxy GetX) HNil (IO a))
      arising from use of `abstract_point' at ...
    In the first argument of `mfix', namely `(abstract_point 7)'
    In a 'do' expression: p' <- mfix (abstract_point 7)
\end{code}
The error message is concise and to the point.

The question arises how we mention pure virtual methods explicitly. We
do not want to reveal pure virtuals by `uses without declaration'. In
particular, there might be no uses at all. In OOHaskell, we simply
constrain @self@. Rather than rewriting the earlier definition of the
@abstract_point@ value, we add an \emph{inapplicable but
  type-constraining} equation:

\begin{Verbatim}[fontsize=\small,commandchars=\\\{\}]
 abstract_point (x_init::a) self 
  | const False (constrain self ::
                 Proxy (  (Proxy GetX, IO a)
                      :*: (Proxy Move, a -> IO ())
                      :*: HNil ))
  = \undefined
\end{Verbatim}

\noindent
(That is, we have written an equation with an always failing guard
(cf.\ @const@~@False@) that nevertheless imposes typing constraints.
The equation evaluates to \undefined, which is Ok because it will
never be chosen anyhow.) The @constrain@ operation processes a record;
in the example it is @self@. An application of the operation must be
annotated with a type for a list of label-component pairs.
The form |constrain| is quite akin to C++ \emph{concepts}
\cite{siek05:_concepts_cpp0x}.
Type-checking the application of @constrain@ implies checking whether
the listed labels occur in the given record, and whether the
components are of the required types.  As we can see in the type
annotation, we let @constrain@ return a type proxy. This makes it
crystal-clear that no interesting computation is performed:
type-checking is of only interest here. (Once again, modest syntactic
sugar could make this idiom look less idiosyncratic, but we are keen
to reveal the true technicalities.)

One possible implementation of @constrain@ is check whether we can
\emph{narrow} the given record such that we end up with the record
type that was described by the result type of @constrain@. Of course,
it is enough to attempt narrowing at the type level alone because we
are not interested in a coerced value here. That is:

\begin{code}
 constrain :: Narrow r l => Record r -> Proxy l
 constrain = const proxy
\end{code}

\noindent
The operation for narrowing, both at the type level and the value
level, is defined in App.~\ref{A:narrow}. Narrowing is a type-driven
projection operation on records, which lives in the \HList\
library. (We can also take co-/contra-variance for method types into
account).


 
%%%%%%%%%%%%%%%%%%%%%%%%%%%%%%%%%%%%%%%%%%%%%%%%%%%%%%%%%%%%%%%%%%%%%%%%%%%%%
%%%%%%%%%%%%%%%%%%%%%%%%%%%%%%%%%%%%%%%%%%%%%%%%%%%%%%%%%%%%%%%%%%%%%%%%%%%%%
%%%%%%%%%%%%%%%%%%%%%%%%%%%%%%%%%%%%%%%%%%%%%%%%%%%%%%%%%%%%%%%%%%%%%%%%%%%%%



\begin{figure*}[t]
\begin{center}
\resizebox{.8\textwidth}{!}{\includegraphics{heavy.pdf}}
\end{center}
\vspace{-33\in}
\caption{Diamond inheritance and beyond}
\label{F:heavy}
\end{figure*}



%%%%%%%%%%%%%%%%%%%%%%%%%%%%%%%%%%%%%%%%%%%%%%%%%%%%%%%%%%%%%%%%%%%%%%%%%%%%%
%%%%%%%%%%%%%%%%%%%%%%%%%%%%%%%%%%%%%%%%%%%%%%%%%%%%%%%%%%%%%%%%%%%%%%%%%%%%%
%%%%%%%%%%%%%%%%%%%%%%%%%%%%%%%%%%%%%%%%%%%%%%%%%%%%%%%%%%%%%%%%%%%%%%%%%%%%%



\subsection{Sophisticated inheritance boils down to record calculus}

In several OO languages, multiple inheritance is allowed. For
instance, in OCaml, the rules are as follows. Only the last definition
of a method is kept: the redefinition in a subclass of a method that
was visible in the parent class overrides the definition in the parent
class. Previous definitions of a method can be reused by binding the
related ancestor using a special \ldots @as@ \ldots notation.  The
bound name is said to be a pseudo value identifier that can only be
used to invoke an ancestor method. Other rules and notations exist for
Eiffel, C++, and so on.

In Haskell, we can handle more than plain multiple inheritance.
We are going to work through a scenario, where a class @heavy_point@
is constructed by inheritance from three different concrete subclasses
of @abstract_point@. The first two
concrete points will be shared in the resulting heavy point, because
we leave open the recursive knot. The third concrete point
does not participate in the open recursion; so it is not shared. See
Fig.~\ref{F:heavy} for an overview.

The object template for heavy points starts as follows:

\begin{code}
 heavy_point x_init color self =
  do
     super1 <- concrete_point1 x_init self
     super2 <- concrete_point2 x_init self
     super3 <- mfix (concrete_point3 x_init)
     ... -- to be continued
\end{code}

\noindent
That is, we bind all ancestor objects for subsequent reference. We
pass @self@ to the first two points, which participate in open
recursion, but we fix the third point in place. A heavy point carries
@print@ and @move@ methods that delegate corresponding messages to all
three points:

\begin{code}
     ... -- continued from above
     let myprint = do
                      putStr "super1: "; (super1 # print)
                      putStr "super2: "; (super2 # print)
                      putStr "super3: "; (super3 # print)
     let mymove  = ( \d -> do
                              super1 # move $ d
                              super2 # move $ d
                              super3 # move $ d )
     return 
       $    print  .=. myprint
      .*.   move   .=. mymove
      .*.   emptyRecord
     ... -- to be continued
\end{code}

\noindent
The three points, with all their many fields and methods, contribute
to the heavy point by means of left-biased union on records, which is
denoted by ``@.<++.@'' below:

\begin{code}
     ... -- continued from above
      .<++. super1
      .<++. super2
      .<++. super3
\end{code}

\noindent
(The routine definition of ``@.<++.@'' is given in App.~\ref{A:hLeftUnion}.)

\myskip

\noindent
It's time for a demo:

\begin{code}
 myDiamondOOP =
  do 
     p <- mfix (heavy_point 42 "blue")
     p # print -- All points still agree!
     p # move $ 2
     p # print -- The third point lacks behind!
\end{code}

\begin{code}
 ghci> myDiamondOOP
 super1: 42
 super2: 42
 super3: 42
 super1: 46
 super2: 46
 super3: 44
\end{code}



%%%%%%%%%%%%%%%%%%%%%%%%%%%%%%%%%%%%%%%%%%%%%%%%%%%%%%%%%%%%%%%%%%%%%%%%%%%%%
%%%%%%%%%%%%%%%%%%%%%%%%%%%%%%%%%%%%%%%%%%%%%%%%%%%%%%%%%%%%%%%%%%%%%%%%%%%%%
%%%%%%%%%%%%%%%%%%%%%%%%%%%%%%%%%%%%%%%%%%%%%%%%%%%%%%%%%%%%%%%%%%%%%%%%%%%%%



\subsection{No need for equi-recursive types}

We saw that our class constructor takes @self@ as an argument, and
returns the record, essentially @self'@, as the result. Our encoding
(for non-abstract classes) requires that @self@ and @self'@ have the
same slots -- but they do not have to be of the same type. This is how
we avoid recursive types. Later on, when we instantiate the class into
an object, we will apply @mfix@ operator -- that is, we will use value
recursion rather than the type recursion. Of course our technique
works if no method type involves the type of the self argument
(\emph{not} counting the implicit self). This is indeed the case in
most OO languages, where we must declare the type of the argument and
the result of a method, and that type must be a concrete type rather
than just `self'.  Functional objects obviously break this rule. We
have found a way to deal with functional objects while avoiding
recursive and existential types, but this topic is outside of the
scope of the present paper.



%%%%%%%%%%%%%%%%%%%%%%%%%%%%%%%%%%%%%%%%%%%%%%%%%%%%%%%%%%%%%%%%%%%%%%%%%%%%%
%%%%%%%%%%%%%%%%%%%%%%%%%%%%%%%%%%%%%%%%%%%%%%%%%%%%%%%%%%%%%%%%%%%%%%%%%%%%%
%%%%%%%%%%%%%%%%%%%%%%%%%%%%%%%%%%%%%%%%%%%%%%%%%%%%%%%%%%%%%%%%%%%%%%%%%%%%%

  


%%%%%%%%%%%%%%%%%%%%%%%%%%%%%%%%%%%%%%%%%%%%%%%%%%%%%%%%%%%%%%%%%%%%%%%%%%%%%
%%%%%%%%%%%%%%%%%%%%%%%%%%%%%%%%%%%%%%%%%%%%%%%%%%%%%%%%%%%%%%%%%%%%%%%%%%%%%
%%%%%%%%%%%%%%%%%%%%%%%%%%%%%%%%%%%%%%%%%%%%%%%%%%%%%%%%%%%%%%%%%%%%%%%%%%%%%
 

                                                                             
\begin{figure*}[t]
\begin{center}
\resizebox{.8\textwidth}{!}{\includegraphics{shapes.pdf}}
\end{center}
\vspace{-33\in}
\caption{The `shapes' benchmark for subtype polymorphism}
\label{F:shapes}
\end{figure*}
                                                                             

 
%%%%%%%%%%%%%%%%%%%%%%%%%%%%%%%%%%%%%%%%%%%%%%%%%%%%%%%%%%%%%%%%%%%%%%%%%%%%%
%%%%%%%%%%%%%%%%%%%%%%%%%%%%%%%%%%%%%%%%%%%%%%%%%%%%%%%%%%%%%%%%%%%%%%%%%%%%%
%%%%%%%%%%%%%%%%%%%%%%%%%%%%%%%%%%%%%%%%%%%%%%%%%%%%%%%%%%%%%%%%%%%%%%%%%%%%%
 

 
\section{The Shapes benchmark}
\label{S:shapes}


Let us now explore the so-called `shapes benchmark'.\footnote{See the
multi-lingual collection `OO Example Code' by Jim Weirich at
\url{http://onestepback.org/articles/poly/}; see also an even heavier
collection `OO Shape Examples' by Chris Rathman at
\url{http://www.angelfire.com/tx4/cus/shapes/}.}  This benchmark (or
OO coding scenario) has a history in evaluating encodings of
subtype polymorphism. The classes that are involved in the
scenario are shown in Fig.~\ref{F:shapes}. There is an abstract class
(or an interface) @Shape@, and their are two subclasses @Rectangle@
and @Circle@. The coding scenario is the following: place different
shape object of different subclasses in a collection and iterate over
the collection to draw each shape object; the drawing
functionality varies per subclass.

We will show that the OOHaskell encoding pleasantly mimics the C++
encoding, while any remaining deviations are appreciated. We will also
discuss some variation points, once we have identified our prime
encoding. Finally, there are various possible encodings of the
scenario that do not deliver an image of Haskell as a true OO language
(including the one in Rathman's suite;
\url{http://www.angelfire.com/tx4/cus/shapes/haskell.html}). We do not
have space to discuss such non-OOHaskell encodings, but we refer to
this paper's source code distribution~\cite{OOHaskell} instead.



%%%%%%%%%%%%%%%%%%%%%%%%%%%%%%%%%%%%%%%%%%%%%%%%%%%%%%%%%%%%%%%%%%%%%%%%%%%%%
%%%%%%%%%%%%%%%%%%%%%%%%%%%%%%%%%%%%%%%%%%%%%%%%%%%%%%%%%%%%%%%%%%%%%%%%%%%%%
%%%%%%%%%%%%%%%%%%%%%%%%%%%%%%%%%%%%%%%%%%%%%%%%%%%%%%%%%%%%%%%%%%%%%%%%%%%%%



\subsection{The C++ reference solution}

We omit the code for the classes of shapes, rectangles and
circles. This is all trivial from a C++ perspective: we use a
pure virtual method for @draw@ in the class @Shape@, which is then
implemented differently in the classes @Rectangle@ and @Circle@.

Here is C++ code to set up an array of (two) shapes:

\begin{code}
   Shape *scribble[2];
   scribble[0] = new Rectangle(10, 20, 5, 6);
   scribble[1] = new Circle(15, 25, 8);
\end{code}

\noindent
We use an array rather than a collection type. We could employ a
collection type from C++'s Standard Template Library with similar
convenience. Here is a for-loop over the array in our C++ code:

\begin{code}
   for (int i = 0; i < 2; i++) {
      scribble[i]->draw();
      scribble[i]->rMoveTo(100, 100);
      scribble[i]->draw();
   }
\end{code}

\noindent
That is, we draw each element, move it relatively to its
origin, and draw it again. If the @draw@ method prints
`progress messages' about what is being drawn, we may see the
following output:

\begin{code}
 Drawing a Rectangle at:(10,20), width 5, height 6
 Drawing a Rectangle at:(110,120), width 5, height 6
 Drawing a Circle at:(15,25), radius 8
 Drawing a Circle at:(115,125), radius 8
\end{code}

\noindent
(Any OO language with parametric polymorphism, or at least polymorphic
arrays, should allow similarly concise code. In a typed language
without parametric polymorphism, we would need to bother about unsafe
down-casts when processing the aggregated objects. Likewise, untyped
languages would risk `message-not-understood' errors.)



%%%%%%%%%%%%%%%%%%%%%%%%%%%%%%%%%%%%%%%%%%%%%%%%%%%%%%%%%%%%%%%%%%%%%%%%%%%%%
%%%%%%%%%%%%%%%%%%%%%%%%%%%%%%%%%%%%%%%%%%%%%%%%%%%%%%%%%%%%%%%%%%%%%%%%%%%%%
%%%%%%%%%%%%%%%%%%%%%%%%%%%%%%%%%%%%%%%%%%%%%%%%%%%%%%%%%%%%%%%%%%%%%%%%%%%%%



\subsection{The OOHaskell transcription}

We also omit the Haskell values for the involved classes since we have
exercised pure virtual methods in Sec.~\ref{S:self}. We only
transcribe the collection code. We start a monadic @do@ sequence to
construct two shape objects~---~just as above:

\begin{code}
 myShapesOOP =
    do
       s1 <- mfix (rectangle (10::Int) (20::Int) 5 6)
       s2 <- mfix (circle (15::Int) 25 8)
       -- to be continued
\end{code}

\noindent
What's different? We use @mfix@ in place of @new@. We use curried
functions instead of C++'s comma notation.  We note that some
constructor arguments are annotated by the @Int@ type because we
preferred to eliminate the implicit polymorphism at this stage.

We continue the monadic @do@ sequence by building an `array' of
shapes:

\begin{code}
       let scribble :: [Shape Int]
           scribble = [narrow s1, narrow s2]
\end{code}

\noindent
In fact, we use a plain Haskell list. The type annotation for the
@scribble@ binding corresponds to the typed variable declaration in
the C++ code. However, the Haskell list construction differs from the
the C++ array construction as follows. In Haskell, we use an
explicit coercion operation, @narrow@ to prepare each shape object for
insertion into the homogeneous list. By contrast, such casting is
\emph{implicit} in the C++ code.

Regarding the class type @Shape@, we note that we have not used
\emph{any} explicit class types in the preceding sections. Mostly, we
do not need them because Haskell's type inference works
fine. (Programmers of C++ and of other mainstream languages have
the habit of writing down types for almost everything.) For the
purpose of \emph{casting}, we require such explicit types in
OOHaskell. They are necessary for steering explicit casting in the
view of programs that otherwise lack pervasive type annotations: So
here is the record type for @Shape@ objects:

\begin{code}
 type Shape a = Record (  (Proxy GetX    , IO a)
                      :*: (Proxy GetY    , IO a)
                      :*: (Proxy SetX    , a -> IO ())
                      :*: (Proxy SetY    , a -> IO ())
                      :*: (Proxy MoveTo  , a -> a -> IO ())
                      :*: (Proxy RMoveTo , a -> a -> IO ())
                      :*: (Proxy Draw    , IO ())
                      :*: HNil )
\end{code}

We finish up the monadic @do@ sequence by iterating over scribble:

\begin{code}
       mapM_ (\shape -> do
                           shape # draw
                           (shape # rMoveTo) 100 100
                           shape # draw)
             scribble
\end{code}

\noindent
Here we use the monadic @mapM_@ operation which only cares about the
effects of the monadic steps, throwing away results. This is really
the Haskell way of iterating over a list with effectful
functions~---~as the counterpart of the for-loop in the C++ code.



%%%%%%%%%%%%%%%%%%%%%%%%%%%%%%%%%%%%%%%%%%%%%%%%%%%%%%%%%%%%%%%%%%%%%%%%%%%%%
%%%%%%%%%%%%%%%%%%%%%%%%%%%%%%%%%%%%%%%%%%%%%%%%%%%%%%%%%%%%%%%%%%%%%%%%%%%%%
%%%%%%%%%%%%%%%%%%%%%%%%%%%%%%%%%%%%%%%%%%%%%%%%%%%%%%%%%%%%%%%%%%%%%%%%%%%%%



\subsection{Narrowing vs.\ heterogeneity vs.\ existentials}

We have employed narrowing to coerce all objects to a common
interface, in fact, to the \emph{same} record type. One might wonder
whether these coercions can be avoided altogether, or whether the
explicit conversions can also be made implicit even in OOHaskell. We
will discuss two techniques, but the conclusion will be that narrowing
is to be preferred.

The first technique is to collect the shape objects, as is, in a
\emph{heterogeneous} list rather than a homogeneous array or list. We
cannot construct such a list with the normal, polymorphic list datatype
constructor, but the \HList\ library comes again to our rescue. 
The scribble construction can now be performed without any
narrowing:

\begin{code}
       let scribble = s1 `HCons` (s2 `HCons` HNil)
\end{code}

\noindent
We cannot use ordinary list-processing function anymore, but the
\HList\ library mimics the normal list-processing API for @HList@s. So
there is also a heterogeneous variation on @mapM_@, namely @hMapM_@,
to be invoked as follows:

\begin{Verbatim}[fontsize=\small,commandchars=\\\{\}]
       hMapM_ (\undefined::FunOnShape) scribble
\end{Verbatim}

\noindent
The first argument of @hMapM_@ is not a function but rather a
\emph{type code}. This is necessary for technical reasons related to
the combination of rank-n polymorphism and ad-hoc
polymorphism.\footnote{A heterogeneous map function can encounter
entities of different types. Hence, its argument function must be
polymorphic on its own (which is different from the normal map
function). The argument function typically uses type classes (say,
ad-hoc polymorphism) to process the entities of different types. The
trouble is that the map function cannot possibly anticipate all the
constraints required by its argument function.  The type-code
technique moves the constraints from the type of the heterogeneous map
function to the interpretation site of the type codes.} The meaning of
each type code must be defined by a dedicated instance of an @Apply@
class for function application. Here is the declaration of the type
code @FunOnShape@ complete with its meaning:

\begin{code}
 data FunOnShape -- a type code only!
\end{code}

\begin{code}
instance ( HasField (Proxy Draw) r (IO ())
         , HasField (Proxy RMoveTo) r (Int -> Int -> IO ())
         )
      => Apply FunOnShape r (IO ())
  where
    apply _ x = do
                   x # draw
                   (x # rMoveTo) 100 100
                   x # draw
\end{code}

\noindent
The @Apply@ instance manifests encoding efforts that we didn't face
for the narrowing-based encoding. Now we have to list the
\emph{method-access constraints} (for ``\#'', i.e., @HasField@) in the
@Apply@ instance. Haskell's type-class system requires us to provide
proper bounds for the instance. One might argue that the form of these
constraints strongly resembles the method types listed in the class
type @Shape@. So one might wonder whether we can somehow use the full
class type in order to constrain the instance.  Haskell won't let us
do that in any reasonable way. (Constraints are not first-class
citizens in Haskell; we can't compute them from types or type
proxies~---~unless we were willing to rely on heavy encoding or
advanced syntactic sugar.) So we are doomed to manually infer such
method-access constraints for each such piece of polymorphic code.

The second technique for avoiding narrowing relies on placing shape
objects in \emph{existentially quantified envelopes}: we do not
coerce, but we wrap:

\begin{code}
       let scribble = [ WrapShape s1 , WrapShape s2 ]
\end{code}

\noindent
The declaration of the @WrapShape@ type depends on the function that
we want to apply to the opaque data. In our case, we can use the
normal @mapM_@ function again; we only need to unwrap the @WrapShape@
constructor prior to method invocations:

\begin{code}
       mapM_ ( \(WrapShape shape) -> do
                  shape # draw
                  (shape # rMoveTo) 100 100
                  shape # draw )
             scribble
\end{code}

\noindent
These operations have to be anticipated in the type bound for
@WrapShape@:

\begin{Verbatim}[fontsize=\small,commandchars=\\\{\}]
 data WrapShape =
  \Forall x. ( HasField (Proxy Draw) x (IO ())
       , HasField (Proxy RMoveTo) x (Int -> Int -> IO ())
       ) => WrapShape x
\end{Verbatim}

\noindent
It becomes evident that this result agrees with the heterogeneity
technique in terms of encoding efforts. In both cases, we need to
identify type-class constraints that correspond to the (potentially)
polymorphic method invocations.

Consequently, the narrowing technique is to be preferred.  We
\emph{could} hide narrowing by eschewing free-wheeling functional
programming and type inference. The required style and API would then
account for hidden narrowing. We do not favour such an approach for
various reasons. Abandoning type inference is in conflict with Haskell
native style.  Making casts implicit introduces the risk that the
programmer can accidentally pass an object of the wrong type.  Making
cast implicit hides the costs that come with casts; we prefer to see
the need for coercions clearly. (In fact, in the presence of multiple
inheritance of classes or interfaces, the implicit cast is absolutely
nontrivial, either for the compiler, or for the run-time system, or
both.)



%%%%%%%%%%%%%%%%%%%%%%%%%%%%%%%%%%%%%%%%%%%%%%%%%%%%%%%%%%%%%%%%%%%%%%%%%%%%%
%%%%%%%%%%%%%%%%%%%%%%%%%%%%%%%%%%%%%%%%%%%%%%%%%%%%%%%%%%%%%%%%%%%%%%%%%%%%%
%%%%%%%%%%%%%%%%%%%%%%%%%%%%%%%%%%%%%%%%%%%%%%%%%%%%%%%%%%%%%%%%%%%%%%%%%%%%%



 
%%%%%%%%%%%%%%%%%%%%%%%%%%%%%%%%%%%%%%%%%%%%%%%%%%%%%%%%%%%%%%%%%%%%%%%%%%%%%
%%%%%%%%%%%%%%%%%%%%%%%%%%%%%%%%%%%%%%%%%%%%%%%%%%%%%%%%%%%%%%%%%%%%%%%%%%%%%
%%%%%%%%%%%%%%%%%%%%%%%%%%%%%%%%%%%%%%%%%%%%%%%%%%%%%%%%%%%%%%%%%%%%%%%%%%%%%
 

 
\section{Final discussion}
\label{S:disc}

We have described an OOP system for Haskell that supports stateful
objects, inheritance and subtype polymorphism. The Shapes Benchmark
demonstrated that our encoding is very close to the textbook OO code
(usually given in C++ or Java tutorials), with pleasant
deviations. The re-implementation of examples from the OCaml object
tutorial demonstrated the faithfulness of our realisation of
objects. (We have opted for OCaml because it is a leading object
system in a functional setting.) We have implemented parameterised
classes, constructor methods, abstract classes, pure virtual methods,
single and multiple inheritance with flexible rules of sharing or
separation of superclasses. Major byproducts are these: extensive type
inference, first-class classes, sharing of the code of methods among
even non-related classes, implicit polymorphism of classes, fully
static type checking.

We have clarified the relation between subtyping and type-class based
polymorphism: the latter can encode the former. We have implemented
OOHaskell with the existing Haskell implementation (GHC), requiring no
extra extensions beyond the commonly implemented ones: multi-parameter
type classes with functional dependencies. It is quite pleasant that
the existing OOHaskell code does not seem to be a burden to
write~---~even in the absence of any syntactic sugar.  Once we
consider open recursion in our setup, it turns out that the object
constructor (i.e., `new') is merely the monadic fix-point operation
(i.e., `mfix'). Just as people have known all the time~---~OO and
recursion are intertwined.

There exists a large body of literature,
e.g.,~\cite{Poll97,AC96,Ohori95,Remy94a,PT94,BM92}. Most often
discussed are pure functional objects. Most often the type systems of
object models are variants of system $F_{\leq}$ (polymorphic
lambda-calculus plus subtyping). Most often objects with open
recursion are represented either with recursive types, or with
existentially quantified types. In the present paper we demonstrated
the encoding of imperative objects with inheritance and polymorphism
in Haskell~---~that is, polymorphic lambda-calculus plus
multi-parameter type classes with functional dependencies. Unlike
$F_{\leq}$, there is no built-in subtyping relation. It is remarkable
that our encoding of imperative objects avoids both recursive types
and existentially quantified types. A particularly interesting case
are functional objects, which seem to require recursive or
existentially-quantified types. Functional objects and preservation of
type inference is to be discussed in a forthcoming paper.  Our current
implementation has strong similarities with prototype-based systems
(such as Self~\cite{Self}) in that mutable fields and method
`pointers' are a part of the same record. This does not have to be the
case~---~and in fact, in the forthcoming paper on pure-functional
objects we separate the two tables (in the manner similar to object
realisations in C++ or Java).

There are some further idioms that complement OOHaskell as a faithful
OOP system.  For instance, we need to take measures to do
instantiation checking for classes at declaration time. Otherwise some
type errors would go unnoticed until the first attempt to programme an
instantiation. We can also model various forms of private and
protected methods and other access modifiers. We can clone objects, we
can let methods return `self', we can operate in the ST monad rather
than the IO monad. The article comes with an extensive collection of
source code~\cite{OOHaskell}, where these and other issues are
covered. The source code also illustrates some cases of depth
subtyping~\cite{Poll97} and statically safe argument
covariance. However, the extent and limitations of our handling of
depth subtyping remains the subject of ongoing research.

We are currently working on some elaborations and advanced topics.
Simple syntactic sugar would make OOP more convenient in Haskell.
Extra effort is needed to provide OOP-like error messages.  This is a
sticking issue that requires major effort, but there is a line of
research being carried out by Sulzmann and others~\cite{SSW04}. A
non-trivial case study is required to demonstrate the scalability of
the approach. The mere compilation time of OOHaskell programs and
their runtime efficiency is challenged by the huge dictionaries that
are implied by our type-class-based approach.  It seems that some
dedicated optimisations will be needed in order to handle the
\HList/OOHaskell style of programming efficiently. An interesting
advanced topic is reflective programming. A simple form of reflection
is readily provided in terms of the type-level encoding of
records. One can iterate over records and their components in a
generic fashion. Other forms of reflection, such as iteration over the
object pool, as needed for dynamic aspect-oriented programming,
requires further effort. Another challenge is to capture reusable
solutions for design problems (as part of design patterns) in Haskell.



%%%%%%%%%%%%%%%%%%%%%%%%%%%%%%%%%%%%%%%%%%%%%%%%%%%%%%%%%%%%%%%%%%%%%%%%%%%%%
%%%%%%%%%%%%%%%%%%%%%%%%%%%%%%%%%%%%%%%%%%%%%%%%%%%%%%%%%%%%%%%%%%%%%%%%%%%%%
%%%%%%%%%%%%%%%%%%%%%%%%%%%%%%%%%%%%%%%%%%%%%%%%%%%%%%%%%%%%%%%%%%%%%%%%%%%%%




%%%%%%%%%%%%%%%%%%%%%%%%%%%%%%%%%%%%%%%%%%%%%%%%%%%%%%%%%%%%%%%%%%%%%%%%%%%%%
%%%%%%%%%%%%%%%%%%%%%%%%%%%%%%%%%%%%%%%%%%%%%%%%%%%%%%%%%%%%%%%%%%%%%%%%%%%%%
%%%%%%%%%%%%%%%%%%%%%%%%%%%%%%%%%%%%%%%%%%%%%%%%%%%%%%%%%%%%%%%%%%%%%%%%%%%%%



{\small 

\subsubsection*{Acknowledgements}
 
We thank Chung-chieh Shan for very helpful discussions. The second
author presented this work at an earlier stage at the WG2.8 meeting
(Functional Programming) in November 2004 at West Point. We are
grateful for feedback at this meeting. We also appreciated feedback from
Robin Green, Bryn Keller, Chris Rath and several other participants
in mailing list or email discussions. 

}



%%%%%%%%%%%%%%%%%%%%%%%%%%%%%%%%%%%%%%%%%%%%%%%%%%%%%%%%%%%%%%%%%%%%%%%%%%%%%
%%%%%%%%%%%%%%%%%%%%%%%%%%%%%%%%%%%%%%%%%%%%%%%%%%%%%%%%%%%%%%%%%%%%%%%%%%%%%
%%%%%%%%%%%%%%%%%%%%%%%%%%%%%%%%%%%%%%%%%%%%%%%%%%%%%%%%%%%%%%%%%%%%%%%%%%%%%



\bibliographystyle{abbrv}
\bibliography{paper}



%%%%%%%%%%%%%%%%%%%%%%%%%%%%%%%%%%%%%%%%%%%%%%%%%%%%%%%%%%%%%%%%%%%%%%%%%%%%%
%%%%%%%%%%%%%%%%%%%%%%%%%%%%%%%%%%%%%%%%%%%%%%%%%%%%%%%%%%%%%%%%%%%%%%%%%%%%%
%%%%%%%%%%%%%%%%%%%%%%%%%%%%%%%%%%%%%%%%%%%%%%%%%%%%%%%%%%%%%%%%%%%%%%%%%%%%%



\renewcommand{\mysize}{\footnotesize}




%%%%%%%%%%%%%%%%%%%%%%%%%%%%%%%%%%%%%%%%%%%%%%%%%%%%%%%%%%%%%%%%%%%%%%%%%%%%%
%%%%%%%%%%%%%%%%%%%%%%%%%%%%%%%%%%%%%%%%%%%%%%%%%%%%%%%%%%%%%%%%%%%%%%%%%%%%%
%%%%%%%%%%%%%%%%%%%%%%%%%%%%%%%%%%%%%%%%%%%%%%%%%%%%%%%%%%%%%%%%%%%%%%%%%%%%%



\appendix

\section{Type-preserving record update}
\label{A:hTPupdateAtLabel}

%\ralf{Some line comments need to be added.}

\begin{code}
 infixr 1 .<.
 (l,v) .<. r = hTPupdateAtLabel l v r
\end{code}

\begin{code}
 hTPupdateAtLabel l (v::v) r = hUpdateAtLabel l v r
  where
   (_::v) = hLookupByLabel l r
\end{code}
 
\begin{code}
 hUpdateAtLabel l v (Record r) = Record (hZip ls vs')
  where
   (ls,vs) = hUnzip r
   n       = hFind l ls
   vs'     = hUpdateAtHNat n v vs
\end{code}

\begin{code}
 hLookupByLabel l (Record r) = v
  where
   (ls,vs) = hUnzip r
   n       = hFind l ls
   v       = hLookupByHNat n vs
\end{code}



%%%%%%%%%%%%%%%%%%%%%%%%%%%%%%%%%%%%%%%%%%%%%%%%%%%%%%%%%%%%%%%%%%%%%%%%%%%%%
%%%%%%%%%%%%%%%%%%%%%%%%%%%%%%%%%%%%%%%%%%%%%%%%%%%%%%%%%%%%%%%%%%%%%%%%%%%%%
%%%%%%%%%%%%%%%%%%%%%%%%%%%%%%%%%%%%%%%%%%%%%%%%%%%%%%%%%%%%%%%%%%%%%%%%%%%%%



\section{Type-driven narrowing for records}
\label{A:narrow}

%\ralf{Some line comments need to be added.}

\begin{code}
 class  Narrow a b
  where narrow :: Record a -> Record b
\end{code}

\begin{code}
 instance Narrow a HNil
  where   narrow _ = emptyRecord
\end{code}

\begin{code}
 instance ( Narrow r r', HExtract r l v
          ) => Narrow r (HCons (l,v) r')
  where
    narrow (Record r) = Record (HCons (l,v) r')
      where
        (Record r')    = narrow (Record r)
        ((l,v)::(l,v)) = hExtract r
\end{code}

\begin{code}
 class  HExtract r l v
  where hExtract :: r -> (l,v)
\end{code}

\begin{code}
 instance ( TypeEq l l1 b, HExtractBool b (HCons (l1,v1) r) l v
          ) => HExtract (HCons (l1,v1) r) l v
  where hExtract = hExtractBool (undefined::b)
\end{code}

\begin{code}
 class HBool b => HExtractBool b r l v
  where hExtractBool :: b -> r -> (l,v)
\end{code}

\begin{code}
 instance TypeCast v1 v => HExtractBool HTrue (HCons (l,v1) r) l v
  where hExtractBool _ (HCons (l,v) _) = (l,typeCast v)
\end{code}

\begin{code}
 instance HExtract r l v => HExtractBool HFalse (HCons (l1,v1) r) l v
  where hExtractBool _ (HCons _ r) = hExtract r
\end{code}



%%%%%%%%%%%%%%%%%%%%%%%%%%%%%%%%%%%%%%%%%%%%%%%%%%%%%%%%%%%%%%%%%%%%%%%%%%%%%
%%%%%%%%%%%%%%%%%%%%%%%%%%%%%%%%%%%%%%%%%%%%%%%%%%%%%%%%%%%%%%%%%%%%%%%%%%%%%
%%%%%%%%%%%%%%%%%%%%%%%%%%%%%%%%%%%%%%%%%%%%%%%%%%%%%%%%%%%%%%%%%%%%%%%%%%%%%



\section{Left-biased union on records}
\label{A:hLeftUnion}

%\ralf{Some line comments need to be added.}

\begin{code}
 class  HLeftUnion r r' r'' | r r' -> r''
  where hLeftUnion :: r -> r' -> r''
\end{code}

\begin{code}
 instance HLeftUnion r (Record HNil) r
  where   hLeftUnion r _ = r
\end{code}

\begin{code}
 instance ( HZip ls vs r
          , HMember l ls b
          , HLeftUnionBool b r l v r'''
          , HLeftUnion (Record r''') (Record r') r''
          )
            => HLeftUnion (Record r) (Record (HCons (l,v) r')) r''
  where
   hLeftUnion (Record r) (Record (HCons (l,v) r')) = r''
    where
     (ls,vs) = hUnzip r
     b       = hMember l ls
     r'''    = hLeftUnionBool b r l v
     r''     = hLeftUnion (Record r''') (Record r')
\end{code}

\begin{code}
 class  HLeftUnionBool b r l v r' | b r l v -> r'
  where hLeftUnionBool :: b -> r -> l -> v -> r'
\end{code}

\begin{code}
 instance HLeftUnionBool HTrue r l v r
    where hLeftUnionBool _ r _ _ = r
\end{code}

\begin{code}
 instance HLeftUnionBool HFalse r l v (HCons (l,v) r)
    where hLeftUnionBool _ r l v = HCons (l,v) r
\end{code}



%%%%%%%%%%%%%%%%%%%%%%%%%%%%%%%%%%%%%%%%%%%%%%%%%%%%%%%%%%%%%%%%%%%%%%%%%%%%%
%%%%%%%%%%%%%%%%%%%%%%%%%%%%%%%%%%%%%%%%%%%%%%%%%%%%%%%%%%%%%%%%%%%%%%%%%%%%%
%%%%%%%%%%%%%%%%%%%%%%%%%%%%%%%%%%%%%%%%%%%%%%%%%%%%%%%%%%%%%%%%%%%%%%%%%%%%%



\section{Illustration of deep subtyping}
\label{A:deep}

Suppose you have a class @cube@ and a subclass @cuboid@, which
overrides one of @cube@'s methods by a version with a co-variant
return type (as in Java 5, for example). Substitutability of cubes by
cuboids does not require specific efforts. However, we can even coere
a cuboid to a cube using deep subtyping, that is, a ``deep'' variation
on App.~\ref{A:narrow}.

\medskip

\noindent
Here is a test case illustrating deep narrow:

\begin{code}
 testDeep
   = do
	(cuboid::cuboid) <- mfix (class_cuboid (10::Int) (20::Int) (30::Int))
	cube <- mfix (class_cube (40::Int))
	let cuboids = [cuboid, deep'narrow cube]
	putStrLn "Volumes of cuboids"
        mapM_ (\cb -> handle_cuboid cb >>= print) cuboids
\end{code}

\noindent


\noindent
See the code distribution for the paper for the specification of
@deep'narrow@. This operation must essentially descent into records
and postfix all ``method returns'' by a narrow operation on the
results.



%%%%%%%%%%%%%%%%%%%%%%%%%%%%%%%%%%%%%%%%%%%%%%%%%%%%%%%%%%%%%%%%%%%%%%%%%%%%%
%%%%%%%%%%%%%%%%%%%%%%%%%%%%%%%%%%%%%%%%%%%%%%%%%%%%%%%%%%%%%%%%%%%%%%%%%%%%%
%%%%%%%%%%%%%%%%%%%%%%%%%%%%%%%%%%%%%%%%%%%%%%%%%%%%%%%%%%%%%%%%%%%%%%%%%%%%%




%%%%%%%%%%%%%%%%%%%%%%%%%%%%%%%%%%%%%%%%%%%%%%%%%%%%%%%%%%%%%%%%%%%%%%%%%%%%%
%%%%%%%%%%%%%%%%%%%%%%%%%%%%%%%%%%%%%%%%%%%%%%%%%%%%%%%%%%%%%%%%%%%%%%%%%%%%%
%%%%%%%%%%%%%%%%%%%%%%%%%%%%%%%%%%%%%%%%%%%%%%%%%%%%%%%%%%%%%%%%%%%%%%%%%%%%%



\end{document}
