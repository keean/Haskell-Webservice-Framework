\section{Before Intro}

We must address relavance and significance: for practioneers, for
the programming language design, and for theory.

The issue of OO and Haskell, OO classes vs. type classes shows up on
Haskell mailing lists and websites with remarkable regularity.
[
the latest ones
\url{http://www.haskell.org//pipermail/haskell/2004-June/014164.html}
\url{http://www.haskell.org/tmrwiki/LearningHaskellNotes#head-f9911730c98633f6182b33ce159a77d72afc6d87}
]

One can't deny deep interest in this subject. The range of approaches and
differences in opinion (see summaries in \cite{MonadReader3} as well as
LearningHaskellNotes above) show that the question is unsettled one.
The often repeated advice is that we can't do the ``classical'' OO in
Haskell. If we wish for OO, it has to be done in a totally different
(not often obvious: Message on Haskel-Cafe, form Alistair
Bayley) way. If we wish to approach the classical OO of Java, C++,
or of Javascript, etc., extensions to Haskell are needed. Such
extensions have indeed been proposed, with the motivation of ...

One of the main motivations, stated in \cite{MonadReader3} is to be able to
bring OO programmers into Haskell fold, or to be able to port existing
OO code to Haskell without totally rewriting it.  ``That said, it
might be that you need to interface with external OO code, or you are
porting an existing program and need to remain faithful to the
original design, if only because you don't understand it well enough
to convert it to a functional design.'' \cite{MonadReader3}.


There is an intellectual challenge as well, of seeing if the
conventional OO can at all be implemented in Haskell. The paper on
imperative programming in Haskell \cite{peytonjoneswadler-popl93}
epitomizes this approach. We should also mention FC++ \cite{fcpp-jfp},
which attempted to see if Haskell Prelude can be implemented in
C++. Clearly it is the (possibility) of a faithful translation rather the
end result of it that is the motivation.

The present paper follows the same intellectual tradition: can we
faithfully implement X in Y (specifically excluding writing of the
compiler for X in Y. Make this a footnote?)? Or, if X is expressible
\cite{Felleisen90} in Y?  For one thing, we settle the question: the
conventional OO is expressible in current Haskell without any new
extensions. It turns out, Haskell98 plus multi-parameter type classes
with functional dependencies are enough. Even overlapping instances
are not essential (yet using them permits a more convenient
representation of labels).  That conclusion has not been known
before\footnote{ Haskell is not even supposed to have extensible
  polymorphic records! At least that was the general belief until a
  few months ago.  There were many debates on the Haskell mailing list
  about adding such records to Haskell. At the Haskell 2003
  workshop~\cite{HW03}, this issue was selected as prime topic for
  discussion.}; and, as being emphasized by many people and reviewers,
is novel and surprising.

Our encoding of objects in this paper is deliberately conventional.
Our goal is not to find some new way to encode objects; rather, find a
way to realize the traditional object encoding based on record
subtyping. One of the main goals of this paper is to be able represent
the conventional OO code, in as straightforward way as possible.
We wish to satisfy the real practical need expressed in
\cite{MonadReader3}, that is [port code without undue mangling -- w/o
  global transformation].

For the sake of simplicity, our object encoding is simple, based on
`flat' extensible records with subtyping. More efficient
representation based on method table and field table is left for
another paper. Although our current encoding is certainly not
optimal, it is conceptually clearer. Furthermore, it is close to the
object encoding used in such languages as Perl, Python, Lua, and thus
helps in an earlier stated goal (correlating mainstream OO with
Haskell).  The encoding is often the first one to be used when adding
OO to the existing language -- and thus seems to be suitable starting
point for us.  Again, for clarity and for easy correlation with
practice, we will be concerend with mutable objects (implemented via
|IORef| or |STRef|). Most OO system in
practical use have mutable state. Functional objects bring quite
an interesting twist and so will be considered in another paper.

From one angle, the present paper can be considered an `inverse' of
FC++. The latter tried to implement in C++ the quintessentials Haskell
features: type inference, higher-order, non-strictness. The present
paper illustrates faithful (to be more precise, in similar syntax and
without requiring global program transformation) representation of one
of the important C++ feature: OO. Section XXX discusses one of the
popular benchmarks -- Shapes. 


One may be tempted to dismiss the paper as ``just type hacking''. We
defend both on the practical grounds, language design grounds, and the
theoretical grounds \footnote{Multi-parameter type classes with
  functional dependencies are \emph{not} a hack: they are
  well-formalized and reasonably understood~\cite{SS04}.}. From the
practical point of view: the fact that we found a quite unexpected
(and unintended) use of a particular language feature does not mean
that the result is practically useless. Template meta-programming in
C++ has been the best known example of such ``type hacking''. And yet
it has lead to |boost|, which has become a de facto tool for modern
C++ programming (Adobe uses Boost:
\url{http://lambda-the-ultimate.org/node/view/563#comment-4531}).
Templates and template meta-programming have changed the very
character of the language (\cite{fcpp-jfp})
[Stroustrup interview? Need some reference. If we
  can't find any, I can use LtU references
\url{http://lambda-the-ultimate.org/node/view/663}
(see comments by Scott Johnson)
\url{http://lambda-the-ultimate.org/node/view/663#comment-5839}
See also:
\url{http://spirit.sourceforge.net/distrib/spirit_1_7_0/libs/spirit/phoenix/doc/preface.html}
] and made generative
research and practice in C++ possible (\cite{DSL-in-three-lang}).

Just as C++ has become the laboratory for generative programming
\cite{DSL-in-three-lang} and lead to such applications as FC++, we contend
that Haskell can become the laboratory for OO design and development.
To extend the motto by Simon Peyton-Jones, Haskell is not only the
best imperative language. It is the best OO language as well.
C++ programmers now routinely use parsing combinators, thanks to the
|boost::spirit| library. Haskell programmers can likewise use OO idioms if it
suits the problem at hand. We can experiment with OO features and gain
experience, without the need to change Haskell compilers.

We shall point out in Sections XXX and XXX (simple objects?),
OOHaskell already exhibits features that are either bleading-edge or
unattainable in mainstream OO languages: for example, first-class
classes and class closures; statically typechecked collection classes
with bounded polymorphism of collection arguments; multiple
inheritance with user-controlled sharing. It is especially remarkable
that these and more familiar object-oriented features are not
introduced by fiat -- we get them for free. For example, the type of a
collection with bounded polymorphism of elements is inferred
automatically by the compiler. Abstract classes cannot be instantiated
not because we say so but because the program will not typecheck
otherwise.  \oleg{Mention more ``major byproducts'' from the sections
  below} We have experienced first-hand the incredible amount of
guidance Haskell type system gives in the design of OO features.

We should not overlook a theoretical contribution. On one hand, it is
indeed well-known that we can build OO systems on top of extensible
records (introduced earlier in the HList paper) with a suitable notion
of subtyping \cite{Cardelli-on-understanding}. Also, an
object-oriented system with `flat' mutable objects is significantly
simpler theoretically; one may think that none of the machinery
[Pierce] of existential or recursive types is needed. We must point
out that even mutable objects implemented on the top of extensible
records are more complex than it may appear. The complexity comes from
supporting inheritance and virtual methods. Even in the case of flat
mutable objects (records of closures), the typing of `self' presents
challenges (seem to require recursive type -- lift phrases from
subsection ``{OOHaskell~---~A challenge}'' below). Furthermore, to implement
a homogeneous list of objects all supporting the same interface (Shape
example), one seem to need existential types. Both recursive types and
existential make inference impossible, in current Haskell. The user
must give explicit type annotations. In the paper we show that we
support open recursion, virtual methods, and homogeneous lists of
diferently implemented Shapes without resorting to recursive object
types and to existentials. Therefore, the object types can be
\emph{inferred}.

Another theoretical problem -- even with mutable objects -- is the
controversy regarding covariance and contravariance of method
arguments \cite{SG04}, \cite{catcall}. Our work shows how to implement
covariant methods (which are considered practically desirable [Eiffel
  FAQ]) and statically (at compile-time) maintain soundness. Alas, due
to the lack of space, we merely skim the topic and refer the reader to
the code (and the next paper).


List all the suboptimal OO encoding, with refs,
and discuss their drawbacks (including earlier Shape examples).
Mention that there are simpler solutions, but they all fall short of
the complete OO encoding. This paper seems to be the one that treats
OO in full generality, including open recursion, interface and
implementation inheritance, subtyping.

A reviewer wrote:
``The encoding is quite simple --- it's surprising that
everything is so easy --- yet not at all obvious. The paper isn't
labelled as a pearl, but would certainly fit well in that category.''

A reviewer wrote:
``The result might seem poor and just containing clever tricks
however it took 10 years to obtain that proof of concepts and this
deserves attention.''


We must mention two other issues: error messages and efficiency of
encoding.

One may think that the inferred types and error messages
would be nearly incomrehensible. That is not actually the case.
We give an example of the inferred type, in Section (about Selfish
types). The type of the constructor, for example, 
essentially lists all the fields of an object, both new and
inherited. The error messages (e.g., from an attempt to instantiate
|abstract_point|, in the same section) succinctly list just the
missing field. In general, the clarity of error messages is undoubtably
is an area that needs more research, and more research is
being done (Sulzmann), which we or compiler writers may take advantage
of. It must be mentioned that error messages in C++ (template
instantiation) can be immensely
verbose, spanning literally 30 and 40 packed lines. And yet boost and
similar libraries that extensively use templates are gaining momentum.

Another issue is the efficiency of the encoding.  In the present
paper, we specifically post-poned any optimizations for the sake of
conceptual clarity.  For example, although record extension is
constant (run-)time, the field/method lookup is linear search. Clearly
a more efficient encoding is possible: one reprentation of the labels
in the \HList\ paper permits the total order among the labels (and label
types! they are singleton types), which in turn, permits construction
of efficient search trees. In this first OOHaskell paper, we chose the
conceptual clarity over such optimizations.


``For these reasons, the method is likely to be of limited value in
practice, unless Haskell compilers start providing special support for
the encoding itself. But showing the usefulness of such an encoding is
the first step towards encouraging compiler writers to do so!''




Todo: 
\begin{enumerate}
\item ``The power of the system is first proved by encoding some very basic
examples from the ocaml tutorial. For this part, page 6 gives the basic
encoding, but pages 7 to 9 just repeat this basic encoding, with no
new information.''
So, drop pages 7-9?
\end{enumerate}

% http://dblp.uni-trier.de


\section{Introduction}

\oleg{Explicitly enumerate the contributions.}

\subsection*{OOP and Haskell~---~Married finally}
\oleg{move this subsection up?}
There is a widely shared perception regarding the relation between
Haskell and object-oriented programming (OOP). That is, Haskell is
thought to really lack OOP essentials. The functionally minded OOP
aficionado could either wish for an extended Haskell, employ a
multi-language setup such as .NET, or disregard Haskell and go for
OCaml instead, which is known to be a mainstream functional
programming language that provides an advanced OOP system. This
perception is without basis. In this paper, we show how to program
with mutable objects, inheritance, subtyping etc.\ \emph{in Haskell}
using only common type-system extensions. More than that, we will end
up with a comparatively rich combination of OO idioms, higher-order
functional programming, and type inference. We therefore think that
(OO)Haskell lends itself as environment for advanced and typed OO
\emph{language design}. We will illustrate OOHaskell with a series of
practical examples as they are commonly found in OO textbooks and
programming language tutorials. We wish to make OOHaskell as easy as
possible to use by an experienced OO programmer who is migrating to
Haskell. We certainly could emulate the OO system of OCaml or Java by
writing a compiler in Haskell for these languages. But that's not the
sort of emulation we aim at. We wish our OOHaskell system to be just
as expressive~\cite{Felleisen90} as the most advanced OO
systems. Informally, we wish our OO code looked `just the same as
native OO code'~---~or perhaps even better, with less syntactic sugar.


\subsection*{OOHaskell~---~A challenge}

We will marry Haskell and OO by going through the following sequence:
extensible polymorphic records, record calculus, objects with mutable
data, classes with subtyping, open recursion, inheritance,
polymorphism. That sounds quite simple until we realise that, for one
thing, Haskell is not even supposed to have extensible polymorphic
records. At least that was the general belief until a few months ago.
There were many debates on the Haskell mailing list about adding such
records to Haskell. At the Haskell 2003 workshop~\cite{HW03}, this
issue was selected as prime topic for discussion. Fortunately, this
problem is resolved. In our very recent work on strongly typed
heterogeneous collections~\cite{HLIST-HW04}, we have provided
collection types such as extensible records using Haskell's type
classes.

There is a second major barrier for OOHaskell: the notion of
`self'. That is, the realisation of objects with inheritance and
polymorphism in terms of records seems to require (equi-) recursive
types, to type `self'~\cite{PT94}.
%
% See p. 29 of the paper and the quotation from Bruce, 1992. Pierce
% seems to agree that recursive types are needed to model inheritance
% of methods involving self. -- beginning of the second paragraph
% on p. 29
%
Such an extension to Haskell was also debated and then rejected
because it will make type-error messages nearly
useless~\cite{Hughes02}. There is an alternative technique for
encoding objects: eschew recursive types in favour of existential
quantification~\cite{PT94}. Unfortunately, the involved higher-ranked
types can not be inferred anymore. Explicit signatures were required,
which, in practical terms, means that the user must explicitly
enumerate all virtual methods in the signature of any function that
operates on an object. This technique cannot be used in OOHaskell
because we would like OOP to be easy to use, first, by Haskell
programmer. That is, we ought to preserve type inference for functions
that use objects. Type inference is the great advantage of Haskell and
ML and is worth fighting for. Fortunately, it \emph{is} possible to
model imperative objects without equi-recursive types, as we will reveal in
this paper.

We want the OOP system for Haskell to be available and usable
\emph{now}. That means that we have to build the OOP system, without
adding any extensions to Haskell. The implementation of our system may
be not for the feeble at heart~---~however, the user of the system
must be able to write conventional OO code without understanding the
complexity of the implementation. At present, error messages belie the
complexity, and this is the topic of future research (and so it is for
C++, where error messages in template meta-programs may span several
hundred lines and be humanly incomprehensible).



\subsection*{Type classes versus object classes~---~Confusions resolved}

At the heart of our approach is the powerful deployment of Haskell's
type classes. It will turn out that we can provide object classes
because of Haskell's type classes. In fact, we build OOHaskell in
terms of Hindley-Milner + multi-parameter type classes + functional
dependencies. (Here we note that this combination is well-formalized
and reasonably understood~\cite{SS04}.) Once we have extensible
records with reusable labels and subtyping, we can model some sort of
objects. The corresponding record types, suitably parameterised, are
OOP-like classes then. Mutable objects can be modelled by using
references such as the @IORef@s provided by Haskell's @IO@ monad.
\oleg{move that up, where we discuss the confusion, etc? OTH, perhaps
  move to the related work?}
We have observed widespread confusions regarding the relation between
Haskell's type classes and the object-oriented notion of classes. At
times these two sorts of classes are said to be very much different,
perhaps even largely orthogonal, incomparable. Elsewhere it is argued
that Haskell's type classes are like Java's interfaces, while
instances are like implementations, but subtyping would be missing in
this picture.  Again, elsewhere it is observed that multi-parameter
classes exhibit some flavour of multi-dispatch in OOP, which does not
get us very far however in the view of other missing OOP
essentials. It is often considered a mistake to
attempt OOP in Haskell, to transcribe Java or C++ classes in Haskell,
to perhaps even try to use use Haskell's type classes to that
end.\footnote{See many mind-boggling discussions on mailing lists:
\url{http://www.cs.mu.oz.au/research/mercury/mailing-lists/mercury-users/mercury-users.0105/0051.html},
\url{http://www.talkaboutprogramming.com/group/comp.lang.functional/messages/47728.html},
\url{http://www.haskell.org/pipermail/haskell/2003-December/013238.html},
\url{http://www.haskell.org/pipermail/haskell-cafe/2004-June/006207.html},
\ldots } We will shed light on the subject matter. That is, we will
effectively use Haskell's type-class system to provide an OOP system
for Haskell. This system will be similar to OCaml's system, which we
view as a very strong existing marriage of functional programming and
OOP.



%%%%%%%%%%%%%%%%%%%%%%%%%%%%%%%%%%%%%%%%%%%%%%%%%%%%%%%%%%%%%%%%%%%%%%%%%%%%%
%%%%%%%%%%%%%%%%%%%%%%%%%%%%%%%%%%%%%%%%%%%%%%%%%%%%%%%%%%%%%%%%%%%%%%%%%%%%%
%%%%%%%%%%%%%%%%%%%%%%%%%%%%%%%%%%%%%%%%%%%%%%%%%%%%%%%%%%%%%%%%%%%%%%%%%%%%%



\subsection*{Other OOP systems for Haskell}
\oleg{move to the related work section. Add references to OO
  encodings, in particular those based on extensible records:
  \cite{Cardelli-on-understanding} and \cite{Remy94a}}
There were attempts to bring OO to Haskell by language extension. An
early attempt is Haskell++~\cite{HS95} by Hughes and Sparud. The
authors motivated their extension by the perception that Haskell lacks
the form of incremental reuse that is offered by inheritance in
object-oriented languages. Our approach uses common extensions of the
Hindley-Milner type system to provide the key OO notions.  So in a
way, Haskell does \emph{not} lack expressiveness for OOP-like
reuse. Haskell's fitness for OOP just had to be discovered, which is
the contribution of this paper. Nordlander has delivered a
comprehensive OOP variation on
Haskell~---~O`Haskell~\cite{Nordlander98,Nordlander02}, which extends
Haskell with reactive objects and subtyping. The subtyping part is a
formidable extension. The reactive object part combines stateful
objects and concurrent execution, again a major extension. Our
development shows that no extension of Haskell is necessary for
stateful objects with a faithful object-oriented type system. Finally,
there is Mondrian~---~the NET-able version of Haskell. In the original
paper on the design and implementation of Mondrian~\cite{MC97}, Meijer
and Claessen write: ``The design of a type system that deals with
subtyping, higher-order functions, and objects is a formidable
challenge ...''. Rather than designing a very complicated language,
the overall principle underlying Mondrian was to obtain a simple
Haskell dialect with an object-oriented flavor. To this end, algebraic
datatypes and type classes were combined into a simple object-oriented
type system with no real subtyping, with completely covariant
type-checking. In Mondrian, runtime errors of the kind ``message not
understood'' are considered a problem akin to partial functions with
non-exhaustive case discriminations. We raise the bar by providing
proper subtyping (``all message will be understood'') and other OOP
concepts in Haskell without extending the Haskell type system.



%%%%%%%%%%%%%%%%%%%%%%%%%%%%%%%%%%%%%%%%%%%%%%%%%%%%%%%%%%%%%%%%%%%%%%%%%%%%%
%%%%%%%%%%%%%%%%%%%%%%%%%%%%%%%%%%%%%%%%%%%%%%%%%%%%%%%%%%%%%%%%%%%%%%%%%%%%%
%%%%%%%%%%%%%%%%%%%%%%%%%%%%%%%%%%%%%%%%%%%%%%%%%%%%%%%%%%%%%%%%%%%%%%%%%%%%%



\subsection*{Plan of the paper}

In Sec.~\ref{S:HList}, we briefly review the \HList\
library~\cite{HLIST-HW04}, which provides extensible polymorphic
heterogeneous records with first-class labels.  In
Sec.~\ref{S:simple}, we introduce more basic OO notions such as
objects and constructors. In Sec.~\ref{S:self}, we describe open
recursion, which allows us to cover rich forms of inheritance.  In
Sec.~\ref{S:shapes}, we handle a prototypical scenario for subtype
polymorphism in detail. In~\ref{S:disc}, we very briefly discuss all
remaining issues~---~including some technicalities, conclusions, and
directions for future work.

\oleg{add why we're using OCaml titorial. Maybe the following has to
  be added at the beginning of the appropriate Section?}

There are many OO systems based on open records, e.g., Perl, Python,
Javascript, Lua, OCaml (check the list!). Of these, only Ocaml is
statically typed. It behoves us therefore to compare OOHaskell with
OCaml: OCaml has a similar motivation to ours (bridging OO and
functional-programming, being able to faithfully port OO programs). On
the other hand, OCaml chose to explicitly add extensible records and
extend its type system with recursive types. The comparison with Ocaml
is meaningful therefore: the same motivation, similar base language
(H-M type system, inference), but different ways of introducing
OO. Therefore, we draw many of the examples from OCaml object
tutorial, to specifically contrast Ocaml and OOHaskell code and to
demonstrate the fact that OCaml examples are expressible in OOHaskell,
roughly in the same syntax. We also use the OCaml object tutorial
because it is clear, comprehensive and concise.

