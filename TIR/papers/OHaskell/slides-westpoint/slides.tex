\documentclass{slides}

\usepackage{times}
\usepackage{latexsym}
\usepackage{graphics}
\usepackage[usenames]{pstcol}
\usepackage{url}
\usepackage{fancyvrb}
\usepackage{ifthen}
\usepackage{comment}
\usepackage{boxedminipage}

\begin{comment}
\newenvironment{myslide}{\begin{slide}\color{Blue}\begin{boxedminipage}{1.1\hsize}\begin{boxedminipage}{1\hsize}\color{Black}
\vspace{-170\in}
}{%
\smallskip
\end{boxedminipage}
\end{boxedminipage}
\end{slide}}
\end{comment}
\begin{comment}
\newenvironment{myslide}{\begin{slide}
}{%
\end{slide}}
\end{comment}
\newenvironment{myslide}{\begin{slide}\color{White}\begin{boxedminipage}{1.1\hsize}\color{Black}
\vspace{-170\in}
}{%
\smallskip
\end{boxedminipage}
\end{slide}}
\newcommand{\almostnoskip}{\topsep1pt \parskip1pt \partopsep1pt}
\newcommand{\negskip}{\vspace{-50\in}}
\newcommand{\noskip}{\topsep1pt \parskip1pt \partopsep1pt}
\newcommand{\littleskip}{\topsep8pt \parskip8pt \partopsep8pt}
\newcommand{\header}[1]{{\large \color{Red} #1}}
\newcommand{\bang}[1]{{\color{Blue} \mytextbf{#1}}}
\newcommand{\inp}[1]{{\color{Brown} \mytextbf{#1}}}
\newcommand{\out}[1]{{\color{Black} \mytextbf{#1}}}
\newcommand{\blau}[1]{{\vspace{-50\in}\normalsize \color{Blue} #1}}
\newcommand{\cod}{c}
\newcommand{\myforall}{\ensuremath{\forall}}
\newcommand{\HList}{\textsc{HList}}
\newcommand{\undefined}{\ensuremath{\bot}}
\newcommand{\farr}{\ensuremath{\to}}
\newcommand{\larr}{\ensuremath{\leftarrow}}
\newcommand{\carr}{\ensuremath{\Rightarrow}}
\renewcommand{\qquad}{\hspace{40\in}}
\newcommand{\flc}{$\=$}
\newcommand{\lc}{$\>$}
\newcommand{\alphaq}{\alpha'}
\newcommand{\sforall}{\overline{\forall}}
\newcommand{\strafunski}{\textit{\textsf{Strafunski}}}
\newcommand{\drift}{\emph{DrIFT}}
\newlength{\basewidth}
\newlength{\topwidth}
\newcommand{\superimpose}[2]{%
  \settowidth{\basewidth}{#1}%
  \settowidth{\topwidth}{#2}%
  \ifthenelse{\lengthtest{\basewidth > \topwidth}}%
    {\makebox[0pt][l]{#1}\makebox[\basewidth]{#2}}%
    {\makebox[0pt][l]{#2}\makebox[\topwidth]{#1}}%
}
\newcommand{\myprompt}{ghci\ensuremath{>}}
\newcommand{\mybar}{\ensuremath{|}}
\newcommand{\mylbrace}{\ensuremath{\{}}
\newcommand{\myrbrace}{\ensuremath{\}}}
\newcommand{\mytextbf}[1]{\texttt{\textbf{#1}}}

\begin{document}

\pagestyle{empty}



\begin{comment}
\begin{slide}
 
Press PgDn to start.
 
\end{slide}
\end{comment}



%%%%%%%%%%%%%%%%%%%%%%%%%%%%%%%%%%%%%%%%%%%%%%%%%%%%%%%%%%%%%%%%%%%%%%%%%%%%%%



\begin{myslide}

\bigskip

\begin{center}
\header{{\itshape\textbf{Haskell's overlooked object system}}}
\end{center}

\bigskip

{\small

\blau{

Oleg Kiselyov, FNMO Center, Monterey, CA\\
Ralf L{\"a}mmel, VUA \& CWI, Amsterdam\\
Keean Schupke, Imperial College, London

}


\bigskip

Subtitles:

{\noskip\begin{itemize}
\item Type-level programming.
\item Programmable type systems in Haskell.
\item Mutable Haskell objects for C++ programmers.
\item The Haskell implementation of the OCaml tutorial.
\item No headache today: functional (immutable) objects omitted.
\end{itemize}}

\medskip

$^*$\ We acknowledge input from Chung-chieh Shan.

}

\medskip

\end{myslide}



%%%%%%%%%%%%%%%%%%%%%%%%%%%%%%%%%%%%%%%%%%%%%%%%%%%%%%%%%%%%%%%%%%%%%%%%%%%%%%



\begin{myslide}

\header{OOHaskell meets OCaml}

{\tiny

OCaml (first tutorial example from Sec 3.1 Classes and objects)

\begin{verbatim}
class point =
    object
      val mutable x = 0
      method get_x = x
      method move d = x <- x + d
    end;;
\end{verbatim}

OOHaskell (uses IORefs and HList's extensible records)

\begin{verbatim}
point =
  do
    x <- newIORef 0
    returnIO
      $  mutableX .=. x
     .*. getX     .=. readIORef x
     .*. moveD    .=. (\d -> modifyIORef x ((+) d))
     .*. emptyRecord
\end{verbatim}

}

\end{myslide}



%%%%%%%%%%%%%%%%%%%%%%%%%%%%%%%%%%%%%%%%%%%%%%%%%%%%%%%%%%%%%%%%%%%%%%%%%%%%%%



\begin{myslide}

\header{OOHaskell uses HList labels}

{\small

There are four models, e.g.:

\begin{verbatim}
data MutableX; mutableX = proxy::Proxy MutableX
data GetX;     getX     = proxy::Proxy GetX
data MoveD;    moveD    = proxy::Proxy MoveD
...
\end{verbatim}

Proxies are defined as follows:

\begin{verbatim}
data Proxy e
instance Show (Proxy e) where show _ = "Proxy"

proxy :: Proxy e
proxy =  undefined

toProxy :: e -> Proxy e
toProxy _ = undefined
\end{verbatim}

}

\end{myslide}



%%%%%%%%%%%%%%%%%%%%%%%%%%%%%%%%%%%%%%%%%%%%%%%%%%%%%%%%%%%%%%%%%%%%%%%%%%%%%%



\begin{myslide}

\header{OOHaskell meets OCaml (cont'd)}

{\tiny

OCaml

\begin{verbatim}
#let p = new point;;
val p : point = <obj>
#p#get_x;;
- : int = 0
#p#move 3;;
- : unit = ()
#p#get_x;;
- : int = 3
\end{verbatim}

OOHaskell

\begin{verbatim}
myFirstOOP = do  p <- point
                 p # getX >>= print
                 p # moveD $ 3
                 p # getX >>= print
ghci> myFirstOOP
0
3
\end{verbatim}

}


\end{myslide}



%%%%%%%%%%%%%%%%%%%%%%%%%%%%%%%%%%%%%%%%%%%%%%%%%%%%%%%%%%%%%%%%%%%%%%%%%%%%%%



\begin{myslide}

\header{What would be a powerful object system?}

\medskip

{\small

\noskip\begin{itemize}
\item Encapsulation
\item Inheritance and overriding
\item Statically typed subtyping
\item Interface polymorphism
\item Private methods
\item Multiple inheritance, diamond inheritance
\item Local objects and classes
\item Constructor expressions
\end{itemize}
}

\medskip

OOHaskell provides these\\
without any ad-hoc extensions,\\
without type system extensions.

\end{myslide}



%%%%%%%%%%%%%%%%%%%%%%%%%%%%%%%%%%%%%%%%%%%%%%%%%%%%%%%%%%%%%%%%%%%%%%%%%%%%%%



\begin{myslide}

\header{Selfish OCaml and OOHaskell}

{\tiny

OCaml

\begin{verbatim}
class printable_point x_init =
   object (s)
     val mutable x = x_init
     method get_x  = x
     method move d = x <- x + d
     method print  = print_int s#get_x
   end;;
\end{verbatim}

OOHaskell

\begin{verbatim}
class_printable_point x_init self
  = do
      x <- newIORef x_init
      returnIO
        $  mutableX .=. x
       .*. getX     .=. readIORef x
       .*. moveD    .=. (\d -> modifyIORef x ((+) d))
       .*. ooprint  .=. ((self # getX ) >>= print)
       .*. emptyRecord
\end{verbatim}

}

\end{myslide}



%%%%%%%%%%%%%%%%%%%%%%%%%%%%%%%%%%%%%%%%%%%%%%%%%%%%%%%%%%%%%%%%%%%%%%%%%%%%%%



\begin{myslide}

\header{The `shapes' OO benchmark}

\bigskip

\url{http://www.angelfire.com/tx4/cus/shapes/}

\bigskip

Involves inheritance, polymorphism, up-cast, ...

\bigskip

{\small

Output behaviour

\begin{verbatim}
Drawing a Rectangle at:(10,20), width 5, height 6
Drawing a Rectangle at:(110,120), width 5, height 6
Drawing a Circle at:(15,25), radius 8
Drawing a Circle at:(115,125), radius 8
Drawing a Rectangle at:(0,0), width 30, height 15
\end{verbatim}

}

\end{myslide}



%%%%%%%%%%%%%%%%%%%%%%%%%%%%%%%%%%%%%%%%%%%%%%%%%%%%%%%%%%%%%%%%%%%%%%%%%%%%%%



\begin{myslide}

\header{Reference implementation in C++}

(Contributed by Jim Weirich)

Shape Class (Shape.h)

{\tiny

\begin{verbatim}
class Shape {

public:
   Shape(int newx, int newy);
   int getX();
   int getY();
   void setX(int newx);
   void setY(int newy);
   void moveTo(int newx, int newy);
   void rMoveTo(int deltax, int deltay);
   virtual void draw();

private:
   int x;
   int y;
};
\end{verbatim}

}

\end{myslide}



%%%%%%%%%%%%%%%%%%%%%%%%%%%%%%%%%%%%%%%%%%%%%%%%%%%%%%%%%%%%%%%%%%%%%%%%%%%%%%



\begin{myslide}

\header{Reference implementation in C++ (cont'd)}

Shape Implementation (Shape.cpp)

{\tiny

\begin{verbatim}
// constructor
Shape::Shape(int newx, int newy) {
   moveTo(newx, newy);
}

// accessors for x & y
int Shape::getX() { return x; }
int Shape::getY() { return y; }
void Shape::setX(int newx) { x = newx; }
void Shape::setY(int newy) { y = newy; }

// move the shape position
void Shape::moveTo(int newx, int newy) {
   setX(newx);
   setY(newy);
}
void Shape::rMoveTo(int deltax, int deltay) {
   moveTo(getX() + deltax, getY() + deltay);
}

// abstract draw method
void Shape::draw() {
}
\end{verbatim}

}

\end{myslide}



%%%%%%%%%%%%%%%%%%%%%%%%%%%%%%%%%%%%%%%%%%%%%%%%%%%%%%%%%%%%%%%%%%%%%%%%%%%%%%



\begin{myslide}

\header{Reference implementation in C++ (cont'd)}

Rectangle Class (Rectangle.h)

{\tiny

\begin{verbatim}
class Rectangle: public Shape {

public:
  Rectangle(int newx, int newy, int newwidth, int newheight);
  int getWidth();
  int getHeight();
  void setWidth(int newwidth);
  void setHeight(int newheight);
  void draw();

private:
  int width;
  int height;
};
\end{verbatim}

}

\end{myslide}



%%%%%%%%%%%%%%%%%%%%%%%%%%%%%%%%%%%%%%%%%%%%%%%%%%%%%%%%%%%%%%%%%%%%%%%%%%%%%%



\begin{myslide}

\header{Reference implementation in C++ (cont'd)}

Rectangle Implementation (Rectangle.cpp)

{\tiny

\begin{verbatim}
// constructor
Rectangle::Rectangle(int newx, int newy, int newwidth
                    , int newheight): Shape(newx, newy) {
   setWidth(newwidth);
   setHeight(newheight);
}

// accessors for width and height
int Rectangle::getWidth() { return width; }
int Rectangle::getHeight() { return height; }
void Rectangle::setWidth(int newwidth)
  { width = newwidth; }
void Rectangle::setHeight(int newheight)
  { height = newheight; }

// draw the rectangle
void Rectangle::draw() {
   cout << "Drawing a Rectangle at:("
        << getX() << "," << getY()
        << "), width " << getWidth()
        << ", height " << getHeight() << endl;
}
\end{verbatim}

}

\end{myslide}



%%%%%%%%%%%%%%%%%%%%%%%%%%%%%%%%%%%%%%%%%%%%%%%%%%%%%%%%%%%%%%%%%%%%%%%%%%%%%%



\begin{myslide}

\header{Reference implementation in C++ (cont'd)}

Circle Interface (Circle.h)

{\tiny

\begin{verbatim}
class Circle: public Shape {

public:
   Circle(int newx, int newy, int newradius);
   int getRadius();
   void setRadius(int newradius);
   void draw();

private:
   int radius;
};
\end{verbatim}

}

\end{myslide}



%%%%%%%%%%%%%%%%%%%%%%%%%%%%%%%%%%%%%%%%%%%%%%%%%%%%%%%%%%%%%%%%%%%%%%%%%%%%%%



\begin{myslide}

\header{Reference implementation in C++ (cont'd)}

Circle Implementation (Circle.cpp)

{\tiny

\begin{verbatim}
// constructor
Circle::Circle(int newx, int newy
              , int newradius): Shape(newx, newy) {
   setRadius(newradius);
}

// accessors for the radius
int Circle::getRadius() { return radius; }
void Circle::setRadius(int newradius) { radius = newradius; }

// draw the circle
void Circle::draw() {
  cout << "Drawing a Circle at:("
       << getX() << "," << getY()
       << "), radius " << getRadius() << endl;
}
\end{verbatim}

}

\end{myslide}



%%%%%%%%%%%%%%%%%%%%%%%%%%%%%%%%%%%%%%%%%%%%%%%%%%%%%%%%%%%%%%%%%%%%%%%%%%%%%%



\begin{myslide}

\header{Reference implementation in C++ (cont'd)}

Try shapes module (Polymorph.cpp)

{\tiny

\begin{verbatim}
int main(void) {
   // set up array to the shapes
   Shape *scribble[2];
   scribble[0] = new Rectangle(10, 20, 5, 6);
   scribble[1] = new Circle(15, 25, 8);

   // iterate through the array
   // and handle shapes polymorphically
   for (int i = 0; i < 2; i++) {
      scribble[i]->draw();
      scribble[i]->rMoveTo(100, 100);
      scribble[i]->draw();
   }

   // call a rectangle specific function
   Rectangle *arec = new Rectangle(0, 0, 15, 15);
   arec->setWidth(30);
   arec->draw();
}
\end{verbatim}

}

\end{myslide}



%%%%%%%%%%%%%%%%%%%%%%%%%%%%%%%%%%%%%%%%%%%%%%%%%%%%%%%%%%%%%%%%%%%%%%%%%%%%%%



\begin{myslide}

\header{Reference implementation in C++ (cont'd)}

Output

{\tiny

\begin{verbatim}
Drawing a Rectangle at:(10,20), width 5, height 6
Drawing a Rectangle at:(110,120), width 5, height 6
Drawing a Circle at:(15,25), radius 8
Drawing a Circle at:(115,125), radius 8
Drawing a Rectangle at:(0,0), width 30, height 15
\end{verbatim}

}

\end{myslide}



%%%%%%%%%%%%%%%%%%%%%%%%%%%%%%%%%%%%%%%%%%%%%%%%%%%%%%%%%%%%%%%%%%%%%%%%%%%%%%



\begin{myslide}

\header{Haskell does not (yet) meet C++}

(contributed by Chris Rathman)

The Shape class

{\tiny

\begin{verbatim}
-- declare method interfaces for the shape superclass
class Shape a where
   getX :: a -> Int
   getY :: a -> Int
   setX :: a -> Int -> a
   setY :: a -> Int -> a
   moveTo :: a -> Int -> Int -> a
   rMoveTo :: a -> Int -> Int -> a
   draw :: a -> IO()
\end{verbatim}

}

\end{myslide}



%%%%%%%%%%%%%%%%%%%%%%%%%%%%%%%%%%%%%%%%%%%%%%%%%%%%%%%%%%%%%%%%%%%%%%%%%%%%%%



\begin{myslide}

\header{Haskell does not (yet) meet C++ (cont'd)}

Manual polymorphism

{\tiny

\begin{verbatim}
-- declare the constructor for the existential type
data ExistentialShape =
   forall a. Shape a => MakeExistentialShape a

-- map the methods for the existential type
instance Shape ExistentialShape where
  getX (MakeExistentialShape a) = getX a
  getY (MakeExistentialShape a) = getY a
  setX (MakeExistentialShape a) newx
    = MakeExistentialShape(setX a newx)
  setY (MakeExistentialShape a) newy
    = MakeExistentialShape(setY a newy)
  moveTo (MakeExistentialShape a) newx newy
    = MakeExistentialShape(moveTo a newx newy)
  rMoveTo (MakeExistentialShape a) deltax deltay
    = MakeExistentialShape(rMoveTo a deltax deltay)
  draw (MakeExistentialShape a) = draw a
\end{verbatim}

}

\end{myslide}



%%%%%%%%%%%%%%%%%%%%%%%%%%%%%%%%%%%%%%%%%%%%%%%%%%%%%%%%%%%%%%%%%%%%%%%%%%%%%%



\begin{myslide}

\header{Haskell does not (yet) meet C++ (cont'd)}

The class of rectangles; and data.

{\tiny

\begin{verbatim}
-- declare method interfaces for rectangle subclass
class Shape a => Rectangle a where
   getWidth :: a -> Int
   getHeight :: a -> Int
   setWidth :: a -> Int -> a
   setHeight :: a -> Int -> a

-- declare the constructor for rectangle class
data RectangleInstance
  = MakeRectangle {x, y, width, height :: Int}
   deriving(Eq, Show)
\end{verbatim}

}

\end{myslide}



%%%%%%%%%%%%%%%%%%%%%%%%%%%%%%%%%%%%%%%%%%%%%%%%%%%%%%%%%%%%%%%%%%%%%%%%%%%%%%



\begin{myslide}

\header{Haskell does not (yet) meet C++ (cont'd)}

One instance for rectangles

{\tiny

\begin{verbatim}
-- define the methods for shape superclass
instance Shape RectangleInstance where
   getX = x
   getY = y
   setX a newx = a {x = newx}
   setY a newy = a {y = newy}
   moveTo a newx newy = a {x = newx, y = newy}
   rMoveTo a deltax deltay
     = a {x = ((getX a) + deltax), y = ((getY a) + deltay)}
   draw a = ... -- too long
\end{verbatim}

}

\end{myslide}



%%%%%%%%%%%%%%%%%%%%%%%%%%%%%%%%%%%%%%%%%%%%%%%%%%%%%%%%%%%%%%%%%%%%%%%%%%%%%%



\begin{myslide}

\header{Haskell does not (yet) meet C++ (cont'd)}

Another instance for rectangles

{\tiny

\begin{verbatim}
-- define the methods for rectangle subclass
instance Rectangle RectangleInstance where
   getWidth = width
   getHeight = height
   setWidth a newwidth = a {width = newwidth}
   setHeight a newheight = a {height = newheight}
\end{verbatim}

}

Likewise for circles.

\end{myslide}



%%%%%%%%%%%%%%%%%%%%%%%%%%%%%%%%%%%%%%%%%%%%%%%%%%%%%%%%%%%%%%%%%%%%%%%%%%%%%%



\begin{myslide}

\header{Haskell does not (yet) meet C++ (cont'd)}

Try shapes

{\tiny

\begin{verbatim}
main =
   do -- handle the shapes polymorphically
      drawloop scribble
      -- handle rectangle specific instance
      draw (setWidth arectangle 30)
   where
      -- create some shape instances (using ex. wrapper)
      scribble = [
         MakeExistentialShape (MakeRectangle 10 20 5 6),
         MakeExistentialShape (MakeCircle 15 25 8)]
      -- create a rectangle instance
      arectangle = (MakeRectangle 0 0 15 15)

-- iterate through the list of shapes and draw
drawloop [] = return True
drawloop (x:xs) =
   do draw x
      draw (rMoveTo x 100 100)
      drawloop xs
\end{verbatim}

}

\end{myslide}



%%%%%%%%%%%%%%%%%%%%%%%%%%%%%%%%%%%%%%%%%%%%%%%%%%%%%%%%%%%%%%%%%%%%%%%%%%%%%%



\begin{myslide}

\header{OOHaskell meets C++}

The class for shapes

{\tiny

\begin{verbatim}
class_shape x_init y_init self
  = do x <- newIORef x_init
       y <- newIORef y_init
       returnIO $
            getX     .=. readIORef x
        .*. getY     .=. readIORef y
        .*. setX     .=. (\newx -> writeIORef x newx)
        .*. setY     .=. (\newy -> writeIORef y newy)
        .*. moveTo   .=. (\newx newy -> do
                                         (self # setX) newx
                                         (self # setY) newy 
                         )
        .*. rMoveTo  .=. (\deltax deltay ->
               do x <- self # getX
                  y <- self # getY
                  (self # moveTo) (x + deltax) (y + deltay) )
        .*. emptyRecord
\end{verbatim}

}

\end{myslide}



%%%%%%%%%%%%%%%%%%%%%%%%%%%%%%%%%%%%%%%%%%%%%%%%%%%%%%%%%%%%%%%%%%%%%%%%%%%%%%



\begin{myslide}

\header{OOHaskell meets C++ (cont'd)}

The interface for all shapes

{\tiny

\begin{verbatim}
type ShapeInterface a
 = Record (  (Proxy GetX    , IO a)
         :*: (Proxy GetY    , IO a)
         :*: (Proxy SetX    , a -> IO ())
         :*: (Proxy SetY    , a -> IO ())
         :*: (Proxy MoveTo  , a -> a -> IO ())
         :*: (Proxy RMoveTo , a -> a -> IO ())
         :*: (Proxy Draw    , IO ())
         :*: HNil )
\end{verbatim}

}

\end{myslide}



%%%%%%%%%%%%%%%%%%%%%%%%%%%%%%%%%%%%%%%%%%%%%%%%%%%%%%%%%%%%%%%%%%%%%%%%%%%%%%



\begin{myslide}

\header{OOHaskell meets C++ (cont'd)}

The class for rectangles

{\tiny

\begin{verbatim}
class_rectangle x y width height self
  = do
      super <- class_shape x y self
      w <- newIORef width
      h <- newIORef height
      returnIO $
           getWidth  .=. readIORef w
       .*. getHeight .=. readIORef h
       .*. setWidth  .=. (\neww -> writeIORef w neww)
       .*. setHeight .=. (\newh -> writeIORef h newh)
       .*. draw      .=. 
           do
              putStr  "Drawing a Rectangle at:(" <<
                      self # getX << ls "," << self # getY <<
                      ls "), width " << self # getWidth <<
                      ls ", height " << self # getHeight <<
                      ls "\n"
       .*. super
\end{verbatim}

}

\end{myslide}



%%%%%%%%%%%%%%%%%%%%%%%%%%%%%%%%%%%%%%%%%%%%%%%%%%%%%%%%%%%%%%%%%%%%%%%%%%%%%%



\begin{myslide}

\header{OOHaskell meets C++ (cont'd)}

The class for circles

{\tiny

\begin{verbatim}
class_circle x y radius self
  = do
      super <- class_shape x y self
      r <- newIORef radius
      returnIO $
           getRadius  .=. readIORef r
       .*. setRadius  .=. (\newr -> writeIORef r newr)
       .*. draw       .=. 
           do
              putStr  "Drawing a Circle at:(" <<
                      self # getX << ls "," << self # getY <<
                      ls "), radius " << self # getRadius <<
                      ls "\n"
       .*. super
\end{verbatim}

}

\end{myslide}



%%%%%%%%%%%%%%%%%%%%%%%%%%%%%%%%%%%%%%%%%%%%%%%%%%%%%%%%%%%%%%%%%%%%%%%%%%%%%%



\begin{myslide}

\header{OOHaskell meets C++ (cont'd)}

Try shapes

{\tiny

\begin{verbatim}
main = do
       -- set up array to the shapes.
       s1 <- mfix (class_rectangle 10 20 5 6)
       s2 <- mfix (class_circle 15 25 8)
       let scribble :: [ShapeInterface Int]
           scribble = [upCast s1, upCast s2]
       
       -- iterate through the array
       -- and handle shapes polymorphically
       mapM_ ( \shape -> do shape # draw
                            (shape # rMoveTo) 100 100
                            shape # draw ) scribble

       -- call a rectangle specific function
       arec <- mfix (class_rectangle 0 10 15 15)
       arec # setWidth $ 30
       arec # draw
\end{verbatim}

}

\end{myslide}



%%%%%%%%%%%%%%%%%%%%%%%%%%%%%%%%%%%%%%%%%%%%%%%%%%%%%%%%%%%%%%%%%%%%%%%%%%%%%%



\begin{myslide}

\header{OOHaskell internals}

\bigskip

\littleskip\begin{itemize}
\item Representation of records
\item Heterogeneous lists
\item Uniqueness of labels
\item Computations on types
\end{itemize}

\end{myslide}



%%%%%%%%%%%%%%%%%%%%%%%%%%%%%%%%%%%%%%%%%%%%%%%%%%%%%%%%%%%%%%%%%%%%%%%%%%%%%%



\begin{myslide}

\header{One model of extensible records}

\vspace{-77\in}

\blau{Records as maps from labels to values}

\smallskip

\begin{Verbatim}[fontfamily=courier,fontsize=\normalsize,commandchars=\\\{\}]
 newtype Record r = Record r
\end{Verbatim}

\smallskip

\begin{Verbatim}[fontfamily=courier,fontsize=\normalsize,commandchars=\\\{\}]
 mkRecord :: (HZip ls vs r, HLabelSet ls)
          \carr r \farr Record r
 mkRecord = Record
\end{Verbatim}

\bigskip

Labels and values are heterogeneous lists.\\
Records are (constrained) heterogeneous lists of pairs.

\end{myslide}



%%%%%%%%%%%%%%%%%%%%%%%%%%%%%%%%%%%%%%%%%%%%%%%%%%%%%%%%%%%%%%%%%%%%%%%%%%%%%%



\begin{myslide}

\bigskip
\bigskip

\blau{Lists as nested, binary, right-associative products}

\medskip

\begin{Verbatim}[fontfamily=courier,fontsize=\small,commandchars=\\\{\}]
 data  HNil      = HNil
 data  HCons e l = HCons e l
 type  e :*: l   = HCons e l
       e .*. l   = HCons e l
 class HList
 instance HList HNil
 instance HList l \carr HList (HCons e l)
\end{Verbatim}

\medskip

The HList library defines list-processing functions HList.\\
(... and on TIPs, TICs, etc.)

\end{myslide}



%%%%%%%%%%%%%%%%%%%%%%%%%%%%%%%%%%%%%%%%%%%%%%%%%%%%%%%%%%%%%%%%%%%%%%%%%%%%%%



\begin{myslide}

\bigskip
\bigskip

\blau{Zipping on HLists}

\medskip

\begin{Verbatim}[fontfamily=courier,fontsize=\tiny,commandchars=\\\!\?]
 class HZip x y l | x y \farr l, l \farr x y
  where hZip :: x \farr y \farr l
        hUnzip :: l \farr (x,y)

 instance HZip HNil HNil HNil
  where hZip HNil HNil = HNil
        hUnzip HNil = (HNil,HNil)

 instance HZip tx ty l
       \carr HZip (HCons hx tx) (HCons hy ty)
                  (HCons (hx,hy) l)
  where
   hZip (HCons hx tx) (HCons hy ty)
      = HCons (hx,hy) (hZip tx ty)
   hUnzip (HCons (hx,hy) l)
        = (HCons hx tx, HCons hy ty)
    where (tx,ty) = hUnzip l
\end{Verbatim}


\end{myslide}



%%%%%%%%%%%%%%%%%%%%%%%%%%%%%%%%%%%%%%%%%%%%%%%%%%%%%%%%%%%%%%%%%%%%%%%%%%%%%%



\begin{myslide}

\bigskip
\bigskip

\blau{No label occurs twice}

\smallskip

\begin{Verbatim}[fontfamily=courier,fontsize=\small,commandchars=\\\{\}]
 class HLabelSet ls
 instance HLabelSet HNil
 instance (HMember x ls HFalse, HLabelSet ls)
       \carr  HLabelSet (HCons x ls)
\end{Verbatim}

\vspace{-77\in}

\blau{Type-level membership test}

\smallskip

\begin{Verbatim}[fontfamily=courier,fontsize=\small,commandchars=\\\{\}]
 class HBool b \carr HMember e l b | e l -> b
 instance HMember e HNil HFalse
 instance ( HEq e e' b
          , HMember e l b', HOr b b' b'' )
       \carr HMember e (HCons e' l) b''
\end{Verbatim}

\vspace{-77\in}

\blau{Type-level equality}

\smallskip

\begin{Verbatim}[fontfamily=courier,fontsize=\small,commandchars=\\\{\}]
 class HBool b \carr HEq x y b | x y \farr b
\end{Verbatim}

\vspace{-77\in}

\blau{Equality on labels boils down to type equality}

\smallskip

\begin{Verbatim}[fontfamily=courier,fontsize=\small,commandchars=\\\{\}]
 instance TypeEq x y b
          \carr HEq (Proxy x) (Proxy y) b
\end{Verbatim}

\end{myslide}



%%%%%%%%%%%%%%%%%%%%%%%%%%%%%%%%%%%%%%%%%%%%%%%%%%%%%%%%%%%%%%%%%%%%%%%%%%%%%%



\begin{myslide}

\header{Building extensible records}

\vspace{-77\in}

\blau{The empty record}

\smallskip

\begin{Verbatim}[fontfamily=courier,fontsize=\small,commandchars=\\\{\}]
emptyRecord = mkRecord $ hZip HNil HNil
\end{Verbatim}

\vspace{-77\in}

\blau{An overloaded consing operation}

\smallskip

\begin{Verbatim}[fontfamily=courier,fontsize=\small,commandchars=\\\{\}]
 (.*.) :: HExtend e l l' \carr e \farr l \farr l'
 (.*.) =  hExtend
\end{Verbatim}

\smallskip

\begin{Verbatim}[fontfamily=courier,fontsize=\small,commandchars=\\\{\}]
 instance ( HZip ls vs r,     HZip ls' vs' r'
          , HExtend l ls ls', HExtend v vs vs'
          , HLabelSet ls'
          ) \carr HExtend (l,v) (Record r) (Record r')
  where
   hExtend (l,v) (Record r) = mkRecord r'
    where
     (ls,vs) = hUnzip r
     ls'     = hExtend l ls
     vs'     = hExtend v vs
     r'      = hZip ls' vs'
\end{Verbatim}

\end{myslide}



%%%%%%%%%%%%%%%%%%%%%%%%%%%%%%%%%%%%%%%%%%%%%%%%%%%%%%%%%%%%%%%%%%%%%%%%%%%%%%



\begin{myslide}

\header{Access operations for extensible records}

\vspace{-77\in}

\blau{Deletion}

\smallskip

\begin{Verbatim}[fontfamily=courier,fontsize=\small,commandchars=\\\{\}]
 (Record r) .-. l = Record r'
   where
    (ls,vs) = hUnzip r
    n       = hFind l ls -- uses HEq on labels
    ls'     = hDeleteAtHNat n ls
    vs'     = hDeleteAtHNat n vs
    r'      = hZip ls' vs'
\end{Verbatim}

\bigskip

Likewise for other record operations.

\end{myslide}



%%%%%%%%%%%%%%%%%%%%%%%%%%%%%%%%%%%%%%%%%%%%%%%%%%%%%%%%%%%%%%%%%%%%%%%%%%%%%%



\begin{myslide}

\header{Up-cast for records}

\vspace{-77\in}

\blau{Casting a record up to a supertype}

\smallskip

\begin{Verbatim}[fontfamily=courier,fontsize=\small,commandchars=\\\{\}]
class UpCast a b where
    upCast :: Record a \farr Record b

instance UpCast a HNil where
    upCast _ = emptyRecord

instance ( UpCast r r'
         , HExtract r l v
         )
           \carr UpCast r (HCons (l,v) r')
  where
    upCast (Record r) = Record (HCons (l,v) r')
      where
        (Record r')    = upCast (Record r)
        ((l,v)::(l,v)) = hExtract r
\end{Verbatim}

\end{myslide}



%%%%%%%%%%%%%%%%%%%%%%%%%%%%%%%%%%%%%%%%%%%%%%%%%%%%%%%%%%%%%%%%%%%%%%%%%%%%%%



\begin{myslide}

\bigskip
\bigskip

\blau{Component extraction from a record}

\bigskip

\begin{Verbatim}[fontfamily=courier,fontsize=\tiny,commandchars=\\\{\}]
class  HExtract r l v
 where hExtract :: r \farr (l,v)

instance ( TypeEq l l1 b
         , HExtractBool b (HCons (l1,v1) r) l v
         ) \carr HExtract (HCons (l1,v1) r) l v
  where
    hExtract = hExtractBool (undefined::b)

class HBool b \carr HExtractBool b r l v
  where hExtractBool :: b \farr r \farr (l,v)

instance TypeCast v1 v
      \carr HExtractBool HTrue (HCons (l,v1) r) l v
  where
    hExtractBool _ (HCons (l,v) _) = (l,typeCast v)

instance HExtract r l v
      \carr HExtractBool HFalse (HCons (l1,v1) r) l v
  where
    hExtractBool _ (HCons _ r) = hExtract r
\end{Verbatim}

\end{myslide}



%%%%%%%%%%%%%%%%%%%%%%%%%%%%%%%%%%%%%%%%%%%%%%%%%%%%%%%%%%%%%%%%%%%%%%%%%%%%%%



\begin{myslide}

\header{Computations on types}

\vspace{-77\in}

\blau{Compute-type level equality}

\smallskip

\begin{Verbatim}[fontfamily=courier,fontsize=\small,commandchars=\\\{\}]
 class HBool b \carr TypeEq x y b | x y \farr b
\end{Verbatim}

\smallskip

\begin{Verbatim}[fontfamily=courier,fontsize=\tiny,commandchars=\\\{\}]
 data Proxy e; proxy :: Proxy e; proxy = \undefined
 proxyEq :: TypeEq t t' b \carr Proxy t \farr Proxy t' \farr b
 proxyEq _ _ = \undefined
\end{Verbatim}

\vspace{-77\in}

\blau{Type-level type-safe cast}

\smallskip

\begin{Verbatim}[fontfamily=courier,fontsize=\small,commandchars=\\\{\}]
 class HMaybe m \carr Cast x m y | x y \farr m
  where cast :: x \farr m y
\end{Verbatim}

\vspace{-77\in}

\blau{Reified type unification}

\smallskip

\begin{Verbatim}[fontfamily=courier,fontsize=\small,commandchars=\\\{\}]
 class TypeCast x y | x \farr y, y \farr x
  where typeCast :: x \farr y
\end{Verbatim}

\end{myslide}



%%%%%%%%%%%%%%%%%%%%%%%%%%%%%%%%%%%%%%%%%%%%%%%%%%%%%%%%%%%%%%%%%%%%%%%%%%%%%%



\begin{myslide}

\header{Want to see an implementation?}

\vspace{-77\in}

\blau{Here is one.}

\vspace{-77\in}

{\tiny (Requires GHC and a separate compilation trick.)}

\bigskip

\begin{Verbatim}[fontfamily=courier,fontsize=\tiny,commandchars=\\\{\}]
  instance TypeEq x x HTrue
  instance (HBool b, TypeCast HFalse b) \carr TypeEq x y b
\end{Verbatim}
\medskip
\begin{Verbatim}[fontfamily=courier,fontsize=\tiny,commandchars=\\\{\}]
  instance Cast x x HJust
   where cast x = HJust x
  instance TypeCast HNothing m \carr Cast x y m
   where cast x = HNothing
\end{Verbatim}
\medskip
\begin{Verbatim}[fontfamily=courier,fontsize=\tiny,commandchars=\\\{\}]
  instance TypeCast x x
   where typeCast = id
\end{Verbatim}

\end{myslide}



%%%%%%%%%%%%%%%%%%%%%%%%%%%%%%%%%%%%%%%%%%%%%%%%%%%%%%%%%%%%%%%%%%%%%%%%%%%%%%



\begin{myslide}

\header{Want to see an implementation?}

\vspace{-77\in}

\blau{Here is another one.}

\vspace{-77\in}

{\tiny (Requires GHC and a `two-class' trick.)}

\bigskip

\begin{Verbatim}[fontfamily=courier,fontsize=\tiny,commandchars=\\\{\}]
 instance TypeCast'  () a b \carr TypeCast a b
    where typeCast x = typeCast' () x
\end{Verbatim}
\smallskip
\begin{Verbatim}[fontfamily=courier,fontsize=\tiny,commandchars=\\\{\}]
 class TypeCast'  t a b | t a \farr b, t b \farr a
  where typeCast'  :: t \farr a \farr b
\end{Verbatim}
\smallskip
\begin{Verbatim}[fontfamily=courier,fontsize=\tiny,commandchars=\\\{\}]
 class TypeCast'' t a b | t a \farr b, t b \farr a
  where typeCast'' :: t \farr a \farr b
\end{Verbatim}
\smallskip
\begin{Verbatim}[fontfamily=courier,fontsize=\tiny,commandchars=\\\{\}]
 instance TypeCast'' t a b \carr TypeCast' t a b
    where typeCast' = typeCast''
\end{Verbatim}
\smallskip
\begin{Verbatim}[fontfamily=courier,fontsize=\tiny,commandchars=\\\{\}]
 instance TypeCast'' () a a
    where typeCast'' _ x  = x
\end{Verbatim}

\end{myslide}



%%%%%%%%%%%%%%%%%%%%%%%%%%%%%%%%%%%%%%%%%%%%%%%%%%%%%%%%%%%%%%%%%%%%%%%%%%%%%%



\begin{myslide}

\header{Yet another implementation}

\vspace{-77\in}

{\tiny (Works fine with modest extensions.)}

\vspace{-42\in}

\blau{Refiy \emph{Data.Typeable}}

\medskip

\begin{Verbatim}[fontfamily=courier,fontsize=\small,commandchars=\\\{\}]
 class TTypeable a b | a \farr b
 instance TTypeable Bool (HCons HZero HNil)
 instance TTypeable Int  (HCons (HSucc HZero) HNil)
 instance (TTypeable a al, TTypeable b bl)
  \carr TTypeable (a \farr b)
        (HCons (HSucc (HSucc HZero))
               (HCons al (HCons bl HNil)))
\end{Verbatim}

\vspace{-77\in}

\blau{Reify type equality}

\medskip

\begin{Verbatim}[fontfamily=courier,fontsize=\small,commandchars=\\\{\}]
 instance ( TTypeable t tt, TTypeable t' tt'
          , HEq tt tt' b ) \carr TypeEq t t' b
\end{Verbatim}


\end{myslide}



%%%%%%%%%%%%%%%%%%%%%%%%%%%%%%%%%%%%%%%%%%%%%%%%%%%%%%%%%%%%%%%%%%%%%%%%%%%%%%



\begin{myslide}

\header{Subtyping for records via projection}

\bigskip

\begin{Verbatim}[fontfamily=courier,fontsize=\small,commandchars=\\\{\}]
 instance
   ( HZip ls vs r'
   , HProjectByLabels ls (Record r) (Record r')
   )
     \carr SubType (Record r) (Record r')
\end{Verbatim}

{\small

Guess what the definition of type equality is.

}

\end{myslide}



%%%%%%%%%%%%%%%%%%%%%%%%%%%%%%%%%%%%%%%%%%%%%%%%%%%%%%%%%%%%%%%%%%%%%%%%%%%%%%



\begin{myslide}

\bigskip
\bigskip
\bigskip

\header{Last slide}

{\small

The rest is engineering:

\begin{itemize}
\item A few more type-level operations on objects.
\item Syntactic sugar (if required) -- local translation is sufficient.
\item Better error messages -- needed for any type-level programming.
\end{itemize}

Please check: \url{http://www.cwi.nl/~ralf/OOHaskell/}

Thank you for your attention.

}

\end{myslide}



%%%%%%%%%%%%%%%%%%%%%%%%%%%%%%%%%%%%%%%%%%%%%%%%%%%%%%%%%%%%%%%%%%%%%%%%%%%%%%



\end{document}
