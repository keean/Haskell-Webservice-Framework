\medskip

\section{Related work}
\label{S:related}

The literature on object representation and encoding is quite
extensive, e.g., ~\cite{Cardelli-on-understanding,
Poll97,AC96,Ohori95,PT94,BM92}.  Often objects are primitive
constructs in the language.  Most often discussed are pure functional
objects. Most often the type systems of object models are variants of
system $F_{\leq}$ (polymorphic lambda-calculus plus subtyping).

\medskip

\subsection{ML-ART}

We identify the {ML-ART} paper \cite{ML-ART} (see also
\cite{RV97}) as the closest to us~---~in motivation and spirit (but
not in the technical approach). We share with Didier R{\'e}my the goal
of discovering the small set of language features that make OO
programming possible. We aim not at introducing objects but at being
able to implement objects~---~as a \emph{library} feature. Therefore,
several OO styles can be implemented, for different classes of users
and classes of problems. One does not need to learn any new language
and can discover OOP progressively. Both {ML-ART} and OOHaskell base
their object systems on polymorphic records (only we use records of
closures, in this paper). Both OOHaskell and{ML-ART} deal with
mutable objects.

What fundamentally sets us apart from {ML-ART} is the different source
language: Haskell. In Haskell, we can implement polymorphic records
natively rather than via an extension. We can avoid row variables and
their related complexities. Our records permit introspection and thus
let us \emph{implement} subtyping. Unlike {ML-ART}, OOHaskell can
compute the most common type of two record types, without any explicit
coercions or user annotations.  Constructing objects via
value recursion is one of three major ways of implementing OOP (the
other two are type recursion and existential abstraction
\cite{PT94}). The value recursion seems simpler but is generally
unsafe, because of the possibility of accessing a slot before it has
been filled in. But in a call-by-name or similar language the
fixpoints are always safe, as mentioned already in \cite{ML-ART}.
 
Our current implementation has strong similarities with
prototype-based systems (such as Self~\cite{Self}) in that mutable
fields and method `pointers' are a part of the same record. This does
not have to be the case~---~and in fact, in our current efforts on
pure-functional objects we separate the two tables (in the manner
similar to object realisations in C++ or Java).



\subsection{Haskell extensions or variations}

There were attempts to bring OO to Haskell by a language extension. An
early attempt is Haskell++~\cite{HS95} by Hughes and Sparud. The
authors motivated their extension by the perception that Haskell lacks
the form of incremental reuse that is offered by inheritance in
object-oriented languages. Our approach uses common extensions of the
Hindley-Milner type system to provide the key OO notions.  So in a
way, Haskell's fitness for OOP just had to be discovered, which is the
contribution of this paper. Nordlander has delivered a comprehensive
OOP variation on
Haskell~---~O`Haskell~\cite{Nordlander98,Nordlander02}, which extends
Haskell with reactive objects and subtyping. The subtyping part is a
formidable extension. The reactive object part combines stateful
objects and concurrent execution, again a major extension. Our
development shows that no extension of Haskell is necessary for
stateful objects with a faithful object-oriented type system. Finally,
there is Mondrian~---~a NET-able variation on Haskell. In the original
paper on the design and implementation of Mondrian~\cite{MC97}, Meijer
and Claessen write: ``The design of a type system that deals with
subtyping, higher-order functions, and objects is a formidable
challenge ...''. Rather than designing a very complicated language,
the overall principle underlying Mondrian was to obtain a simple
Haskell dialect with an object-oriented flavor. To this end, algebraic
datatypes and type classes were combined into a simple object-oriented
type system with no real subtyping, with completely covariant
type-checking. In Mondrian, runtime errors of the kind ``message not
understood'' are considered a problem akin to partial functions with
non-exhaustive case discriminations. We raise the bar by providing
proper subtyping (``all message will be understood'') and other OOP
concepts in Haskell without extending the Haskell type system.



\subsection{Other OO encodings for Haskell}

The exercise of encoding some tenets of OO in Haskell seems to be a
favourite pastime of Haskell aficionados. The most pressing motivation
for such efforts has been the goal to import foreign libraries or
components into Haskell~\cite{FLMPJ99,SPJ01,PC03}. This problem domain
makes simplifying assumptions when compared to actual OO program
development in Haskell:

\begin{itemize}\noskip
\item Object state does not reside in Haskell data.
\item There are only (opaque) object ids referring to the foreign site.
\item (I.e., state is solely accessed through methods (``properties'').
\item Haskell methods are (often generated) stubs for foreign code.
\item As a result, such OO styles just deal with interfaces.
\item Also, no actual (sub)classes are written by the programmer.
\end{itemize}

One approach is to use phantom types for recording inheritance
relationships \cite{FLMPJ99}. One represents each interface by an
(empty) datatype with a type parameter for extension. After due
consideration, it turns out that this approach is a restricted version
of what Burton called ``type extension through polymorphism'': even
records can be made extensible through the provision of a polymorphic
dummy field~\cite{Burton90}. Once we do not maintain Haskell data for
objects, there is no need to maintain a record type, but the extension
point is a left over, and it becomes a phantom.  We ``re-generalize''
the phantom approach in the appendix. Its remaining problems are: lack
of support for multiple inheritance, lack of proper encapsulation
(which could be fixed at the expense of self-reference problems), code
bloat for accessors in subclasses, a closed-world assumption on base
classes, and the use of existentials for coercion to common base types
(cf.\ Sec.~\ref{S:ex} for the implied problems).

Another approach is to set up a Haskell type class to represent the
subtyping relationship among interfaces~\cite{SPJ01,PC03} while each
interface is modelled as a dedicated (empty) Haskell type. We enhance
this approach by state in the appendix. This second approach fixes
some of the aforementioned problems: multiple inheritance is possible;
there is no closed-world assumption for base classes, but the other
problems remain. There is an additional problem with code bloat
related to the expression of transitive subtyping relationships.

We do not further discuss more untyped and less extensible variations
where OO subclasses do not amount to distinguished Haskell types,
where they are modelled as constructors of a ``base-class
type''. There also exist very restrictive variations where only
abstract-to-concrete inheritance relationships are allowed. As a final
point, the published Haskell reference solution for the Shapes
benchmark
\url{http://www.angelfire.com/tx4/cus/shapes/} is a simple-to-understand
code that does not attempt to maximise reuse among data declarations
and accessors. It also uses existentials for handling subtyping.
