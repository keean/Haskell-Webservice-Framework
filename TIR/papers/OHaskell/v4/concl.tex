\medskip
 
\section{Concluding remarks}
\label{S:concl}

We have described an OO system for Haskell that supports stateful
objects, inheritance and subtype polymorphism. We have demonstrated
that our encoding is very close to the textbook OO code (usually given
in C++ or Java tutorials), with pleasant deviations. The
re-implementation of examples from the OCaml object tutorial
demonstrated the faithfulness of our realization of objects. (We have
opted for OCaml because it is a leading object system in a functional
setting.) We have implemented parameterized classes, constructor
methods, abstract classes, pure virtual methods, single and multiple
inheritance with flexible rules of sharing or separation of
superclasses. Some major byproducts are these: extensive type
inference, first-class classes, implicit polymorphism of classes, an
option of implicit coercion of objects to their most common supertype
without any type annotations.

We have clarified the relation between subtyping and type-class based
polymorphisms: the latter can encode the former. We have implemented
OOHaskell with the existing Haskell implementation (GHC), requiring no
extra extensions beyond the commonly implemented ones: multi-parameter
type classes with functional dependencies. It is quite pleasant that
the existing OOHaskell code does not seem to be a burden to
write~---~even in the absence of any syntactic sugar.

We have implemented OO as a library feature -- based on the
polymorphic records with introspection and subtyping provided by the
\HList\ library.

There are some further idioms that complement OOHaskell as a faithful
OOP system, as substantiated by the paper's source distribution. For
instance, we need to take measures to do instantiation checking for
classes at declaration time. Otherwise some type errors would go
unnoticed until the first attempt to program an instantiation. We can
also model various forms of private and protected methods and other
access modifiers. We can clone objects, we can let methods return
`self', we can operate in the ST monad rather than the IO
monad. OOHaskell allows for the implementation often desirable
covariant methods~\cite{SG04,catcall} and static guarantees for their
soundness. OOHaskell also supports some forms of depth
subtyping~\cite{Poll97}. The extent and limitations of our handling of
depth subtyping as well as co-/contra-variance schemes remains a
subject for future work.

\begin{comment}
At present, error messages belie the
complexity, and this is the topic of future research (and so it is for
C++, where error messages in template meta-programs may span several
hundred lines and be humanly incomprehensible).
\end{comment}

Further topics for future work are the following. Simple syntactic
sugar would make OOP more convenient in Haskell, in particular, the
inferred types can really benefit from prettier-printing.  Extra
effort is needed to provide OOP-like error messages.  This is a
sticking issue that requires major effort, but there is a line of
research being carried out by Sulzmann and others~\cite{SSW04}.

An interesting advanced topic is reflective programming. A simple form
of reflection is readily provided in terms of the type-level encoding
of records. One can iterate over records and their components in a
generic fashion. Other forms of reflection, such as iteration over the
object pool, as needed for dynamic aspect-oriented programming,
requires further effort. Another challenge is to capture reusable
solutions for design problems (as part of design patterns) in Haskell.

An important issue is the efficiency of the encoding.  For example,
although record extension is constant (run-)time, the field/method
lookup is linear search. Clearly a more efficient encoding is
possible: one representation of the labels in the \HList\ paper
permits the total order among the labels types, which in turn, permits
construction of efficient search trees. In this first OOHaskell paper,
we chose the conceptual clarity over such optimizations.  A
non-trivial case study is required to demonstrate the scalability of
the approach. The mere compilation time of OOHaskell programs and
their runtime efficiency is challenged by the huge dictionaries that
are implied by our type-class-based approach. It is quite likely that,
as one reviewer has observed, large-scale efficient
\HList/OOHaskell style of programming will need dedicated compiler
optimizations. ``But showing the usefulness of such an encoding is the
first step towards encouraging compiler writers to do so!''
