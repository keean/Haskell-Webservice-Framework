\section{Detailed rationale for OOHaskell}
\label{S:rationale}



%%%%%%%%%%%%%%%%%%%%%%%%%%%%%%%%%%%%%%%%%%%%%%%%%%%%%%%%%%%%%%%%%%%%%%%%%%%%%%%
%%%%%%%%%%%%%%%%%%%%%%%%%%%%%%%%%%%%%%%%%%%%%%%%%%%%%%%%%%%%%%%%%%%%%%%%%%%%%%%
%%%%%%%%%%%%%%%%%%%%%%%%%%%%%%%%%%%%%%%%%%%%%%%%%%%%%%%%%%%%%%%%%%%%%%%%%%%%%%%



\subsection{Enduring attention, unabated controversy}

The issue of OO and Haskell, OO classes vs. type classes shows up on
Haskell mailing lists and websites with remarkable regularity. The
topic has attracted the attention of noted theoreticians
\cite{HS95,GJ96}, language implementors \cite{FLMPJ99, SPJ01}, as well
as many practitioners (\cite{MonadReader3}, Haskell-Cafe, June 2005).
The multitude of approaches and the differences in opinion indicate
that the question is an unsettled one. The reviews of the previous
version of this paper submitted to ICFP05 ranged from dismissive to
highly enthusiastic.

Haskell is thought to really lack OOP essentials. The functionally
minded OOP aficionado could either wish for an extended Haskell (and
several such extensions have indeed been proposed, see
Sec.~\ref{S:related}), employ a multi-language setup such as .NET, or
disregard Haskell and go for OCaml instead, which is known to be a
mainstream functional programming language that provides an advanced
OOP system.

We have observed widespread confusions regarding the relation between
Haskell's type classes and the object-oriented notion of classes. At
times these two sorts of classes are said to be very much different,
perhaps even largely orthogonal, incomparable. Elsewhere it is argued
that Haskell's type classes are like Java's interfaces, while
instances are like implementations, but subtyping would be missing in
this picture.  Again, elsewhere it is observed that multi-parameter
classes exhibit some flavour of multi-dispatch in OOP, which does not
get us very far however in the view of other missing OOP
essentials. It is often considered a mistake to attempt OOP in
Haskell, to transcribe Java or C++ classes in Haskell, to perhaps even
try to use use Haskell's type classes to that end.\footnote{\small See
many mind-boggling discussions on mailing lists:
\url{http://www.cs.mu.oz.au/research/mercury/mailing-lists/mercury-users/mercury-users.0105/0051.html},
\url{http://www.talkaboutprogramming.com/group/comp.lang.functional/messages/47728.html},
\url{http://www.haskell.org/pipermail/haskell/2003-December/013238.html},
\url{http://www.haskell.org/pipermail/haskell-cafe/2004-June/006207.html},
\url{http://www.haskell.org//pipermail/haskell/2004-June/014164.html},
\ldots } 


\begin{comment}
We will shed light on the subject matter. That is, we will
effectively use Haskell's type-class system to provide an OOP system
for Haskell. This system will be similar to OCaml's system, which we
view as a very strong existing marriage of functional programming and
OOP.
\end{comment}



%%%%%%%%%%%%%%%%%%%%%%%%%%%%%%%%%%%%%%%%%%%%%%%%%%%%%%%%%%%%%%%%%%%%%%%%%%%%%%%
%%%%%%%%%%%%%%%%%%%%%%%%%%%%%%%%%%%%%%%%%%%%%%%%%%%%%%%%%%%%%%%%%%%%%%%%%%%%%%%
%%%%%%%%%%%%%%%%%%%%%%%%%%%%%%%%%%%%%%%%%%%%%%%%%%%%%%%%%%%%%%%%%%%%%%%%%%%%%%%



\subsection{Faithful encoding as an intellectual challenge}

There is an intellectual challenge of seeing if the conventional OO
can at all be implemented in Haskell (short of writing a compiler for
an OO language in Haskell). Peyton Jones and Wadler's paper on
imperative programming in Haskell \cite{peytonjoneswadler-popl93}
epitomizes such an intellectual tradition. Another example is FC++
\cite{fcpp-jfp}, which implements in C++ the quintessential Haskell
features: type inference, higher-order functions, non-strictness. The
present paper, conversely, faithfully (i.e., in a similar syntax and
without global program transformation) realizes a principal C++ trait,
OOP.

The question of expressibility \cite{Felleisen90} of OO features such
as interface and implementation inheritance, subtyping, virtual
methods, open recursion in Haskell is significantly more complex than
one may hope. There have been several attempts, which~---~although may
appear adequate for a restricted problem at hand~---~clearly fall
short of the complete OO encoding. We discuss a few simple solutions
in the appendix; the full code accompanying this paper includes more.

In this paper we have settled the question that hierto has been open.
The conventional OO in its full generality \emph{is} expressible in
current Haskell without any new extensions. It turns out, Haskell~98
plus multi-parameter type classes with functional dependencies are
sufficient. Even overlapping instances are not essential (yet using
them permits a more convenient representation of labels).  That
conclusion has not been known before; and, as being emphasized by many
readers of the draft, is novel and surprising.



%%%%%%%%%%%%%%%%%%%%%%%%%%%%%%%%%%%%%%%%%%%%%%%%%%%%%%%%%%%%%%%%%%%%%%%%%%%%%%%
%%%%%%%%%%%%%%%%%%%%%%%%%%%%%%%%%%%%%%%%%%%%%%%%%%%%%%%%%%%%%%%%%%%%%%%%%%%%%%%
%%%%%%%%%%%%%%%%%%%%%%%%%%%%%%%%%%%%%%%%%%%%%%%%%%%%%%%%%%%%%%%%%%%%%%%%%%%%%%%



\subsection{Haskell's fitness regarding OO}

We build OOHaskell in terms of Hindley-Milner + multi-parameter type
classes + functional dependencies. This combination is well-formalized
and reasonably understood~\cite{SS04}.

Not only OOHaskell provides the conventional OOP; it already exhibits
features that are either bleeding-edge or unattainable in mainstream
OO languages: for example, first-class classes and class closures;
statically type-checked collection classes with bounded polymorphism
of implicit collection arguments; multiple inheritance with
user-controlled sharing. It is especially remarkable that these and
more familiar object-oriented features are not introduced by fiat --
we get them for free. For example, the type of a collection with
bounded polymorphism of elements is inferred automatically by the
compiler. Abstract classes cannot be instantiated not because we say
so but because the program will not typecheck otherwise.


\begin{comment}
At the heart of our approach is the powerful deployment of Haskell's
type classes. It will turn out that we can provide object classes
because of Haskell's type classes. In fact,  Once we have extensible
records with reusable labels and subtyping, we can model some sort of
objects. The corresponding record types, suitably parameterised, are
OOP-like classes then. Mutable objects can be modelled by using
references such as the @IORef@s provided by Haskell's @IO@ monad.
\end{comment}



%%%%%%%%%%%%%%%%%%%%%%%%%%%%%%%%%%%%%%%%%%%%%%%%%%%%%%%%%%%%%%%%%%%%%%%%%%%%%%%
%%%%%%%%%%%%%%%%%%%%%%%%%%%%%%%%%%%%%%%%%%%%%%%%%%%%%%%%%%%%%%%%%%%%%%%%%%%%%%%
%%%%%%%%%%%%%%%%%%%%%%%%%%%%%%%%%%%%%%%%%%%%%%%%%%%%%%%%%%%%%%%%%%%%%%%%%%%%%%%



\subsection{A conventional object encoding}

An ICFP reviewer wrote: ``The encoding is quite simple~---~it's
surprising that everything is so easy~---~yet not at all obvious.''
Our representation of objects and their types is \emph{deliberately}
straightforward: polymorphic extensible records of closures. A more
efficient representation based on separate method and field tables (as
in C++ and Java) is possible in principle. Although our current
encoding is certainly not optimal, it is conceptually clearer. This
encoding is used in such languages as Perl, Python, Lua~---~and is
often the first one chosen when adding OO to an existing language. We
want the OOP system for Haskell to be available and usable
\emph{now}. Therefore, we have to get by with the existing Haskell
system (GHC) as it is.



%%%%%%%%%%%%%%%%%%%%%%%%%%%%%%%%%%%%%%%%%%%%%%%%%%%%%%%%%%%%%%%%%%%%%%%%%%%%%%%
%%%%%%%%%%%%%%%%%%%%%%%%%%%%%%%%%%%%%%%%%%%%%%%%%%%%%%%%%%%%%%%%%%%%%%%%%%%%%%%
%%%%%%%%%%%%%%%%%%%%%%%%%%%%%%%%%%%%%%%%%%%%%%%%%%%%%%%%%%%%%%%%%%%%%%%%%%%%%%%



\subsection{Practical use of OOHaskell idioms}

One of the main goals of this paper is to be able represent the
conventional OO code, in as straightforward way as possible.  We
illustrate OOHaskell with a series of practical examples as they are
commonly found in OO textbooks and programming language tutorials.
The implementation of our system may be not for the feeble at
heart~---~however, the user of the system must be able to write
conventional OO code without understanding the complexity of the
implementation.  Since most OO system in practical use have mutable
state, we will be concerned with mutable objects, implemented via
|IORef| or |STRef|. Functional objects bring quite an interesting
twist, which lends itself as a topic for future work.

%% \ralf{We face a confusion in terminology.
%% When we say encoding, we either refer to the fundamental 
%% OO encoding that we in turn encode in Haskell, or we might
%% just refer to the surface encoding of OO programs. 
%% I contend that we really need to be careful here.}


%%%%%%%%%%%%%%%%%%%%%%%%%%%%%%%%%%%%%%%%%%%%%%%%%%%%%%%%%%%%%%%%%%%%%%%%%%%%%%%
%%%%%%%%%%%%%%%%%%%%%%%%%%%%%%%%%%%%%%%%%%%%%%%%%%%%%%%%%%%%%%%%%%%%%%%%%%%%%%%
%%%%%%%%%%%%%%%%%%%%%%%%%%%%%%%%%%%%%%%%%%%%%%%%%%%%%%%%%%%%%%%%%%%%%%%%%%%%%%%



\subsection{Is there any ``type hacking'' involved?}

One may be tempted to suspect that OOHaskell relies on ``type
hacking''. We defend both on the practical grounds, language design,
and the theoretical grounds.

Foremost, multi-parameter type classes with functional dependencies
are \emph{not} a hack: they are well-formalized and reasonably
understood~\cite{SS04}. The fact that we found a quite unexpected (and
unintended) use of a particular language feature does not mean that
the result is practically useless. Template meta-programming in C++
has been the best known example of such ``type hacking''. And yet it
has lead to |boost|, which has become a de facto tool for modern C++
programming.
%
%\footnote{Adobe uses boost:
%\url{http://lambda-the-ultimate.org/node/view/563\#comment-4531}}
%
Templates and template meta-programming have changed the very
character of the language~\cite{fcpp-jfp} and made generative
programming research and practice in C++ possible
\cite{DSL-in-three-lang,siek05:_concepts_cpp0x}.

\begin{comment}
[Stroustrup interview?
  Need some reference. If we can't find any, I can use LtU references
  \url{http://lambda-the-ultimate.org/node/view/663} (see comments by
  Scott Johnson)
  \url{http://lambda-the-ultimate.org/node/view/663#comment-5839} See
  also:
  \url{http://spirit.sourceforge.net/distrib/spirit_1_7_0/libs/spirit/phoenix/doc/preface.html}
] 


An ICFP reviewer wrote:
``The result might seem poor and just containing clever tricks
however it took 10 years to obtain that proof of concepts and this
deserves attention.''
\end{comment}


%%%%%%%%%%%%%%%%%%%%%%%%%%%%%%%%%%%%%%%%%%%%%%%%%%%%%%%%%%%%%%%%%%%%%%%%%%%%%%%
%%%%%%%%%%%%%%%%%%%%%%%%%%%%%%%%%%%%%%%%%%%%%%%%%%%%%%%%%%%%%%%%%%%%%%%%%%%%%%%
%%%%%%%%%%%%%%%%%%%%%%%%%%%%%%%%%%%%%%%%%%%%%%%%%%%%%%%%%%%%%%%%%%%%%%%%%%%%%%%



\subsection{Theoretical contribution}

As far as OOP is concerned, OOHaskell intentionally resembles OCaml
(or, its predecessor, ML-ART \cite{ML-ART}).  ML-ART adds several
extensions to ML to implement objects: records with polymorphic access
and extension, projective records, recursive types, implicit
existential and universal types. Polymorphic records and existentials
are the main ones. As the paper \cite{ML-ART} says, none of the
extensions are new, but their combination is original and ``provides
just enough power to program objects in a flexible and elegant way.''

We make the same claim for OOHaskell, but using a quite different set
of features. We do rely on polymorphic records~---~which, although
have the same behavior, are very different from those in ML-ART. Our
polymorphic records have no special row variables and avoid
impredicative types. The fact that such records are realizable in
Haskell at all has been unexpected and unknown, until the \HList\
paper less than a year ago.\footnote{Again, there were many debates on
the Haskell mailing list about adding extensible records to
Haskell. At the Haskell 2003 workshop~\cite{HW03} this issue was
selected as prime topic for discussion.} Unlike ML-ART~\cite{ML-ART},
we do \emph{not} rely on existential or implicitly universal types,
nor recursive types. We use the value recursion instead (incidentally,
objects in ML-ART can be mutable too). As the consequence of the
simplicity of the type theory for our implementation, the types of
objects and object-manipulating functions and collections can be
\emph{inferred}, without any type annotations.

% safety: See ML-ART paper, p. 9, end of section 3.1
% and Sectiom 3.7
%
% See p. 29 of the paper and the quotation from Bruce, 1992. Pierce
% seems to agree that recursive types are needed to model inheritance
% of methods involving self. -- beginning of the second paragraph
% on p. 29
%

\begin{comment}
Such an extension (equi-recursive types)
 to Haskell was also debated and then rejected
because it will make type-error messages nearly
useless~\cite{Hughes02}. There is an alternative technique for
encoding objects: eschew recursive types in favour of existential
quantification~\cite{PT94}. Unfortunately, the involved higher-ranked
types can not be inferred anymore. Explicit signatures were required,
which, in practical terms, means that the user must explicitly
enumerate all virtual methods in the signature of any function that
operates on an object. This technique cannot be used in OOHaskell
because we would like OOP to be easy to use, first, by Haskell
programmer. That is, we ought to preserve type inference for functions
that use objects. Type inference is the great advantage of Haskell and
ML and is worth fighting for. 
\end{comment}

\begin{comment}
\ralf{Variance: since we don't do in the paper, this is odd.
Either we should dive into it in the advanced section,
or leave it out.}
\oleg{We have the code for that, it's already in the archive. I submit
  we still can mention it, just to keep people interested and keep
  asking questions.}

Another theoretical problem -- even with mutable objects -- is the
controversy regarding covariance and contravariance of method
arguments \cite{SG04}, \cite{catcall}. Our work shows how to implement
often desirable covariant methods and statically guarantee
soundness. Due to the lack of space, we merely skim the topic and
refer the reader to the code (and the next paper).
\end{comment}



%%%%%%%%%%%%%%%%%%%%%%%%%%%%%%%%%%%%%%%%%%%%%%%%%%%%%%%%%%%%%%%%%%%%%%%%%%%%%%%
%%%%%%%%%%%%%%%%%%%%%%%%%%%%%%%%%%%%%%%%%%%%%%%%%%%%%%%%%%%%%%%%%%%%%%%%%%%%%%%
%%%%%%%%%%%%%%%%%%%%%%%%%%%%%%%%%%%%%%%%%%%%%%%%%%%%%%%%%%%%%%%%%%%%%%%%%%%%%%%



\subsection{A sandbox for language design}

Just as C++ has become the laboratory for generative programming
\cite{DSL-in-three-lang} and lead to such applications as FC++, we
contend that Haskell can become the laboratory for OO design and
development.  To extend the motto by Simon Peyton-Jones, Haskell is
not only the best imperative language, it is the best OO language too.
C++ programmers now routinely use parsing combinators, thanks to the
|boost::spirit| library. Haskell programmers can likewise use OO
idioms if it suits the problem at hand. We can experiment with OO
features, without the need to change Haskell compilers.

Our OOHaskell library ~\cite{OOHaskell} ended up to be a comparatively rich
combination of OO idioms, higher-order functional programming, and
type inference.  We have experienced first-hand the incredible amount
of guidance Haskell type system gives in the design of OO features.
We therefore think that (OO)Haskell lends itself as an environment for
advanced and typed OO \emph{language design}.
