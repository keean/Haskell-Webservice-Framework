\documentclass[nocopyrightspace,preprint]{sigplan-proc}

\setlength{\parskip}{6pt}
\setlength{\parsep}{3pt}

% \documentclass[onecolumn,11pt,preprint]{sigplanconf}
% \documentclass{llncs}
% \documentclass{article}

\usepackage{code}
%\usepackage{amsmath}
\usepackage{times}
\usepackage{comment}
\usepackage{url}
\usepackage{fancyvrb}
\usepackage{graphics}
\usepackage{boxedminipage}

\usepackage[bookmarks=false,%
            colorlinks,linkcolor=black,urlcolor=blue,%
            pdfauthor={Oleg and Ralf},%
            pdftitle={OOHaskell}]{hyperref}

\DefineShortVerb{\|}
%\DefineVerbatimEnvironment{code}{Verbatim}{xleftmargin=\mathindent,commandchars=\\\{\},fontsize=\small}
% \setlength{\parskip}{0pt}
%\newlength{\mathindent}
%\setlength{\mathindent}{1em}


% Macros for notes to each other in the text
\newcommand{\keean}[1]{{\it [Keean says: #1]}}
\newcommand{\oleg}[1]{{\it [Oleg says: #1]}}
\newcommand{\ralf}[1]{{\it [Ralf says: #1]}}

\newcommand{\noskip}{\topsep0pt \parskip0pt \partopsep0pt}
\newcommand{\w}[1]{\textit{#1}}
\newcommand{\cmt}[1]{\mbox{\textrm{\emph{#1}}}}
\newcommand{\mysize}{\small}
\newcommand{\myskip}{\smallskip}
\newcommand{\antiskip}{\vspace{-25\in}}
\newcommand{\myinput}[1]{The content of this section is not yet released.}
\newcommand{\HList}{\textsc{HList}}
\newcommand{\undefined}{\ensuremath{\bot}}
\newcommand{\Forall}{\ensuremath{\forall}}
 

\begin{document}

%\conferenceinfo{WXYZ '05}{date, City.} 
%\copyrightyear{2005} 
%\copyrightdata{supplied by printer} 
 
\title{Haskell's overlooked object system\\
{\small ---~DRAFT OF \today~---}\vspace{-77\in}}

\author{Oleg Kiselyov\\FNMOC, Monterey, CA
%Fleet Numerical Meteorology and Oceanography Center, Monterey, CA
\and
Ralf L{\"a}mmel\\Microsoft Corp., Redmond, WA}


%\authorinfo{Oleg Kiselyov}{FNMO Center, Monterey, CA}{}
%\authorinfo{Ralf L{\"a}mmel}{Microsoft Corp., Redmond, WA}{}


\maketitle

\begin{abstract}

Haskell provides type-class-based bounded polymorphism as opposed to
subtype polymorphism of object-oriented languages such as Java and
OCaml. It is a contentious question whether Haskell (alone or with
extensions) can fully support conventional object-oriented programming
with encapsulation, mutable state, inheritance, overriding, statically
checked subtyping, and so on.

We have constructively shown that Haskell \emph{as it is} supports all
the conventional OO features plus more advanced ones, including
first-class lexically scoped classes, implicitly polymorphic classes,
flexible multiple inheritance, type inference for objects and
object-operating functions, an ability to put objects of distinct
types to a homogeneous list, whose type is derived automatically. Many
of the features we get ``for free'': the type system of Haskell turns
out to be a great help and a guide rather than a hindrance.

The OO features are introduced in Haskell as the OOHaskell
\emph{library}, non-trivially based on the \HList\ library of
extensible polymorphic records with first-class labels and
subtyping. The library sample code, which is patterned after the
examples found in OO textbooks and programming language tutorials,
including OCaml object tutorial, demonstrates that OO code translates
into OOHaskell in an intuition-preserving way: essentially
expression-by-expression, without requiring global transformations.

OOHaskell lends itself as a sandbox for typed OO language design.

\end{abstract}

\makeatactive



%%%%%%%%%%%%%%%%%%%%%%%%%%%%%%%%%%%%%%%%%%%%%%%%%%%%%%%%%%%%%%%%%%%%%%%%%%%%%
%%%%%%%%%%%%%%%%%%%%%%%%%%%%%%%%%%%%%%%%%%%%%%%%%%%%%%%%%%%%%%%%%%%%%%%%%%%%%
%%%%%%%%%%%%%%%%%%%%%%%%%%%%%%%%%%%%%%%%%%%%%%%%%%%%%%%%%%%%%%%%%%%%%%%%%%%%%



\section{Introduction}

Is Haskell fit for OOP? This topic is raised time and again on
functional programming mailing lists, on websites, and in verbal
communication with remarkable regularity. There appear to be several
reasons for such popularity:

\smallskip

\begin{itemize}

\item In an \emph{intellectual} sense, we may wonder whether Haskell's
advanced type system is expressive enough to model object types,
inheritance, subtyping, virtual methods, etc. There is an established
scientific pessimism regarding this question.

\smallskip

\item In a \emph{practical} sense, can we faithfully transport
imperative OO designs from, say, C++, Java, C\#, VB to Haskell --
without totally rewriting the design and without foreign-language
interfacing?

\smallskip

\item From a \emph{language design} perspective, we know how to use
Haskell for prototyping semantics and for encoding various abstraction
mechanisms, but can we also leverage Haskell as a sandbox for design
of typed object-oriented languages so that we can play with new ideas
without the immediate need to write or modify a compiler.

\smallskip

\item In an \emph{educational} sense, is there anything important that
more or less advanced functional and object-oriented programmers can
learn about Haskell's type system and about OO by looking into the
pros and cons of different OO encoding options?

\end{itemize}

\smallskip

This paper delivers substantiated, positive answers to these
questions. We describe OOHaskell~---~a Haskell-based library for (as
of today: imperative) OO programming in Haskell. OOHaskell uncovers
Haskell's overlooked object system. The key to this result is a good
deal of exploitation of Haskell's breath-taking type system
\emph{combined} with a careful identification of a suitable object
encoding. OOHaskell builds upon our previous work on heterogeneous
collections~\cite{HLIST-HW04} (the HList library), while we had to
discover several new techniques to make non-trivial objects work.


\medskip

\subsection*{Road-map of this paper}

\begin{itemize}
\item Sec.~\ref{S:shapes}: We walk through an illustrative OO example.
\item Sec.~\ref{S:rationale}: We give a detailed rationale for OOHaskell.
\item Sec.~\ref{S:basics}: We describe basic OOHaskell programming idioms.
\item Sec.~\ref{S:types}: We discuss typing issues~---~inference, errors, etc.
\item Sec.~\ref{S:strengths}: We highlight some of the strengths of OOHaskell.
\item Sec.~\ref{S:related}: We review related work.
\item Sec.~\ref{S:concl}: The paper is concluded.
\end{itemize}
%
There is an extended code distribution coming with this paper~\cite{OOHaskell}.

\begin{comment}

In Sec.~\ref{S:HList}, we briefly review the \HList\
library~\cite{HLIST-HW04}, which provides extensible polymorphic
heterogeneous records with first-class labels.  In
Sec.~\ref{S:simple}, we introduce more basic OO notions such as
objects and constructors. In Sec.~\ref{S:self}, we describe open
recursion, which allows us to cover rich forms of inheritance.  In
Sec.~\ref{S:shapes}, we handle a prototypical scenario for subtype
polymorphism in detail. In~\ref{S:disc}, we very briefly discuss all
remaining issues~---~including some technicalities, conclusions, and
directions for future work.

\end{comment}
  


%%%%%%%%%%%%%%%%%%%%%%%%%%%%%%%%%%%%%%%%%%%%%%%%%%%%%%%%%%%%%%%%%%%%%%%%%%%%%
%%%%%%%%%%%%%%%%%%%%%%%%%%%%%%%%%%%%%%%%%%%%%%%%%%%%%%%%%%%%%%%%%%%%%%%%%%%%%
%%%%%%%%%%%%%%%%%%%%%%%%%%%%%%%%%%%%%%%%%%%%%%%%%%%%%%%%%%%%%%%%%%%%%%%%%%%%%
 

                                                                             
\begin{figure*}[t]
\begin{center}
\resizebox{.8\textwidth}{!}{\includegraphics{shapes.pdf}}
\end{center}
\vspace{-33\in}
\caption{The `shapes' benchmark for subtype polymorphism}
\label{F:shapes}
\end{figure*}
                                                                             

 
%%%%%%%%%%%%%%%%%%%%%%%%%%%%%%%%%%%%%%%%%%%%%%%%%%%%%%%%%%%%%%%%%%%%%%%%%%%%%
%%%%%%%%%%%%%%%%%%%%%%%%%%%%%%%%%%%%%%%%%%%%%%%%%%%%%%%%%%%%%%%%%%%%%%%%%%%%%
%%%%%%%%%%%%%%%%%%%%%%%%%%%%%%%%%%%%%%%%%%%%%%%%%%%%%%%%%%%%%%%%%%%%%%%%%%%%%
 

 
\section{The Shapes benchmark}
\label{S:shapes}


Let us now explore the so-called `shapes benchmark'.\footnote{See the
multi-lingual collection `OO Example Code' by Jim Weirich at
\url{http://onestepback.org/articles/poly/}; see also an even heavier
collection `OO Shape Examples' by Chris Rathman at
\url{http://www.angelfire.com/tx4/cus/shapes/}.}  This benchmark (or
OO coding scenario) has a history in evaluating encodings of
subtype polymorphism. The classes that are involved in the
scenario are shown in Fig.~\ref{F:shapes}. There is an abstract class
(or an interface) @Shape@, and their are two subclasses @Rectangle@
and @Circle@. The coding scenario is the following: place different
shape object of different subclasses in a collection and iterate over
the collection to draw each shape object; the drawing
functionality varies per subclass.

We will show that the OOHaskell encoding pleasantly mimics the C++
encoding, while any remaining deviations are appreciated. We will also
discuss some variation points, once we have identified our prime
encoding. Finally, there are various possible encodings of the
scenario that do not deliver an image of Haskell as a true OO language
(including the one in Rathman's suite;
\url{http://www.angelfire.com/tx4/cus/shapes/haskell.html}). We do not
have space to discuss such non-OOHaskell encodings, but we refer to
this paper's source code distribution~\cite{OOHaskell} instead.



%%%%%%%%%%%%%%%%%%%%%%%%%%%%%%%%%%%%%%%%%%%%%%%%%%%%%%%%%%%%%%%%%%%%%%%%%%%%%
%%%%%%%%%%%%%%%%%%%%%%%%%%%%%%%%%%%%%%%%%%%%%%%%%%%%%%%%%%%%%%%%%%%%%%%%%%%%%
%%%%%%%%%%%%%%%%%%%%%%%%%%%%%%%%%%%%%%%%%%%%%%%%%%%%%%%%%%%%%%%%%%%%%%%%%%%%%



\subsection{The C++ reference solution}

We omit the code for the classes of shapes, rectangles and
circles. This is all trivial from a C++ perspective: we use a
pure virtual method for @draw@ in the class @Shape@, which is then
implemented differently in the classes @Rectangle@ and @Circle@.

Here is C++ code to set up an array of (two) shapes:

\begin{code}
   Shape *scribble[2];
   scribble[0] = new Rectangle(10, 20, 5, 6);
   scribble[1] = new Circle(15, 25, 8);
\end{code}

\noindent
We use an array rather than a collection type. We could employ a
collection type from C++'s Standard Template Library with similar
convenience. Here is a for-loop over the array in our C++ code:

\begin{code}
   for (int i = 0; i < 2; i++) {
      scribble[i]->draw();
      scribble[i]->rMoveTo(100, 100);
      scribble[i]->draw();
   }
\end{code}

\noindent
That is, we draw each element, move it relatively to its
origin, and draw it again. If the @draw@ method prints
`progress messages' about what is being drawn, we may see the
following output:

\begin{code}
 Drawing a Rectangle at:(10,20), width 5, height 6
 Drawing a Rectangle at:(110,120), width 5, height 6
 Drawing a Circle at:(15,25), radius 8
 Drawing a Circle at:(115,125), radius 8
\end{code}

\noindent
(Any OO language with parametric polymorphism, or at least polymorphic
arrays, should allow similarly concise code. In a typed language
without parametric polymorphism, we would need to bother about unsafe
down-casts when processing the aggregated objects. Likewise, untyped
languages would risk `message-not-understood' errors.)



%%%%%%%%%%%%%%%%%%%%%%%%%%%%%%%%%%%%%%%%%%%%%%%%%%%%%%%%%%%%%%%%%%%%%%%%%%%%%
%%%%%%%%%%%%%%%%%%%%%%%%%%%%%%%%%%%%%%%%%%%%%%%%%%%%%%%%%%%%%%%%%%%%%%%%%%%%%
%%%%%%%%%%%%%%%%%%%%%%%%%%%%%%%%%%%%%%%%%%%%%%%%%%%%%%%%%%%%%%%%%%%%%%%%%%%%%



\subsection{The OOHaskell transcription}

We also omit the Haskell values for the involved classes since we have
exercised pure virtual methods in Sec.~\ref{S:self}. We only
transcribe the collection code. We start a monadic @do@ sequence to
construct two shape objects~---~just as above:

\begin{code}
 myShapesOOP =
    do
       s1 <- mfix (rectangle (10::Int) (20::Int) 5 6)
       s2 <- mfix (circle (15::Int) 25 8)
       -- to be continued
\end{code}

\noindent
What's different? We use @mfix@ in place of @new@. We use curried
functions instead of C++'s comma notation.  We note that some
constructor arguments are annotated by the @Int@ type because we
preferred to eliminate the implicit polymorphism at this stage.

We continue the monadic @do@ sequence by building an `array' of
shapes:

\begin{code}
       let scribble :: [Shape Int]
           scribble = [narrow s1, narrow s2]
\end{code}

\noindent
In fact, we use a plain Haskell list. The type annotation for the
@scribble@ binding corresponds to the typed variable declaration in
the C++ code. However, the Haskell list construction differs from the
the C++ array construction as follows. In Haskell, we use an
explicit coercion operation, @narrow@ to prepare each shape object for
insertion into the homogeneous list. By contrast, such casting is
\emph{implicit} in the C++ code.

Regarding the class type @Shape@, we note that we have not used
\emph{any} explicit class types in the preceding sections. Mostly, we
do not need them because Haskell's type inference works
fine. (Programmers of C++ and of other mainstream languages have
the habit of writing down types for almost everything.) For the
purpose of \emph{casting}, we require such explicit types in
OOHaskell. They are necessary for steering explicit casting in the
view of programs that otherwise lack pervasive type annotations: So
here is the record type for @Shape@ objects:

\begin{code}
 type Shape a = Record (  (Proxy GetX    , IO a)
                      :*: (Proxy GetY    , IO a)
                      :*: (Proxy SetX    , a -> IO ())
                      :*: (Proxy SetY    , a -> IO ())
                      :*: (Proxy MoveTo  , a -> a -> IO ())
                      :*: (Proxy RMoveTo , a -> a -> IO ())
                      :*: (Proxy Draw    , IO ())
                      :*: HNil )
\end{code}

We finish up the monadic @do@ sequence by iterating over scribble:

\begin{code}
       mapM_ (\shape -> do
                           shape # draw
                           (shape # rMoveTo) 100 100
                           shape # draw)
             scribble
\end{code}

\noindent
Here we use the monadic @mapM_@ operation which only cares about the
effects of the monadic steps, throwing away results. This is really
the Haskell way of iterating over a list with effectful
functions~---~as the counterpart of the for-loop in the C++ code.



%%%%%%%%%%%%%%%%%%%%%%%%%%%%%%%%%%%%%%%%%%%%%%%%%%%%%%%%%%%%%%%%%%%%%%%%%%%%%
%%%%%%%%%%%%%%%%%%%%%%%%%%%%%%%%%%%%%%%%%%%%%%%%%%%%%%%%%%%%%%%%%%%%%%%%%%%%%
%%%%%%%%%%%%%%%%%%%%%%%%%%%%%%%%%%%%%%%%%%%%%%%%%%%%%%%%%%%%%%%%%%%%%%%%%%%%%



\subsection{Narrowing vs.\ heterogeneity vs.\ existentials}

We have employed narrowing to coerce all objects to a common
interface, in fact, to the \emph{same} record type. One might wonder
whether these coercions can be avoided altogether, or whether the
explicit conversions can also be made implicit even in OOHaskell. We
will discuss two techniques, but the conclusion will be that narrowing
is to be preferred.

The first technique is to collect the shape objects, as is, in a
\emph{heterogeneous} list rather than a homogeneous array or list. We
cannot construct such a list with the normal, polymorphic list datatype
constructor, but the \HList\ library comes again to our rescue. 
The scribble construction can now be performed without any
narrowing:

\begin{code}
       let scribble = s1 `HCons` (s2 `HCons` HNil)
\end{code}

\noindent
We cannot use ordinary list-processing function anymore, but the
\HList\ library mimics the normal list-processing API for @HList@s. So
there is also a heterogeneous variation on @mapM_@, namely @hMapM_@,
to be invoked as follows:

\begin{Verbatim}[fontsize=\small,commandchars=\\\{\}]
       hMapM_ (\undefined::FunOnShape) scribble
\end{Verbatim}

\noindent
The first argument of @hMapM_@ is not a function but rather a
\emph{type code}. This is necessary for technical reasons related to
the combination of rank-n polymorphism and ad-hoc
polymorphism.\footnote{A heterogeneous map function can encounter
entities of different types. Hence, its argument function must be
polymorphic on its own (which is different from the normal map
function). The argument function typically uses type classes (say,
ad-hoc polymorphism) to process the entities of different types. The
trouble is that the map function cannot possibly anticipate all the
constraints required by its argument function.  The type-code
technique moves the constraints from the type of the heterogeneous map
function to the interpretation site of the type codes.} The meaning of
each type code must be defined by a dedicated instance of an @Apply@
class for function application. Here is the declaration of the type
code @FunOnShape@ complete with its meaning:

\begin{code}
 data FunOnShape -- a type code only!
\end{code}

\begin{code}
instance ( HasField (Proxy Draw) r (IO ())
         , HasField (Proxy RMoveTo) r (Int -> Int -> IO ())
         )
      => Apply FunOnShape r (IO ())
  where
    apply _ x = do
                   x # draw
                   (x # rMoveTo) 100 100
                   x # draw
\end{code}

\noindent
The @Apply@ instance manifests encoding efforts that we didn't face
for the narrowing-based encoding. Now we have to list the
\emph{method-access constraints} (for ``\#'', i.e., @HasField@) in the
@Apply@ instance. Haskell's type-class system requires us to provide
proper bounds for the instance. One might argue that the form of these
constraints strongly resembles the method types listed in the class
type @Shape@. So one might wonder whether we can somehow use the full
class type in order to constrain the instance.  Haskell won't let us
do that in any reasonable way. (Constraints are not first-class
citizens in Haskell; we can't compute them from types or type
proxies~---~unless we were willing to rely on heavy encoding or
advanced syntactic sugar.) So we are doomed to manually infer such
method-access constraints for each such piece of polymorphic code.

The second technique for avoiding narrowing relies on placing shape
objects in \emph{existentially quantified envelopes}: we do not
coerce, but we wrap:

\begin{code}
       let scribble = [ WrapShape s1 , WrapShape s2 ]
\end{code}

\noindent
The declaration of the @WrapShape@ type depends on the function that
we want to apply to the opaque data. In our case, we can use the
normal @mapM_@ function again; we only need to unwrap the @WrapShape@
constructor prior to method invocations:

\begin{code}
       mapM_ ( \(WrapShape shape) -> do
                  shape # draw
                  (shape # rMoveTo) 100 100
                  shape # draw )
             scribble
\end{code}

\noindent
These operations have to be anticipated in the type bound for
@WrapShape@:

\begin{Verbatim}[fontsize=\small,commandchars=\\\{\}]
 data WrapShape =
  \Forall x. ( HasField (Proxy Draw) x (IO ())
       , HasField (Proxy RMoveTo) x (Int -> Int -> IO ())
       ) => WrapShape x
\end{Verbatim}

\noindent
It becomes evident that this result agrees with the heterogeneity
technique in terms of encoding efforts. In both cases, we need to
identify type-class constraints that correspond to the (potentially)
polymorphic method invocations.

Consequently, the narrowing technique is to be preferred.  We
\emph{could} hide narrowing by eschewing free-wheeling functional
programming and type inference. The required style and API would then
account for hidden narrowing. We do not favour such an approach for
various reasons. Abandoning type inference is in conflict with Haskell
native style.  Making casts implicit introduces the risk that the
programmer can accidentally pass an object of the wrong type.  Making
cast implicit hides the costs that come with casts; we prefer to see
the need for coercions clearly. (In fact, in the presence of multiple
inheritance of classes or interfaces, the implicit cast is absolutely
nontrivial, either for the compiler, or for the run-time system, or
both.)



%%%%%%%%%%%%%%%%%%%%%%%%%%%%%%%%%%%%%%%%%%%%%%%%%%%%%%%%%%%%%%%%%%%%%%%%%%%%%
%%%%%%%%%%%%%%%%%%%%%%%%%%%%%%%%%%%%%%%%%%%%%%%%%%%%%%%%%%%%%%%%%%%%%%%%%%%%%
%%%%%%%%%%%%%%%%%%%%%%%%%%%%%%%%%%%%%%%%%%%%%%%%%%%%%%%%%%%%%%%%%%%%%%%%%%%%%

\section{Detailed rationale for OOHaskell}
\label{S:rationale}



%%%%%%%%%%%%%%%%%%%%%%%%%%%%%%%%%%%%%%%%%%%%%%%%%%%%%%%%%%%%%%%%%%%%%%%%%%%%%%%
%%%%%%%%%%%%%%%%%%%%%%%%%%%%%%%%%%%%%%%%%%%%%%%%%%%%%%%%%%%%%%%%%%%%%%%%%%%%%%%
%%%%%%%%%%%%%%%%%%%%%%%%%%%%%%%%%%%%%%%%%%%%%%%%%%%%%%%%%%%%%%%%%%%%%%%%%%%%%%%



\subsection{Enduring attention, unabated controversy}

The issue of OO and Haskell, OO classes vs. type classes shows up on
Haskell mailing lists and websites with remarkable regularity. The
topic has attracted the attention of noted theoreticians
\cite{HS95,GJ96}, language implementors \cite{FLMPJ99, SPJ01}, as well
as many practitioners (\cite{MonadReader3}, Haskell-Cafe, June 2005).
The multitude of approaches and the differences in opinion indicate
that the question is an unsettled one. The reviews of the previous
version of this paper submitted to ICFP05 ranged from dismissive to
highly enthusiastic.

Haskell is thought to really lack OOP essentials. The functionally
minded OOP aficionado could either wish for an extended Haskell (and
several such extensions have indeed been proposed, see
Sec.~\ref{S:related}), employ a multi-language setup such as .NET, or
disregard Haskell and go for OCaml instead, which is known to be a
mainstream functional programming language that provides an advanced
OOP system.

We have observed widespread confusions regarding the relation between
Haskell's type classes and the object-oriented notion of classes. At
times these two sorts of classes are said to be very much different,
perhaps even largely orthogonal, incomparable. Elsewhere it is argued
that Haskell's type classes are like Java's interfaces, while
instances are like implementations, but subtyping would be missing in
this picture.  Again, elsewhere it is observed that multi-parameter
classes exhibit some flavour of multi-dispatch in OOP, which does not
get us very far however in the view of other missing OOP
essentials. It is often considered a mistake to attempt OOP in
Haskell, to transcribe Java or C++ classes in Haskell, to perhaps even
try to use use Haskell's type classes to that end.\footnote{\small See
many mind-boggling discussions on mailing lists:
\url{http://www.cs.mu.oz.au/research/mercury/mailing-lists/mercury-users/mercury-users.0105/0051.html},
\url{http://www.talkaboutprogramming.com/group/comp.lang.functional/messages/47728.html},
\url{http://www.haskell.org/pipermail/haskell/2003-December/013238.html},
\url{http://www.haskell.org/pipermail/haskell-cafe/2004-June/006207.html},
\url{http://www.haskell.org//pipermail/haskell/2004-June/014164.html},
\ldots } 


\begin{comment}
We will shed light on the subject matter. That is, we will
effectively use Haskell's type-class system to provide an OOP system
for Haskell. This system will be similar to OCaml's system, which we
view as a very strong existing marriage of functional programming and
OOP.
\end{comment}



%%%%%%%%%%%%%%%%%%%%%%%%%%%%%%%%%%%%%%%%%%%%%%%%%%%%%%%%%%%%%%%%%%%%%%%%%%%%%%%
%%%%%%%%%%%%%%%%%%%%%%%%%%%%%%%%%%%%%%%%%%%%%%%%%%%%%%%%%%%%%%%%%%%%%%%%%%%%%%%
%%%%%%%%%%%%%%%%%%%%%%%%%%%%%%%%%%%%%%%%%%%%%%%%%%%%%%%%%%%%%%%%%%%%%%%%%%%%%%%



\subsection{Faithful encoding as an intellectual challenge}

There is an intellectual challenge of seeing if the conventional OO
can at all be implemented in Haskell (short of writing a compiler for
an OO language in Haskell). Peyton Jones and Wadler's paper on
imperative programming in Haskell \cite{peytonjoneswadler-popl93}
epitomizes such an intellectual tradition. Another example is FC++
\cite{fcpp-jfp}, which implements in C++ the quintessential Haskell
features: type inference, higher-order functions, non-strictness. The
present paper, conversely, faithfully (i.e., in a similar syntax and
without global program transformation) realizes a principal C++ trait,
OOP.

The question of expressibility \cite{Felleisen90} of OO features such
as interface and implementation inheritance, subtyping, virtual
methods, open recursion in Haskell is significantly more complex than
one may hope. There have been several attempts, which~---~although may
appear adequate for a restricted problem at hand~---~clearly fall
short of the complete OO encoding. We discuss a few simple solutions
in the appendix; the full code accompanying this paper includes more.

In this paper we have settled the question that hierto has been open.
The conventional OO in its full generality \emph{is} expressible in
current Haskell without any new extensions. It turns out, Haskell~98
plus multi-parameter type classes with functional dependencies are
sufficient. Even overlapping instances are not essential (yet using
them permits a more convenient representation of labels).  That
conclusion has not been known before; and, as being emphasized by many
readers of the draft, is novel and surprising.



%%%%%%%%%%%%%%%%%%%%%%%%%%%%%%%%%%%%%%%%%%%%%%%%%%%%%%%%%%%%%%%%%%%%%%%%%%%%%%%
%%%%%%%%%%%%%%%%%%%%%%%%%%%%%%%%%%%%%%%%%%%%%%%%%%%%%%%%%%%%%%%%%%%%%%%%%%%%%%%
%%%%%%%%%%%%%%%%%%%%%%%%%%%%%%%%%%%%%%%%%%%%%%%%%%%%%%%%%%%%%%%%%%%%%%%%%%%%%%%



\subsection{Haskell's fitness regarding OO}

We build OOHaskell in terms of Hindley-Milner + multi-parameter type
classes + functional dependencies. This combination is well-formalized
and reasonably understood~\cite{SS04}.

Not only OOHaskell provides the conventional OOP; it already exhibits
features that are either bleeding-edge or unattainable in mainstream
OO languages: for example, first-class classes and class closures;
statically type-checked collection classes with bounded polymorphism
of implicit collection arguments; multiple inheritance with
user-controlled sharing. It is especially remarkable that these and
more familiar object-oriented features are not introduced by fiat --
we get them for free. For example, the type of a collection with
bounded polymorphism of elements is inferred automatically by the
compiler. Abstract classes cannot be instantiated not because we say
so but because the program will not typecheck otherwise.


\begin{comment}
At the heart of our approach is the powerful deployment of Haskell's
type classes. It will turn out that we can provide object classes
because of Haskell's type classes. In fact,  Once we have extensible
records with reusable labels and subtyping, we can model some sort of
objects. The corresponding record types, suitably parameterised, are
OOP-like classes then. Mutable objects can be modelled by using
references such as the @IORef@s provided by Haskell's @IO@ monad.
\end{comment}



%%%%%%%%%%%%%%%%%%%%%%%%%%%%%%%%%%%%%%%%%%%%%%%%%%%%%%%%%%%%%%%%%%%%%%%%%%%%%%%
%%%%%%%%%%%%%%%%%%%%%%%%%%%%%%%%%%%%%%%%%%%%%%%%%%%%%%%%%%%%%%%%%%%%%%%%%%%%%%%
%%%%%%%%%%%%%%%%%%%%%%%%%%%%%%%%%%%%%%%%%%%%%%%%%%%%%%%%%%%%%%%%%%%%%%%%%%%%%%%



\subsection{A conventional object encoding}

An ICFP reviewer wrote: ``The encoding is quite simple~---~it's
surprising that everything is so easy~---~yet not at all obvious.''
Our representation of objects and their types is \emph{deliberately}
straightforward: polymorphic extensible records of closures. A more
efficient representation based on separate method and field tables (as
in C++ and Java) is possible in principle. Although our current
encoding is certainly not optimal, it is conceptually clearer. This
encoding is used in such languages as Perl, Python, Lua~---~and is
often the first one chosen when adding OO to an existing language. We
want the OOP system for Haskell to be available and usable
\emph{now}. Therefore, we have to get by with the existing Haskell
system (GHC) as it is.



%%%%%%%%%%%%%%%%%%%%%%%%%%%%%%%%%%%%%%%%%%%%%%%%%%%%%%%%%%%%%%%%%%%%%%%%%%%%%%%
%%%%%%%%%%%%%%%%%%%%%%%%%%%%%%%%%%%%%%%%%%%%%%%%%%%%%%%%%%%%%%%%%%%%%%%%%%%%%%%
%%%%%%%%%%%%%%%%%%%%%%%%%%%%%%%%%%%%%%%%%%%%%%%%%%%%%%%%%%%%%%%%%%%%%%%%%%%%%%%



\subsection{Practical use of OOHaskell idioms}

One of the main goals of this paper is to be able represent the
conventional OO code, in as straightforward way as possible.  We
illustrate OOHaskell with a series of practical examples as they are
commonly found in OO textbooks and programming language tutorials.
The implementation of our system may be not for the feeble at
heart~---~however, the user of the system must be able to write
conventional OO code without understanding the complexity of the
implementation.  Since most OO system in practical use have mutable
state, we will be concerned with mutable objects, implemented via
|IORef| or |STRef|. Functional objects bring quite an interesting
twist, which lends itself as a topic for future work.

%% \ralf{We face a confusion in terminology.
%% When we say encoding, we either refer to the fundamental 
%% OO encoding that we in turn encode in Haskell, or we might
%% just refer to the surface encoding of OO programs. 
%% I contend that we really need to be careful here.}


%%%%%%%%%%%%%%%%%%%%%%%%%%%%%%%%%%%%%%%%%%%%%%%%%%%%%%%%%%%%%%%%%%%%%%%%%%%%%%%
%%%%%%%%%%%%%%%%%%%%%%%%%%%%%%%%%%%%%%%%%%%%%%%%%%%%%%%%%%%%%%%%%%%%%%%%%%%%%%%
%%%%%%%%%%%%%%%%%%%%%%%%%%%%%%%%%%%%%%%%%%%%%%%%%%%%%%%%%%%%%%%%%%%%%%%%%%%%%%%



\subsection{Is there any ``type hacking'' involved?}

One may be tempted to suspect that OOHaskell relies on ``type
hacking''. We defend both on the practical grounds, language design,
and the theoretical grounds.

Foremost, multi-parameter type classes with functional dependencies
are \emph{not} a hack: they are well-formalized and reasonably
understood~\cite{SS04}. The fact that we found a quite unexpected (and
unintended) use of a particular language feature does not mean that
the result is practically useless. Template meta-programming in C++
has been the best known example of such ``type hacking''. And yet it
has lead to |boost|, which has become a de facto tool for modern C++
programming.
%
%\footnote{Adobe uses boost:
%\url{http://lambda-the-ultimate.org/node/view/563\#comment-4531}}
%
Templates and template meta-programming have changed the very
character of the language~\cite{fcpp-jfp} and made generative
programming research and practice in C++ possible
\cite{DSL-in-three-lang,siek05:_concepts_cpp0x}.

\begin{comment}
[Stroustrup interview?
  Need some reference. If we can't find any, I can use LtU references
  \url{http://lambda-the-ultimate.org/node/view/663} (see comments by
  Scott Johnson)
  \url{http://lambda-the-ultimate.org/node/view/663#comment-5839} See
  also:
  \url{http://spirit.sourceforge.net/distrib/spirit_1_7_0/libs/spirit/phoenix/doc/preface.html}
] 


An ICFP reviewer wrote:
``The result might seem poor and just containing clever tricks
however it took 10 years to obtain that proof of concepts and this
deserves attention.''
\end{comment}


%%%%%%%%%%%%%%%%%%%%%%%%%%%%%%%%%%%%%%%%%%%%%%%%%%%%%%%%%%%%%%%%%%%%%%%%%%%%%%%
%%%%%%%%%%%%%%%%%%%%%%%%%%%%%%%%%%%%%%%%%%%%%%%%%%%%%%%%%%%%%%%%%%%%%%%%%%%%%%%
%%%%%%%%%%%%%%%%%%%%%%%%%%%%%%%%%%%%%%%%%%%%%%%%%%%%%%%%%%%%%%%%%%%%%%%%%%%%%%%



\subsection{Theoretical contribution}

As far as OOP is concerned, OOHaskell intentionally resembles OCaml
(or, its predecessor, ML-ART \cite{ML-ART}).  ML-ART adds several
extensions to ML to implement objects: records with polymorphic access
and extension, projective records, recursive types, implicit
existential and universal types. Polymorphic records and existentials
are the main ones. As the paper \cite{ML-ART} says, none of the
extensions are new, but their combination is original and ``provides
just enough power to program objects in a flexible and elegant way.''

We make the same claim for OOHaskell, but using a quite different set
of features. We do rely on polymorphic records~---~which, although
have the same behavior, are very different from those in ML-ART. Our
polymorphic records have no special row variables and avoid
impredicative types. The fact that such records are realizable in
Haskell at all has been unexpected and unknown, until the \HList\
paper less than a year ago.\footnote{Again, there were many debates on
the Haskell mailing list about adding extensible records to
Haskell. At the Haskell 2003 workshop~\cite{HW03} this issue was
selected as prime topic for discussion.} Unlike ML-ART~\cite{ML-ART},
we do \emph{not} rely on existential or implicitly universal types,
nor recursive types. We use the value recursion instead (incidentally,
objects in ML-ART can be mutable too). As the consequence of the
simplicity of the type theory for our implementation, the types of
objects and object-manipulating functions and collections can be
\emph{inferred}, without any type annotations.

% safety: See ML-ART paper, p. 9, end of section 3.1
% and Sectiom 3.7
%
% See p. 29 of the paper and the quotation from Bruce, 1992. Pierce
% seems to agree that recursive types are needed to model inheritance
% of methods involving self. -- beginning of the second paragraph
% on p. 29
%

\begin{comment}
Such an extension (equi-recursive types)
 to Haskell was also debated and then rejected
because it will make type-error messages nearly
useless~\cite{Hughes02}. There is an alternative technique for
encoding objects: eschew recursive types in favour of existential
quantification~\cite{PT94}. Unfortunately, the involved higher-ranked
types can not be inferred anymore. Explicit signatures were required,
which, in practical terms, means that the user must explicitly
enumerate all virtual methods in the signature of any function that
operates on an object. This technique cannot be used in OOHaskell
because we would like OOP to be easy to use, first, by Haskell
programmer. That is, we ought to preserve type inference for functions
that use objects. Type inference is the great advantage of Haskell and
ML and is worth fighting for. 
\end{comment}

\begin{comment}
\ralf{Variance: since we don't do in the paper, this is odd.
Either we should dive into it in the advanced section,
or leave it out.}
\oleg{We have the code for that, it's already in the archive. I submit
  we still can mention it, just to keep people interested and keep
  asking questions.}

Another theoretical problem -- even with mutable objects -- is the
controversy regarding covariance and contravariance of method
arguments \cite{SG04}, \cite{catcall}. Our work shows how to implement
often desirable covariant methods and statically guarantee
soundness. Due to the lack of space, we merely skim the topic and
refer the reader to the code (and the next paper).
\end{comment}



%%%%%%%%%%%%%%%%%%%%%%%%%%%%%%%%%%%%%%%%%%%%%%%%%%%%%%%%%%%%%%%%%%%%%%%%%%%%%%%
%%%%%%%%%%%%%%%%%%%%%%%%%%%%%%%%%%%%%%%%%%%%%%%%%%%%%%%%%%%%%%%%%%%%%%%%%%%%%%%
%%%%%%%%%%%%%%%%%%%%%%%%%%%%%%%%%%%%%%%%%%%%%%%%%%%%%%%%%%%%%%%%%%%%%%%%%%%%%%%



\subsection{A sandbox for language design}

Just as C++ has become the laboratory for generative programming
\cite{DSL-in-three-lang} and lead to such applications as FC++, we
contend that Haskell can become the laboratory for OO design and
development.  To extend the motto by Simon Peyton-Jones, Haskell is
not only the best imperative language, it is the best OO language too.
C++ programmers now routinely use parsing combinators, thanks to the
|boost::spirit| library. Haskell programmers can likewise use OO
idioms if it suits the problem at hand. We can experiment with OO
features, without the need to change Haskell compilers.

Our OOHaskell library ~\cite{OOHaskell} ended up to be a comparatively rich
combination of OO idioms, higher-order functional programming, and
type inference.  We have experienced first-hand the incredible amount
of guidance Haskell type system gives in the design of OO features.
We therefore think that (OO)Haskell lends itself as an environment for
advanced and typed OO \emph{language design}.

\medskip

\section{Basic OOHaskell programming idioms}
\label{S:basics}

We start with the basics of OO: objects as capsules of mutable data
and methods, access control, constructor methods, self references,
single inheritance with extension or overriding. We will focus here
on the explanation of all technicalities that were not yet
illustrated in the shapes example.



%%%%%%%%%%%%%%%%%%%%%%%%%%%%%%%%%%%%%%%%%%%%%%%%%%%%%%%%%%%%%%%%%%%%%%%%%%%%%
%%%%%%%%%%%%%%%%%%%%%%%%%%%%%%%%%%%%%%%%%%%%%%%%%%%%%%%%%%%%%%%%%%%%%%%%%%%%%
%%%%%%%%%%%%%%%%%%%%%%%%%%%%%%%%%%%%%%%%%%%%%%%%%%%%%%%%%%%%%%%%%%%%%%%%%%%%%

\medskip

% \subsection{Adoption of the OCaml tutorial}

There are many OO systems based on open records, e.g., Perl, Python,
Javascript, Lua, and OCaml. Of these, only OCaml is statically
typed. OCaml (to be precise, its predecessor {ML-ART}) are close to
OOHaskell in motivation, of introducing objects as a library in a
strongly-typed functional language with inference. The implementation
of that library and the set of features used or required are quite
different (Sec.~\ref{S:related}), which makes the comparison with OCaml
meaningful. Therefore, we draw many of the examples from OCaml object
tutorial, to specifically contrast OCaml and OOHaskell code and to
demonstrate the fact that OCaml examples are expressible in OOHaskell,
roughly in the same syntax. We also use the OCaml object tutorial
because it is clear, comprehensive and concise.  We are keen to mimic
the OCaml code from the tutorial in some cases because this suggests a
direct, local translation.

Quoting from~\cite[\S\,3.1]{OCaml}:\footnote{Throughout the paper and
the source code distribution, we took the liberty to rename some
identifiers and to massage some subminor details while quoting
portions of the OCaml tutorial.}

\begin{quote}\itshape\small
``The class @point@ below defines one instance variable @varX@ and two
methods @getX@ and @moveX@. The initial value of the instance variable
is @0@. The variable @varX@ is declared mutable, so the method @moveX@
can change its value.''
\end{quote}

\begin{code}
 class point =
   object
     val mutable varX = 0
     method getX      = varX
     method moveX d   = varX <- varX + d
   end;;
\end{code}



%%%%%%%%%%%%%%%%%%%%%%%%%%%%%%%%%%%%%%%%%%%%%%%%%%%%%%%%%%%%%%%%%%%%%%%%%%%%%
%%%%%%%%%%%%%%%%%%%%%%%%%%%%%%%%%%%%%%%%%%%%%%%%%%%%%%%%%%%%%%%%%%%%%%%%%%%%%
%%%%%%%%%%%%%%%%%%%%%%%%%%%%%%%%%%%%%%%%%%%%%%%%%%%%%%%%%%%%%%%%%%%%%%%%%%%%%

\medskip

\subsection{Objects as HList records}

The transcription to Haskell starts with the declaration of all the
labels that occur in the OCaml code. The \HList\ library readily
offers 4 different models of labels. In all cases, labels are Haskell
values that are distinguished by their Haskell \emph{type}. We choose
the following model:
%
\begin{itemize}
\item The value of a label is ``\undefined''.
\item The type of a label is a \emph{proxy} for an \emph{empty} type (except ``\undefined'').
\end{itemize}

\medskip

\begin{code}
 data VarX;  varX  = proxy :: Proxy VarX 
 data GetX;  getX  = proxy :: Proxy GetX
 data MoveX; moveX = proxy :: Proxy MoveX
\end{code}

GHC allows us to define such empty algebraic datatypes. We note that
simple syntactic sugar can reduce the length of these one-liners
dramatically in case this is considered an issue.

We assume the following definitions of proxies:

\begin{Verbatim}[fontsize=\small,commandchars=\\\{\}]
 data Proxy e      -- \cmt{A proxy type is an empty phantom type.}
 proxy :: Proxy e  -- \cmt{A proxy value is just ``\undefined''.}
 proxy = \undefined
\end{Verbatim}

The \emph{explicit} declaration of OOHaskell labels blends perfectly
with Haskell's scoping rules and its module concept. If different
modules with various record types want to share labels, then they have
to agree on a declaration site that they all import. All models of
\HList\ labels support labels as first-class citizens. In particular,
we can pass them to functions. The ``labels as type proxies'' idea is
the basis for defining record operations since they can thereby
\emph{dispatch} on labels in type-class-based functionality. We refer
to the HList paper for details~\cite{HLIST-HW04}.

\noindent
The earlier @point@ class is defined as the following Haskell value:

\begin{code}
 point = 
   do
      x <- newIORef 0
      returnIO
        $  varX  .=. x
       .*. getX  .=. readIORef x
       .*. moveX .=. (\d -> do modifyIORef x ((+) d))
       .*. emptyRecord
\end{code}
%%% $
\noindent
Note how the Haskell code mimics the OCaml code. We use Haskell's
@IORefs@ to model mutable variables. (We do not use any magic of the
IO monad. We could as well use the simpler ST monad, which is very
well formalised~\cite{LPJ95}. In fact, the code distribution for the
paper explores this option, too.) The @point@ class stands revealed as
a monadic @do@ sequence that first creates an @IORef@ for the mutable
variable, and then returns a record for the new @point@ object.  The
record provides access to the public methods of the object and to the
@IORefs@ for public mutable variables.

Let's now instantiate the @point@ class and invoke some methods. To
provide a reference, we include the log of an OCaml session, which
shows some inputs and the responses of the OCaml interpreter:

\begin{code}
 let p = new point;;
 val p : point = <obj>
 p#getX;;
 - : int = 0
 p#moveX 3;;
 - : unit = ()
 p#getX;;
 - : int = 3
\end{code}
%% $
\noindent
In Haskell, we capture this program in a monadic @do@ sequence because
method invocations can involve side effects. Hence:

\begin{code}
 myFirstOOP =
  do
     p <- point -- no need for new!
     p # getX >>= Prelude.print
     p # moveX $ 3
     p # getX >>= Prelude.print
\end{code}
%%% $
\noindent
The following Haskell session agrees with the OCaml version:

\begin{code}
 ghci> myFirstOOP
 0
 3
\end{code}




%%%%%%%%%%%%%%%%%%%%%%%%%%%%%%%%%%%%%%%%%%%%%%%%%%%%%%%%%%%%%%%%%%%%%%%%%%%%%
%%%%%%%%%%%%%%%%%%%%%%%%%%%%%%%%%%%%%%%%%%%%%%%%%%%%%%%%%%%%%%%%%%%%%%%%%%%%%
%%%%%%%%%%%%%%%%%%%%%%%%%%%%%%%%%%%%%%%%%%%%%%%%%%%%%%%%%%%%%%%%%%%%%%%%%%%%%

\medskip

\subsection{Access control}

We note that the variable @varX@ is public~---~just as in the OCaml
code. Hence, we can manipulate @varX@ directly:

\begin{code}
 mySecondOOP =
  do 
     p <- point
     writeIORef (p # varX) 42
     p # getX >>= Prelude.print
 ghci> mySecondOOP
 0
 42
\end{code}

\noindent
Making the mutable variable private is no problem at all. We simply do
not provide direct access to the IORef in the record, i.e., we omit
the variable @varX@. (This was illustrated in the shapes example.) 
Using the delete operation for record components, we can also restrict
access after the fact.



%%%%%%%%%%%%%%%%%%%%%%%%%%%%%%%%%%%%%%%%%%%%%%%%%%%%%%%%%%%%%%%%%%%%%%%%%%%%%
%%%%%%%%%%%%%%%%%%%%%%%%%%%%%%%%%%%%%%%%%%%%%%%%%%%%%%%%%%%%%%%%%%%%%%%%%%%%%
%%%%%%%%%%%%%%%%%%%%%%%%%%%%%%%%%%%%%%%%%%%%%%%%%%%%%%%%%%%%%%%%%%%%%%%%%%%%%

\medskip

\subsection{Tailored object construction}

Quoting from~\cite[\S\,3.1]{OCaml}:

\begin{quote}\itshape\small
``The class @point@ can also be abstracted over the initial value of
@varX@.  The parameter @x_init@ is, of course, visible in the whole
body of the definition, including methods. For instance, the method
@getOffset@ in the class below returns the position of the object
relative to its initial position.''
\end{quote}

\begin{code}
 class para_point x_init =
   object
     val mutable varX = x_init
     method getX      = varX
     method getOffset = varX - x_init
     method moveX d   = varX <- varX + d
   end;;
\end{code}

\noindent
Non-parameterised classes are represented as computations in Haskell.
Consequently, parameterised classes are represented as monadic
functions (i.e., functions that return computations). Hence, the
parameter @x_init@ ends up as a plain function argument:

\begin{code}
 para_point x_init
   = do
        x <- newIORef x_init
        returnIO
          $  varX      .=. x
         .*. getX      .=. readIORef x
         .*. getOffset .=. queryIORef x (\v -> v - x_init)
         .*. moveX     .=. (\d -> modifyIORef x ((+) d))
         .*. emptyRecord
\end{code}
%%% $
\medskip

Quoting from~\cite[\S\,3.1]{OCaml}:

\begin{quote}\itshape\small
``Expressions can be evaluated and bound before defining the object
body of the class. This is useful to enforce invariants. For instance,
points can be automatically adjusted to the nearest point on a grid,
as follows:''
\end{quote}

\begin{code}
 class adjusted_point x_init =
   let origin = (x_init / 10) * 10 in
   object
     val mutable varX = origin
     method getX      = varX
     method getOffset = varX - origin
     method moveX d   = varX <- varX + d
   end;;
\end{code}

\noindent
This ability is akin to functionality in constructor methods known
from mainstream OO languages. As suggested by OCaml tutorial, we use
local lets to carry out the constructor computations ``prior'' to
returning the constructed object:

\begin{code}
 adjusted_point x_init
   = do
        let origin = (x_init `div` 10) * 10
        x <- newIORef origin
        returnIO
          $  varX      .=. x
         .*. getX      .=. readIORef x
         .*. getOffset .=. queryIORef x (\v -> v - origin)
         .*. moveX     .=. (\d -> modifyIORef x ((+) d))
         .*. emptyRecord
\end{code}
%%% $
\noindent
(The fact whether such lets are computed \emph{at all} depends, of
course, on the strictness of the program due to Haskell's lazyness. So
``prior'' is not meant in a temporal sense.)



%%%%%%%%%%%%%%%%%%%%%%%%%%%%%%%%%%%%%%%%%%%%%%%%%%%%%%%%%%%%%%%%%%%%%%%%%%%%%
%%%%%%%%%%%%%%%%%%%%%%%%%%%%%%%%%%%%%%%%%%%%%%%%%%%%%%%%%%%%%%%%%%%%%%%%%%%%%
%%%%%%%%%%%%%%%%%%%%%%%%%%%%%%%%%%%%%%%%%%%%%%%%%%%%%%%%%%%%%%%%%%%%%%%%%%%%%

\medskip

\subsection{Open recursion}

Another basic tenet of OO is to send messages to `self'.  One
\emph{could} simulate such selfish messages by implementing methods as
regular mutually recursive functions. Sending a message to `self' will
then clearly look differently than sending a message to another
object. A more important problem with that naive solution is the
closed nature of the method recursion. Overriding methods in
subclasses will not affect this recursion, drastically limiting
subtype polymorphism.

So we need to bind `self' explicitly. Consequently, object templates
need to be in the style of `open recursion': they take self and
construct (some part of) self.

Quoting from~\cite[\S\,3.2]{OCaml}:

\begin{quote}\itshape\small
``A method or an initialiser can send messages to self (that is, the
current object). For that, self must be explicitly bound, here to the
variable @s@ (@s@ could be any identifier, even though we will often
choose the name @self@.) ... Dynamically, the variable @s@ is bound at
the invocation of a method. In particular, when the class
@printable_point@ is inherited, the variable @s@ will be correctly
bound to the object of the subclass.''
\end{quote}

\begin{code}
 class printable_point x_init =
   object (s)
     val mutable varX = x_init
     method getX      = varX
     method moveX d   = varX <- varX + d
     method print = print_int s#getX
   end;;
\end{code}

\noindent
Again, this OCaml code is transcribed to Haskell very directly. A
noteworthy and appreciated deviation is that @s@ ends up as just an
\emph{ordinary} argument of the monadic function for constructing
printable point objects:

\begin{code}
 printable_point x_init s =
   do
      x <- newIORef x_init
      returnIO
        $  varX  .=. x
       .*. getX  .=. readIORef x
       .*. moveX .=. (\d -> modifyIORef x ((+) d))
       .*. print .=. ((s # getX ) >>= Prelude.print)
       .*. emptyRecord
\end{code}
%%% $
\noindent
Object creation and invocation looks as follows in OCaml:

\begin{code}
 let p = new printable_point 7;;
 val p : printable_point = <obj>
 p#moveX 2;;
 - : unit = ()
 p#print;;
 9- : unit = ()
\end{code}

\noindent
We note that @s@ does not show up in the OCaml line that constructs a
point @p@ with the @new@ construct, but it is clear that the recursive
knot is tied right there. The Haskell code makes this really explicit.
We do not use any special @new@ construct. We simply use the (monadic)
fix-point operation as is:

\begin{code}
 mySelfishOOP =
   do
      p <- mfix (printable_point 7)
      p # moveX $ 2
      p # print
\end{code}
%%% $
\begin{code}
 ghci> mySelfishOOP
 9
\end{code}



%%%%%%%%%%%%%%%%%%%%%%%%%%%%%%%%%%%%%%%%%%%%%%%%%%%%%%%%%%%%%%%%%%%%%%%%%%%%%
%%%%%%%%%%%%%%%%%%%%%%%%%%%%%%%%%%%%%%%%%%%%%%%%%%%%%%%%%%%%%%%%%%%%%%%%%%%%%
%%%%%%%%%%%%%%%%%%%%%%%%%%%%%%%%%%%%%%%%%%%%%%%%%%%%%%%%%%%%%%%%%%%%%%%%%%%%%

\medskip

\subsection{Single inheritance with extension}

Quoting from~\cite[\S\,3.7]{OCaml}:

\begin{quote}\itshape\small
``We illustrate inheritance by defining a class of colored points that
inherits from the class of points. This class has all instance
variables and all methods of class @point@, plus a new instance
variable @color@, and a new method @getColor@.''
\end{quote}

\begin{code}
 class colored_point x (color : string) =
   object
     inherit point x
     val color = color
     method getColor = color
   end;;
\end{code}

\begin{code}
 let p' = new colored_point 5 "red";;
 val p' : colored_point = <obj>
\end{code}

\begin{code} 
 p'#getX, p'#getColor;;
 - : int * string = (5, "red")
\end{code}

\noindent
(We only consider width subtyping at this stage.)  The following
Haskell version does not refer to a special @inherit@ construct. We
rather compose a computation. That is, to construct a colored point,
we instantiate the superclass while maintaining open recursion, and
the obtained record is extended by the new method:

\begin{code}
 colored_point x_init (color::String) self =
   do
        p <- printable_point x_init self
        returnIO $ getColor .=. (returnIO color) .*. p
 myColoredOOP =
   do
      p' <- mfix (colored_point 5 "red")
      x  <- p' # getX
      c  <- p' # getColor
      Prelude.print (x,c)
 ghci>  myColoredOOP
 (5,"red")
\end{code}
%%% $


%%%%%%%%%%%%%%%%%%%%%%%%%%%%%%%%%%%%%%%%%%%%%%%%%%%%%%%%%%%%%%%%%%%%%%%%%%%%%
%%%%%%%%%%%%%%%%%%%%%%%%%%%%%%%%%%%%%%%%%%%%%%%%%%%%%%%%%%%%%%%%%%%%%%%%%%%%%
%%%%%%%%%%%%%%%%%%%%%%%%%%%%%%%%%%%%%%%%%%%%%%%%%%%%%%%%%%%%%%%%%%%%%%%%%%%%%

\medskip

\subsection{Single inheritance with overriding}

We can also override methods and refer to the implementation of a
method in the superclass (akin to the @super@ construct in OCaml and
other languages). This is illustrated with a subclass of
@colored_point@ whose @print@ method is more informative:

\begin{code}
 colored_point' x_init color self =
   do
      super <- colored_point x_init color self
      return $  print .=. (
              do putStr "so far - "; p # print
                 putStr "color  - "; Prelude.print color )
            .<. p
\end{code}
%%% $
\noindent
The first step in the monadic @do@ sequence constructs an
old-fashioned colored point, and binds it to @super@ for further
reference. (Note: @super@ is just a variable not an extra construct.) 
The second step in the monadic @do@ sequence returns @super@ but
updated as far as the @print@ method is concerned. The \HList\
operation ``@.<.@'' denotes type-preserving record update as opposed
to record extension. This operation ``@.<.@'' rather than the familiar
``@.*.@'' makes the overriding explicit (as it is in |C#|, for
example). We could also use a more hybrid record operation, which
carries out extension in case the given label does not yet occur in
the given record, while it falls back to type-preserving update
otherwise. The latter operation would let us model the implicit
overriding in C++ and Java.

Here is a demo of inheritance with override:

\begin{code}
 myOverridingOOP =
   do
      p  <- mfix (colored_point' 5 "red")
      p  # print
 ghci> myOverridingOOP
 so far - 5
 color  - "red"
\end{code}

\medskip

\section{Discussion of typing issues}
\label{S:types}

We have played strong so far by leaving everything to type inference,
by not writing down a single piece of type signature or type
annotation (except perhaps for resolving unwanted polymorphism).
Eventually, programmers want to look at inferred types, they might
want to declare types, they certainly have to understand type
errors. These are the issues that we will discuss first. Later, we will
also compare some models of dealing with subtype polymorphism, which
differ regarding the kind and amount of type information that has to
be expressed by the programmer.



%%%%%%%%%%%%%%%%%%%%%%%%%%%%%%%%%%%%%%%%%%%%%%%%%%%%%%%%%%%%%%%%%%%%%%%%%%%%%
%%%%%%%%%%%%%%%%%%%%%%%%%%%%%%%%%%%%%%%%%%%%%%%%%%%%%%%%%%%%%%%%%%%%%%%%%%%%%
%%%%%%%%%%%%%%%%%%%%%%%%%%%%%%%%%%%%%%%%%%%%%%%%%%%%%%%%%%%%%%%%%%%%%%%%%%%%%



\medskip

\subsection{Type inference}

Readers of a previous version of the paper have wondered whether the
inferred types would be perhaps incomprehensible. That is actually not
the case (Likewise for error messages as reviewed later.)

Let us infer the result of constructing a colored point:
 
\begin{code}
 ghci6.4> :t mfix $ colored_point (1::Int) "red"
 mfix $ colored_point (1::Int) "red" ::
        IO (Record 
            (HCons (Proxy GetColor, IO String)
             (HCons (Proxy VarX, IORef Int)
              (HCons (Proxy GetX, IO Int)
               (HCons (Proxy MoveX, Int -> IO ())
                (HCons (Proxy Print, IO ())
                 HNil))))))
\end{code} 

The type is pretty readable, even though it reveals the underlying
representation of records (as a heterogeneous list of label-value
pairs), and it also gives away the proxy-based model for labels.  It
is reasonable to expect that a more customisable `pretty printer' for
types could easily present the result of type inference as
follows:

\begin{code}
 ghci6.6> :t mfix $ colored_point (1::Int) "red"
 mfix $ colored_point (1::Int) "red" ::
        IO ( Record (
               GetColor :=: IO String
           :*: VarX     :=: IORef Int
           :*: GetX     :=: IO Int
           :*: MoveX    :=: Int -> IO ()
           :*: Print    :=: IO ()
           :*: HNil ))
\end{code} 

In this object construction example, we had all polymorphism
resolved. Also, the recursive knot was tied. Let's consider type
inference for the case of more polymorphic expressions. In
fact, here is the type of class @colored_point@:

\begin{code}
 ghc6.6> :t colored_point
 ( Num a
 , HasField GetX r (IO a1)
 , Show a1
 ) => a
   -> String
   -> r
   -> IO ( Record (
             GetColor :=: IO String
         :*: VarX     :=: IORef a
         :*: GetX     :=: IO a
         :*: MoveX    :=: a -> IO ()
         :*: Print    :=: IO ()
         :*: EmptyRecord ))
\end{code}

That is, the type of the constructor essentially lists all the fields
of an object, both new and inherited. Assumptions about @self@ are
expressed as constraints on the type variable @r@. The constructor
@colored_point@ refers to @getX@ (through @self@), and hence this
reference implies a constraint of the form
@HasField@~@GetX@~@r@~@(IO@~@a1)@. We note the polymorphism in the
coordinate type for the point; cf.\ @a@ (and @a1@). Since arithmetic
is performed on @GetX@, this implies bounded polymorphism: only @Num@
types are permitted. Interestingly, type inference does not infer the
fact that @a@ and @a1@ eventually must be the same. (@r@ with the
@HasField GetX r (IO a1)@ constraint is part of the result record
whose @GetX@ component is said to be of type @a@.)  This equality is
not inferred because we have not (yet) taught the type system about
the law that record extension preserves the type of previous
components.

We must admit that we have assumed a \emph{relatively} eager instance
selection in the previous session. The hugs implementation of Haskell
is (more than) eager enough. The recent versions of GHC have become
more and more lazy. In a session with contemporary GHC (6.4) we would
additionally see the following constraints, which deal with the
uniqueness of label sets as they are encountered during record
extension:

\begin{Verbatim}[fontsize=\small,commandchars=\\\{\}]
 HRLabelSet (HCons (Proxy MoveX, a -> IO ())
            (HCons (Proxy Print, IO ()) HNil)),
 \cmt{likewise for} MoveX, Print, GetX
 \cmt{likewise for} MoveX, Print, GetX, VarX
 \cmt{likewise for} MoveX, Print, GetX, VarX, GetColor
\end{Verbatim}
 
Inspection of the @HRLabelSet@ instances would reveal that these
constraints are all satisfied, no matter how the type variable @a@ is
instantiated. No ingenuity is required. A simple form of strictness
analysis is sufficient. Alas, GHC is consistently lazy in resolving
even such constraints. This is the issue we intend to bring to the GHC
implementors.

The type without the @HRLabelSet@ constraints looks very
reasonable. The type explicitly lists all the fields and the types of
their values. The type is actually quite readable. Because the type
lists both the new fields and all inherited fields, the type could
even be used for the simple implementation of a class browser in an
IDE. (The class browser does not need to figure out the set of all
methods in a class by itself: the compiler has already done that, and
expressed in the type.)

We must mention that the issue of the object types, as inferred vs. as
they are actually shown to the user existed for OCaml, and it has been
resolved as we hypothesised it could for GHC. Although objects types
shown by OCaml are quite concise, that has not always been the
case. In the {ML-ART} system, the predecessor of OCaml with no
syntactic sugar~\cite{ML-ART} (Section 3):

\begin{quote}\small
``objects have anonymous, long, and often recursive
types that describe all methods that the object can receive. Thus, we
usually do not show the inferred types of programs in order to
emphasize object and inheritance encoding rather than typechecking
details. This is quite in a spirit of ML where type information is
optional and is mainly used for documentation or in module
interfaces. Except when trying top-level examples, or debugging, the
user does not often wish to see the inferred types of his programs in
a batch compiler.''
\end{quote}



%%%%%%%%%%%%%%%%%%%%%%%%%%%%%%%%%%%%%%%%%%%%%%%%%%%%%%%%%%%%%%%%%%%%%%%%%%%%%
%%%%%%%%%%%%%%%%%%%%%%%%%%%%%%%%%%%%%%%%%%%%%%%%%%%%%%%%%%%%%%%%%%%%%%%%%%%%%
%%%%%%%%%%%%%%%%%%%%%%%%%%%%%%%%%%%%%%%%%%%%%%%%%%%%%%%%%%%%%%%%%%%%%%%%%%%%%



\medskip

\subsection{Type errors}

We now briefly illustrate type errors in OOHaskell programming.  To
this end, we will experiment with virtual methods.

In OCaml, one can declare a method without actually defining it, using
the keyword @virtual@. A class containing virtual methods must be
flagged @virtual@, and cannot be instantiated. Virtual methods will be
implemented in subclasses. Virtual classes still define type
abbreviations. Here is a virtual class:

\begin{code}
 class virtual abstract_point x_init =
   object (self)
     val mutable varX = x_init
     method print = print_int self#getX
     method virtual getX : int
     method virtual moveX : int -> unit
   end;;
\end{code}

\noindent
In C++, one calls such methods \emph{pure} virtual methods and classes
that cannot be instantiated are called abstract. In Java, we can flag
classes as being abstract. In Haskell, we do not need any special
constructs. A virtual method is simply not defined, and that's it:

\begin{code}
 abstract_point x_init self =
   do
      x <- newIORef x_init
      returnIO $
           varX   .=. x
       .*. print  .=. (self # getX >>= Prelude.print )
       .*. emptyRecord
\end{code}
%%% $

\noindent
This specific class cannot be instantiated with @mfix@ because @getX@
is used but not defined. It is worth quoting an error message:

\begin{code}
 No instance for (HasField (Proxy GetX) HNil (IO a))
   arising from use of `abstract_point' at ...
 In the first argument of `mfix', namely `(abstract_point 7)'
 In a 'do' expression: p' <- mfix (abstract_point 7)
\end{code}

The error message is concise and to the point. The error message
succinctly list just the missing field. In general, the clarity of
error messages is undoubtedly an area that needs more research, and
such research is being done~\cite{SSW04}, which we or compiler writers
may take advantage of. It must be mentioned that error messages in C++
(template instantiation) can be immensely verbose, spanning literally
30 and 40 packed lines. And yet boost and similar libraries that
extensively use templates are gaining momentum.


%%%%%%%%%%%%%%%%%%%%%%%%%%%%%%%%%%%%%%%%%%%%%%%%%%%%%%%%%%%%%%%%%%%%%%%%%%%%%
%%%%%%%%%%%%%%%%%%%%%%%%%%%%%%%%%%%%%%%%%%%%%%%%%%%%%%%%%%%%%%%%%%%%%%%%%%%%%
%%%%%%%%%%%%%%%%%%%%%%%%%%%%%%%%%%%%%%%%%%%%%%%%%%%%%%%%%%%%%%%%%%%%%%%%%%%%%



\medskip

\subsection{Explicit self constraints}

The type-error discussion incidentally disclosed OOHaskell's ease with
virtual methods. The Haskell type system effectively prevents us from
instantiating classes which use methods neither they nor their parents
have defined. There arises the question of the explicit designation of
a method as pure virtual (even if it does not happen to be used in
the class itself).

We can simply constrain @self@. As a matter of discipline, we do not
want to rewrite the earlier definition of the @abstract_point@ value.
Instead, we add an \emph{inapplicable} equation whose only purpose
is to impose a type constraint on the class:

\begin{Verbatim}[fontsize=\small,commandchars=\\\{\}]
 abstract_point (x_init::a) self 
  | const False (constrain self ::
                 Proxy (  (Proxy GetX, IO a)
                      :*: (Proxy MoveX, a -> IO ())
                      :*: HNil ))
  = \undefined
\end{Verbatim}

\noindent
(That is, we have written an equation with an always failing guard
(cf.\ @const@~@False@) that nevertheless imposes typing constraints.
The equation evaluates to \undefined, which is Ok because it will
never be chosen anyhow.) The @constrain@ operation processes a record,
i.e., @self@. An application of the operation must be
annotated with a type for a list of label-component pairs.  The form
|constrain| is quite akin to C++ \emph{concepts}
\cite{siek05:_concepts_cpp0x}. Type-checking the application of
@constrain@ implies checking whether the listed labels occur in the
given record, and whether the components are of the required types.
As we can see in the type annotation, we let @constrain@ return a type
proxy. This makes it crystal-clear that no interesting computation is
performed: type-checking is of only interest here. (Once again, modest
syntactic sugar could make this idiom look less idiosyncratic, but we
are keen to reveal the true technicalities.)

One possible implementation of @constrain@ is to check whether we can
\emph{narrow} the argument record to the result record
type.  Of course, it is enough to attempt narrowing at the type level
alone because we are not interested in a coerced value here. That is:

\begin{code}
 constrain :: Narrow r l => Record r -> Proxy l
 constrain = const proxy
\end{code}

\noindent
Narrowing is a type-driven projection operation on records, which
lives in the \HList\ library. (We can also take co-/contra-variance
for method types into account).


%%%%%%%%%%%%%%%%%%%%%%%%%%%%%%%%%%%%%%%%%%%%%%%%%%%%%%%%%%%%%%%%%%%%%%%%%%%%%
%%%%%%%%%%%%%%%%%%%%%%%%%%%%%%%%%%%%%%%%%%%%%%%%%%%%%%%%%%%%%%%%%%%%%%%%%%%%%
%%%%%%%%%%%%%%%%%%%%%%%%%%%%%%%%%%%%%%%%%%%%%%%%%%%%%%%%%%%%%%%%%%%%%%%%%%%%%



\medskip

\subsection{Explicit typing and casting}

There are situations when the explicit declaration of types can be
necessary, when programming idioms like casting (aka narrowing,
coercion) are to be applied. We look back to the shapes sample for a
scenario. We recall the part of the OOHaskell program that placed
different shapes in a normal, homogeneous Haskell list:

\begin{code}
 let scribble = lubCast (HCons s1 (HCons s2 HNil))
\end{code}

While this is our \emph{favourite} model of dealing with the
combination subtyping polymorphism + collections, one can think of a
less automatic approach. So rather than computing the least upper
bound (whatever it happens to be), we could want to explicitly cast
each single shape to a fixed upper bound type. OOHaskell offers this
idiom as well:

\begin{Verbatim}[fontsize=\small,commandchars=\\\{\}]
 let scribble :: [Shape Int] -- Shape \cmt{to be defined}
     scribble = [narrow s1, narrow s2]
\end{Verbatim}

An explicit coercion operation, @narrow@, now prepares each shape
object for insertion into the homogeneous list. As an aside, such
casting is \emph{implicit} in the C++ code from which we started --
but must be \emph{explicit} in OCaml (where OOHaskell's favourable
|lubCast| is not supported by its type system).

Here is the record type for @Shape@ objects.

\begin{code}
 type Shape a = Record (  (Proxy GetX    , IO a)
                      :*: (Proxy GetY    , IO a)
                      :*: (Proxy SetX    , a -> IO ())
                      :*: (Proxy SetY    , a -> IO ())
                      :*: (Proxy MoveTo  , a -> a -> IO ())
                      :*: (Proxy RMoveTo , a -> a -> IO ())
                      :*: (Proxy Draw    , IO ())
                      :*: HNil )
\end{code}

We note that we have included the virtual @draw@ operation because it
is a part of the interface that is used in the loop over scribble.



%%%%%%%%%%%%%%%%%%%%%%%%%%%%%%%%%%%%%%%%%%%%%%%%%%%%%%%%%%%%%%%%%%%%%%%%%%%%%
%%%%%%%%%%%%%%%%%%%%%%%%%%%%%%%%%%%%%%%%%%%%%%%%%%%%%%%%%%%%%%%%%%%%%%%%%%%%%
%%%%%%%%%%%%%%%%%%%%%%%%%%%%%%%%%%%%%%%%%%%%%%%%%%%%%%%%%%%%%%%%%%%%%%%%%%%%%



\medskip

\subsection{Trading heterogeneity for subtyping}

So far we have shown two (more reasonable) ways to build a container
over objects of different subtypes.  Nevertheless, it is very
insightful to consider other techniques, be it just to see where they
fall short.

Rather than casting all elements of an emerging list to the
\emph{same} record type, we may wonder whether these coercions can
be avoided altogether.

The first technique is to collect the shape objects, as is, in a
\emph{heterogeneous} list rather than a homogeneous array or list. We
cannot construct such a list with the normal, polymorphic list
datatype constructor, but the \HList\ library comes again to our
rescue. The scribble construction can now be performed without any
narrowing:

\begin{code}
 let scribble = s1 `HCons` (s2 `HCons` HNil)
\end{code}

\noindent
We cannot use ordinary list-processing function anymore, but the
\HList\ library mimics the normal list-processing API for @HList@s. So
there is also a heterogeneous variation on @mapM_@, namely @hMapM_@,
to be invoked as follows:

\begin{Verbatim}[fontsize=\small,commandchars=\\\{\}]
 hMapM_ (\undefined::FunOnShape) scribble
\end{Verbatim}

\noindent
The first argument of @hMapM_@ is not a function but rather a
\emph{type code}. This is necessary for technical reasons related to
the combination of rank-n polymorphism and ad-hoc
polymorphism.\footnote{\small A heterogeneous map function can encounter
entities of different types. Hence, its argument function must be
polymorphic on its own (which is different from the normal map
function). The argument function typically uses type classes (say,
ad-hoc polymorphism) to process the entities of different types. The
trouble is that the map function cannot possibly anticipate all the
constraints required by its argument function.  The type-code
technique moves the constraints from the type of the heterogeneous map
function to the interpretation site of the type codes.} The meaning of
each type code must be defined by a dedicated instance of an @Apply@
class for function application. Here is the declaration of the type
code @FunOnShape@ complete with its meaning:

\begin{code}
 data FunOnShape -- a type code only!
\end{code}

\begin{code}
instance ( HasField (Proxy Draw) r (IO ())
         , HasField (Proxy RMoveTo) r (Int -> Int -> IO ())
         )
      => Apply FunOnShape r (IO ())
  where
    apply _ x = do
                   x # draw
                   (x # rMoveTo) 100 100
                   x # draw
\end{code}

\noindent
The @Apply@ instance manifests encoding efforts that we did not face
for the casting-based techniques. Now we have to list the
\emph{method-access constraints} (for ``\#'', i.e., @HasField@) in the
@Apply@ instance. Haskell's type-class system requires us to provide
proper bounds for the instance.

One might argue that the form of these constraints strongly resembles
the method types listed in the class type @Shape@. So one might wonder
whether we can somehow use the full class type in order to constrain
the instance.  Haskell will not let us do that in any reasonable
way. (Constraints are not first-class citizens in Haskell; we cannot
compute them from types or type proxies~---~unless we were willing to
rely on heavy encoding or advanced syntactic sugar.) So we are doomed
to manually infer such method-access constraints for each such piece
of polymorphic code.



%%%%%%%%%%%%%%%%%%%%%%%%%%%%%%%%%%%%%%%%%%%%%%%%%%%%%%%%%%%%%%%%%%%%%%%%%%%%%
%%%%%%%%%%%%%%%%%%%%%%%%%%%%%%%%%%%%%%%%%%%%%%%%%%%%%%%%%%%%%%%%%%%%%%%%%%%%%
%%%%%%%%%%%%%%%%%%%%%%%%%%%%%%%%%%%%%%%%%%%%%%%%%%%%%%%%%%%%%%%%%%%%%%%%%%%%%



\medskip

\subsection{The use of existentials}
\label{S:ex}

The second (again suboptimal) technique for avoiding casts relies on
placing shape objects in \emph{existentially quantified envelopes}: we
do not coerce, but we wrap:

\begin{code}
 let scribble = [ WrapShape s1 , WrapShape s2 ]
\end{code}

\noindent
The declaration of the @WrapShape@ type depends on the function that
we want to apply to the opaque data. In our case, we can use the
normal @mapM_@ function again; we only need to unwrap the @WrapShape@
constructor prior to method invocations:

\begin{code}
 mapM_ ( \(WrapShape shape) -> do
             shape # draw
             (shape # rMoveTo) 100 100
             shape # draw )
         scribble
\end{code}

\noindent
These operations have to be anticipated in the type bound for
@WrapShape@:

\begin{Verbatim}[fontsize=\small,commandchars=\\\{\}]
 data WrapShape =
  \Forall x. ( HasField (Proxy Draw) x (IO ())
       , HasField (Proxy RMoveTo) x (Int -> Int -> IO ())
       ) => WrapShape x
\end{Verbatim}

\noindent
It becomes evident that this result agrees with the heterogeneity
technique in terms of encoding efforts. In both cases, we need to
identify type-class constraints that correspond to the (potentially)
polymorphic method invocations. This is a show stopper. So the use of
explicit casting (@narrow@) or more implicit LUB construction
(@lubCast@) is clearly to be preferred.


\medskip

\section{Illustrative strengths of OOHaskell}
\label{S:strengths}

We would like to highlight some pieces of expressiveness and issues of
convenience implied by our use of Haskell as the base language for OO
programming. Most notably, labels, classes, and methods are all
first-class citizens in OOHaskell, which can be passed as arguments
and returned as results. The combined list that follows is not covered
by any mainstream OO language. This strongly suggests that (OO)Haskell
lends itself as prime environment for typed object-oriented language
design.



%%%%%%%%%%%%%%%%%%%%%%%%%%%%%%%%%%%%%%%%%%%%%%%%%%%%%%%%%%%%%%%%%%%%%%%%%%%%%
%%%%%%%%%%%%%%%%%%%%%%%%%%%%%%%%%%%%%%%%%%%%%%%%%%%%%%%%%%%%%%%%%%%%%%%%%%%%%
%%%%%%%%%%%%%%%%%%%%%%%%%%%%%%%%%%%%%%%%%%%%%%%%%%%%%%%%%%%%%%%%%%%%%%%%%%%%%



\medskip

\subsection{Functions polymorphic in classes}

Consider the following example:

\begin{code}
 myFirstClassOOP point_class =
   do
      p <- mfix (point_class 7)
      p # moveX $ 35
      p # print
 > myFirstClassOOP printable_point
 42
\end{code}
%%% $

\noindent
That is, we have parameterised the function @myFirstClassOOP@ with
respect to a class. We pass @myFirstClassOOP@ a constructor function
(a `class'), which, when instantiated, creates an object with the
slots |move| and |print|. We can pass @myFirstClassOOP@ any
constructor of the class printable point --- or of any other class
provided that the required slots are present. The latter constraint is
statically verified. For instance, our earlier colored point is still
a printable point:

\begin{code}
 ghci> myFirstClassOOP $ flip colored_point "red"
 so far - 42
 color  - "red"
\end{code}
%%% $


%%%%%%%%%%%%%%%%%%%%%%%%%%%%%%%%%%%%%%%%%%%%%%%%%%%%%%%%%%%%%%%%%%%%%%%%%%%%%
%%%%%%%%%%%%%%%%%%%%%%%%%%%%%%%%%%%%%%%%%%%%%%%%%%%%%%%%%%%%%%%%%%%%%%%%%%%%%
%%%%%%%%%%%%%%%%%%%%%%%%%%%%%%%%%%%%%%%%%%%%%%%%%%%%%%%%%%%%%%%%%%%%%%%%%%%%%



\medskip

\subsection{Reusable methods without hosts}

Methods too are first-class citizens in OOHaskell. With a bit of
parameterisation and higher-order functional programming, we can
programme methods outside of any hosting class. Such methods can be
reused across classes without any inheritance relationship.  For
instance, let's identify a method @print_getX@ that can be shared by
all objects that have at least the method @getX@ of type
@Show@~@a@~@=>@~@IO@~@a@~---~regardless of any inheritance
relationships:

\begin{code}
 print_getX self = ((self # getX ) >>= Prelude.print)
\end{code}

\noindent
We can update the code for @printable_point@ as follows:

\begin{code}
 -- before
 ... .*. print    .=. ((s # getX ) >>= Prelude.print)
 -- after
 ... .*. print    .=. print_getX s
\end{code}



%%%%%%%%%%%%%%%%%%%%%%%%%%%%%%%%%%%%%%%%%%%%%%%%%%%%%%%%%%%%%%%%%%%%%%%%%%%%%
%%%%%%%%%%%%%%%%%%%%%%%%%%%%%%%%%%%%%%%%%%%%%%%%%%%%%%%%%%%%%%%%%%%%%%%%%%%%%
%%%%%%%%%%%%%%%%%%%%%%%%%%%%%%%%%%%%%%%%%%%%%%%%%%%%%%%%%%%%%%%%%%%%%%%%%%%%%



\medskip

\subsection{Arbitrarily nested object generators}

Quoting from~\cite[\S\,3.1]{OCaml}:

\begin{quote}\itshape\small
``The evaluation of the body of a class only takes place at object
creation time.  Therefore, in the following example, the instance
variable @varX@ is initialised to different values for two different
objects.''
\end{quote}

\begin{code}
 let x0 = ref 0;;
 val x0 : int ref = {contents = 0}
\end{code}

\begin{code}
 class incrementing_point :
   object
     val mutable varX = incr x0; !x0
     method getX      = varX
     method moveX d   = varX <- varX + d
   end;;
\end{code}

\begin{code}
 new incrementing_point#getX;;
 - : int = 1
 new incrementing_point#getX;;
 - : int = 2
\end{code}

\noindent
Before we transcribe the use of this OCaml idiom to Haskell, we
observe that we can view the body of a class as the body of a
constructor method. Then, any mutable variable that is used along
subsequent invocations of the constructor functionality can be viewed
as belonging to a \emph{class object}.

So we arrive at a nested object generator:

\begin{code}
 incrementing_point = 
   do 
      x0 <- newIORef 0
      returnIO (
        do modifyIORef x0 (+1)
           x <- readIORef x0 >>= newIORef
           returnIO
             $  varX  .=. x
            .*. getX  .=. readIORef x
            .*. moveX .=. (\d -> modifyIORef x ((+) d))
            .*. emptyRecord)
\end{code}
%%% $
\noindent
In general, such nesting could be any number of levels deep since we
just use normal Haskell scopes. In the example, at the outer level, we
do the computation for the point template; at the inner level, we
perform the computation that constructs points themselves. This value
deserves a more OOP-biased name:

\begin{code}
 makeIncrementingPointClass = incrementing_point
\end{code}



%%%%%%%%%%%%%%%%%%%%%%%%%%%%%%%%%%%%%%%%%%%%%%%%%%%%%%%%%%%%%%%%%%%%%%%%%%%%%
%%%%%%%%%%%%%%%%%%%%%%%%%%%%%%%%%%%%%%%%%%%%%%%%%%%%%%%%%%%%%%%%%%%%%%%%%%%%%
%%%%%%%%%%%%%%%%%%%%%%%%%%%%%%%%%%%%%%%%%%%%%%%%%%%%%%%%%%%%%%%%%%%%%%%%%%%%%



\medskip

\subsection{Class closures}

We use the class object from the previous example:

\begin{code}
 myNestedOOP =
   do
      localClass <- makeIncrementingPointClass
      localClass >>= ( # getX ) >>= Prelude.print
      localClass >>= ( # getX ) >>= Prelude.print
 ghci> myNestedOOP
 1
 2
\end{code}
%%% $
\noindent
We effectively created a class in a scope, and then exported it,
closing over a locally-scoped variable. We cannot do such a class
closure in Java! Java supports anonymous objects, but not anonymous
first-class classes. (Nested classes in Java must be linked to an
object of the enclosing class.) C++ is nowhere close to such an
ability.



%%%%%%%%%%%%%%%%%%%%%%%%%%%%%%%%%%%%%%%%%%%%%%%%%%%%%%%%%%%%%%%%%%%%%%%%%%%%%
%%%%%%%%%%%%%%%%%%%%%%%%%%%%%%%%%%%%%%%%%%%%%%%%%%%%%%%%%%%%%%%%%%%%%%%%%%%%%
%%%%%%%%%%%%%%%%%%%%%%%%%%%%%%%%%%%%%%%%%%%%%%%%%%%%%%%%%%%%%%%%%%%%%%%%%%%%%



\medskip

\subsection{Implicit polymorphism}

The class of printable points, given earlier, is polymorphic with
regard to the point's coordinate~---~without our contribution. This is
a fine difference between the OCaml model and our Haskell
transcription. In OCaml's definition of @printable_point@, the
parameter @x_init@ was of the type @int@~---~because the operation
@(+)@ in OCaml can deal with integers only. Our points are
polymorphic~---~a point's coordinate can be any @Num@-ber, for
example, an @Int@ or a @Double@. Here is an example to illustrate
that:

\begin{code}
 myPolyOOP =
   do
      p  <- mfix (printable_point (1::Int))
      p' <- mfix (printable_point (1::Double))
      p  # moveX $ 2
      p' # moveX $ 2.5
      p  # print
      p' # print
\end{code}

\noindent
Our points are actually \emph{bounded} polymorphic. The point
coordinate may be of any type that implements addition. Until very
recently, one could not express this in Java and in C\#. Expressing
bounded polymorphism in C++ is possible with significant
contortions. In (OO)Haskell, we did not have to do anything at
all. Bounded polymorphism (aka, generics) are available in Ada95,
Eiffel and a few other languages. However, in those languages, the
polymorphic type and the type bounds must be declared
\emph{explicitly}. In (OO)Haskell, the type system \emph{infers} the
(bounded) polymorphism on its own.

Of course, implicit polymorphism does not injur static typing. (This
is unlike the poor men's implementation of polymorphic collections,
e.g., in Java @<@ 1.5, which up-casts all the items to the most
general type, @Object@, when inserting elements into the collection,
and which attempts runtime-checked downcasts when accessing elements.) 
Indeed, if we confuse @Int@s and @Double@s in the above code, say we
attempt ``@p@~@#@~@moveX@~@$@~@2.5@'', then we get a type error saying
that @Int@ is not the same as @Double@.

%%% $

 
%%%%%%%%%%%%%%%%%%%%%%%%%%%%%%%%%%%%%%%%%%%%%%%%%%%%%%%%%%%%%%%%%%%%%%%%%%%%%
%%%%%%%%%%%%%%%%%%%%%%%%%%%%%%%%%%%%%%%%%%%%%%%%%%%%%%%%%%%%%%%%%%%%%%%%%%%%%
%%%%%%%%%%%%%%%%%%%%%%%%%%%%%%%%%%%%%%%%%%%%%%%%%%%%%%%%%%%%%%%%%%%%%%%%%%%%%



\begin{figure}[t]
\begin{center}
\resizebox{.45\textwidth}{!}{\includegraphics{heavy.pdf}}
\end{center}
\caption{A complex inheritance scenario}
\label{F:heavy}
\end{figure}



%%%%%%%%%%%%%%%%%%%%%%%%%%%%%%%%%%%%%%%%%%%%%%%%%%%%%%%%%%%%%%%%%%%%%%%%%%%%%
%%%%%%%%%%%%%%%%%%%%%%%%%%%%%%%%%%%%%%%%%%%%%%%%%%%%%%%%%%%%%%%%%%%%%%%%%%%%%
%%%%%%%%%%%%%%%%%%%%%%%%%%%%%%%%%%%%%%%%%%%%%%%%%%%%%%%%%%%%%%%%%%%%%%%%%%%%%



\subsection{Programmer-defined inheritance}

In several OO languages, multiple inheritance is allowed. For
instance, in OCaml, the rules are as follows. Only the last definition
of a method is kept: the redefinition in a subclass of a method that
was visible in the parent class overrides the definition in the parent
class. Previous definitions of a method can be reused by binding the
related ancestor using a special \ldots @as@ \ldots notation.  The
bound name is said to be a pseudo value identifier that can only be
used to invoke an ancestor method. Other rules and notations exist for
Eiffel, C++, and so on.

In OOHaskell, the programmer can set up inheritance in flexible ways
taking advantage of the (typed) record calculus. We are going to work
through a scenario, where a class @heavy_point@ is constructed by
inheritance from three different concrete subclasses of
@abstract_point@. The first two concrete points will be shared in the
resulting heavy point, because we leave open the recursive knot. The
third concrete point does not participate in the open recursion; so it
is not shared. See Fig.~\ref{F:heavy} for an overview.

The object template for heavy points starts as follows:

\begin{code}
 heavy_point x_init color self =
  do
     super1 <- concrete_point1 x_init self
     super2 <- concrete_point2 x_init self
     super3 <- mfix (concrete_point3 x_init)
     ... -- to be continued
\end{code}

\noindent
That is, we bind all ancestor objects for subsequent reference. We
pass @self@ to the first two points, which participate in open
recursion, but we fix the third point in place. (Hence, we also
exercise object composition.) A heavy point carries @print@ and
@moveX@ methods that delegate corresponding messages to all three
points:

\begin{code}
     ... -- continued from above
     let myprint = do
                      putStr "super1: "; (super1 # print)
                      putStr "super2: "; (super2 # print)
                      putStr "super3: "; (super3 # print)
     let mymove  = ( \d -> do
                              super1 # moveX $ d
                              super2 # moveX $ d
                              super3 # moveX $ d )
     return 
       $    print  .=. myprint
      .*.   moveX  .=. mymove
      .*.   emptyRecord
     ... -- to be continued
\end{code}

\noindent
The three points, with all their many fields and methods, contribute
to the heavy point by means of left-biased union on records, which is
denoted by ``@.<++.@'' below:

\begin{code}
     ... -- continued from above
      .<++. super1
      .<++. super2
      .<++. super3
\end{code}

Here is a demo:

\begin{code}
 myDiamondOOP =
  do 
     p <- mfix (heavy_point 42 "blue")
     p # print -- All points still agree!
     p # moveX $ 2
     p # print -- The third point lacks behind!
 ghci> myDiamondOOP
 super1: 42
 super2: 42
 super3: 42
 super1: 46
 super2: 46
 super3: 44
\end{code}



%%%%%%%%%%%%%%%%%%%%%%%%%%%%%%%%%%%%%%%%%%%%%%%%%%%%%%%%%%%%%%%%%%%%%%%%%%%%%
%%%%%%%%%%%%%%%%%%%%%%%%%%%%%%%%%%%%%%%%%%%%%%%%%%%%%%%%%%%%%%%%%%%%%%%%%%%%%
%%%%%%%%%%%%%%%%%%%%%%%%%%%%%%%%%%%%%%%%%%%%%%%%%%%%%%%%%%%%%%%%%%%%%%%%%%%%%
 
\medskip

\subsection{Width and depth subtyping}
\label{A:deep}

Suppose you have a class @cube@ and a subclass @cuboid@, which
overrides one of @cube@'s methods by a version with a co-variant
return type (as in Java 5, for example). Substitutability of cubes by
cuboids does not require any casting effort or otherwise. However, we
can coerce a cuboid to become \emph{exactly} of type cube through a
narrowing operation that involves depth subtyping:

\begin{code}
 testDeep = do
   (cuboid::cuboid) <- mfix (class_cuboid
                         (10::Int) (20::Int) (30::Int))
   cube <- mfix (class_cube (40::Int))
   let cuboids = [cuboid, deep'narrow cube]
   putStrLn "Volumes of cuboids"
   mapM_ (\cb -> handle_cuboid cb >>= print) cuboids
\end{code}

The operation @deep'narrow@ must essentially descent into records and
postfix all ``method returns'' by a narrow operation on the results.
There is no magic about the definition of deep subtyping. It is
essentially just another record operation that is driven by the
structure of method types; see the source distribution~\cite{OOHaskell}.



%%%%%%%%%%%%%%%%%%%%%%%%%%%%%%%%%%%%%%%%%%%%%%%%%%%%%%%%%%%%%%%%%%%%%%%%%%%%%
%%%%%%%%%%%%%%%%%%%%%%%%%%%%%%%%%%%%%%%%%%%%%%%%%%%%%%%%%%%%%%%%%%%%%%%%%%%%%
%%%%%%%%%%%%%%%%%%%%%%%%%%%%%%%%%%%%%%%%%%%%%%%%%%%%%%%%%%%%%%%%%%%%%%%%%%%%%


\begin{comment}
\subsection{No equi-recursive types}

Actually, the value recursion IS danagerous, see Remy's Section 3.7. Our
saving grace is that in a call-by-name language fixed-point is always
safe (i.e., it loops). This may be considered just as bad as a
run-time error, but it doesn't force an undefined behavior and
security compromises.

\begin{code}
printable_point1 x_init s =
   do
      x <- newIORef x_init
      s # print
      returnIO
        $  varX  .=. x
       .*. getX  .=. readIORef x
       .*. moveX .=. (\d -> modifyIORef x ((+) d))
       .*. print .=. ((s # getX ) >>= Prelude.print)
       .*. emptyRecord

-- Note that 'mfix' plays the role of 'new' in the OCaml code...
mySelfishOOP1 =
   do
      p <- mfix (printable_point1 7)
      p # moveX $ 2
      p # print
\end{code}
%%% $

When we try mySelfishOOP1, we get |Exception: <<Loop>>|.



It is absolutely not obvious that an object system based on record
calculus can avoid equi-recursive types (whose use is known to
challenge type inference and to complicate a type system
considerably). For instance, Remy's ML-ART calls for recursive types:
``Since objects must be able to send messages to themselves, either
the objects or the functions that send messages to objects have
recursive types''~\cite{ML-ART}.


\ralf{To be honest, I do *not* understand why Remy is right.}

\ralf{Sorry, I also have to give up on the following bits.
I just don't get the fully story.
Let's sort it out on the phone.}

Hence, does OOHaskell circumvent equi-recursive types? We saw that our
class constructor takes @self@ as an argument, and returns the record,
essentially @self'@, as the result. Our encoding (for non-abstract
classes) requires that @self@ and @self'@ have the same slots~---~but
they do not have to be of the same type. This is how we avoid
recursive types. Later on, when we instantiate the class into an
object, we will apply @mfix@ operator -- that is, we will use value
recursion rather than the type recursion. This operation
% mfix, HasField constraint that guarantee mfix is applied to
% something of a record type, the absence of recursive types
% See ML-ART, end of Section 3.1
guarantees safety~\cite{ML-ART} that is, no method accesses fields
that are not filled in yet.

Of course our technique requires that no method type involves the type
of the self argument (\emph{not} counting the implicit self). This is
indeed the case in most OO languages, where we must declare the type
of the argument and the result of a method, and that type must be a
concrete type rather than just `self'.\footnote{\small Functional
objects obviously break this rule. We have found a way to deal with
functional objects while avoiding recursive and existential types, but
this topic is outside of the scope of the present paper.}

\ralf{Perhaps this should go elsewhere}

There is a related issue: the tension between nominal and structural
types. Our record types and therefore our classes provide structural
subtype polymorphism~---~just as in OCaml. Many other OO languages
prefer nominal classes. In OOHaskell, it is trivial to make nominal
distinctions by wrapping structural record types in Haskell
@newtype@s.  One can easily derive a definition for @(#)@ and other
record operations if necessary so that we can operate on nominal types
with the same easy as on structural types. Not surprisingly, we
\emph{must} introduce nominal types in case we deal with recursive
class structures. A type system \emph{with} equi-recursive types would
be slightly more convenient in this respect, while it was injure type
inference at the same time. The source distribution demonstrates the
nominal approach~\cite{OOHaskell}.

\end{comment}

\medskip

\section{Related work}
\label{S:related}

The literature on object representation and encoding is quite
extensive, e.g., ~\cite{Cardelli-on-understanding,
Poll97,AC96,Ohori95,PT94,BM92}.  Often objects are primitive
constructs in the language.  Most often discussed are pure functional
objects. Most often the type systems of object models are variants of
system $F_{\leq}$ (polymorphic lambda-calculus plus subtyping).

\medskip

\subsection{ML-ART}

We identify the {ML-ART} paper \cite{ML-ART} (see also
\cite{RV97}) as the closest to us~---~in motivation and spirit (but
not in the technical approach). We share with Didier R{\'e}my the goal
of discovering the small set of language features that make OO
programming possible. We aim not at introducing objects but at being
able to implement objects~---~as a \emph{library} feature. Therefore,
several OO styles can be implemented, for different classes of users
and classes of problems. One does not need to learn any new language
and can discover OOP progressively. Both {ML-ART} and OOHaskell base
their object systems on polymorphic records (only we use records of
closures, in this paper). Both OOHaskell and{ML-ART} deal with
mutable objects.

What fundamentally sets us apart from {ML-ART} is the different source
language: Haskell. In Haskell, we can implement polymorphic records
natively rather than via an extension. We can avoid row variables and
their related complexities. Our records permit introspection and thus
let us \emph{implement} subtyping. Unlike {ML-ART}, OOHaskell can
compute the most common type of two record types, without any explicit
coercions or user annotations.  Constructing objects via
value recursion is one of three major ways of implementing OOP (the
other two are type recursion and existential abstraction
\cite{PT94}). The value recursion seems simpler but is generally
unsafe, because of the possibility of accessing a slot before it has
been filled in. But in a call-by-name or similar language the
fixpoints are always safe, as mentioned already in \cite{ML-ART}.
 
Our current implementation has strong similarities with
prototype-based systems (such as Self~\cite{Self}) in that mutable
fields and method `pointers' are a part of the same record. This does
not have to be the case~---~and in fact, in our current efforts on
pure-functional objects we separate the two tables (in the manner
similar to object realisations in C++ or Java).



\subsection{Haskell extensions or variations}

There were attempts to bring OO to Haskell by a language extension. An
early attempt is Haskell++~\cite{HS95} by Hughes and Sparud. The
authors motivated their extension by the perception that Haskell lacks
the form of incremental reuse that is offered by inheritance in
object-oriented languages. Our approach uses common extensions of the
Hindley-Milner type system to provide the key OO notions.  So in a
way, Haskell's fitness for OOP just had to be discovered, which is the
contribution of this paper. Nordlander has delivered a comprehensive
OOP variation on
Haskell~---~O`Haskell~\cite{Nordlander98,Nordlander02}, which extends
Haskell with reactive objects and subtyping. The subtyping part is a
formidable extension. The reactive object part combines stateful
objects and concurrent execution, again a major extension. Our
development shows that no extension of Haskell is necessary for
stateful objects with a faithful object-oriented type system. Finally,
there is Mondrian~---~a NET-able variation on Haskell. In the original
paper on the design and implementation of Mondrian~\cite{MC97}, Meijer
and Claessen write: ``The design of a type system that deals with
subtyping, higher-order functions, and objects is a formidable
challenge ...''. Rather than designing a very complicated language,
the overall principle underlying Mondrian was to obtain a simple
Haskell dialect with an object-oriented flavor. To this end, algebraic
datatypes and type classes were combined into a simple object-oriented
type system with no real subtyping, with completely covariant
type-checking. In Mondrian, runtime errors of the kind ``message not
understood'' are considered a problem akin to partial functions with
non-exhaustive case discriminations. We raise the bar by providing
proper subtyping (``all message will be understood'') and other OOP
concepts in Haskell without extending the Haskell type system.



\subsection{Other OO encodings for Haskell}

The exercise of encoding some tenets of OO in Haskell seems to be a
favourite pastime of Haskell aficionados. The most pressing motivation
for such efforts has been the goal to import foreign libraries or
components into Haskell~\cite{FLMPJ99,SPJ01,PC03}. This problem domain
makes simplifying assumptions when compared to actual OO program
development in Haskell:

\begin{itemize}\noskip
\item Object state does not reside in Haskell data.
\item There are only (opaque) object ids referring to the foreign site.
\item (I.e., state is solely accessed through methods (``properties'').
\item Haskell methods are (often generated) stubs for foreign code.
\item As a result, such OO styles just deal with interfaces.
\item Also, no actual (sub)classes are written by the programmer.
\end{itemize}

One approach is to use phantom types for recording inheritance
relationships \cite{FLMPJ99}. One represents each interface by an
(empty) datatype with a type parameter for extension. After due
consideration, it turns out that this approach is a restricted version
of what Burton called ``type extension through polymorphism'': even
records can be made extensible through the provision of a polymorphic
dummy field~\cite{Burton90}. Once we do not maintain Haskell data for
objects, there is no need to maintain a record type, but the extension
point is a left over, and it becomes a phantom.  We ``re-generalize''
the phantom approach in the appendix. Its remaining problems are: lack
of support for multiple inheritance, lack of proper encapsulation
(which could be fixed at the expense of self-reference problems), code
bloat for accessors in subclasses, a closed-world assumption on base
classes, and the use of existentials for coercion to common base types
(cf.\ Sec.~\ref{S:ex} for the implied problems).

Another approach is to set up a Haskell type class to represent the
subtyping relationship among interfaces~\cite{SPJ01,PC03} while each
interface is modelled as a dedicated (empty) Haskell type. We enhance
this approach by state in the appendix. This second approach fixes
some of the aforementioned problems: multiple inheritance is possible;
there is no closed-world assumption for base classes, but the other
problems remain. There is an additional problem with code bloat
related to the expression of transitive subtyping relationships.

We do not further discuss more untyped and less extensible variations
where OO subclasses do not amount to distinguished Haskell types,
where they are modelled as constructors of a ``base-class
type''. There also exist very restrictive variations where only
abstract-to-concrete inheritance relationships are allowed. As a final
point, the published Haskell reference solution for the Shapes
benchmark
\url{http://www.angelfire.com/tx4/cus/shapes/} is a simple-to-understand
code that does not attempt to maximise reuse among data declarations
and accessors. It also uses existentials for handling subtyping.

\medskip
 
\section{Concluding remarks}
\label{S:concl}

We have described an OO system for Haskell that supports stateful
objects, inheritance and subtype polymorphism. We have demonstrated
that our encoding is very close to the textbook OO code (usually given
in C++ or Java tutorials), with pleasant deviations. The
re-implementation of examples from the OCaml object tutorial
demonstrated the faithfulness of our realization of objects. (We have
opted for OCaml because it is a leading object system in a functional
setting.) We have implemented parameterized classes, constructor
methods, abstract classes, pure virtual methods, single and multiple
inheritance with flexible rules of sharing or separation of
superclasses. Some major byproducts are these: extensive type
inference, first-class classes, implicit polymorphism of classes, an
option of implicit coercion of objects to their most common supertype
without any type annotations.

We have clarified the relation between subtyping and type-class based
polymorphisms: the latter can encode the former. We have implemented
OOHaskell with the existing Haskell implementation (GHC), requiring no
extra extensions beyond the commonly implemented ones: multi-parameter
type classes with functional dependencies. It is quite pleasant that
the existing OOHaskell code does not seem to be a burden to
write~---~even in the absence of any syntactic sugar.

We have implemented OO as a library feature -- based on the
polymorphic records with introspection and subtyping provided by the
\HList\ library.

There are some further idioms that complement OOHaskell as a faithful
OOP system, as substantiated by the paper's source distribution. For
instance, we need to take measures to do instantiation checking for
classes at declaration time. Otherwise some type errors would go
unnoticed until the first attempt to program an instantiation. We can
also model various forms of private and protected methods and other
access modifiers. We can clone objects, we can let methods return
`self', we can operate in the ST monad rather than the IO
monad. OOHaskell allows for the implementation often desirable
covariant methods~\cite{SG04,catcall} and static guarantees for their
soundness. OOHaskell also supports some forms of depth
subtyping~\cite{Poll97}. The extent and limitations of our handling of
depth subtyping as well as co-/contra-variance schemes remains a
subject for future work.

\begin{comment}
At present, error messages belie the
complexity, and this is the topic of future research (and so it is for
C++, where error messages in template meta-programs may span several
hundred lines and be humanly incomprehensible).
\end{comment}

Further topics for future work are the following. Simple syntactic
sugar would make OOP more convenient in Haskell, in particular, the
inferred types can really benefit from prettier-printing.  Extra
effort is needed to provide OOP-like error messages.  This is a
sticking issue that requires major effort, but there is a line of
research being carried out by Sulzmann and others~\cite{SSW04}.

An interesting advanced topic is reflective programming. A simple form
of reflection is readily provided in terms of the type-level encoding
of records. One can iterate over records and their components in a
generic fashion. Other forms of reflection, such as iteration over the
object pool, as needed for dynamic aspect-oriented programming,
requires further effort. Another challenge is to capture reusable
solutions for design problems (as part of design patterns) in Haskell.

An important issue is the efficiency of the encoding.  For example,
although record extension is constant (run-)time, the field/method
lookup is linear search. Clearly a more efficient encoding is
possible: one representation of the labels in the \HList\ paper
permits the total order among the labels types, which in turn, permits
construction of efficient search trees. In this first OOHaskell paper,
we chose the conceptual clarity over such optimizations.  A
non-trivial case study is required to demonstrate the scalability of
the approach. The mere compilation time of OOHaskell programs and
their runtime efficiency is challenged by the huge dictionaries that
are implied by our type-class-based approach. It is quite likely that,
as one reviewer has observed, large-scale efficient
\HList/OOHaskell style of programming will need dedicated compiler
optimizations. ``But showing the usefulness of such an encoding is the
first step towards encouraging compiler writers to do so!''


    

%%%%%%%%%%%%%%%%%%%%%%%%%%%%%%%%%%%%%%%%%%%%%%%%%%%%%%%%%%%%%%%%%%%%%%%%%%%%%
%%%%%%%%%%%%%%%%%%%%%%%%%%%%%%%%%%%%%%%%%%%%%%%%%%%%%%%%%%%%%%%%%%%%%%%%%%%%%
%%%%%%%%%%%%%%%%%%%%%%%%%%%%%%%%%%%%%%%%%%%%%%%%%%%%%%%%%%%%%%%%%%%%%%%%%%%%%



{\small 

\subsubsection*{Acknowledgements}
 
We thank Keean Schupke for his major contributions to HList and
OOHaskell. We thank Chung-chieh Shan for very helpful discussions.  We
also gratefully acknowledge feedback from Robin Green, Bryn Keller,
Chris Rathman and several other participants in mailing list or email
discussions. The second author presented this work at an earlier stage
at the WG2.8 meeting (Functional Programming) in November 2004 at West
Point. We are grateful for feedback received at this meeting.

}



%%%%%%%%%%%%%%%%%%%%%%%%%%%%%%%%%%%%%%%%%%%%%%%%%%%%%%%%%%%%%%%%%%%%%%%%%%%%%
%%%%%%%%%%%%%%%%%%%%%%%%%%%%%%%%%%%%%%%%%%%%%%%%%%%%%%%%%%%%%%%%%%%%%%%%%%%%%
%%%%%%%%%%%%%%%%%%%%%%%%%%%%%%%%%%%%%%%%%%%%%%%%%%%%%%%%%%%%%%%%%%%%%%%%%%%%%

{\small

\bibliographystyle{abbrv}
\bibliography{paper}

}


%%%%%%%%%%%%%%%%%%%%%%%%%%%%%%%%%%%%%%%%%%%%%%%%%%%%%%%%%%%%%%%%%%%%%%%%%%%%%
%%%%%%%%%%%%%%%%%%%%%%%%%%%%%%%%%%%%%%%%%%%%%%%%%%%%%%%%%%%%%%%%%%%%%%%%%%%%%
%%%%%%%%%%%%%%%%%%%%%%%%%%%%%%%%%%%%%%%%%%%%%%%%%%%%%%%%%%%%%%%%%%%%%%%%%%%%%



\renewcommand{\mysize}{\footnotesize}

\appendix

\bigskip

{\large Two ``poor mens''' approaches follow.}




%%%%%%%%%%%%%%%%%%%%%%%%%%%%%%%%%%%%%%%%%%%%%%%%%%%%%%%%%%%%%%%%%%%%%%%%%%%%%
%%%%%%%%%%%%%%%%%%%%%%%%%%%%%%%%%%%%%%%%%%%%%%%%%%%%%%%%%%%%%%%%%%%%%%%%%%%%%
%%%%%%%%%%%%%%%%%%%%%%%%%%%%%%%%%%%%%%%%%%%%%%%%%%%%%%%%%%%%%%%%%%%%%%%%%%%%%



\appendix

\section{Type-preserving record update}
\label{A:hTPupdateAtLabel}

%\ralf{Some line comments need to be added.}

\begin{code}
 infixr 1 .<.
 (l,v) .<. r = hTPupdateAtLabel l v r
\end{code}

\begin{code}
 hTPupdateAtLabel l (v::v) r = hUpdateAtLabel l v r
  where
   (_::v) = hLookupByLabel l r
\end{code}
 
\begin{code}
 hUpdateAtLabel l v (Record r) = Record (hZip ls vs')
  where
   (ls,vs) = hUnzip r
   n       = hFind l ls
   vs'     = hUpdateAtHNat n v vs
\end{code}

\begin{code}
 hLookupByLabel l (Record r) = v
  where
   (ls,vs) = hUnzip r
   n       = hFind l ls
   v       = hLookupByHNat n vs
\end{code}



%%%%%%%%%%%%%%%%%%%%%%%%%%%%%%%%%%%%%%%%%%%%%%%%%%%%%%%%%%%%%%%%%%%%%%%%%%%%%
%%%%%%%%%%%%%%%%%%%%%%%%%%%%%%%%%%%%%%%%%%%%%%%%%%%%%%%%%%%%%%%%%%%%%%%%%%%%%
%%%%%%%%%%%%%%%%%%%%%%%%%%%%%%%%%%%%%%%%%%%%%%%%%%%%%%%%%%%%%%%%%%%%%%%%%%%%%



\section{Type-driven narrowing for records}
\label{A:narrow}

%\ralf{Some line comments need to be added.}

\begin{code}
 class  Narrow a b
  where narrow :: Record a -> Record b
\end{code}

\begin{code}
 instance Narrow a HNil
  where   narrow _ = emptyRecord
\end{code}

\begin{code}
 instance ( Narrow r r', HExtract r l v
          ) => Narrow r (HCons (l,v) r')
  where
    narrow (Record r) = Record (HCons (l,v) r')
      where
        (Record r')    = narrow (Record r)
        ((l,v)::(l,v)) = hExtract r
\end{code}

\begin{code}
 class  HExtract r l v
  where hExtract :: r -> (l,v)
\end{code}

\begin{code}
 instance ( TypeEq l l1 b, HExtractBool b (HCons (l1,v1) r) l v
          ) => HExtract (HCons (l1,v1) r) l v
  where hExtract = hExtractBool (undefined::b)
\end{code}

\begin{code}
 class HBool b => HExtractBool b r l v
  where hExtractBool :: b -> r -> (l,v)
\end{code}

\begin{code}
 instance TypeCast v1 v => HExtractBool HTrue (HCons (l,v1) r) l v
  where hExtractBool _ (HCons (l,v) _) = (l,typeCast v)
\end{code}

\begin{code}
 instance HExtract r l v => HExtractBool HFalse (HCons (l1,v1) r) l v
  where hExtractBool _ (HCons _ r) = hExtract r
\end{code}



%%%%%%%%%%%%%%%%%%%%%%%%%%%%%%%%%%%%%%%%%%%%%%%%%%%%%%%%%%%%%%%%%%%%%%%%%%%%%
%%%%%%%%%%%%%%%%%%%%%%%%%%%%%%%%%%%%%%%%%%%%%%%%%%%%%%%%%%%%%%%%%%%%%%%%%%%%%
%%%%%%%%%%%%%%%%%%%%%%%%%%%%%%%%%%%%%%%%%%%%%%%%%%%%%%%%%%%%%%%%%%%%%%%%%%%%%



\section{Left-biased union on records}
\label{A:hLeftUnion}

%\ralf{Some line comments need to be added.}

\begin{code}
 class  HLeftUnion r r' r'' | r r' -> r''
  where hLeftUnion :: r -> r' -> r''
\end{code}

\begin{code}
 instance HLeftUnion r (Record HNil) r
  where   hLeftUnion r _ = r
\end{code}

\begin{code}
 instance ( HZip ls vs r
          , HMember l ls b
          , HLeftUnionBool b r l v r'''
          , HLeftUnion (Record r''') (Record r') r''
          )
            => HLeftUnion (Record r) (Record (HCons (l,v) r')) r''
  where
   hLeftUnion (Record r) (Record (HCons (l,v) r')) = r''
    where
     (ls,vs) = hUnzip r
     b       = hMember l ls
     r'''    = hLeftUnionBool b r l v
     r''     = hLeftUnion (Record r''') (Record r')
\end{code}

\begin{code}
 class  HLeftUnionBool b r l v r' | b r l v -> r'
  where hLeftUnionBool :: b -> r -> l -> v -> r'
\end{code}

\begin{code}
 instance HLeftUnionBool HTrue r l v r
    where hLeftUnionBool _ r _ _ = r
\end{code}

\begin{code}
 instance HLeftUnionBool HFalse r l v (HCons (l,v) r)
    where hLeftUnionBool _ r l v = HCons (l,v) r
\end{code}



%%%%%%%%%%%%%%%%%%%%%%%%%%%%%%%%%%%%%%%%%%%%%%%%%%%%%%%%%%%%%%%%%%%%%%%%%%%%%
%%%%%%%%%%%%%%%%%%%%%%%%%%%%%%%%%%%%%%%%%%%%%%%%%%%%%%%%%%%%%%%%%%%%%%%%%%%%%
%%%%%%%%%%%%%%%%%%%%%%%%%%%%%%%%%%%%%%%%%%%%%%%%%%%%%%%%%%%%%%%%%%%%%%%%%%%%%



\section{Illustration of deep subtyping}
\label{A:deep}

Suppose you have a class @cube@ and a subclass @cuboid@, which
overrides one of @cube@'s methods by a version with a co-variant
return type (as in Java 5, for example). Substitutability of cubes by
cuboids does not require specific efforts. However, we can even coere
a cuboid to a cube using deep subtyping, that is, a ``deep'' variation
on App.~\ref{A:narrow}.

\medskip

\noindent
Here is a test case illustrating deep narrow:

\begin{code}
 testDeep
   = do
	(cuboid::cuboid) <- mfix (class_cuboid (10::Int) (20::Int) (30::Int))
	cube <- mfix (class_cube (40::Int))
	let cuboids = [cuboid, deep'narrow cube]
	putStrLn "Volumes of cuboids"
        mapM_ (\cb -> handle_cuboid cb >>= print) cuboids
\end{code}

\noindent


\noindent
See the code distribution for the paper for the specification of
@deep'narrow@. This operation must essentially descent into records
and postfix all ``method returns'' by a narrow operation on the
results.



%%%%%%%%%%%%%%%%%%%%%%%%%%%%%%%%%%%%%%%%%%%%%%%%%%%%%%%%%%%%%%%%%%%%%%%%%%%%%
%%%%%%%%%%%%%%%%%%%%%%%%%%%%%%%%%%%%%%%%%%%%%%%%%%%%%%%%%%%%%%%%%%%%%%%%%%%%%
%%%%%%%%%%%%%%%%%%%%%%%%%%%%%%%%%%%%%%%%%%%%%%%%%%%%%%%%%%%%%%%%%%%%%%%%%%%%%


\end{document}
