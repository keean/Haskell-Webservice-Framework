\medskip

\section{An example~---~the `shapes' benchmark}
\label{S:shapes}

We face an abstract class `shapes' and two concrete subclasses
`rectangle' and `circle'. Shapes maintain coordinates and other data
as state. Shapes can be moved around and drawn. The exercise shall be
to place objects of \emph{different} shapes in a collection and to
iterate over them as to draw the shapes. It turns out that this is a
crisp benchmark problem for subtyping polymorphism, virtual methods,
and class inheritance.\footnote{\small This benchmark problem has been
designed by Jim Weirich and deeply explored by him and Chris
Rathman. See the multi-lingual collection `OO Example Code' by Jim
Weirich at \url{http://onestepback.org/articles/poly/}; see also an
even heavier collection `OO Shape Examples' by Chris Rathman at
\url{http://www.angelfire.com/tx4/cus/shapes/}.}

\smallskip

In C++, the base class for shapes is defined as follows:

\begin{Verbatim}[fontsize=\small,commandchars=\|\!\@]
 class Shape {
 public:
  Shape(int newx, int newy) {
    x = newx; y = newy;
  }
\end{Verbatim}

\begin{Verbatim}[fontsize=\small,commandchars=\|\!\@]
  // |cmt!Accessors@
  int getX() { return x; }
  int getY() { return y; }
  void setX(int newx) { x = newx; }
  void setY(int newy) { y = newy; }
\end{Verbatim}

\begin{Verbatim}[fontsize=\small,commandchars=\|\!\@]
  // |cmt!Move shape relatively@
  void rMoveTo(int deltax, int deltay) {
    setX(getX() + deltax);
    setY(getY() + deltay);
  }
\end{Verbatim}

\begin{Verbatim}[fontsize=\small,commandchars=\|\!\@]
  // |cmt!An abstract draw method@
  virtual void draw() = 0;
 private:
   int x;
   int y;
 }
\end{Verbatim}

The @x@, @y@ coordinates are made private, while public accessors are
provided. The @rMoveTo@ method will be inherited to the specific shape
subclasses. The @draw@ method is virtual and even abstract; hence
subclasses must implement @draw@ eventually.

For brevity, we elide the derivation of specific subclasses:

\begin{code}
 class Rectangle : public Shape {
 public:
  Rectangle(int newx, int newy, int newwidth, int newheight)
           : Shape(newx, newy) { ... }
  ...
 }
\end{code}

\begin{code}
 class Circle : public Shape {
  Circle(int newx, int newy, int newradius)
        : Shape(newx, newy) { ... }
  ...
 }
\end{code}

We iterate over some shapes as follows:

\begin{code}
 Shape *scribble[2];
 scribble[0] = new Rectangle(10, 20, 5, 6);
 scribble[1] = new Circle(15, 25, 8);
 for (int i = 0; i < 2; i++) {
   scribble[i]->draw();
   scribble[i]->rMoveTo(100, 100);
   scribble[i]->draw();
 }
\end{code}

We note that this loop exercises subtyping polymorphism; the actual
@draw@ method, that is called, differs per element in the array. The
program run produces this output~---~subject to suitable
implementations of the (elided) @draw@ methods:

\begin{code}
 Drawing a Rectangle at:(10,20), width 5, height 6
 Drawing a Rectangle at:(110,120), width 5, height 6
 Drawing a Circle at:(15,25), radius 8
 Drawing a Circle at:(115,125), radius 8
\end{code}

We will now show an OOHaskell encoding, which happens to pleasantly
mimic the C++ encoding, while any remaining deviations are
appreciated. Most notably, we are going to leverage type inference.
We are not going to define \emph{any} type~---~nevertheless the code
will be fully statically typed, of course.

\smallskip

Here is the OOHaskell rendering of the shape class:

\begin{Verbatim}[fontsize=\small,commandchars=\|\{\}]
 -- |cmt{Object generator for shapes}
 shape newx newy self
   = do
	-- |cmt{New references for private state}
        x <- newIORef newx
        y <- newIORef newy
\end{Verbatim}

\begin{Verbatim}[fontsize=\small,commandchars=\|\{\}]
	-- |cmt{Return object as record of methods}
        return $  getX     .=. readIORef x
              .*. getY     .=. readIORef y
              .*. setX     .=. (\newx -> writeIORef x newx)
              .*. setY     .=. (\newy -> writeIORef y newy)
              .*. rMoveTo  .=. (\deltax deltay -> 
                            do
                              x <- self # getX
                              y <- self # getY
                              (self # setX) (x + deltax)
                              (self # setY) (y + deltay) )
         .*. emptyRecord
\end{Verbatim}

Classes become \emph{functions that take constructor arguments plus a
self reference and that return a computation whose result is the new
object~---~a record of methods}. We can refer to other methods of the
same object by invoking @self@; cf.\ @self@~@#@~@getX@ and others.

We use the extensible records of the HList library~\cite{HLIST-HW04},
hence:
%
\begin{itemize}\noskip
\item @emptyRecord@ denotes what the name promises.
\item @(.*.)@ is right-associative record extension.
\item @(.=.)@ is the component constructor: \w{label} @.=.@ \w{value}.
\item Labels are defined according to a trivial scheme; see later.
\end{itemize}

We note that the abstract @draw@ method is not mentioned in the
OOHaskell code because we do not dare declaring its type. (We will
later show how to add a static type constraint so that we can rule out
instantiation of @shape@ and any subclass not defining @draw@.)

As in the C++ case, we only sketch derivation of the sublasses:

\begin{code}
 rectangle newx newy newwidth newheight self
  = do  super <- shape newx newy self
        ...
        return ... .*. super
\end{code}

\begin{code}
 circle newx newy newradius self
  = do  super <- shape newx newy self
        ...
        return ... .*. super
\end{code}

This snippet illustrates the essence of inheritance in OOHaskell.
Object generation for the ancestor class is made part of the monadic
sequence; @self@ is passed to the ancestor as well.

Here is the OOHaskell rendering of a loop over different shapes:

\begin{Verbatim}[fontsize=\small,commandchars=\|\{\}]
 -- |cmt{Object construction and invocation as a monadic sequence}
 myOOP = do
          -- |cmt{Take fixpoint of object generators}
          s1 <- mfix (rectangle (10::Int) (20::Int) 5 6)
          s2 <- mfix (circle (15::Int) 25 8)
          -- |cmt{Create a list of different shapes}
          let scribble = lubCast (HCons s1 (HCons s2 HNil))
          -- |cmt{Loop over list with normal monadic map}
          mapM_ (\shape -> do
                            shape # draw
                            (shape # rMoveTo) 100 100
                            shape # draw)
                scribble
\end{Verbatim}

The use of @mfix@ (an analogue of @new@) reflects that object
generators take self and construct (part of) it. (Open recursion
enables inheritance.) We cannot \emph{directly} place rectangles and
circles in a normal Haskell list~---~as their types differ. So instead
we detour through a heterogeneous list (cf.\ @HCons@ and @HNil@), which
however is casted immediately to a normal, homogeneous list by means
of a least-upper bound operation; cf.\ @lubCast@. Incidentally, if the
`intersection' of the type of the objects @s1@ and @s2@ does not
include @draw@, we get a static type error which literally says so. As
a result, the original for-loop can be carried out in the native
Haskell way: a normal (monadic) list map over a normal list of shapes.
