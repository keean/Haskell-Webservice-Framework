\documentclass{article}

\usepackage{times}
\usepackage{code}
\usepackage{fancyvrb}
\usepackage{comment}
\usepackage{boxedminipage}


% Macros for notes to each other in the text
\newcommand{\keean}[1]{{\it [Keean says: #1]}}
\newcommand{\oleg}[1]{{\it [Oleg says: #1]}}
\newcommand{\ralf}[1]{{\it [Ralf says: #1]}}

\begin{document}
 
\title{Haskell with dependent types}
 
%\author{}

\date{Draft; \today}

\maketitle

\begin{abstract}

We describe a simple extension of Haskell, DHaskell, to support
dependently typed programming. DHaskell allows the programmer to avoid
the tedious coding that results from using Haskell's type-class system
for faked dependently typed programming. Compared to other dependently
typed programming languages, DHaskell provides a straigthforward
integration with normal Haskell programming and type-level
computations, at the cost of some limitations. We lay out the schemes
that are used in a preprocessor-based implementation of DHaskell.
\end{abstract}

% Switch on the 'at' sign
\makeatactive
 


\section{Introduction}



\section{The obligatory \texttt{append} example}

We illustrate dependent typing in DHaskell by means of a trivial
example that we adopt from Xi's seminal work on DML. We are going to
reconstruct the following @append@ function:

\begin{code}
 fun ('a) append xs ys =
  case xs of [] => ys | x :: xs => x :: append xs ys
 withtype {m:nat,n:nat} <m> =>
         'a list(m) -> 'a list(n) -> 'a list(m+n)
\end{code}

This version of @append@ operates 

We develop this
example step by step starting from a version that involves type-level
computations. The example is about the static guarantee that the
length of an appended list is always the sum of the length of its
operands.

\medskip

\noindent
This is the normal @List@ type:

\begin{code}
 data List a = Nil | Cons a (List a)
\end{code}

\noindent
This is the normal @append@ function:

\begin{code}
 append :: List a -> List a -> List a
 append Nil x = x
 append (Cons h t) x = Cons h (append t x)
\end{code}

\noindent
Now suppose that we would like to know for sure that:
\[\begin{array}{l}
\forall\,\alpha :: *.\ \forall\,l_1, l_2 :: \texttt{List}\ \alpha.\\ 
\texttt{length}\ (\texttt{append}\ l_1\ l_2) == 
\texttt{add}\ (\texttt{length}\ l_1)\ (\texttt{length}\ l_2)
\end{array}\]

\noindent
(We assume that all involved functions terminate.)

\medskip

\noindent
It is trivial to prove that the abovementioned property holds. A paper
proof of the property would be by induction on the list structure for
@append@'s first operand. Here we assume the following definitions of
the referenced functions:

\begin{code}
 -- Naturals
 data Nat = Zero | Succ Nat
\end{code}

\begin{code}
 -- Addition of naturals
 add :: Nat -> Nat -> Nat
 add Zero n = n
 add (Succ n) n' = Succ (add n n')
\end{code}

\begin{code}
 -- Length of lists
 length :: List a -> Nat
 length Nil = Zero
 length (Cons h t) = Succ (length t)
\end{code}


\paragraph*{The induction proof}

\noindent
\begin{itemize}
%
\item Base case: $l_1 == \texttt{Nil}$ 
\[\begin{array}{p{1in}ll}
& & \texttt{length}\ (\texttt{append}\ \texttt{Nil}\ l_2)\\
\mbox{(1st eq.\ \texttt{append})}
& ==
& \texttt{length}\ l_2
\\
\mbox{}\\
& & \texttt{add}\ (\texttt{length}\ \texttt{Nil})\ (\texttt{length}\ l_2)\\
\mbox{(1st eq.\ \texttt{length})}
& ==
& \texttt{add}\ \texttt{Zero}\ (\texttt{length}\ l_2)
\\
\mbox{(1st eq.\ \texttt{add})}
& ==
& \texttt{length}\ l_2
\end{array}\]
%
%
%
\item Induction step: $l_1$ is of the form $\texttt{Cons}\ a\ b$
\[\begin{array}{p{1in}ll}
& & \texttt{length}\ (\texttt{append}\ (\texttt{Cons}\ a\ b)\ l_2)\\
\mbox{(2nd eq.\ \texttt{append})}
& ==
& \texttt{length}\ (\texttt{Cons}\ a\ (\texttt{append}\ b\ l_2))
\\
\mbox{(2nd eq.\ \texttt{length})}
& ==
& \texttt{Succ}\ (\texttt{length}\ (\texttt{append}\ b\ l_2))
\\
\mbox{}\\
& & \texttt{add}\ (\texttt{length}\ (\texttt{Cons}\ a\ b))\ (\texttt{length}\ l_2)\\
\mbox{(2nd eq.\ \texttt{length})}
& ==
& \texttt{add}\ (\texttt{Succ}\ (\texttt{length}\ b))\ (\texttt{length}\ l_2)
\\
\mbox{(2nd eq.\ \texttt{add})}
& ==
& \texttt{Succ}\ (\texttt{add}\ (\texttt{length}\ b)\ (\texttt{length}\ l_2))
\\
\mbox{(ind.\ hypoth.)}
& ==
& \texttt{Succ}\ (\texttt{length}\ (\texttt{append}\ b\ l_2))
\end{array}\]
%
\end{itemize}



%%%%%%%%%%%%%%%%%%%%%%%%%%%%%%%%%%%%%%%%%%%%%%%%%%%%%%%%%%%%%%%%%%%%%%%%%%%%%%



\begin{comment}
\begin{code}
 data List (n::Nat) a = Nil where n=0
                      | Cons a (List m a) where n=m+1
\end{code}

\begin{code}
 append ::  List a l -> List a m -> List a (l+m)
 append (Cons x xs) ys = Cons x (append xs ys)
 append Nil ys = Nil 
\end{code}
\end{comment}



%%%%%%%%%%%%%%%%%%%%%%%%%%%%%%%%%%%%%%%%%%%%%%%%%%%%%%%%%%%%%%%%%%%%%%%%%%%%%%
%%%%%%%%%%%%%%%%%%%%%%%%%%%%%%%%%%%%%%%%%%%%%%%%%%%%%%%%%%%%%%%%%%%%%%%%%%%%%%
%%%%%%%%%%%%%%%%%%%%%%%%%%%%%%%%%%%%%%%%%%%%%%%%%%%%%%%%%%%%%%%%%%%%%%%%%%%%%%



\section{Preprocessing semantics}

\section{Examples}

\section{Related work}

\section{Concluding remarks}


%\bibliographystyle{abbrv}
%\bibliography{paper}
 
\end{document}
