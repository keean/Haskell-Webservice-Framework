\documentclass{slides}

\usepackage{times}
\usepackage{comment}
\usepackage{boxedminipage}
\usepackage{alltt}
\usepackage{graphics}
\usepackage[usenames]{pstcol}
\usepackage{url}
\usepackage{fancyvrb}
\usepackage{ifthen}
\usepackage{code}


\newcommand{\pure}{\textsc{PURE}}
\newcommand{\littleskip}{\topsep6pt \parskip6pt \partopsep6pt}
\newcommand{\noskip}{\topsep0pt \parskip0pt \partopsep0pt}
\newcommand{\header}[1]{{\large\scshape \color{Red} #1} \medskip }
\newcommand{\blau}[1]{{\color{Blue} #1} \medskip }
\newenvironment{myslide}{\begin{slide}\color{Blue}\begin{boxedminipage}{1.1\hsize}\begin{boxedminipage}{1\hsize}\color{Black}
\vspace{-170\in}
}{%
\smallskip
\end{boxedminipage}
\end{boxedminipage}
\end{slide}}


\begin{document}

\pagestyle{empty}

% Switch on the 'at' sign
\makeatactive


%\begin{comment}
\begin{slide}

Press PgDn to start.

\end{slide}
%\end{comment}



%%%%%%%%%%%%%%%%%%%%%%%%%%%%%%%%%%%%%%%%%%%%%%%%%%%%%%%%%%%%%%%%%%%%%%%%%%%%%%



\begin{myslide}

\bigskip

\header{
\begin{center}
Preprocessing support\\ for type-level programming in Haskell
\end{center}
}

\vspace{200\in}

\blau{Ralf L{\"a}mmel\\
Vrije Universiteit Amsterdam\\
Centrum voor Wiskunde en Informatica}

\bigskip
Joint work with\\
Oleg Kiselyov, FNMO Center, Monterey, CA\\
Keean Schupke, Imperial College, London

\bigskip

\end{myslide}



%%%%%%%%%%%%%%%%%%%%%%%%%%%%%%%%%%%%%%%%%%%%%%%%%%%%%%%%%%%%%%%%%%%%%%%%%%%%%%



\begin{myslide}

\header{Motivation}

\blau{The HList library$^*$ for heterogeneous collections}

{\small

\littleskip\begin{itemize}
\item Heterogeneous lists (nested, right-associative products)
\item Heterogeneous arrays (indexed by type-level Nats)
\item Type-driven access operations
\item Type-indexed products and co-products
\item Extensible records and variants
\end{itemize}

Issue: Current library code uses verbose type-class code.

\bigskip

$^*$\ Come and attend the Haskell Workshop 2004; next week!!!

}

\end{myslide}



%%%%%%%%%%%%%%%%%%%%%%%%%%%%%%%%%%%%%%%%%%%%%%%%%%%%%%%%%%%%%%%%%%%%%%%%%%%%%%



\begin{myslide}

\header{Illustration of verbosity}

{\tiny

\blau{Normal append in Haskell's Prelude}

\vspace{-42\in}

\begin{code}
append :: [a] -> [a] -> [a]
append [] l = l
append (x:l) l' = x : append l l'
\end{code}

\blau{Type-level append in HList library}

\vspace{-42\in}

\begin{code}
data HNil = HNil
data HCons h t = HCons h t
\end{code}
\begin{code}
class  HAppend l l' l'' | l l' -> l''
 where hAppend :: l -> l' -> l''
\end{code}
\begin{code}
instance HAppend HNil l l where hAppend HNil l = l
\end{code}
\begin{code} 
instance HAppend l l' l''
      => HAppend (HCons x l) l' (HCons x l'')
 where   hAppend (HCons x l) l' = HCons x (hAppend l l')
\end{code}

}

\end{myslide}



%%%%%%%%%%%%%%%%%%%%%%%%%%%%%%%%%%%%%%%%%%%%%%%%%%%%%%%%%%%%%%%%%%%%%%%%%%%%%%



\begin{myslide}

\header{All verbosity disclosed}

{\tiny

\begin{code}
data HNil = HNil
data HCons h t = HCons h t

class HList l
instance HList HNil
instance HList t => HList (HCons h t)

class (HList l, HList l', HList l'')
   =>  HAppend l l' l'' | l l' -> l''
 where hAppend :: l -> l' -> l''
 
instance HList l => HAppend HNil l l
 where hAppend HNil l = l
 
instance (HList l, HAppend l l' l'')
      => HAppend (HCons x l) l' (HCons x l'')
 where   hAppend (HCons x l) l' = HCons x (hAppend l l')
\end{code}

}

\end{myslide}



%%%%%%%%%%%%%%%%%%%%%%%%%%%%%%%%%%%%%%%%%%%%%%%%%%%%%%%%%%%%%%%%%%%%%%%%%%%%%%



\begin{myslide}

\header{Plan of attack}

{\small

\littleskip
\begin{enumerate}
\item Reuse many `normal' algebraic datatypes.
\item Reuse many `normal' function definitions.
\item Design functions with type-level `contracts'.

\vspace{-42\in}
{\tiny
\begin{Verbatim}[fontseries=normal,fontsize=\tiny,commandchars=\\\{\}]
myPickyFun :: ... -> r
myPickyFun ... = ... 
 \textbf{require} ... -- contract
\end{Verbatim}
}
\item Preprocessing of \texttt{myPickyFun}
\begin{enumerate}
\item Reify types referred to in contract.
\item Reify functions used in contract.
\item \texttt{myPickyFun} becomes a constrained function.

\vspace{-42\in}
{\tiny\begin{Verbatim}
myPickyFun :: ( ... {- contract -} ... )
           => ... -> r
myPickyFun ... = ... 
\end{Verbatim}
}
\end{enumerate}
\end{enumerate}

}

\end{myslide}



%%%%%%%%%%%%%%%%%%%%%%%%%%%%%%%%%%%%%%%%%%%%%%%%%%%%%%%%%%%%%%%%%%%%%%%%%%%%%%



\begin{myslide}

\header{The \emph{equal length} sample$^*$}

\blau{A `run-time-checked' append}

{\tiny

\begin{code}
rtcAppend :: [a] -> [a] -> Maybe [a]
rtcAppend lst1 lst2
   = if length lst3 == add (length lst1) (length lst2)
        then Just lst3
        else Nothing
 where
  lst3 = append lst1 lst2
\end{code}

}

\bigskip

{\tiny

$^*$\ Let's not try HLists immediately, which are non-standard.\\
\phantom{$^*$}\ We will get there.

}

\end{myslide}



%%%%%%%%%%%%%%%%%%%%%%%%%%%%%%%%%%%%%%%%%%%%%%%%%%%%%%%%%%%%%%%%%%%%%%%%%%%%%%



\begin{myslide}

\header{The equal-length sample cont'd}

\blau{A `compile-time-checked' append}

\begin{Verbatim}[fontseries=normal,fontsize=\small,commandchars=\\\{\}]
ctcAppend :: [a] -> [a] -> [a]
ctcAppend lst1 lst2 = lst3
 where
  lst3 = append lst1 lst2
 \textbf{require}
  length lst3 == add (length lst1) (length lst2) 
\end{Verbatim}

{\small

`\textbf{require}' triggers: reification of naturals, lists and the
involved operations. The operation \texttt{ctcAppend} can be
viewed as a constrained, partially type-level operation.

}

\end{myslide}



%%%%%%%%%%%%%%%%%%%%%%%%%%%%%%%%%%%%%%%%%%%%%%%%%%%%%%%%%%%%%%%%%%%%%%%%%%%%%%



\begin{myslide}

\header{The result of preprocessing}

\blau{Type-level lists$^*$}

{\small

\begin{code}
data Nil a    = Nil
data Cons b a = Cons a (b a)
 
class List ( lst :: * -> * )
instance List Nil
instance List b => List (Cons b)
\end{code}

\bigskip

$^*$\ List functor reified.

}

\end{myslide}


%%%%%%%%%%%%%%%%%%%%%%%%%%%%%%%%%%%%%%%%%%%%%%%%%%%%%%%%%%%%%%%%%%%%%%%%%%%%%%



\begin{myslide}

\header{The result of preprocessing cont'd}

\blau{Type-level append$^*$}

{\tiny

\begin{code}
class  (List l1, List l2, List l3)
   =>  Append l1 l2 l3 | l1 l2 -> l3
 where append :: l1 a -> l2 a -> l3 a

instance List x => Append Nil x x
 where append Nil x = x

instance Append t x t'
      => Append (Cons t) x (Cons t')
 where   append (Cons h t) x = Cons h (append t x)
\end{code}

\bigskip

$^*$\ We have seen that.\\
\phantom{$^*$}\ Continue like this for naturals and all that.

}

\end{myslide}


%%%%%%%%%%%%%%%%%%%%%%%%%%%%%%%%%%%%%%%%%%%%%%%%%%%%%%%%%%%%%%%%%%%%%%%%%%%%%%



\begin{myslide}

\header{The result of preprocessing cont'd}

\blau{The `compile-time-checked' append}

{\tiny

\begin{code}
ctcAppend :: ( Append lst1 lst2 lst3
             , Length lst1 nat1 
             , Length lst2 nat2
             , Length lst3 nat3
             , Add nat1 nat2 nat12
             , Eq nat3 nat12 True
             ) => lst1 a -> lst2 a -> lst3 a
ctcAppend = append
\end{code}

}

\end{myslide}



%%%%%%%%%%%%%%%%%%%%%%%%%%%%%%%%%%%%%%%%%%%%%%%%%%%%%%%%%%%%%%%%%%%%%%%%%%%%%%



\begin{myslide}

\header{Relevant preprocessing idioms}

{\small

\noskip\begin{description}
\item[Functor] Data types are reified in their functorial form.
\item[Constructor] Each constructor gives rise to one datatype.
\item[Signature] Function signatures become class declarations.
\item[Equation] Equations become instances.
\item[Require1] Function with `require' is defined by a single equation.
\item[Require2] Condition becomes constraint of host function.
\item[Case] Reification of case expressions requires helper classes.
\item[Linear] We only use linear instances (equations are linear anyway).
\item[Eager] Type-level functions are not lazy.
\item[Close] Pretend closed type classes for non-overloaded functions.
\item[Times] Reification of classes requires duplicated parameters.
\item[Para] Reify type parameters as per constructor parameters.
\item[Cast] Reify existential castables as constructor parameters.
\end{description}

}

\end{myslide}



%%%%%%%%%%%%%%%%%%%%%%%%%%%%%%%%%%%%%%%%%%%%%%%%%%%%%%%%%%%%%%%%%%%%%%%%%%%%%%



\begin{myslide}

\header{The \emph{HList} sample}

\blau{An ambiguous look-up operation}

{\tiny

\begin{code}
-- Weakly typed heterogeneous lists
type HList    = [Castable]
data Castable = forall x. Typeable x => Castable x 
unCastable :: Typeable x => Castable -> Maybe x
unCastable (Castable x) = cast x

-- Look up all elements of a given type
hOccursMany :: Typeable a => HList -> [a]
hOccursMany [] = []
hOccursMany (h:t) = case unCastable h of
                      Just a  -> a : hOccursMany t
                      Nothing -> hOccursMany t
\end{code}

}

\end{myslide}



%%%%%%%%%%%%%%%%%%%%%%%%%%%%%%%%%%%%%%%%%%%%%%%%%%%%%%%%%%%%%%%%%%%%%%%%%%%%%%



\begin{myslide}

\header{The \emph{HList} sample cont'd}

\blau{A `run-time-checked' unambiguous look-up operation}

\medskip

\begin{Verbatim}[fontseries=normal,fontsize=\small,commandchars=\\\{\}]
rtcHOccurs :: Typeable a => HList -> Maybe a
rtcHOccurs hl = if length many == Succ Zero 
                  then Just (head many)
                  else Nothing
 where
  many = hOccursMany hl
\end{Verbatim}

\end{myslide}



%%%%%%%%%%%%%%%%%%%%%%%%%%%%%%%%%%%%%%%%%%%%%%%%%%%%%%%%%%%%%%%%%%%%%%%%%%%%%%



\begin{myslide}

\header{The \emph{HList} sample cont'd}

\blau{A `compile-time-checked' look-up operation}

\medskip

\begin{Verbatim}[fontseries=normal,fontsize=\small,commandchars=\\\{\}]
ctcHOccurs :: Typeable a => HList -> a
ctcHOccurs hl = head many
 where
  many = hOccursMany hl
 \textbf{require}
  length many == Succ Zero
\end{Verbatim}

\end{myslide}



%%%%%%%%%%%%%%%%%%%%%%%%%%%%%%%%%%%%%%%%%%%%%%%%%%%%%%%%%%%%%%%%%%%%%%%%%%%%%%



\begin{myslide}

\header{The result of preprocessing}

\blau{The `compile-time-checked' look-up operation}

{\small

\begin{code}
ctcHOccurs :: ( HOccursMany a hl lst
              , Length lst len
              , Eq len (Succ Zero) True
              , Head lst
              )
           => hl -> a
ctcHOccurs hl = head many
 where
  many = hOccursMany hl
\end{code}

}

\end{myslide}



%%%%%%%%%%%%%%%%%%%%%%%%%%%%%%%%%%%%%%%%%%%%%%%%%%%%%%%%%%%%%%%%%%%%%%%%%%%%%%



\begin{myslide}

\header{Reification of {\textrm{\texttt{hOccursMany}}}}

\blau{Difficult!}

\vspace{-42\in}

{\small

\littleskip\begin{itemize}
%
\item We need to reify existentially qualified castables.\\
Use heterogeneous lists.
%
\begin{Verbatim}[fontseries=normal,fontsize=\tiny,commandchars=\\\{\}]
data HNil = HNil
data HCons h t = HCons h t
\end{Verbatim}
%
\item We need to reify type-safe cast.\\
Well, we managed to lift boilerplate's cast to the type-level.
%
\begin{Verbatim}[fontseries=normal,fontsize=\tiny,commandchars=\\\{\}]
class Maybe m => Cast x m y | x y -> m
 where cast :: x -> m y
\end{Verbatim}
%
\item We need to perform case on type-level maybies.\\
Easy!
%
\begin{Verbatim}[fontseries=normal,fontsize=\tiny,commandchars=\\\{\}]
class (Maybe m, HList hl, List lst)
   =>  HOccursManyCase m a hl lst | m hl a -> lst
 where hOccursManyCase :: m a -> hl -> lst a
\end{Verbatim}
%
\end{itemize}

}

\end{myslide}



%%%%%%%%%%%%%%%%%%%%%%%%%%%%%%%%%%%%%%%%%%%%%%%%%%%%%%%%%%%%%%%%%%%%%%%%%%%%%%



\begin{myslide}

\header{The result of preprocessing cont'd}

\blau{Type-level ambiguous look-up operation}

{\tiny

\begin{code}
class (HList hl, List lst)
   =>  HOccursMany a hl lst | hl a -> lst
 where
  hOccursMany :: hl -> lst a

instance HOccursMany a HNil Nil
 where
  hOccursMany HNil = Nil

instance (Cast h m a, HOccursManyCase m a t lst)
      =>  HOccursMany a (HCons h t) lst
 where
  hOccursMany (HCons h t) = hOccursManyCase (cast h) t
\end{code}

}

\end{myslide}



\begin{myslide}

\header{cont'd}

{\tiny

\begin{code}
class (HList hl, List lst)
   =>  HOccursManyCase m a hl lst | m hl a -> lst
 where
  hOccursManyCase :: m a -> hl -> lst a

instance (HList hl, List lst, HOccursMany a hl lst)
      =>  HOccursManyCase Just a hl (Cons lst)
 where 
  hOccursManyCase (Just a) hl = Cons a (hOccursMany hl) 

instance (HList hl, List lst, HOccursMany a hl lst)
      =>  HOccursManyCase Nothing a hl lst
 where
  hOccursManyCase Nothing hl = hOccursMany hl
\end{code}

}

\end{myslide}



%%%%%%%%%%%%%%%%%%%%%%%%%%%%%%%%%%%%%%%%%%%%%%%%%%%%%%%%%%%%%%%%%%%%%%%%%%%%%%



\begin{slide}

\medskip

Thank you for your attention.\\
End of slide show.

\bigskip

{\tiny

Spare slides:
\littleskip\begin{itemize}
\item Type-level verification.
\item Stanamic operations.
\end{itemize}

}

\medskip

\end{slide}



%%%%%%%%%%%%%%%%%%%%%%%%%%%%%%%%%%%%%%%%%%%%%%%%%%%%%%%%%%%%%%%%%%%%%%%%%%%%%%



\begin{myslide}

\header{Beyond compile-time checks}

\blau{Verification~---~Wanted!}

{\small

\[\begin{array}{l}
\forall\,\alpha :: *.\ \forall\,l_1, l_2 :: \texttt{List}\ \alpha.\\ 
\texttt{add}\ (\texttt{length}\ l_1)\ (\texttt{length}\ l_2) == 
\texttt{length}\ (\texttt{append}\ l_1\ l_2)
\end{array}\]

}

\end{myslide}



%%%%%%%%%%%%%%%%%%%%%%%%%%%%%%%%%%%%%%%%%%%%%%%%%%%%%%%%%%%%%%%%%%%%%%%%%%%%%%



\begin{myslide}

\bigskip
\bigskip
\bigskip

\blau{Proof by induction on \texttt{append}'s first operand}

{

\tiny

\begin{itemize}
%
\item Base case: $l_1 == \texttt{Nil}$ 

$\begin{array}{p{1.5in}ll}
& & \texttt{length}\ (\texttt{append}\ \texttt{Nil}\ l_2)\\
\mbox{(base \texttt{append})}
& ==
& \texttt{length}\ l_2
\\
\mbox{}\\
& & \texttt{add}\ (\texttt{length}\ \texttt{Nil})\ (\texttt{length}\ l_2)\\
\mbox{(base \texttt{length})}
& ==
& \texttt{add}\ \texttt{Zero}\ (\texttt{length}\ l_2)
\\
\mbox{(base \texttt{add})}
& ==
& \texttt{length}\ l_2
\end{array}$

%
%
%
\item Induction step: $l_1$ is of the form $\texttt{Cons}\ a\ b$

$\begin{array}{p{1.5in}ll}
& & \texttt{length}\ (\texttt{append}\ (\texttt{Cons}\ a\ b)\ l_2)\\
\mbox{step \texttt{append})}
& ==
& \texttt{length}\ (\texttt{Cons}\ a\ (\texttt{append}\ b\ l_2))
\\
\mbox{step \texttt{length})}
& ==
& \texttt{Succ}\ (\texttt{length}\ (\texttt{append}\ b\ l_2))
\\
\mbox{}\\
& & \texttt{add}\ (\texttt{length}\ (\texttt{Cons}\ a\ b))\ (\texttt{length}\ l_2)\\
\mbox{(step \texttt{length})}
& ==
& \texttt{add}\ (\texttt{Succ}\ (\texttt{length}\ b))\ (\texttt{length}\ l_2)
\\
\mbox{(step \texttt{add})}
& ==
& \texttt{Succ}\ (\texttt{add}\ (\texttt{length}\ b)\ (\texttt{length}\ l_2))
\\
\mbox{(ind.\ hypoth.)}
& ==
& \texttt{Succ}\ (\texttt{length}\ (\texttt{append}\ b\ l_2))
\end{array}$

\end{itemize}

}

\end{myslide}



%%%%%%%%%%%%%%%%%%%%%%%%%%%%%%%%%%%%%%%%%%%%%%%%%%%%%%%%%%%%%%%%%%%%%%%%%%%%%%



\begin{myslide}

\header{From `require' to `with'}

\blau{Lists with a natural as index}

\begin{Verbatim}[fontseries=normal,fontsize=\tiny,commandchars=\\\{\}]
data List <x::Nat> a
   = Nil                      \textbf{with} x = Zero
   | Cons a (List <y::Nat> a) \textbf{with} x = Succ y
\end{Verbatim}

\blau{A verified append on the indexed lists}

\begin{Verbatim}[fontseries=normal,fontsize=\tiny,commandchars=\\\{\}]
append :: List <len1> a -> List <len2> a -> List <len3> a
 \textbf{with} 
  len3 == add len1 len2
append Nil x = x
append (Cons h t) x = Cons h (append t x)
\end{Verbatim}

\blau{Example of an impossible flaw}

\begin{Verbatim}[fontseries=normal,fontsize=\tiny,commandchars=\\\{\}]
append (Cons h t) x = append t x -- does not type check
\end{Verbatim}

\end{myslide}



%%%%%%%%%%%%%%%%%%%%%%%%%%%%%%%%%%%%%%%%%%%%%%%%%%%%%%%%%%%%%%%%%%%%%%%%%%%%%%



\begin{myslide}

\header{Type-level programming idiom}

\blau{Encoding indexed types}

\medskip

\begin{Verbatim}[fontseries=normal,fontsize=\tiny]
data Nil len a    = Nil
data Cons b len a = Cons a (b a)
class List ( lst :: * -> * )
instance List (Nil Zero)
instance List (b len) => List (Cons (b len) (Succ len))
\end{Verbatim}

\end{myslide}



%%%%%%%%%%%%%%%%%%%%%%%%%%%%%%%%%%%%%%%%%%%%%%%%%%%%%%%%%%%%%%%%%%%%%%%%%%%%%%



\begin{myslide}

\header{Type-level programming idiom}

\blau{Encoding verification conditions as class constraints}

\medskip

\begin{Verbatim}[fontseries=normal,fontsize=\tiny]
class ( List (l1 len1)
      , List (l2 len2)
      , List (l3 len3)
      , Add len1 len2 len3
      )
   => Append l1 len1 l2 len2 l3 len3
    | l1 len1 l2 len2 -> l3 len3
 where
  append :: l1 len1 a -> l2 len2 a -> l3 len3 a
\end{Verbatim}

\end{myslide}



%%%%%%%%%%%%%%%%%%%%%%%%%%%%%%%%%%%%%%%%%%%%%%%%%%%%%%%%%%%%%%%%%%%%%%%%%%%%%%



\begin{myslide}

\header{Type-level programming idiom}

\blau{Encoding the induction proof}

\begin{Verbatim}[fontseries=normal,fontsize=\tiny]
instance (Nat len, List (x len))
      =>  Append Nil Zero x len x len
 where   
  append Nil x = x

instance Append t len x xlen t' len'
      => Append (Cons (t len))   (Succ len)
                x                xlen
                (Cons (t' len')) (Succ len')
 where
  append (Cons h t) x = Cons h (append t x)
\end{Verbatim}

\end{myslide}



%%%%%%%%%%%%%%%%%%%%%%%%%%%%%%%%%%%%%%%%%%%%%%%%%%%%%%%%%%%%%%%%%%%%%%%%%%%%%%



\begin{myslide}

\header{Where are we?}

\begin{itemize}
\item We can distinguish subsets of normal types.
\item We can compile-time program with type equality.
\item We can compile-time guard function applications.
\item We can compile-time verify function definitions.
\item We can turn reified values into normal values.
\item Can we also go the other way around? (Yes!)
\item Can we mix type-level and normal values? (Yes!)
\end{itemize}

\end{myslide}



%%%%%%%%%%%%%%%%%%%%%%%%%%%%%%%%%%%%%%%%%%%%%%%%%%%%%%%%%%%%%%%%%%%%%%%%%%%%%%



\begin{myslide}

\header{Mix type-level and normal values}

\blau{Stanamic naturals}

\begin{Verbatim}[fontseries=normal,fontsize=\small]
data VNat    = VZero | VSucc VNat
data TZero   = TZero
data TSucc n = TSucc n

class Nat n
instance Nat VNat
instance Nat TZero
instance Nat n => Nat (TSucc n)
\end{Verbatim}

\end{myslide}



%%%%%%%%%%%%%%%%%%%%%%%%%%%%%%%%%%%%%%%%%%%%%%%%%%%%%%%%%%%%%%%%%%%%%%%%%%%%%%



\begin{myslide}

\header{Turn reified values into normal ones}

\blau{Fold over type-level naturals}

\begin{Verbatim}[fontseries=normal,fontsize=\tiny]
class TNat n
 where
  onTNat :: n -> r -> (forall n'. TNat n' => n' -> r) -> r
  
instance TNat TZero
 where
  onTNat _ r _ = r

instance TNat n => TNat (TSucc n)
 where
  onTNat (TSucc n) _ f = f n
\end{Verbatim}

\medskip

\blau{Example: extraction of normal naturals}

\begin{Verbatim}[fontseries=normal,fontsize=\tiny]
tn2vn :: TNat n => n -> VNat
tn2vn tn = onTNat tn VZero (\tn' -> VSucc (tn2vn tn'))
\end{Verbatim}

\end{myslide}



%%%%%%%%%%%%%%%%%%%%%%%%%%%%%%%%%%%%%%%%%%%%%%%%%%%%%%%%%%%%%%%%%%%%%%%%%%%%%%



\begin{myslide}

\header{Process normal values as reified ones}

\blau{Example: roundtrip}

\begin{Verbatim}[fontseries=normal,fontsize=\tiny]
vn2tn :: VNat -> (forall n. TNat n => n -> w) -> w
vn2tn VZero f      = f TZero
vn2tn (VSucc vn) f = vn2tn vn (\tn -> f (TSucc tn))

ghci-or-hugs> vn2tn (VSucc (VSucc VZero)) tn2vn
VSucc (VSucc VZero)
\end{Verbatim}

\end{myslide}



%%%%%%%%%%%%%%%%%%%%%%%%%%%%%%%%%%%%%%%%%%%%%%%%%%%%%%%%%%%%%%%%%%%%%%%%%%%%%%



\begin{slide}

\medskip

Thank you for your attention.

\medskip

End of slide show.

\medskip

\end{slide}



%%%%%%%%%%%%%%%%%%%%%%%%%%%%%%%%%%%%%%%%%%%%%%%%%%%%%%%%%%%%%%%%%%%%%%%%%%%%%%



\end{document}
