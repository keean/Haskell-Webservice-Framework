\documentclass{slides}

\usepackage{times}
\usepackage{comment}
\usepackage{boxedminipage}
\usepackage{alltt}
\usepackage{graphics}
\usepackage[usenames]{pstcol}
\usepackage{url}
\usepackage{fancyvrb}
\usepackage{ifthen}


\newcommand{\pure}{\textsc{PURE}}
\newcommand{\noskip}{\topsep0pt \parskip0pt \partopsep0pt}
\newcommand{\header}[1]{{\large\scshape \color{Red} #1} \medskip }
\newcommand{\blau}[1]{{\color{Blue} #1} \medskip }
\newenvironment{myslide}{\begin{slide}\color{Blue}\begin{boxedminipage}{1.1\hsize}\begin{boxedminipage}{1\hsize}\color{Black}
\vspace{-170\in}
}{%
\smallskip
\end{boxedminipage}
\end{boxedminipage}
\end{slide}}


\begin{document}

\pagestyle{empty}



%\begin{comment}
\begin{slide}

Press PgDn to start.

\end{slide}
%\end{comment}



%%%%%%%%%%%%%%%%%%%%%%%%%%%%%%%%%%%%%%%%%%%%%%%%%%%%%%%%%%%%%%%%%%%%%%%%%%%%%%



\begin{myslide}

\bigskip

\header{
\begin{center}
Type-level computations in Haskell
\end{center}
}

\vspace{200\in}

\blau{Ralf L{\"a}mmel\\
Vrije Universiteit Amsterdam\\
Centrum voor Wiskunde en Informatica}

\bigskip
Joint work with\\
Oleg Kiselyov, FNMO Center, Monterey, CA\\
Keean Schupke, Imperial College, London

\bigskip
\bigskip

(Subtitle: Secrets on the DHaskell preprocessor.)

\bigskip

\end{myslide}



%%%%%%%%%%%%%%%%%%%%%%%%%%%%%%%%%%%%%%%%%%%%%%%%%%%%%%%%%%%%%%%%%%%%%%%%%%%%%%



\begin{myslide}

\header{One motivating example}

Heterogenous collections in a database API

\bigskip

\blau{SQL}

\begin{Verbatim}[fontseries=normal,fontsize=\tiny]
SELECT key,name FROM Animal WHERE breed = 'sheep';
\end{Verbatim}

\blau{Haskell}

\begin{Verbatim}[fontseries=normal,fontsize=\tiny]
selectBreed b = -- argument b for the breed
  do r1 <- table animalTable
     r2 <- restrict r1 (\r -> r .!. breed `SQL.eq` b)
     r3 <- project r2 (key .*. name .*. HNil)
     doSelect r3
\end{Verbatim}

\end{myslide}



%%%%%%%%%%%%%%%%%%%%%%%%%%%%%%%%%%%%%%%%%%%%%%%%%%%%%%%%%%%%%%%%%%%%%%%%%%%%%%



\begin{myslide}

\header{Types involved in the DB example}

\blau{Reified database schema}

\begin{Verbatim}[fontseries=normal,fontsize=\tiny]
type AnimalSchema =
   Tkey   :=: Attribute AnimalId SqlInteger :*:
   Tname  :=: Attribute String   SqlVarchar :*:
   Tbreed :=: Attribute Breed    SqlVarchar :*:
   Tprice :=: Attribute Float    SqlNumeric :*:
   Tfarm  :=: Attribute FarmId   SqlInteger :*: HNil
\end{Verbatim}

\blau{Computed (say inferred) type of query}

\begin{Verbatim}[fontseries=normal,fontsize=\tiny]
selectBreed :: Breed -> Query [
     Tkey   :=: AnimalId :*:
     Tname  :=: String   :*: HNil ]
\end{Verbatim}

\end{myslide}



%%%%%%%%%%%%%%%%%%%%%%%%%%%%%%%%%%%%%%%%%%%%%%%%%%%%%%%%%%%%%%%%%%%%%%%%%%%%%%



\begin{myslide}

\header{Another motivating example}

Compile-time grammar processing

\bigskip

\blau{Normal datatypes to be reified}

\begin{Verbatim}[fontseries=normal,fontsize=\tiny]
data Stm   = Assign Ident Exp | Sequ Stm Stm
data Exp   = Var Ident | Zero | Succ Exp
type Ident = String
\end{Verbatim}

{\small

\blau{Opportunities}

\noskip\begin{itemize}
\item Test for properties: reducedness, first-order, BNF, ...
\item Transformations: annotation, BNF$\leftrightarrow$EBNF, ...
\item Generators: parsing, pretty printing, serialisation
\end{itemize}

}

\end{myslide}



%%%%%%%%%%%%%%%%%%%%%%%%%%%%%%%%%%%%%%%%%%%%%%%%%%%%%%%%%%%%%%%%%%%%%%%%%%%%%%



\begin{myslide}

\header{Plan of the talk}

\begin{itemize}
\item Examples of type-level computations in Haskell
\item Idioms for type-level programming in Haskell
\item The \emph{D}Haskell language extension
\item Occasional references to other languages
\end{itemize}

\end{myslide}



%%%%%%%%%%%%%%%%%%%%%%%%%%%%%%%%%%%%%%%%%%%%%%%%%%%%%%%%%%%%%%%%%%%%%%%%%%%%%%



\begin{myslide}

\header{Heterogeneous collections}

\blau{Weakly typed heterogeneous lists}

\begin{Verbatim}[fontseries=normal,fontsize=\tiny]
type HList = [Dynamic]

angus = [ toDyn (Key 42)
        , toDyn (Name "Angus")
        , toDyn Cow
        , toDyn (Price 75.5) ] 
\end{Verbatim}

\blau{Type-driven computations}

\begin{Verbatim}[fontseries=normal,fontsize=\tiny]
hOccursMany :: Typeable e => HList -> [e]
hOccursMany [] = []
hOccursMany (h:t) = case fromDynamic h of
                     Just e  -> e : hOccursMany t
                     Nothing -> hOccursMany t

ghci-or-hugs> hOccursMany angus :: [Breed]
[Cow]
\end{Verbatim}

\end{myslide}



%%%%%%%%%%%%%%%%%%%%%%%%%%%%%%%%%%%%%%%%%%%%%%%%%%%%%%%%%%%%%%%%%%%%%%%%%%%%%%



\begin{myslide}

\header{Sigh! Run-time checks everywhere}

\blau{Return an unambiguous occurrence for the type}

\begin{Verbatim}[fontseries=normal,fontsize=\small]
 hOccurs :: Typeable e => HList -> e
 hOccurs l = case hOccursMany l of
              [e] -> e
              _   -> error "hOccurs"
\end{Verbatim}

\end{myslide}



%%%%%%%%%%%%%%%%%%%%%%%%%%%%%%%%%%%%%%%%%%%%%%%%%%%%%%%%%%%%%%%%%%%%%%%%%%%%%%



\begin{myslide}

\header{Type-level programming idiom}

\blau{Reify datatype constructors as types}

\begin{Verbatim}[fontseries=normal,fontsize=\tiny]
-- Heterogeneous lists as nested tuples
data HNil      = HNil
data HCons e l = HCons e l

-- A dedicated class
class HList l
instance HList HNil
instance HList l => HList (HCons e l)
\end{Verbatim}

\medskip

{\small

Notice: this type allows for heterogeneous lists.

}

\end{myslide}



%%%%%%%%%%%%%%%%%%%%%%%%%%%%%%%%%%%%%%%%%%%%%%%%%%%%%%%%%%%%%%%%%%%%%%%%%%%%%%



\begin{myslide}

\header{Type-level programming idiom}

\blau{Turn function signatures into classes}

\begin{Verbatim}[fontseries=normal,fontsize=\tiny]
-- A function signature
hOccursMany :: Typeable e => HList -> [e]

-- The corresponding class
class  HList l => HOccursMany e l
 where hOccursMany :: l -> [e]
\end{Verbatim}

\medskip

{\small

(Instances to be disclosed soon.)

}

\end{myslide}



%%%%%%%%%%%%%%%%%%%%%%%%%%%%%%%%%%%%%%%%%%%%%%%%%%%%%%%%%%%%%%%%%%%%%%%%%%%%%%



\begin{myslide}

\header{Typeful heterogeneous lists}

\blau{Convenience notation}

\begin{Verbatim}[fontseries=normal,fontsize=\tiny]
type e :*: l = HCons e l  -- type-level constructor
e .*. l      = HCons e l  -- value-level constructor
\end{Verbatim}

\medskip

\blau{Typeful animals}

\begin{Verbatim}[fontseries=normal,fontsize=\tiny]
type Animal =
  Key :*: Name :*: Breed :*: Price :*: HNil

angus :: Animal -- optional type declaration
angus =  Key 42
     .*. Name "Angus"
     .*. Cow
     .*. Price 75.5
     .*. HNil
\end{Verbatim}

\end{myslide}



%%%%%%%%%%%%%%%%%%%%%%%%%%%%%%%%%%%%%%%%%%%%%%%%%%%%%%%%%%%%%%%%%%%%%%%%%%%%%%



\begin{myslide}

\header{Type-level programming idiom}

\blau{Turn function equations into class instances}

\bigskip

\begin{Verbatim}[fontseries=normal,fontsize=\tiny]
-- The equation for the empty input list
hOccursMany [] = []

-- The corresponding instance
instance HOccursMany e HNil
 where   hOccursMany HNil = []
\end{Verbatim}

\end{myslide}



%%%%%%%%%%%%%%%%%%%%%%%%%%%%%%%%%%%%%%%%%%%%%%%%%%%%%%%%%%%%%%%%%%%%%%%%%%%%%%



\begin{myslide}

\header{Turn equations into instances cont'd}

\blau{The idiosyncratic value-level case for non-empty lists}

\begin{Verbatim}[fontseries=normal,fontsize=\tiny]
hOccursMany (h:t) = case fromDynamic h of
                     Just e  -> e : hOccursMany t
                     Nothing -> hOccursMany t
\end{Verbatim}

\medskip

\blau{Transposition}

\begin{Verbatim}[fontseries=normal,fontsize=\tiny]
instance ( HOccursMany e l, HList l )
      =>   HOccursMany e (HCons e l)
 where     hOccursMany (HCons e l) = e:hOccursMany l
\end{Verbatim}
 
\begin{Verbatim}[fontseries=normal,fontsize=\tiny]
instance ( HOccursMany e l, HList l )
      =>   HOccursMany e (HCons e' l)
 where     hOccursMany (HCons _ l) = hOccursMany l
\end{Verbatim}

\medskip

{\small

Note overlaps and non-linearity in the instances.

}

\end{myslide}



%%%%%%%%%%%%%%%%%%%%%%%%%%%%%%%%%%%%%%%%%%%%%%%%%%%%%%%%%%%%%%%%%%%%%%%%%%%%%%



\begin{myslide}

\header{Type-level programming idiom}

\blau{Observe type equality}

\bigskip

\begin{Verbatim}[fontseries=normal,fontsize=\small]
class HBool b
   => TypeEq x y b
    | x y -> b
\end{Verbatim}


\end{myslide}



%%%%%%%%%%%%%%%%%%%%%%%%%%%%%%%%%%%%%%%%%%%%%%%%%%%%%%%%%%%%%%%%%%%%%%%%%%%%%%



\begin{myslide}

\header{Type-level programming idiom}

\blau{Branch according to type-level Boolean}

\bigskip

\begin{Verbatim}[fontseries=normal,fontsize=\small]
instance ( TypeEq e h b
         , HOccursManyBool b e (HCons h t)
         )
      =>   HOccursMany e (HCons h t)
 where     hOccursMany l@(HCons h t) = l'
            where b  = typeEq (head l') h 
                  l' = hOccursManyBool b l
\end{Verbatim}

\end{myslide}



%%%%%%%%%%%%%%%%%%%%%%%%%%%%%%%%%%%%%%%%%%%%%%%%%%%%%%%%%%%%%%%%%%%%%%%%%%%%%%



\begin{myslide}

\header{Type-level programming idiom}

\blau{Compile-time constraints as method-less classes}

\bigskip

\begin{Verbatim}[fontseries=normal,fontsize=\tiny]
-- Check for lack of occurrence
class HOccursNot e l
instance HOccursNot e HNil
instance Fail (TypeFound e) => HOccursNot e (HCons e l)
instance HOccursNot e l     => HOccursNot e (HCons e' l)

class Fail x      -- meant to fail
data TypeFound e  -- error message
\end{Verbatim}

\end{myslide}



%%%%%%%%%%%%%%%%%%%%%%%%%%%%%%%%%%%%%%%%%%%%%%%%%%%%%%%%%%%%%%%%%%%%%%%%%%%%%%



\begin{myslide}

\header{Yeah! Compile-time checks}

\blau{Return an unambiguous occurrence for the type}

\begin{Verbatim}[fontseries=normal,fontsize=\tiny]
class  HOccurs e l
 where hOccurs :: l -> e

instance (HList l, HOccursNot e l)
      =>  HOccurs e (HCons e l)
 where    hOccurs (HCons e _) = e

instance (HOccurs e l, HList l)
      =>  HOccurs e (HCons e' l)
 where    hOccurs (HCons _ l) = hOccurs l
\end{Verbatim}

\medskip

{\small

Any application of \texttt{hOccurs} is statically checked to return an
unambiguous occurrence for the type of interest (modulo halting
problem regarding partiality).

}

\end{myslide}



%%%%%%%%%%%%%%%%%%%%%%%%%%%%%%%%%%%%%%%%%%%%%%%%%%%%%%%%%%%%%%%%%%%%%%%%%%%%%%



\begin{myslide}

\header{Securing the weakly typed solution}

\blau{Recall the value-level code}

\begin{Verbatim}[fontseries=normal,fontsize=\small]
 hOccurs :: Typeable a => HList -> a
 hOccurs l = case hOccursMany l of
              [x] -> Just a
              _   -> error "hOccurs"
\end{Verbatim}

\medskip

\blau{Transposition to the type-level}

{\small

\noskip\begin{itemize}
\item The polymorphic list structure needs to be reified.
\item The error string becomes an `error type'.
\item The error case becomes a `failing instance'.
\end{itemize}

}

\end{myslide}



%%%%%%%%%%%%%%%%%%%%%%%%%%%%%%%%%%%%%%%%%%%%%%%%%%%%%%%%%%%%%%%%%%%%%%%%%%%%%%



\begin{myslide}

\header{Reification of polymorphic lists}

\blau{Normal list datatype}

\begin{Verbatim}[fontseries=normal,fontsize=\tiny]
data List a = Nil | Cons a (List a)
\end{Verbatim}

\medskip

\blau{Reified version}

\begin{Verbatim}[fontseries=normal,fontsize=\tiny]
data Nil a    = Nil
data Cons b a = Cons a (b a)
\end{Verbatim}
\medskip
\begin{Verbatim}[fontseries=normal,fontsize=\tiny]
class    List ( lst :: * -> * )
instance List Nil
instance List b => List (Cons b)
\end{Verbatim}

\end{myslide}



%%%%%%%%%%%%%%%%%%%%%%%%%%%%%%%%%%%%%%%%%%%%%%%%%%%%%%%%%%%%%%%%%%%%%%%%%%%%%%



\begin{myslide}

\header{A benchmark for dependent typing}

\blau{Normal append}

\begin{Verbatim}[fontseries=normal,fontsize=\tiny]
append :: [a] -> [a] -> [a]
append [] x = x
append (h : t) x = h : (append t x)
\end{Verbatim}

\medskip

\blau{Run-time side condition}

\begin{Verbatim}[fontseries=normal,fontsize=\tiny]
guardedAppend :: [a] -> [a] -> Maybe [a]
guardedAppend lst1 lst2
   = if length lst3 == add (length lst1) (length lst2) 
        then Just lst3
        else Nothing
 where
  lst3 = append lst1 lst2
\end{Verbatim}

\end{myslide}



%%%%%%%%%%%%%%%%%%%%%%%%%%%%%%%%%%%%%%%%%%%%%%%%%%%%%%%%%%%%%%%%%%%%%%%%%%%%%%



\begin{myslide}

\header{Type-level guarding}

\blau{Encoding in DHaskell}

\begin{Verbatim}[fontseries=normal,fontsize=\small,commandchars=\\\{\}]
guardedAppend :: [a] -> [a] -> [a]
guardedAppend lst1 lst2 = lst3
 where
  lst3 = append lst1 lst2
 \textbf{require}
  length lst3 == add (length lst1) (length lst2) 
\end{Verbatim}

\bigskip

{\small

To compile away the DHaskell extension, we need to lift \texttt{[]},
\texttt{append}, \texttt{add}, and \texttt{length}, and we need to
define a constrained function \texttt{guardedAppend}.

}

\medskip

(Disclaimer: notation in flux)

\end{myslide}



%%%%%%%%%%%%%%%%%%%%%%%%%%%%%%%%%%%%%%%%%%%%%%%%%%%%%%%%%%%%%%%%%%%%%%%%%%%%%%



\begin{myslide}

\header{Lifting value-level operations}

\blau{Normal list append}

\begin{Verbatim}[fontseries=normal,fontsize=\tiny]
append :: [a] -> [a] -> [a]
append [] x = x
append (h : t) x = h : (append t x)
\end{Verbatim}

\blau{Type-level append}

\begin{Verbatim}[fontseries=normal,fontsize=\tiny]
class (List l1, List l2, List l3)
   =>  Append l1 l2 l3 | l1 l2 -> l3
 where append :: l1 a -> l2 a -> l3 a

instance List x
      => Append Nil x x
 where   append Nil x = x

instance Append t x t'
      => Append (Cons t) x (Cons t')
 where   append (Cons h t) x = Cons h (append t x)
\end{Verbatim}

\end{myslide}



%%%%%%%%%%%%%%%%%%%%%%%%%%%%%%%%%%%%%%%%%%%%%%%%%%%%%%%%%%%%%%%%%%%%%%%%%%%%%%



\begin{myslide}

\header{Type-level side condition}

\blau{Turn \textbf{require} into constraints}

\begin{Verbatim}[fontseries=normal,fontsize=\tiny]
guardedAppend :: ( Append lst1 lst2 lst3
                 , Length lst1 nat1
                 , Length lst2 nat2
                 , Length lst3 nat3
                 , Add nat1 nat2 nat3
                 ) => lst1 a -> lst2 a -> lst3 a
guardedAppend = append
\end{Verbatim}

\blau{Use type-level naturals}

\begin{Verbatim}[fontseries=normal,fontsize=\tiny]
data Zero
data Succ n
class Nat n
instance Nat Zero
instance Nat n => Nat (Succ n)
\end{Verbatim}

\end{myslide}



%%%%%%%%%%%%%%%%%%%%%%%%%%%%%%%%%%%%%%%%%%%%%%%%%%%%%%%%%%%%%%%%%%%%%%%%%%%%%%



\begin{myslide}

\header{From guarding to verification}

\blau{Wanted!}

{\tiny

\[\begin{array}{l}
\forall\,\alpha :: *.\ \forall\,l_1, l_2 :: \texttt{List}\ \alpha.\\ 
\texttt{add}\ (\texttt{length}\ l_1)\ (\texttt{length}\ l_2) == 
\texttt{length}\ (\texttt{append}\ l_1\ l_2)
\end{array}\]

}

\end{myslide}



%%%%%%%%%%%%%%%%%%%%%%%%%%%%%%%%%%%%%%%%%%%%%%%%%%%%%%%%%%%%%%%%%%%%%%%%%%%%%%



\begin{myslide}

\bigskip
\bigskip
\bigskip

\blau{Proof by induction on \texttt{append}'s first operand}

{

\tiny

\begin{itemize}
%
\item Base case: $l_1 == \texttt{Nil}$ 

$\begin{array}{p{1.5in}ll}
& & \texttt{length}\ (\texttt{append}\ \texttt{Nil}\ l_2)\\
\mbox{(base \texttt{append})}
& ==
& \texttt{length}\ l_2
\\
\mbox{}\\
& & \texttt{add}\ (\texttt{length}\ \texttt{Nil})\ (\texttt{length}\ l_2)\\
\mbox{(base \texttt{length})}
& ==
& \texttt{add}\ \texttt{Zero}\ (\texttt{length}\ l_2)
\\
\mbox{(base \texttt{add})}
& ==
& \texttt{length}\ l_2
\end{array}$

%
%
%
\item Induction step: $l_1$ is of the form $\texttt{Cons}\ a\ b$

$\begin{array}{p{1.5in}ll}
& & \texttt{length}\ (\texttt{append}\ (\texttt{Cons}\ a\ b)\ l_2)\\
\mbox{step \texttt{append})}
& ==
& \texttt{length}\ (\texttt{Cons}\ a\ (\texttt{append}\ b\ l_2))
\\
\mbox{step \texttt{length})}
& ==
& \texttt{Succ}\ (\texttt{length}\ (\texttt{append}\ b\ l_2))
\\
\mbox{}\\
& & \texttt{add}\ (\texttt{length}\ (\texttt{Cons}\ a\ b))\ (\texttt{length}\ l_2)\\
\mbox{(step \texttt{length})}
& ==
& \texttt{add}\ (\texttt{Succ}\ (\texttt{length}\ b))\ (\texttt{length}\ l_2)
\\
\mbox{(step \texttt{add})}
& ==
& \texttt{Succ}\ (\texttt{add}\ (\texttt{length}\ b)\ (\texttt{length}\ l_2))
\\
\mbox{(ind.\ hypoth.)}
& ==
& \texttt{Succ}\ (\texttt{length}\ (\texttt{append}\ b\ l_2))
\end{array}$

\end{itemize}

}

\end{myslide}



%%%%%%%%%%%%%%%%%%%%%%%%%%%%%%%%%%%%%%%%%%%%%%%%%%%%%%%%%%%%%%%%%%%%%%%%%%%%%%



\begin{myslide}

\header{Verification in DHaskell}

\blau{Lists with a natural as index}

\begin{Verbatim}[fontseries=normal,fontsize=\tiny,commandchars=\\\{\}]
data List <x::Nat> a
   = Nil                      \textbf{with} x = Zero
   | Cons a (List <y::Nat> a) \textbf{with} x = Succ y
\end{Verbatim}

\blau{A verified append on the indexed lists}

\begin{Verbatim}[fontseries=normal,fontsize=\tiny,commandchars=\\\{\}]
append :: List <len1> a -> List <len2> a -> List <len3> a
 \textbf{with} 
  len3 == add len1 len2
append Nil x = x
append (Cons h t) x = Cons h (append t x)
\end{Verbatim}

\blau{Example of an impossible flaw}

\begin{Verbatim}[fontseries=normal,fontsize=\tiny,commandchars=\\\{\}]
append (Cons h t) x = append t x -- does not type check
\end{Verbatim}

\medskip

(Disclaimer: notation in flux)

\end{myslide}



%%%%%%%%%%%%%%%%%%%%%%%%%%%%%%%%%%%%%%%%%%%%%%%%%%%%%%%%%%%%%%%%%%%%%%%%%%%%%%



\begin{myslide}

\header{Type-level programming idiom}

\blau{Encoding indexed types}

\medskip

\begin{Verbatim}[fontseries=normal,fontsize=\tiny]
data Nil len a    = Nil
data Cons b len a = Cons a (b a)
class List ( lst :: * -> * )
instance List (Nil Zero)
instance List (b len) => List (Cons (b len) (Succ len))
\end{Verbatim}

\end{myslide}



%%%%%%%%%%%%%%%%%%%%%%%%%%%%%%%%%%%%%%%%%%%%%%%%%%%%%%%%%%%%%%%%%%%%%%%%%%%%%%



\begin{myslide}

\header{Type-level programming idiom}

\blau{Encoding verification conditions as class constraints}

\medskip

\begin{Verbatim}[fontseries=normal,fontsize=\tiny]
class ( List (l1 len1)
      , List (l2 len2)
      , List (l3 len3)
      , Add len1 len2 len3
      )
   => Append l1 len1 l2 len2 l3 len3
    | l1 len1 l2 len2 -> l3 len3
 where
  append :: l1 len1 a -> l2 len2 a -> l3 len3 a
\end{Verbatim}

\end{myslide}



%%%%%%%%%%%%%%%%%%%%%%%%%%%%%%%%%%%%%%%%%%%%%%%%%%%%%%%%%%%%%%%%%%%%%%%%%%%%%%



\begin{myslide}

\header{Type-level programming idiom}

\blau{Encoding the structural induction proof}

\begin{Verbatim}[fontseries=normal,fontsize=\tiny]
instance (Nat len, List (x len))
      =>  Append Nil Zero x len x len
 where   
  append Nil x = x

instance Append t len x xlen t' len'
      => Append (Cons (t len))   (Succ len)
                x                xlen
                (Cons (t' len')) (Succ len')
 where
  append (Cons h t) x = Cons h (append t x)
\end{Verbatim}

\end{myslide}



%%%%%%%%%%%%%%%%%%%%%%%%%%%%%%%%%%%%%%%%%%%%%%%%%%%%%%%%%%%%%%%%%%%%%%%%%%%%%%



\begin{myslide}

\header{Where are we?}

\begin{itemize}
\item We can distinguish subsets of normal types.
\item We can compile-time program with type equality.
\item We can compile-time guard function applications.
\item We can compile-time verify function definitions.
\item We can turn reified values into normal values.
\item Can we also go the other way around? (Yes!)
\item Can we mix type-level and normal values? (Yes!)
\end{itemize}

\end{myslide}



%%%%%%%%%%%%%%%%%%%%%%%%%%%%%%%%%%%%%%%%%%%%%%%%%%%%%%%%%%%%%%%%%%%%%%%%%%%%%%



\begin{myslide}

\header{Mix type-level and normal values}

\blau{Stanamic naturals}

\begin{Verbatim}[fontseries=normal,fontsize=\small]
data VNat    = VZero | VSucc VNat
data TZero   = TZero
data TSucc n = TSucc n

class Nat n
instance Nat VNat
instance Nat TZero
instance Nat n => Nat (TSucc n)
\end{Verbatim}

\end{myslide}



%%%%%%%%%%%%%%%%%%%%%%%%%%%%%%%%%%%%%%%%%%%%%%%%%%%%%%%%%%%%%%%%%%%%%%%%%%%%%%



\begin{myslide}

\header{Turn reified values into normal ones}

\blau{Fold over type-level naturals}

\begin{Verbatim}[fontseries=normal,fontsize=\tiny]
class TNat n
 where
  onTNat :: n -> r -> (forall n'. TNat n' => n' -> r) -> r
  
instance TNat TZero
 where
  onTNat _ r _ = r

instance TNat n => TNat (TSucc n)
 where
  onTNat (TSucc n) _ f = f n
\end{Verbatim}

\medskip

\blau{Example: extraction of normal naturals}

\begin{Verbatim}[fontseries=normal,fontsize=\tiny]
tn2vn :: TNat n => n -> VNat
tn2vn tn = onTNat tn VZero (\tn' -> VSucc (tn2vn tn'))
\end{Verbatim}

\end{myslide}



%%%%%%%%%%%%%%%%%%%%%%%%%%%%%%%%%%%%%%%%%%%%%%%%%%%%%%%%%%%%%%%%%%%%%%%%%%%%%%



\begin{myslide}

\header{Process normal values as reified ones}

\blau{Example: roundtrip}

\begin{Verbatim}[fontseries=normal,fontsize=\tiny]
vn2tn :: VNat -> (forall n. TNat n => n -> w) -> w
vn2tn VZero f      = f TZero
vn2tn (VSucc vn) f = vn2tn vn (\tn -> f (TSucc tn))

ghci-or-hugs> vn2tn (VSucc (VSucc VZero)) tn2vn
VSucc (VSucc VZero)
\end{Verbatim}

\end{myslide}



%%%%%%%%%%%%%%%%%%%%%%%%%%%%%%%%%%%%%%%%%%%%%%%%%%%%%%%%%%%%%%%%%%%%%%%%%%%%%%



\begin{slide}

\medskip

Thank you for your attention.

\medskip

End of slide show.

\medskip

\end{slide}



%%%%%%%%%%%%%%%%%%%%%%%%%%%%%%%%%%%%%%%%%%%%%%%%%%%%%%%%%%%%%%%%%%%%%%%%%%%%%%



\end{document}
