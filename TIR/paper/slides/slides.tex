\documentclass{slides}

\usepackage{times}
\usepackage{latexsym}
\usepackage{graphics}
\usepackage[usenames]{pstcol}
\usepackage{url}
\usepackage{fancyvrb}
\usepackage{ifthen}
\usepackage{comment}
\usepackage{boxedminipage}

\begin{comment}
\newenvironment{myslide}{\begin{slide}\color{Blue}\begin{boxedminipage}{1.1\hsize}\begin{boxedminipage}{1\hsize}\color{Black}
\vspace{-170\in}
}{%
\smallskip
\end{boxedminipage}
\end{boxedminipage}
\end{slide}}
\end{comment}
\begin{comment}
\newenvironment{myslide}{\begin{slide}
}{%
\end{slide}}
\end{comment}
\newenvironment{myslide}{\begin{slide}\color{White}\begin{boxedminipage}{1.1\hsize}\color{Black}
\vspace{-170\in}
}{%
\smallskip
\end{boxedminipage}
\end{slide}}
\newcommand{\almostnoskip}{\topsep1pt \parskip1pt \partopsep1pt}
\newcommand{\negskip}{\vspace{-50\in}}
\newcommand{\noskip}{\topsep1pt \parskip1pt \partopsep1pt}
\newcommand{\littleskip}{\topsep8pt \parskip8pt \partopsep8pt}
\newcommand{\header}[1]{{\large \color{Red} #1}}
\newcommand{\bang}[1]{{\color{Blue} \mytextbf{#1}}}
\newcommand{\inp}[1]{{\color{Brown} \mytextbf{#1}}}
\newcommand{\out}[1]{{\color{Black} \mytextbf{#1}}}
\newcommand{\blau}[1]{{\vspace{-50\in}\normalsize \color{Blue} #1}}
\newcommand{\cod}{c}
\newcommand{\myforall}{\ensuremath{\forall}}
\newcommand{\HList}{\textsc{HList}}
\newcommand{\undefined}{\ensuremath{\bot}}
\newcommand{\farr}{\ensuremath{\to}}
\newcommand{\larr}{\ensuremath{\leftarrow}}
\newcommand{\carr}{\ensuremath{\Rightarrow}}
\renewcommand{\qquad}{\hspace{40\in}}
\newcommand{\flc}{$\=$}
\newcommand{\lc}{$\>$}
\newcommand{\alphaq}{\alpha'}
\newcommand{\sforall}{\overline{\forall}}
\newcommand{\strafunski}{\textit{\textsf{Strafunski}}}
\newcommand{\drift}{\emph{DrIFT}}
\newlength{\basewidth}
\newlength{\topwidth}
\newcommand{\superimpose}[2]{%
  \settowidth{\basewidth}{#1}%
  \settowidth{\topwidth}{#2}%
  \ifthenelse{\lengthtest{\basewidth > \topwidth}}%
    {\makebox[0pt][l]{#1}\makebox[\basewidth]{#2}}%
    {\makebox[0pt][l]{#2}\makebox[\topwidth]{#1}}%
}
\newcommand{\myprompt}{ghci\ensuremath{>}}
\newcommand{\mybar}{\ensuremath{|}}
\newcommand{\mylbrace}{\ensuremath{\{}}
\newcommand{\myrbrace}{\ensuremath{\}}}
\newcommand{\mytextbf}[1]{\texttt{\textbf{#1}}}

\begin{document}

\pagestyle{empty}



\begin{comment}
\begin{slide}
 
Press PgDn to start.
 
\end{slide}
\end{comment}



%%%%%%%%%%%%%%%%%%%%%%%%%%%%%%%%%%%%%%%%%%%%%%%%%%%%%%%%%%%%%%%%%%%%%%%%%%%%%%



\begin{myslide}

\bigskip

\begin{center}
\header{{\itshape\normalsize \textbf{Strongly Typed Heterogeneous Collections}}}
\end{center}

\bigskip
\bigskip

{\small

\blau{

Oleg Kiselyov, FNMO Center, Monterey, CA\\
Ralf L{\"a}mmel, VUA \& CWI, Amsterdam\\
Keean Schupke, Imperial College, London

}


\bigskip

Subtitles:

{\littleskip\begin{itemize}
\item A useful application of type-class programming
\item Progress regarding the quest for better Haskell records
\item Programmable type systems in Haskell
\end{itemize}}

}

\bigskip
\bigskip


\end{myslide}



%%%%%%%%%%%%%%%%%%%%%%%%%%%%%%%%%%%%%%%%%%%%%%%%%%%%%%%%%%%%%%%%%%%%%%%%%%%%%%



\begin{myslide}

\header{Main deliverable: the \HList\ library}

{\small

\littleskip\begin{itemize}
\item Typeful heterogeneous lists~---~the base type
\item Heterogeneous arrays~---~access via Church numerals
\item Type-driven access operations
\item TIPs~---~Type-Indexed Products
\item TICs~---~Type-Indexed Co-products
\item Extensible records~---~different models available
\item Variants
\end{itemize}

}

\end{myslide}



%%%%%%%%%%%%%%%%%%%%%%%%%%%%%%%%%%%%%%%%%%%%%%%%%%%%%%%%%%%%%%%%%%%%%%%%%%%%%%



\begin{myslide}

\header{Application opportunities}

{\small

\littleskip\begin{itemize}
\item Symbol tables with different entries
\item XML processing~---~view content as nested collection
\item DB access~---~rows are heterogenous lists
\item An advanced object system
\item Keyword arguments and polyvariadic functions
\item Typeful foreign-function interfaces
\item Grammar processing
\item Type-indexed datatypes
\end{itemize}

}

\end{myslide}



%%%%%%%%%%%%%%%%%%%%%%%%%%%%%%%%%%%%%%%%%%%%%%%%%%%%%%%%%%%%%%%%%%%%%%%%%%%%%%



\begin{myslide}

\header{\HList's API}

\vspace{-66\in}

\blau{Heterogeneous lists}

\medskip

\begin{Verbatim}[fontfamily=courier,fontsize=\small,commandchars=\\\{\}]
hHead   :: HHead l h  \carr l \farr h
hTail   :: HTail l l' \carr l \farr l'
hAppend :: HAppend l l' l'' \carr l \farr l' \farr l''
hFoldr  :: HFoldr f v l r \carr f \farr v \farr l \farr r
...
\end{Verbatim}

\end{myslide}



%%%%%%%%%%%%%%%%%%%%%%%%%%%%%%%%%%%%%%%%%%%%%%%%%%%%%%%%%%%%%%%%%%%%%%%%%%%%%%



\begin{myslide}

\header{\HList's API}

\vspace{-66\in}

\blau{Heterogeneous arrays}

\medskip

\begin{Verbatim}[fontfamily=courier,fontsize=\small,commandchars=\\\{\}]
 hLookupByHNat   :: HLookupByHNat n l e     
                 \carr n \farr l \farr e
\end{Verbatim}

\medskip

\begin{Verbatim}[fontfamily=courier,fontsize=\small,commandchars=\\\{\}]
 hDeleteAtHNat   :: HDeleteAtHNat n l l'    
                 \carr n \farr l \farr l'
\end{Verbatim}

\medskip

\begin{Verbatim}[fontfamily=courier,fontsize=\small,commandchars=\\\{\}]
 hUpdateAtHNat   :: HUpdateAtHNat n e l l'  
                 \carr n \farr e \farr l \farr l'
\end{Verbatim}

\medskip

\begin{Verbatim}[fontfamily=courier,fontsize=\small,commandchars=\\\{\}]
 hProjectByHNats :: HProjectByHNats ns l l' 
                 \carr ns \farr l \farr l'
\end{Verbatim}

\medskip

\begin{Verbatim}[fontfamily=courier,fontsize=\small,commandchars=\\\{\}]
 hSplitByHNats   :: HSplitByHNats' ns r l' l'' 
                 \carr ns \farr l \farr (l', l'')
\end{Verbatim}

\end{myslide}



%%%%%%%%%%%%%%%%%%%%%%%%%%%%%%%%%%%%%%%%%%%%%%%%%%%%%%%%%%%%%%%%%%%%%%%%%%%%%%



\begin{myslide}

\header{\HList's API}

\vspace{-66\in}

\blau{Type-driven access}

\medskip

\begin{Verbatim}[fontfamily=courier,fontsize=\small,commandchars=\\\{\}]
  hOccursMany  :: HOccursMany e l  \carr l \farr [e]
  hOccursMany1 :: HOccursMany1 e l \carr l \farr (e, [e])
  hOccursOpt   :: HOccursOpt e l   \carr l \farr Maybe e
  hOccursFst   :: HOccursFst e l   \carr l \farr e
  hOccurs      :: HOccurs e l      \carr l \farr e
  ...
\end{Verbatim}

\vspace{-66\in}

\blau{Layered library}

\medskip

\begin{Verbatim}[fontfamily=courier,fontsize=\small,commandchars=\\\{\}]
 -- Map type to index in a list
 class HNat n \carr HType2HNat e l n | e l \farr n
\end{Verbatim}

{

\small

Use array-like operations to implement type-based operations.

}

\end{myslide}



%%%%%%%%%%%%%%%%%%%%%%%%%%%%%%%%%%%%%%%%%%%%%%%%%%%%%%%%%%%%%%%%%%%%%%%%%%%%%%



\begin{myslide}

\header{Plan of the talk}

{\small

\littleskip\begin{itemize}
\item Recall pre-\HList\ (say weakly-typed) situation 
\item Reify list structure
\item Implement type-driven access operations
\item Address issues of overlapping instances
\item Details of extensible records
\item Details of type-indexed products
\item Handle subtyping
\item Idioms for type annotation and improvement
\item Disclose type-level type-safe cast etc.
\end{itemize}

}

\end{myslide}



%%%%%%%%%%%%%%%%%%%%%%%%%%%%%%%%%%%%%%%%%%%%%%%%%%%%%%%%%%%%%%%%%%%%%%%%%%%%%%



\begin{myslide}

\header{Weakly typed heterogeneous lists}

\bigskip

\begin{Verbatim}[fontfamily=courier,fontsize=\small,commandchars=\\\{\}]
 type HList    = [Castable]
 data Castable = \myforall x. Typeable x \carr Castable x 
 unCastable :: Typeable x \carr Castable \farr Maybe x
 unCastable (Castable x) = cast x
\end{Verbatim}

\vspace{-66\in}

\blau{The weakly typed cow Angus}

\medskip

\begin{Verbatim}[fontfamily=courier,fontsize=\small,commandchars=\\\{\}]
 angus = [ Castable (Key 42)
         , Castable (Name "Angus")
         , Castable Cow
         , Castable (Price 75.5) ] 
\end{Verbatim}

\bigskip

\begin{Verbatim}[fontfamily=courier,fontsize=\tiny,commandchars=\\\{\}]
 newtype Key     = Key Integer
 newtype Name    = Name String
 data    Breed   = Cow | Sheep
 newtype Price   = Price Float
 data    Disease = BSE | FM
\end{Verbatim}

\end{myslide}



%%%%%%%%%%%%%%%%%%%%%%%%%%%%%%%%%%%%%%%%%%%%%%%%%%%%%%%%%%%%%%%%%%%%%%%%%%%%%%



\begin{myslide}

\header{Sufficiently typed type-driven access}

\blau{Look up all elements of a given type}

\medskip

\begin{Verbatim}[fontfamily=courier,fontsize=\small,commandchars=\\\{\}]
 hOccursMany :: Typeable a \carr HList \farr [a]
 hOccursMany [] = []
 hOccursMany (h:t)
   = case unCastable h of
       Just a  \farr a : hOccursMany t
       Nothing \farr hOccursMany t
\end{Verbatim}

\vspace{-66\in}

\blau{Demo}

\medskip

\begin{Verbatim}[fontfamily=courier,fontsize=\tiny,commandchars=\\\{\}]
 \myprompt \inp{hOccursMany angus :: [Breed]}
 \out{[Cow]}
\end{Verbatim}

\end{myslide}



%%%%%%%%%%%%%%%%%%%%%%%%%%%%%%%%%%%%%%%%%%%%%%%%%%%%%%%%%%%%%%%%%%%%%%%%%%%%%%



\begin{myslide}

\header{Insufficiently typed type-driven access}

\blau{Unambiguous look-up}

\medskip

\begin{Verbatim}[fontfamily=courier,fontsize=\small,commandchars=\\\{\}]
 hOccurs :: Typeable a \carr HList \farr Maybe a
 hOccurs hl = case hOccursMany hl of
               [x] \farr Just x
               _   \farr Nothing
\end{Verbatim}

\vspace{-66\in}

\blau{Demo~---~we are lucky}

\medskip

\begin{Verbatim}[fontfamily=courier,fontsize=\tiny,commandchars=\\\{\}]
 \myprompt \inp{hOccurs angus :: Maybe Breed}
 \out{Just Cow}
\end{Verbatim}

\end{myslide}



%%%%%%%%%%%%%%%%%%%%%%%%%%%%%%%%%%%%%%%%%%%%%%%%%%%%%%%%%%%%%%%%%%%%%%%%%%%%%%



\begin{myslide}

\header{Strongly typed heterogeneous lists}

\vspace{-66\in}

\blau{The strongly typed cow Angus}

\medskip

\begin{Verbatim}[fontfamily=courier,fontsize=\small,commandchars=\\\{\}]
 angus =  Key 42
      .*. Name "Angus"
      .*. Cow
      .*. Price 75.5
      .*. HNil
\end{Verbatim}

\vspace{-66\in}

\blau{Lists as nested, binary, right-associative products}

\medskip

\begin{Verbatim}[fontfamily=courier,fontsize=\small,commandchars=\\\{\}]
 data  HNil      = HNil
 data  HCons e l = HCons e l
 type  e :*: l   = HCons e l
       e .*. l   = HCons e l
 class HList
 instance HList HNil
 instance HList l \carr HList (HCons e l)
\end{Verbatim}

\end{myslide}



%%%%%%%%%%%%%%%%%%%%%%%%%%%%%%%%%%%%%%%%%%%%%%%%%%%%%%%%%%%%%%%%%%%%%%%%%%%%%%



\begin{myslide}

\header{Type-level \emph{hOccursMany}}

\vspace{-77\in}

\blau{Signature $\leadsto$ class}

\smallskip

\begin{Verbatim}[fontfamily=courier,fontsize=\small,commandchars=\\\{\}]
 class  HOccursMany e l
  where hOccursMany :: l \farr [e]
\end{Verbatim}

\vspace{-77\in}

\blau{``\emph{[]}'' equation $\leadsto$ \emph{HNil} instance}

\smallskip

\begin{Verbatim}[fontfamily=courier,fontsize=\small,commandchars=\\\{\}]
 instance HOccursMany e \textbf{HNil}
  where   hOccursMany _ = []
\end{Verbatim}

\vspace{-77\in}

\blau{Type case $\leadsto$ overlapping instances}

\smallskip

\begin{Verbatim}[fontfamily=courier,fontsize=\small,commandchars=\\\{\}]
 instance HOccursMany e l
       \carr \textbf{HOccursMany e (HCons e l)}
  where hOccursMany (HCons e l) = e : hOccursMany l

 instance HOccursMany e l
       \carr \textbf{HOccursMany e (HCons e' l)}
  where hOccursMany (HCons _ l) = hOccursMany l
\end{Verbatim}

\end{myslide}



%%%%%%%%%%%%%%%%%%%%%%%%%%%%%%%%%%%%%%%%%%%%%%%%%%%%%%%%%%%%%%%%%%%%%%%%%%%%%%



\begin{myslide}

\header{Unambiguous type-driven look-up}

\bigskip

\begin{Verbatim}[fontfamily=courier,commandchars=\\\{\}]
 class  HOccurs e l
  where hOccurs :: l \farr e  
\end{Verbatim}

\bigskip

Challenges

\begin{itemize}
\item Documentation of type errors
\item Test for absence of type
\end{itemize}

\end{myslide}



%%%%%%%%%%%%%%%%%%%%%%%%%%%%%%%%%%%%%%%%%%%%%%%%%%%%%%%%%%%%%%%%%%%%%%%%%%%%%%



\begin{myslide}

\header{Unambiguous type-driven look-up (cont'd)}

\vspace{-77\in}

\blau{The \emph{HNil} instance for type-error documentation}

\smallskip

\begin{Verbatim}[fontfamily=courier,fontsize=\small,commandchars=\\\{\}]
 instance Fail (TypeNotFound e) \carr HOccurs e HNil
  where hOccurs = \undefined
\end{Verbatim}

\vspace{-77\in}

\blau{A type-level error code}

\smallskip

\begin{Verbatim}[fontfamily=courier,fontsize=\small,commandchars=\\\{\}]
 data TypeNotFound e -- no values, no operations! 
\end{Verbatim}

\vspace{-77\in}

\blau{Any Prelude should have it!}

\smallskip

\begin{Verbatim}[fontfamily=courier,fontsize=\small,commandchars=\\\{\}]
 class Fail x -- no methods, no instances!
\end{Verbatim}

\vspace{-77\in}

\blau{Demo}

\smallskip

\begin{Verbatim}[fontfamily=courier,fontsize=\small,commandchars=\\\{\}]
 \myprompt \inp{hOccurs (HCons True HNil) :: Int}
 \out{No instance for (\textbf{Fail (TypeNotFound Int)})}
\end{Verbatim}


\end{myslide}



%%%%%%%%%%%%%%%%%%%%%%%%%%%%%%%%%%%%%%%%%%%%%%%%%%%%%%%%%%%%%%%%%%%%%%%%%%%%%%



\begin{myslide}

\header{Unambiguous type-driven look-up (cont'd)}

\vspace{-77\in}

\blau{`Type found' instance}

\smallskip

\begin{Verbatim}[fontfamily=courier,fontsize=\small,commandchars=\\\{\}]
 instance \textbf{HOccursNot e l} \carr HOccurs e (HCons e l)
  where hOccurs (HCons e _) = e
\end{Verbatim}

\vspace{-77\in}

\blau{`Type not yet found' instance}

\smallskip

\begin{Verbatim}[fontfamily=courier,fontsize=\small,commandchars=\\\{\}]
 instance HOccurs e l \carr HOccurs e (HCons e' l)
  where hOccurs (HCons _ l) = hOccurs l
\end{Verbatim}

\end{myslide}



%%%%%%%%%%%%%%%%%%%%%%%%%%%%%%%%%%%%%%%%%%%%%%%%%%%%%%%%%%%%%%%%%%%%%%%%%%%%%%



\begin{myslide}

\header{Unambiguous type-driven look-up (cont'd)}

\vspace{-77\in}

\blau{A class for testing the absence of a type}

\bigskip

\begin{Verbatim}[fontfamily=courier,fontsize=\small,commandchars=\\\{\}]
 data TypeFound e     -- for a failure instance
\end{Verbatim}

\medskip

\begin{Verbatim}[fontfamily=courier,fontsize=\small,commandchars=\\\{\}]
 class HOccursNot e l -- no methods!
\end{Verbatim}

\medskip

\begin{Verbatim}[fontfamily=courier,fontsize=\small,commandchars=\\\{\}]
 instance HOccursNot e HNil
\end{Verbatim}

\medskip

\begin{Verbatim}[fontfamily=courier,fontsize=\small,commandchars=\\\{\}]
 instance HOccursNot e l
       \carr HOccursNot e (HCons e' l)
\end{Verbatim}

\medskip

\begin{Verbatim}[fontfamily=courier,fontsize=\small,commandchars=\\\{\}]
 instance Fail (TypeFound e)
       \carr HOccursNot e (HCons e l)
\end{Verbatim}

\end{myslide}




%%%%%%%%%%%%%%%%%%%%%%%%%%%%%%%%%%%%%%%%%%%%%%%%%%%%%%%%%%%%%%%%%%%%%%%%%%%%%%



\begin{myslide}

\header{Let's try to get rid of overlapping}

\blau{Why?}

\medskip

\begin{itemize}
\item Overlapping instances generally debated
\item Overlapping + functional dependencies problematic
\item Lazy instance selection required
\item Weak correspondence value vs. type level
\end{itemize}

\end{myslide}



%%%%%%%%%%%%%%%%%%%%%%%%%%%%%%%%%%%%%%%%%%%%%%%%%%%%%%%%%%%%%%%%%%%%%%%%%%%%%%



\begin{myslide}

\header{Type-level \emph{hOccurs}~---~with type equality}

\vspace{-77\in}

\blau{One \emph{HCons} instance}

\bigskip

\begin{Verbatim}[fontfamily=courier,fontsize=\small,commandchars=\\\{\}]
 instance ( TypeEq e e' b
          , \textbf{HOccursBool} b e (HCons e' l) 
          )
            \carr HOccurs e (HCons e' l)
  where
   hOccurs (HCons e' l) = e
    where
     e = hOccursBool b (HCons e' l)
     b = typeEq e e'
\end{Verbatim}

{\small

Case-discriminate on type-level Booleans in \emph{HOccursBool}.

}

\end{myslide}



%%%%%%%%%%%%%%%%%%%%%%%%%%%%%%%%%%%%%%%%%%%%%%%%%%%%%%%%%%%%%%%%%%%%%%%%%%%%%%



\begin{myslide}

\header{Type-level \emph{hOccurs}~---~with cast}

\vspace{-77\in}

\blau{Type-level maybies}

\begin{Verbatim}[fontfamily=courier,fontsize=\small,commandchars=\\\{\}]
 class HMaybe (hmaybe :: * \farr *)
 instance HMaybe HNothing
 instance HMaybe HJust
\end{Verbatim}

\vspace{-77\in}

\blau{One \emph{HCons} instance}

\bigskip

\begin{Verbatim}[fontfamily=courier,fontsize=\small,commandchars=\\\{\}]
 instance ( Cast e' m e
          , \textbf{HOccursMaybe} m e l 
          )
            \carr HOccurs e (HCons e' l)
  where
   hOccurs (HCons e' l) = hOccursMaybe (cast e') l
\end{Verbatim}

{\small

Case-discriminate on type-level maybies in \emph{HOccursMaybe}.

}

\end{myslide}



%%%%%%%%%%%%%%%%%%%%%%%%%%%%%%%%%%%%%%%%%%%%%%%%%%%%%%%%%%%%%%%%%%%%%%%%%%%%%%



\begin{myslide}

\header{Intermediate summary}

{\small

We can treat nested products like lists and arrays.

We can handle all kinds of type-driven access operations.

We can get rid of overlapping instances.

We have to disclose classes \emph{TypeEq} and \emph{Cast}.

Verbosity can be attacked by simple preprocessing.

\bigskip
\bigskip

Let's do extensible records now.

}

\end{myslide}



%%%%%%%%%%%%%%%%%%%%%%%%%%%%%%%%%%%%%%%%%%%%%%%%%%%%%%%%%%%%%%%%%%%%%%%%%%%%%%



\begin{myslide}

\header{Haskell's nonextensible records}

\vspace{-66\in}

\blau{Unpriced cows get stuck}

\medskip

\begin{Verbatim}[fontfamily=courier,fontsize=\tiny,commandchars=\\\!\?]
 data Unpriced = Unpriced { key   :: Integer
                          , name  :: String
                          , breed :: Breed }
\end{Verbatim}

\smallskip

\begin{Verbatim}[fontfamily=courier,fontsize=\tiny,commandchars=\\\!\?]
 unpricedAngus = Unpriced { key    = 42
                          , name   = "Angus"
                          , breed  = Cow }
\end{Verbatim}

\vspace{-66\in}

\blau{Demo~---~not much is feasible}

\smallskip

\begin{Verbatim}[fontfamily=courier,fontsize=\small,commandchars=\\\!\?]
 \myprompt \inp!breed unpricedAngus?
 \out!Cow?
 \myprompt \inp!unpricedAngus { breed = Sheep }?
 \out!Unpriced{key=42,name="Angus",breed=Sheep}?
\end{Verbatim}

{\small

No extensibility~---~unless we use polymorphic dummy fields.\\
No reuse of labels. Neither can we treat labels as data.

}

\end{myslide}



%%%%%%%%%%%%%%%%%%%%%%%%%%%%%%%%%%%%%%%%%%%%%%%%%%%%%%%%%%%%%%%%%%%%%%%%%%%%%%



\begin{myslide}

\header{\HList's extensible records}

\vspace{-77\in}

\blau{Record labels live in namespaces}

\smallskip

\begin{Verbatim}[fontfamily=courier,fontsize=\small,commandchars=\\\{\}]
 data FootNMouth = FootNMouth  -- a namespace
\end{Verbatim}

\vspace{-77\in}

\blau{Record labels are values}

\smallskip

\begin{Verbatim}[fontfamily=courier,fontsize=\small,commandchars=\\\{\}]
 key   = firstLabel FootNMouth "key"
 name  = nextLabel  key        "name"
 breed = nextLabel  name       "breed"
 price = nextLabel  breed      "price"  
\end{Verbatim}

\vspace{-77\in}

\blau{The unpriced cow Angus who won't get stuck}

\smallskip

\begin{Verbatim}[fontfamily=courier,fontsize=\small,commandchars=\\\{\}]
 unpricedAngus =  key    .=. (42::Integer)
              .*. name   .=. "Angus"
              .*. breed  .=. Cow
              .*. emptyRecord
\end{Verbatim}

\end{myslide}



%%%%%%%%%%%%%%%%%%%%%%%%%%%%%%%%%%%%%%%%%%%%%%%%%%%%%%%%%%%%%%%%%%%%%%%%%%%%%%



\begin{myslide}

\header{\HList's extensible records (cont'd)}

\vspace{-77\in}

\blau{\emph{show} looks good.}

\smallskip

\begin{Verbatim}[fontfamily=courier,fontsize=\small,commandchars=\\\!\?]
 \myprompt \inp!unpricedAngus?
 Record{key=42,name="Angus",breed=Cow}
\end{Verbatim}

\vspace{-77\in}

\blau{Selection works fine.}

\begin{Verbatim}[fontfamily=courier,fontsize=\small,commandchars=\\\|\?]
 \myprompt \inp|unpricedAngus .!. breed?
 \out|Cow?
\end{Verbatim}

\vspace{-77\in}

\blau{Update works fine.}

\begin{Verbatim}[fontfamily=courier,fontsize=\small,commandchars=\\\!\?]
 \myprompt \inp!unpricedAngus .@. breed .=. Sheep?
 \out!Record{key=42,name="Angus",breed=Sheep}?
\end{Verbatim}

\vspace{-77\in}

\blau{Finally, extensions works, too.}

\begin{Verbatim}[fontfamily=courier,fontsize=\small,commandchars=\\\!\?]
 \myprompt \inp!price .=. 8.8 .*. unpricedAngus?
 \out!Record{price=8.8,key=42,name="Angus",breed=Cow}?
\end{Verbatim}

\end{myslide}



%%%%%%%%%%%%%%%%%%%%%%%%%%%%%%%%%%%%%%%%%%%%%%%%%%%%%%%%%%%%%%%%%%%%%%%%%%%%%%



\begin{myslide}

\header{One model of extensible records}

\vspace{-77\in}

\blau{Labels using type-level naturals}

\smallskip

\begin{Verbatim}[fontfamily=courier,fontsize=\tiny,commandchars=\\\{\}]
 data HNat x \carr Label x ns = Label x ns String
 firstLabel = Label hZero
 nextLabel (Label x ns _) = Label (hSucc x) ns
\end{Verbatim}

\vspace{-77\in}

\blau{Records as maps from labels to values}

\smallskip

\begin{Verbatim}[fontfamily=courier,fontsize=\tiny,commandchars=\\\{\}]
 newtype Record r = Record r
\end{Verbatim}

\smallskip

\begin{Verbatim}[fontfamily=courier,fontsize=\tiny,commandchars=\\\{\}]
 mkRecord :: (HZip ls vs r, HLabelSet ls) \carr r \farr Record r
 mkRecord = Record
\end{Verbatim}

\vspace{-77\in}

\blau{Helper functionality}

\smallskip

\begin{Verbatim}[fontfamily=courier,fontsize=\tiny,commandchars=\\\!\?]
 class HZip x y l | x y \farr l, l \farr x y
  where hZip :: x \farr y \farr l
        hUnzip :: l \farr (x,y)
\end{Verbatim}

\smallskip

\begin{Verbatim}[fontfamily=courier,fontsize=\tiny,commandchars=\\\{\}]
 class HLabelSet ls -- no label occurs twice
 class HBool b \carr HMember e l b | e l \farr b
 class HBool b \carr HEq x y b | x y \farr b
\end{Verbatim}

\smallskip

\begin{Verbatim}[fontfamily=courier,fontsize=\tiny,commandchars=\\\{\}]
 instance HEq x x' b \carr HEq (Label x ns) (Label x' ns) b
\end{Verbatim}

\end{myslide}



%%%%%%%%%%%%%%%%%%%%%%%%%%%%%%%%%%%%%%%%%%%%%%%%%%%%%%%%%%%%%%%%%%%%%%%%%%%%%%



\begin{myslide}

\header{Access operations for extensible records}

\vspace{-66\in}

\blau{Deletion}

\begin{Verbatim}[fontfamily=courier,fontsize=\small,commandchars=\\\{\}]
 (Record r) .-. l = Record r'
   where
    (ls,vs) = hUnzip r
    n       = hFind l ls -- uses HEq on labels
    ls'     = hDeleteAtHNat n ls
    vs'     = hDeleteAtHNat n vs
    r'      = hZip ls' vs'
\end{Verbatim}

\vspace{-66\in}

\blau{Rename}

\begin{Verbatim}[fontfamily=courier,fontsize=\small,commandchars=\\\{\}]
 hRenameLabel l l' r = r'' where
   v   = r  .@. l        -- look up by label
   r'  = r  .-. l        -- delete at label
   r'' = l' .=. v .*. r' -- re-add component
\end{Verbatim}

\end{myslide}



%%%%%%%%%%%%%%%%%%%%%%%%%%%%%%%%%%%%%%%%%%%%%%%%%%%%%%%%%%%%%%%%%%%%%%%%%%%%%%



\begin{myslide}

\header{\HList's type-indexed products}

\vspace{-77\in}

\blau{Angus goes tipy}

\smallskip

\begin{Verbatim}[fontfamily=courier,fontsize=\small,commandchars=\\\{\}]
 \myprompt \inp{let myTipyCow = TIP angus}
\end{Verbatim}

\vspace{-77\in}

\blau{Sad Angus suffers BSE now}

\smallskip

\begin{Verbatim}[fontfamily=courier,fontsize=\small,commandchars=\\\{\}]
 \myprompt \inp{BSE .*. myTipyCow}
 \out{TIP (HCons BSE ...)}
\end{Verbatim}

\vspace{-77\in}

\blau{Angus is really a cow}

\smallskip

\begin{Verbatim}[fontfamily=courier,fontsize=\small,commandchars=\\\{\}]
 \myprompt \inp{Sheep .*. myTipyCow}
 \out{No instance for (Fail (TypeFound Breed))}
\end{Verbatim}

\end{myslide}



%%%%%%%%%%%%%%%%%%%%%%%%%%%%%%%%%%%%%%%%%%%%%%%%%%%%%%%%%%%%%%%%%%%%%%%%%%%%%%



\begin{myslide}

\header{The implementation of TIPs}

\vspace{-77\in}

\blau{Use a newtype to record TIP status}

\smallskip

\begin{Verbatim}[fontfamily=courier,fontsize=\small,commandchars=\\\{\}]
 newtype TIP l  = TIP l -- to be constrained
 unTIP  (TIP l) = l
\end{Verbatim}
\smallskip
\begin{Verbatim}[fontfamily=courier,fontsize=\small,commandchars=\\\{\}]
 mkTIP :: HTypeIndexed l \carr l \farr TIP l
 mkTIP = TIP
\end{Verbatim}

\vspace{-77\in}

\blau{The TIP constraint}

\smallskip

\begin{Verbatim}[fontfamily=courier,fontsize=\small,commandchars=\\\{\}]
 class HTypeIndexed l
 instance HTypeIndexed HNil
 instance ( HOccursNot e l
          , HTypeIndexed l
          )
           \carr HTypeIndexed (HCons e l)
\end{Verbatim}

\end{myslide}



%%%%%%%%%%%%%%%%%%%%%%%%%%%%%%%%%%%%%%%%%%%%%%%%%%%%%%%%%%%%%%%%%%%%%%%%%%%%%%



\begin{myslide}

\header{Subtyping? Here we go.}

\vspace{-77\in}

\blau{Subtyping for TIPs via occurrence check}

\smallskip

\begin{Verbatim}[fontfamily=courier,fontsize=\tiny,commandchars=\\\{\}]
 class SubType l l'
 instance SubType (TIP l) (TIP HNil)
 instance ( HOccurs e l
          , SubType (TIP l) (TIP l')
          )
            \carr SubType (TIP l) (TIP (HCons e l'))
\end{Verbatim}

\vspace{-77\in}

\blau{Subtyping for records via projection}

\smallskip

\begin{Verbatim}[fontfamily=courier,fontsize=\tiny,commandchars=\\\{\}]
 instance ( HZip ls vs r'
          , HProjectByLabels ls (Record r) (Record r')
          )
            \carr SubType (Record r) (Record r')
\end{Verbatim}

{\small

Guess what the definition of type equality is.

}

\end{myslide}



%%%%%%%%%%%%%%%%%%%%%%%%%%%%%%%%%%%%%%%%%%%%%%%%%%%%%%%%%%%%%%%%%%%%%%%%%%%%%%



\begin{myslide}

\header{An idiom for constraint annotation}

\vspace{-77\in}

\blau{The verbose way}

\smallskip

\begin{Verbatim}[fontfamily=courier,fontsize=\small,commandchars=\\\{\}]
 animalKey :: ( SubType l (TIP Animal) -- extra
              , HOccurs Key l          -- implied
              ) \carr l \farr Key
 animalKey = hOccurs
\end{Verbatim}

\vspace{-77\in}

\blau{The concise way}

\smallskip

\begin{Verbatim}[fontfamily=courier,fontsize=\small,commandchars=\\\{\}]
 animalKey l = hOccurs (animalish l) :: Key
\end{Verbatim}

\smallskip

\begin{Verbatim}[fontfamily=courier,fontsize=\small,commandchars=\\\{\}]
 animalish :: SubType l (TIP Animal) \carr l \farr l
 animalish = id
\end{Verbatim}

\vspace{-77\in}

\blau{Demo}

\smallskip

\begin{Verbatim}[fontfamily=courier,fontsize=\tiny,commandchars=\\\{\}]
 \myprompt \inp{animalKey myTipyCow}
 \out{Key 42}
 \myprompt animalKey (Key 42 .*. emptyTIP)
 \out{No instances for (Fail (TypeNotFound Price),}
 \out{                  Fail (TypeNotFound Breed),}
 \out{                  Fail (TypeNotFound Name))}
\end{Verbatim}

\end{myslide}



%%%%%%%%%%%%%%%%%%%%%%%%%%%%%%%%%%%%%%%%%%%%%%%%%%%%%%%%%%%%%%%%%%%%%%%%%%%%%%



\begin{myslide}

\header{Type-improvement constraints}

\vspace{-77\in}

\blau{Too much polymorphism}

\smallskip

\begin{Verbatim}[fontfamily=courier,fontsize=\small,commandchars=\\\{\}]
 \myprompt \inp{hOccurs (HCons True HNil)}
 \out{No instance for (HOccurs e (HCons Bool HNil))}
\end{Verbatim}

\vspace{-77\in}

\blau{Use unconditional type cast for type improvement}

\smallskip

\begin{Verbatim}[fontfamily=courier,fontsize=\tiny,commandchars=\\\{\}]
 instance \textbf{TypeCast e' e}
       \carr HOccurs e (TIP (\textbf{HCons e' HNil}))
 where    hOccurs (TIP (HCons e' _)) = typeCast e'
\end{Verbatim}
\smallskip
\begin{Verbatim}[fontfamily=courier,fontsize=\tiny,commandchars=\\\{\}]
instance HOccurs e (HCons x (HCons y l))
      \carr HOccurs e (TIP (\textbf{HCons x (HCons} y l)))
 where hOccurs (TIP l) = hOccurs l
\end{Verbatim}

\vspace{-77\in}

\blau{Instance-level type improvement works!}

\smallskip

\begin{Verbatim}[fontfamily=courier,fontsize=\small,commandchars=\\\{\}]
 \myprompt \inp{hOccurs (True .*. emptyTIP)}
 \out{True}
\end{Verbatim}

\end{myslide}



%%%%%%%%%%%%%%%%%%%%%%%%%%%%%%%%%%%%%%%%%%%%%%%%%%%%%%%%%%%%%%%%%%%%%%%%%%%%%%



\begin{myslide}

\header{Computations on types}

\vspace{-66\in}

\blau{Compute-type level equality}

\smallskip

\begin{Verbatim}[fontfamily=courier,fontsize=\small,commandchars=\\\{\}]
 class HBool b \carr TypeEq x y b | x y \farr b
\end{Verbatim}

\smallskip

\begin{Verbatim}[fontfamily=courier,fontsize=\tiny,commandchars=\\\{\}]
 data Proxy e; proxy :: Proxy e; proxy = \undefined
 proxyEq :: TypeEq t t' b \carr Proxy t \farr Proxy t' \farr b
 proxyEq _ _ = \undefined
\end{Verbatim}

\vspace{-77\in}

\blau{Type-level type-safe cast}

\smallskip

\begin{Verbatim}[fontfamily=courier,fontsize=\small,commandchars=\\\{\}]
 class HMaybe m \carr Cast x m y | x y \farr m
  where cast :: x \farr m y
\end{Verbatim}

\vspace{-77\in}

\blau{Reified type unification}

\smallskip

\begin{Verbatim}[fontfamily=courier,fontsize=\small,commandchars=\\\{\}]
 class TypeCast x y | x \farr y, y \farr x
  where typeCast :: x \farr y
\end{Verbatim}

\end{myslide}



%%%%%%%%%%%%%%%%%%%%%%%%%%%%%%%%%%%%%%%%%%%%%%%%%%%%%%%%%%%%%%%%%%%%%%%%%%%%%%



\begin{myslide}

\header{Want to see an implementation?}

\vspace{-77\in}

\blau{Here is one.}

\vspace{-77\in}

{\tiny (Requires GHC and a separate compilation trick.)}

\bigskip

\begin{Verbatim}[fontfamily=courier,fontsize=\tiny,commandchars=\\\{\}]
  instance TypeEq x x HTrue
  instance (HBool b, TypeCast HFalse b) \carr TypeEq x y b
\end{Verbatim}
\medskip
\begin{Verbatim}[fontfamily=courier,fontsize=\tiny,commandchars=\\\{\}]
  instance Cast x x HJust
   where cast x = HJust x
  instance TypeCast HNothing m \carr Cast x y m
   where cast x = HNothing
\end{Verbatim}
\medskip
\begin{Verbatim}[fontfamily=courier,fontsize=\tiny,commandchars=\\\{\}]
  instance TypeCast x x
   where typeCast = id
\end{Verbatim}

\end{myslide}



%%%%%%%%%%%%%%%%%%%%%%%%%%%%%%%%%%%%%%%%%%%%%%%%%%%%%%%%%%%%%%%%%%%%%%%%%%%%%%



\begin{myslide}

\header{Want to see an implementation?}

\vspace{-77\in}

\blau{Here is another one.}

\vspace{-77\in}

{\tiny (Requires GHC and a `two-class' trick.)}

\bigskip

\begin{Verbatim}[fontfamily=courier,fontsize=\tiny,commandchars=\\\{\}]
 instance TypeCast'  () a b \carr TypeCast a b
    where typeCast x = typeCast' () x
\end{Verbatim}
\smallskip
\begin{Verbatim}[fontfamily=courier,fontsize=\tiny,commandchars=\\\{\}]
 class TypeCast'  t a b | t a \farr b, t b \farr a
  where typeCast'  :: t \farr a \farr b
\end{Verbatim}
\smallskip
\begin{Verbatim}[fontfamily=courier,fontsize=\tiny,commandchars=\\\{\}]
 class TypeCast'' t a b | t a \farr b, t b \farr a
  where typeCast'' :: t \farr a \farr b
\end{Verbatim}
\smallskip
\begin{Verbatim}[fontfamily=courier,fontsize=\tiny,commandchars=\\\{\}]
 instance TypeCast'' t a b \carr TypeCast' t a b
    where typeCast' = typeCast''
\end{Verbatim}
\smallskip
\begin{Verbatim}[fontfamily=courier,fontsize=\tiny,commandchars=\\\{\}]
 instance TypeCast'' () a a
    where typeCast'' _ x  = x
\end{Verbatim}

\end{myslide}



%%%%%%%%%%%%%%%%%%%%%%%%%%%%%%%%%%%%%%%%%%%%%%%%%%%%%%%%%%%%%%%%%%%%%%%%%%%%%%



\begin{myslide}

\header{Yet another implementation}

\vspace{-77\in}

{\tiny (Works fine with modest extensions.)}

\vspace{-42\in}

\blau{Refiy \emph{Data.Typeable}}

\medskip

\begin{Verbatim}[fontfamily=courier,fontsize=\small,commandchars=\\\{\}]
 class TTypeable a b | a \farr b
 instance TTypeable Bool (HCons HZero HNil)
 instance TTypeable Int  (HCons (HSucc HZero) HNil)
 instance (TTypeable a al, TTypeable b bl)
  \carr TTypeable (a \farr b)
        (HCons (HSucc (HSucc HZero))
               (HCons al (HCons bl HNil)))
\end{Verbatim}

\vspace{-77\in}

\blau{Reify type equality}

\medskip

\begin{Verbatim}[fontfamily=courier,fontsize=\small,commandchars=\\\{\}]
 instance ( TTypeable t tt, TTypeable t' tt'
          , HEq tt tt' b ) \carr TypeEq t t' b
\end{Verbatim}

\end{myslide}



%%%%%%%%%%%%%%%%%%%%%%%%%%%%%%%%%%%%%%%%%%%%%%%%%%%%%%%%%%%%%%%%%%%%%%%%%%%%%%



\begin{myslide}

\bigskip

\header{Conclusion and future work}

\medskip

{\small

\littleskip\begin{itemize}

\item \HList\ is readily useful.

\item Haskell's type system is open.

\item Do we want a marriage of reification, macros, templates?

\item A preprocessor for type-level programming!

\begin{Verbatim}[fontseries=normal,fontsize=\tiny,commandchars=\\\{\}]
hOccurs :: Typeable a \carr HList \farr a
hOccurs hl = head many
 where many = hOccursMany hl
 \textbf{require} length many == 1
\end{Verbatim}

\end{itemize}

Thank you for your attention.

}

\end{myslide}



%%%%%%%%%%%%%%%%%%%%%%%%%%%%%%%%%%%%%%%%%%%%%%%%%%%%%%%%%%%%%%%%%%%%%%%%%%%%%%



\begin{myslide}

\header{The future of Haskell}

\vspace{-77\in}

\blau{Getting rid of overlapping instances}

\medskip

\begin{Verbatim}[fontseries=normal,fontsize=\small,commandchars=\\\{\}]
 -- A good reason for overlapping?
 class Show x 
 instance Show [x]
 instance Show [Char]

 -- Rather do it like this!
 instance ( TypeEq x Char bool
          , ShowList bool x 
          ) => Show [x]

 class ShowList bool x
 instance ShowList HTrue Char -- handle String
 instance ShowList HFalse x   -- normal lists
\end{Verbatim}

\end{myslide}



\begin{myslide}

\header{Getting rid of overlapping instances cont'd}

\medskip

\begin{Verbatim}[fontseries=normal,fontsize=\small,commandchars=\\\{\}]
 -- Another good reason for overlapping?
 class Foo x
 instance Foo Int
 instance Foo Char
 instance Foo a -- otherwise?

 -- Rather do it like this!
 instance 
   ( HType2HNat a (Int :*: Char :*: HNil) nat
   , Foo' a nat
   ) => Foo a
 class Foo' a nat
 instance Foo' a (HSucc HZero)
 instance Foo' a (HSucc (HSucc HZero))
 instance Foo' a HZero -- otherwise?
\end{Verbatim}


\end{myslide}



%%%%%%%%%%%%%%%%%%%%%%%%%%%%%%%%%%%%%%%%%%%%%%%%%%%%%%%%%%%%%%%%%%%%%%%%%%%%%%



\end{document}
